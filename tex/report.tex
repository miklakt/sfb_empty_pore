\documentclass[12pt,a4paper]{article}
%%%%%%%%%%%%%%%%%%%%%%%%%%%%%%%%%%%%%%%%%%%%%%%%%%%%%%%%%%%%%%%%%%%%%%%%%%%%%%%%
\usepackage{graphicx, overcite}
\graphicspath{.}

\begin{document}

\title{Diffusion through polymer brush membrane and polymer brush pore.}

\author{Mikhail Y.Laktionov}

\maketitle

\begin{abstract}
Effective diffusion coefficient has been calculated using insertion free energy profiles for planar polymer brush.
\end{abstract}

\section{Insertion free energy calculation}

Using the results from our previous paper [] we can calculate polymer density profiles for polymer brush. 
As has been shown the results of SF-SCF numerical are close enough to SS-SCF analytical results this allows to use less computationally demanding analytical SS-SCF approach.

A polymer brush pore has been calculated using SF-SCF approach.

We suggest that free energy penalty can be split in two terms:

\begin{equation}
    F = \Pi(\phi, \chi_{PS}) \cdot V + \gamma(\phi, \chi_{ads}) \cdot S
    \label{fe}
\end{equation}

where $\Pi$ is osmotic pressure,
$\gamma$ is surface related coefficient (the excess of surface free energy per unit area),
$\phi$ is local polymer density,
$\chi_{PS}$ is Flory parameter for polymer-solvent interaction,
$\chi_{ads} \equiv \chi_{PC} - \chi_{PS}(1-\phi)$ is universal measure of particle attractiveness to a polymer brush (free energy change upon replacement of a contact of monomer unit with solvent by a contact with the surface).

In the paper[] we have shown that both terms depend on local polymer only, thus knowing the polymer density and interaction parameters we can calculate insertion free energy. 

The first osmotic term in eq \ref{fe} can be calculated using analytical expression for osmotic pressure

\begin{equation}
	\Pi(z)= k_BT[-\ln(1-\phi(z)) - \phi(z) -\chi_{PS}\phi^2(z)]
	\label{pi}
\end{equation}

to calculate excess surface free energy we have to calculate $\gamma(\phi, \chi_{ads})$, as there is no analytical expression to calculate $\gamma$, we use the next expression obtained by fitting SF-SCF numerical data

\begin{equation}
    \gamma(\chi_{ads}, \phi)=(\chi_{ads}-\chi_{ads}^{(crit)})(a_1\phi+a_2\phi^{2})
    \label{gamma_fit}
\end{equation}
    
where $\chi_{ads}=\phi_{PC}-\phi_{PS}(1-\phi)$ and $a_1,a_2$ are numerical fitting parameters and we neglected higher order terms in $\phi$ in the last bracket in eq \ref{gamma_fit}. The value of $\chi_{ads}^{(crit)}$ depends on the local properties of the lattice where polymer chain performs its random walk. For the simple cubic lattice $\chi_{ads}^{(crit)}=6ln(5/6)$.
The rough estimate of $a_1$ and $a_2$ are $0.16$ and $0.08$, respectively.

\section{Planar polymer brush membrane}

Consider planar polymer brush membrane \ref{fig:membrane}, let us examine effective diffusion coefficient. Here we will not take into account any possible interaction between the film (or any other substrate the chains are grafted on) and a particle. 

For the next calculation we selected degree of polymerization $N=300$ and grafting density $\sigma = 0.02$, so that the parameters are consistent with the ones found in real biological systems like nucleopores[].

\begin{figure}
    \center
    \includegraphics[width =2in]{membrane.pdf}
    \caption{Planar polymer brush membrane}
    \label{fig:membrane}
\end{figure}

The brush thickness and polymer density distribution depend on the solvent regime. Smooth and wide distribution of the polymer density corresponds to good solvent regimes $\chi_{PS} \leq 0.5$, whilst in poor solvent regime boxlike distribution observed (fig \ref{fig:phi_planar}). Clearly, free energy penalty profiles in the poor solvent regime will also appear boxlike.

\begin{figure}
    \center
    \includegraphics[width =\textwidth]{phi_plain.pdf}
    \caption{Polymer density distribution for planar polymer brush with $N=300$, $\sigma = 0.02$ at different solvent regimes.}
    \label{fig:phi_planar}
\end{figure}

Polymer density distributions allow us to calculate insertion free energy profiles, that are also depend on particle size and interaction parameter of polymer-particle interaction $\chi_{PC}$. 

Take arbitrary particle size $d=10$ and a set of polymer-particle interaction parameters from repulsive to inert and attractive. In the figure \ref{fig:fe_planar} the frames show insertion free energy profile at different solvent regimes from a good solvent regime on the left to poorer ones on the right. Line colors are corresponds to different $\chi_{PC}$. 
One can see that in the poor solvent free energy profiles are boxlike for every $\chi_{PC}$, the magnitude is also increasing when $\chi_{PS}$ is increasing.

\begin{figure}
    \center
    \includegraphics[width =\textwidth]{free_energy.pdf}
    \caption{Insertion free energy penalty for planar polymer brush with $N=300$, $\sigma = 0.02$ at different solvent regimes and polymer-particle interaction parameters.}
    \label{fig:fe_planar}
\end{figure}

To calculate effective diffusion $D_{eff}$ through the membrane we mirror free energy profile to make it an even function and than we integrate obtained function using the next equation

\begin{equation}
    D_{eff} = D \frac{1}{2H}\left(\int_{-H}^{H}e^{U(x)}dx\right)^{-1}
\end{equation}

where $H$ is polymer brush thickness. Here we consider diffusion coefficient $D=1$.

The dependency of effective diffusion $D_{eff}$ on polymer-particle interaction parameter $\chi_{PC}$ is shown on figure \ref{fig:D_eff_planar}. 
In the poor solvent (red line) the dependency in semi-log plot is almost straight line which is expected consider boxlike free energy profile. In good and theta solvent we can see a plato when $\chi_{PC}$ decreasing, means diminishing results in diffusion rate the higher affinity of a particle to the polymer is. 

\begin{figure}
    \center
    \includegraphics[width =\textwidth]{D_eff_membrane.pdf}
    \caption{The dependency of effective diffusion $D_{eff}$ on polymer-particle interaction parameter $\chi_{PC}$ for planar polymer brush membrane with $N=300$, $\sigma = 0.02$ at different solvent regimes.}
    \label{fig:D_eff_planar}
\end{figure}

\begin{figure}
    \center
    \includegraphics[width =\textwidth]{D_eff_membrane_2.pdf}
    \caption{The dependency of effective diffusion $D_{eff}$ on polymer-solvent interaction parameter $\chi_{PS}$ for planar polymer brush membrane with $N=300$, $\sigma = 0.02$ at different particle affinity.}
    \label{fig:D_eff_planar_2}
\end{figure}

\section{Pore with a polymer brush interior}



\begin{figure}
    \center
    \includegraphics[width =2in]{pore.pdf}
    \caption{Planar polymer brush membrane}
    \label{fig:membrane}
\end{figure}

\begin{figure}
    \center
    \includegraphics[width =\textwidth]{pore_central_axis_phi.pdf}
    \caption{Polymer density distribution in pore center axis ($r=0$) $N=300$, $\sigma = 0.02$ at different solvent regimes.}
    \label{fig:phi_pore}
\end{figure}


\begin{figure}
    \center
    \includegraphics[width =\textwidth]{pore_central_fe.pdf}
    \caption{Free energy penalty to insert a particle of $d=10$ through the central axis of a pore interiored with a polymer brush $N=300$, $\sigma = 0.02$ at different solvent regimes and polymer-particle interaction parameters.}
    \label{fig:fe_planar}
\end{figure}


\end{document}