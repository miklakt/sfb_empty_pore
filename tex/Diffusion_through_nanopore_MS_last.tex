\documentclass[12pt, a4paper]{article}
\usepackage{graphicx}
\usepackage{amsmath, amssymb, amsfonts, mathtools}
\usepackage{subcaption}
\usepackage[
backend=biber,
natbib=true,
style=numeric,
sorting=none,
doi=false,
isbn=false,
url=false,
eprint=false
]{biblatex}
\usepackage{xcolor}
\usepackage{bm}

\usepackage{lineno}
\linenumbers

\newcommand\todo[1]{\textcolor{red}{#1}}

\addbibresource{biblio.bib}
\title{A polymer filling enhances the rate and selectivity of colloid permeation across mesopores}

\author{Mikhail Y. Laktionov$^1$, Leonid I.Klushin$^{2,5}$,\\Ralf P.Richter$^3$, France A.M. Leermakers$^4$, Oleg V.Borisov$^1$\\
$^{1}$CNRS, Universit\'e de Pau et des Pays de l'Adour UMR 5254,\\
Institut des Sciences Analytiques et de Physico-Chimie\\
pour l'Environnement et les Mat\'eriaux, 64053 Pau, France \\
$^{2}$Branch of Petersburg Nuclear Physics Institute\\
named by B.P.Konstantinov \\
of National Research Centre "Kurchatov Institute",\\
Institute of Macromolecular Compounds,\\
199004 St.Petersburg, Russia,\\
$^{3}$University of Leeds, School of Biomedical Sciences, \\
Faculty of Biological Sciences, 
School of Physics and Astronomy, \\
Faculty of Engineering and Physical Sciences,\\  
Astbury Centre for Structural Molecular Biology,\\ 
and Bragg Center for Materials Research,\\ 
Leeds, LS2 9JT, United Kingdom\\
$^{4}$ University of Wageningen, The Netherlands\\
$^{5}$ American University of Beirut, Department of Physics, Lebanon
}
\date{}

\begin{document}
\maketitle

\begin{abstract}

Mesoporous membranes are emerging as new materials with potential applications in sensing and separation devices.
The nuclear envelope of eukaryotic cells provides a striking example of a functional mesoporous membrane, where diffusive transport is mediated by nuclear pore complexes (NPCs).
Transport across NPCs is mediated by a pore-filling meshwork of naturally-disordered proteins (FG-domains of nucleoporins) anchored to the pore walls, and is highly selective.
Even colloids much smaller than the inner diameter of the NPC are effectively blocked from transport, but some larger colloids with distinct surface features can readily permeate.
Simplistically, one may expect any polymer meshwork to slow down colloid movement as the steric constraints imposed by the polymer meshwork hinder permeation.
However, we demonstrate how a rationally designed polymer filling can instead increase the permeation rate, by an order of magnitude and more, compared to an empty pore.
Such enhanced permeation is achieved with a polymer phase that attracts the colloid and extends beyond the confines of the mesopore channel itself, thus maximising colloid capture for diffusive transport across the pore.
We further define how polymer-filled mesopores can be designed to effectively gate colloids according to their size and surface features. 
This combination of features renders mesopores promising as highly selective separation devices.
It also provides a physical explanation for the basic mechanism of nuclear pore permselectivity.
\end{abstract}

%Comment by RR on the storyline: Can we contrast our results with other separation devices, and spell what our mesopores enable that was not possible before? This would help spelling out the significance of our findings.

%%%%%%%%%%
\section{INTRODUCTION}
%%%%%%%%%%

Polymer-modified mesoporous materials and membranes belong to a new class of functional nanostructured materials with great potential in a number of technologies. 

The interaction of (macro)molecules and nanocolloidal particles with porous media, as well as their transport through porous membranes, are important elements of many technological processes (chromatography, heterogeneous catalysis, micro- and ultrafiltration, protein separation and purification, etc.) and, therefore, have been the subject of intensive research for more than sixty years \cite{Watson1959, Rout2003, Huang2023, Uredat2024}.

Advances in macromolecular chemistry have made it possible to significantly improve functional properties of materials with mesopores (i.e., pores with a diameter between a few and many tens of nanometers) by modifying them with polymers of various chemical nature anchored to  the pore walls.
Thus, a fuzzy meshwork of solvated polymers is formed filling the entire pore volume or just the near-wall regions, depending on the molecular mass and conformational state of the polymer chains.
The interaction of this polymer meshwork with colloids, that is, nanoparticles or (macro)molecules in the solution phase, essentially determines the absorption and separation properties of the polymer-modified mesoporous materials and membranes.
These interactions can be attractive or repulsive, and controlled by a broad spectrum of external stimuli \cite{Jeong2002, Lee2010, Low2019}, 
such as temperature                     \cite{Stetsyshyn2020}, 
pH and/or ionic strength of the medium  \cite{Dai2008, Zhang2005}, 
ion valency and specificity             \cite{Zhulina1999, Robertson2021},
electric fields                         \cite{Lokuge2005}, 
solvent composition                     \cite{Halperin2011, Darabi2016}, 
or complex biological stimuli           \cite{Ikeda2010, Lu2003}.
This opens up a unique opportunity for highly selective and controlled uptake and transport of colloids through polymer-filled mesoscopic channels.

Past experimental and theoretical efforts have focused on systems where an external stimulus triggered the transient opening of a polymer free path, typically along the center of the pore, to gate colloid transport.
However, the polymer phase itself can potentially also provide high selectivity to colloids as a function of their size and attraction by the polymer.
We thus hypothesized that even mesopores that are filled by a polymer meshwork across their entire cross-section can effectively gate colloid transport.
If successful, this approach would enable more robust gating as it does not rely on careful tuning of the diameter of a polymer-free channel, and higher transport rates as the full pore cross-section can potentially participate in colloid transport.

Nature provides a case in point.
Nuclear pore complexes (NPCs) perforate the nuclear envelope of eukaryotic cells and control the bulk transport of proteins and nucleic acids between the nucleus and the cytoplasm.
This process enables the spatial separation of gene transcription (in the nucleus) and translation into proteins (in the cytosol), and thus is critical for the ordered course of gene expression.
Each NPC forms a cylindrical channel, measuring approximately 40-60 nm in diameter and 40-95 nm in length (depending on the species \cite{Yang1998, Beck2004, VonAppen2015, Alberts2015, Hayama2017, Holzer2018}).
The channel is filled with a meshwork of several 100 natively disordered protein domains rich in phenylalainine-glycine dipeptides (FG domains) that are anchored to the channel walls \cite{Holzer2018, Ori2013, Rout2000, Dickmanns2015}.
Collectively, the FG domain meshwork provides remarkable gating function: biocolloids of 5 nm (i.e., just a tenth of the pore diameter) or more in hydrodynamic diameter are effectively blocked, except for some dedicated transport factors (importins and exportins, alone and in complex with cargo) which bind to the FG domains and can undergo rapid permeation. 

%RR: The below section is rather long, and may be shortened? Also need to review reference list.

Several independent strands of evidence indicate that the mechanism of diffusive transport across NPCs is based on rather generic physical principles, whereas the exact chemical makeup of the polymers and colloids is secondary for function.
Firstly, despite significant variations in the molecular building blocks of the NPC and the transport factors across distant eukaryotic taxa, NPCs consistently fulfill the same functional role \cite{DeGrasse2009, Maimon2012, Ori2013, Hayama2017, Yaron2018, Holzer2018}.
Secondly, the NPC can gate diffusive colloid transport similarly well in both directions.
Whilst the native cell is capable of directed transport of cargo against a concentration gradient \cite{Rout2003, Tijana2017}, this function is not an intrinsic part of the NPC itself but relies on energy derived from GTP hydrolysis by soluble intracellular proteins \cite{Lowe2015, Yang2004} and can even be reversed through cell engineering without modifying the NPC structure \cite{Nachury1999, Sakiyama2016}.
Thirdly, the binding behaviour of transport factors to assemblies of purified FG domains could be reproduced by simple physical models that treat FG domains as regular flexible polymers and transport factors as spherical colloids with a homogeneous surface. This approach provided faithful predictions even though it ignored the detailed arrangement of interaction sites along FG domains and on the transport factor surface. 
Fourthly, recent work with a range of mutants of green fluorescent protein (GFP) demonstrated that NPCs exhibit a wide and continuous spectrum of permeabilities as a function of colloid surface properties, and earlier studies with non-interacting colloids similarly evidenced a wide and continuous spectrum of permeabilities as a function of colloid size.
Additionally, certain native proteins with affinity to FG-domains, such as $\beta$-catenin, can translocate through the pore without the need for a transport factor, moving from the cytoplasm to the nucleoplasm along a concentration gradient due to their continuous binding to chromatin \cite{Rout2003}. This suggests that a fine balance of many individually weak physicochemical (e.g., electrostatic, hydrophobic, aromatic stacking, ...) interactions between polymers and biocolloids dictates the gating behaviour, rather than a few highly specific biochemical interactions.

%A similar structural motif was recently found in the internal channels of microtubules (about 15 nm in diameter) decorated with so-called microtubule intrinsic proteins (MEPs), presumably modifying microtubule stability and rigidity \cite{Mukhopadhyay2001}.

However, we currently lack an understanding of the relationship between the molecular architecture of the polymer brush filling the pore and its ability to transport colloids with high selectivity and rate. Here, we develop a theoretical approach to reveal the physical mechanisms of diffusive colloid transport across polymer-filled mesopores.
A meshwork of flexible polymers effectively increases the local viscosity and thereby slows down transport of colloids compared to an open pore. 
On the other hand, an attractive polymer phase recruits colloids into the pore, thus increasing colloid transport, and such recruitment is further enhanced when attractive polymers extend outside the pore lumen. 
Intriguingly, the solvent strength through its influence on the density and compactness of the polymer meshwork impacts all of these effects. 
Using a self-consistent field approach, we define how solvent quality and colloid attraction to the polymer may be tuned to maximize the transport rate (even beyond the rate for an open pore) and to achieve highly selective colloidal transport with respect to particle size or affinity for the polymer.


%%%%%%%%%%
\section{RESULTS}
%%%%%%%%%%


%%%%%%%%%%
\subsection{Defining the transport scenario}
%%%%%%%%%%

Salient features of our simulated mesopore are illustrated in Figure~\ref{fig:colloid_transport}.
The pore has a cylindrical shape with radius $r_{\text{p}}$.
It perforates an otherwise impermeable, planar membrane of thickness $L$, and thus is the sole conduit for colloids between two semi-infinite solution reservoirs.
Flexible polymer chains are end-grafted to the inner pore walls, at a density sufficient to form a polymer brush that fills the entire pore cross-section.

We will focus on a pore with a set radius and length, and polymers with a set length and grafting density (Figure~\ref{fig:colloid_transport}).
Whilst the selected values are inspired by the nuclear pore complex (see Supplementary Methods 1), we expect that our findings will be of rather general validity so they can be applied to the performance analysis and rational design of mesopores with other geometries or polymer fillings.

Our aim is to understand how the colloid size, and the affinity of the colloid for the polymer, define the transport of colloids across the pore.
Colloids are taken to be spherical in shape, with diameter $d$.
The interaction strength (contact free energy) between a polymer segment and the surface of the colloid is represented by the Flory-Huggins parameter $\chi_{\text{PC}}$.

\begin{figure}
    \centering
    \includegraphics[width = 0.7\linewidth]{fig/pore_cartoon.png}
    \caption{
        Schematic illustration of colloid diffusive transport through a pore filled with a polymer brush. 
        The brush is formed by linear polymer chains (red strands) with a degree of polymerization $N$, uniformly grafted with grafting density $\sigma$ 
        to the inner surface of a cylindrical pore.
        The pore radius is $r_{\text{p}}$ and the thickness of the impermeable membrane is $L$.
        Polymer chains are flexible with a statistical segment length $a$ and volume $\sim a^3$. 
        Spherical colloids with diameter $d$ are free to diffuse in the surrounding solvent.
        Below all length scales are normalized by the statistical segment length $a$, surfaces by $a^2$ and volumes by $a^3$.
        As a model pore, we set $L = 2r_{\text{p}} = 56$, $\sigma = 0.02$ and $N = 300$.
        With $a = 0.8 {\text{ nm}}$, these parameters reproduce the basic features of nuclear pore complexes.
          }
    \label{fig:colloid_transport}
\end{figure}

To understand how the polymer brush affects the transport of colloids, we consider the stationary diffusive flux of colloids through the pore and analyze how it depends on the parameters of the pore, the brush, and the colloid.
We consider unidirectional transport of colloid particles driven solely by the concentration difference across the membrane and focus on the fundamental mechanisms of diffusion mediated by particle-polymer interactions.
The colloid concentration is set to zero and $\Delta c$ far away from the membrane (at $z\rightarrow\pm\infty$, respectively).
We assume axial (cylindrical) symmetry of the flow in the pore.
Together with the stationary conditions, this implies that parameters relevant to colloid transport depend on the axial coordinate $z$ and the radial coordinate $r$, but not on the azimuthal angle.


%%%%%%%%%%
\subsection{Colloid transport is defined by the sum of resistances of regions outside and inside the pore}
%%%%%%%%%%

%%%%%%%%%%
\subsubsection{Empty pore as a reference case}
%%%%%%%%%%

A natural reference is the diffusive flux through a bare pore, which itself limits the transport of solutes \cite{Deen1987, Sun2024}.
Lord Rayleigh analyzed the flux of point-like solute particles through a circular pore in a planar membrane of negligible thickness \cite{Strutt1878}.
In this simplest case, the equiconcentration surfaces are oblate spheroids, and the streamlines form confocal hyperboloids of revolution \cite{Cooke1966}.
The net flux through the pore is given by
\begin{equation}
    J=2D_0r_{\text{p}}\Delta c,
    \label{eq:flux_Ral}
\end{equation}
where $D_0$ is the diffusion coefficient of the colloid in plain solvent.

A membrane of finite thickness $L$ allows an approximate analytical solution (valid over the full range of $\frac{L}{r_{\text{p}}}$ ratios) \cite{Brunn1984}:
\begin{equation}
    J=\frac{2 D_0 r_{\text{p}}}{1+\cfrac{2L}{\pi r_{\text{p}}}}\Delta c.
    \label{eq:flux_finlength}
\end{equation}

Introducing the resistance $R$ to colloid flow $J = \frac{\Delta c}{R}$ provides a natural interpretation of Eq.~(\ref{eq:flux_finlength}) in terms of the total resistance of the pore:
\begin{equation}
    R = \frac{L}{D_0 \pi r_{\text{p}}^{2}} + \frac{1}{2 D_0 r_{\text{p}}} = R_{\text{int}}^{0} + R_{\text{ext}}^{0},
    \label{eq:resistance}
\end{equation}
where the superscript '0' refers to the bare pore.
The first term in Eq.~(\ref{eq:resistance}) represents the resistance of the interior of the empty pore.
The second term corresponds to the Rayleigh resistance for a pore of infinitesimal thickness (Eq.~(\ref{eq:flux_Ral})), which accounts for the effects of convergent flow toward the pore entrance and its symmetric counterpart at the pore exit.
Inside the pore, the flow lines are approximately axial.
In the empty pore scenario, the inverse of the diffusion constant ($\rho_0=D_0^{-1}$) represents the resistivity of the medium both inside and outside the pore.
Naturally,  the resistance of a pore in a thin membrane $L \ll r_{\text{p}}$ is determined by the resistance of the exterior region, $R \approx R_{\text{ext}}^{0}$.
In contrast, for long pores ($L \gg r_{\text{p}}$) the resistance of the inner region becomes dominant, such that $R \approx R_{\text{int}}^{0}$.

The finite size of colloids affects the diffusive flux in two ways.
First, the excluded volume reduces the effective pore radius $(r_{\text{p}} - d/2\rightarrow r_{\text{p}})$ and increases pore length $(L + d \rightarrow L)$ \cite{Renkin1954, Beck1970, Bungay1973, Anderson1974, Brenner1977}.
Second, the presence of the pore walls entails some additional drag \cite{Ladenburg1907, Faxen1922, Haberman1958}.
This effect can be neglected here as the presence of the polymer brush screens hydrodynamics and thus requires a different kind of drag analysis, as discussed below.


%%%%%%%%%%
\subsubsection{A polymer filling affects the resistance of the pore itself, and also of regions outside the pore}
%%%%%%%%%%

Conformations adopted by overlapping polymer chains grafted to the pore walls are controlled by strong intermolecular interactions that depend on the solvent quality.
The solvent quality is here quantified by the Flory-Huggins parameter $\chi_{\text{PS}}$.
Values of $\chi_{\text{PS}}<0.5$ and $\chi_{\text{PS}}>0.5$ correspond to good and poor solvent, respectively, whereas $\chi_{\text{PS}}=0.5$ represents the ideal (or $\theta$-)solvent.

The polymer density profile $\phi(z,r)$ in the pore was calculated by the two-gradient self-consistent field numerical method of Scheutjens and Fleer (SF-SCF; see Supplementary Methods 2-3).
In Figure \ref{fig:phi_hm_grid}, one can appreciate the expected increase in polymer concentration inside the por inside the pore with decreasing solvent quality (increasing $\chi_{\text{PS}}$).
With the selected pore and polymer parameters (Figure~\ref{fig:colloid_transport}), the polymer brush fills the entire pore cross-section within the full range of solvent qualities explored ($\chi_{\text{PS}}\le0.9$), so that colloids need to navigate the polymer meshwork to move across the membrane.
For wider pores, shorter polymers and/or lower grafting densities, an open channel free of polymer may appear in the pore center, as discussed in detail elsewhere~\cite{Laktionov2021}.
This scenario would result in a distinct permeation behavior, as colloids could move through the pore without traversing the polymer brush.
This case is not considered here.

\begin{figure}
    \centering
    \includegraphics[width = 0.7\linewidth]{fig/phi_hm_grid.png}
    \caption{
    Maps of the polymer volume fraction $\phi(r,z)$ for a polymer brush in a cylindrical pore with solvent quality ranging from good (upper left panel) to poor (lower right panel), as quantified by the Flory-Huggins parameter $\chi_{\text{PS}}$.
    Polymer volume fractions are mapped in cylindrical coordinates (as shown by $rz$-coordinate arrows), color coded as indicated and with selected iso-concentration lines. The blank space corresponds to pure solvent; the membrane is colored green.
    Pore and brush parameters are as given in Figure~\ref{fig:colloid_transport}.
    }
    \label{fig:phi_hm_grid}
\end{figure}

Figure \ref{fig:phi_hm_grid} further illustrates that whilst the brush remains confined within the pore lumen in poor solvent ($\chi_{\text{PS}}=0.9$) it protrudes substantially into the surrounding space in ideal and good solvent ($\chi_{\text{PS}}\le0.5$), thus forming fringes on either side of the pore.
The polymer brush therefore will impact on the resistance to the colloid flow within as well as outside the pore, such that
\begin{equation}
    R=R_{\text{int}}+R_{\text{ext}},
    \label{eq:R_tot_tot}
\end{equation}
with $R_{\text{int}}\rightarrow R_{\text{int}}^{0}$ and $R_{\text{ext}}\rightarrow R_{\text{ext}}^{0}$ in the limit of the empty pore.


%%%%%%%%%%
\subsection{Insertion free energy and diffusivity control diffusive transport}
%%%%%%%%%%

Zooming in to the local scale, we can analyze how colloids are accumulated in or depleted from due to attractive or repulsive interactions, respectively, with the polymer meshwork, and how the meshwork affects the colloid's local mobility.


%%%%%%%%%%
\subsubsection{Leading contributions to the insertion free energy}
%%%%%%%%%%

The position-dependent insertion free energy $\Delta F(r,z)$ is the work required to move a colloid from the exterior solution into the polymer brush.
For colloids that are significantly smaller than the size of the pore, the insertion free energy is determined entirely by the local polymer concentration (i.e., wall effects can be neglected), and comprises two distinct contributions:
\begin{eqnarray}
    \Delta F = \Delta F_{\text{osm}} + \Delta F_{\text{sur}},
    \label{eq:Delta_F}
    \\
    \Delta F_{\text{osm}}(r,z) = \int_{V} \Pi(r',z') dV', \nonumber
    \\
    \Delta F_{\text{sur}}(r,z) = \oint_{S} \gamma (r',z') dS'. \nonumber
\end{eqnarray}
The coordinates $(r,z)$ refer to the center of the colloid, whilst the insertion free energy is obtained by integrating over the volume and surface of the colloid, respectively.
Here and below, all the free energy values are normalized by the thermal energy unit $k_{\text{B}}T$.

The osmotic contribution, $\Delta F_{\text{osm}}$, is proportional to the colloid volume and accounts for the work against excess osmotic pressure upon insertion of the colloid into the brush.
The local osmotic pressure (normalized by $k_\text{B} T$) is calculated from the local polymer concentration as
\begin{equation}
    \begin{aligned}
        \Pi(r,z)=  \phi(r,z)\frac{\partial f\{\phi(r,z)\}}{\partial \phi(r,z)} - f\{\phi(r,z)\}= 
        \\
        [-\ln(1-\phi(r,z)) - \phi(r,z) -\chi_{\text{PS}}\phi^2(r,z)],
    \end{aligned}
    \label{eq:osmotic}
\end{equation}
where 
$$
f\{\phi(r,z)\}=(1-\phi(r,z))\ln(1-\phi(r,z)) +\chi_{\text{PS}}\phi(r,z)(1-\phi(r,z))
$$
is the mean-field Flory expression for the interaction free energy per unit volume of the polymer solution of concentration (volume fraction) $\phi(r,z)$.
As long as the osmotic pressure inside the brush is positive, $\Delta F_{\text{osm}}$ is also positive and provides a dominant contribution for sufficiently large particles.

The interfacial contribution, $\Delta F_{\text{sur}}$, is proportional to the colloid surface area, with the surface tension $\gamma (r,z)$ approximated as
\begin{gather}
     \gamma (r,z)= \frac{1}{6}(\chi_{\text{ads}} - \chi_{\text{ads}}^{\text{crit}})\phi^{\ast}(r,z),
    \label{eq:chi_ads} 
    \\
    \text{with } \chi_{\text{ads}} = \chi_{\text{PC}} - \chi_{\text{PS}}(1-\phi^{\ast}), \text{ and } \phi^{\ast}(r,z)= (b_{0} + b_{1}\chi_{\text{PC}})\phi(r,z).
    \nonumber
\end{gather}
Here $\gamma$ is the change in the free energy of a unit area upon replacement of a contact of the colloid with solvent by a contact with a polymer solution of concentration $\phi(r,z)$.
The coefficients $b_0$ and $b_1$ are parameters that account for depletion or accumulation of the polymer in the vicinity of the colloid surface (Supplementary Methods 4-5).

Depending on the relative strengths of polymer-colloid ($\chi_{\text{PC}}$) and polymer-solvent ($\chi_{\text{PS}}$) interactions, the sign of $\gamma$ can be either positive or negative.
If the particle surface is repulsive ($\chi_{\text{ads}} \geq 0$) or even weakly attractive for polymers ($\chi_{\text{ads}}^{\text{crit}} \leq \chi_{\text{ads}} < 0$), then, due to steric constraints imposed by the impermeable surface, the available conformations of the polymer are restricted, leading to polymer depletion near the particle surface and $\gamma > 0$.
At the critical adsorption condition $\chi_{\text{ads}} = \chi_{\text{ads}}^{\text{crit}}$, the losses in conformational entropy caused by the presence of the surface are exactly balanced by the free energy gain from monomer-surface contacts, causing $\gamma$ to vanish \cite{Fleer1993,Birshtein1979,Birshtein1983,Eisenriegler1982}.
Finally, $\gamma < 0$ at $\chi_{\text{ads}} < \chi_{\text{ads}^{\text{crit}}}$.

In what follows, we applied an approximate analytical scheme to evaluate the insertion free energy $\Delta F(r,z)$ as $\Delta F\{\phi(r,z)\}$, where $\phi(r,z)$ is the polymer density distribution in a colloid-free brush calculated with the SF-SCF approach.
The colloid is thus considered as a 'probe' which does not perturb the global polymer concentration distribution $\phi(r,z)$.
With this scheme, we can evaluate the insertion free energy at any position of the colloid in the brush, including off the pore axis (Supplementary Method 6).
Comparison of the approximate analytical approach with direct SF-SCF calculations of the insertion free energy for colloids on the pore axis demonstrated good quantitative agreement (Supplementary Method 5), justifying the use of the more versatile analytical approach.

Figure~\ref{fig:DeltaF_map} illustrates how colloids may be either repelled or attracted by the polymer meshwork, depending on the balance of osmotic and interfacial contributions to $\Delta F$.
Since the polymer concentration in the pore is strongly inhomogeneous, the net insertion free energy $\Delta F(r,z)$ may exhibit quite large spatial variations.
For example, the brush shown in Figure~\ref{fig:DeltaF_map} at $\chi_{\text{PC}}=-0.75$ simultaneously exhibits attraction ($\Delta F<0$) at the loose fringes and repulsion ($\Delta F>0$) inside the pore where polymer concentration is high.

\begin{figure}
    \centering
    \includegraphics[width = 0.7\linewidth]{fig/free_energy_hm.png}
    \caption{
    Maps of the particle insertion free energy $\Delta F(r,z)$ for a range of polymer-colloid interaction strengths.
    The polymer-colloid interaction strength is quantified by the Flory-Huggins parameter $\chi_{\text{PC}}$ ranging from -0.50 (least attractive) to -1.25 (most attractive), as indicated.
    Insertion free energies are mapped in cylindrical coordinates, color coded as indicated.
    Pore and brush parameters are as given in Figure~\ref{fig:colloid_transport}; $\chi_{\text{PS}}=0.5$ and $d=8$.
    }
    \label{fig:DeltaF_map}
\end{figure}


%%%%%%%%%%
\subsubsection{Local colloid mobility is determined by the diameter of the colloid and the polymer mesh size and by attraction strength}
%%%%%%%%%%

The crowded polymers naturally decrease the diffusion of colloids.
A polymer brush is effectively described as an inhomogeneous semi-dilute solution with a concentration-dependent correlation length (mesh size) $\xi(\phi)$.
Colloids of size $d > \xi$ experience additional friction as they are trapped by the polymer meshwork.
As a result, diffusion is slowed compared to pure solvent, leading to a position-dependent diffusion coefficient ($D(r,z) < D_0$).

According to a scaling theory by Cai et al. \cite{Cai2011} for diffusion of non-sticky particles in a semi-dilute polymer solution, the mobility of a colloid scales as $D\sim D_0 (\xi/d)^2\ll D_0$ for $d\gg \xi$, while small particles diffuse virtually unimpeded ($D\sim D_0$ for $d\ll \xi$). We here use a simple interpolation formula to capture the diffusion coefficient across the full range of relevant colloid sizes relative to the correlation length $d / \xi$:

\begin{equation}
    D\{\phi(r,z)\} = \frac{D_0}{1+[\alpha d / \xi\{\phi(r,z)\}]^2} \approx \frac{D_0}{1+[\alpha d \phi(r,z)]^2} .
    \label{eq:Rubinstein}
\end{equation}

The correlation length $\xi$ in Eq.~(\ref{eq:Rubinstein}), is controlled by the local polymer concentration $\phi(r,z)$.
The scaling relation between the correlation length and the polymer concentration depends on the solvent quality \cite{DeGennes1979}.
In the term on the right hand side, we have approximated the dependence of the mesh size on polymer concentration by $\xi \cong \phi^{-1}$, valid close to $\theta$-solvent conditions in a mean-field regime.
The coefficient $\alpha$ in Eq.~(\ref{eq:Rubinstein}) is a numerical pre-factor.
In section 2.7, we estimate $\alpha = 5.5$ by comparing our theoretical stationary flux predictions to experimental data on the flux of non-sticky colloids of different sizes through nuclear pore  complexes.
In the following sections we consistently use this value for numerical estimates of the pore permeability and permselectivity.  

Recent theoretical, experimental and simulation studies extended the scaling approach of Cai et al. \cite{Cai2011} to include the effects of polymer-colloid attraction \cite{Yamamoto2018, Carroll2018}.
In the limit where proper diffusion of polymer chains is negligible (relevant to our case of pore-anchored chains), the characteristic desorption time of a single polymer-colloid contact $\tau_\text{des}(\epsilon)$ appears as an extra timescale.
Here, $\epsilon > 0$ is the absolute value of the activation energy to break a contact (normalized by the thermal energy $k_\text{B} T$).
The polymer-colloid attraction may be considered weak if the desorption time is much smaller than the self-diffusion time.
Typically, this happens when $\epsilon \lesssim k_\text{B} T$, and in this case attractive forces only modify local friction by affecting the local packing of polymers around the particle.
The magnitude of this effect is estimated to be small, reducing the diffusivity by at most a few 10\% \cite{Yamamoto2011}.
On the other hand, when $\epsilon$ amounts to several $k_\text{B} T$ units, then the desorption time grows exponentially with $\epsilon$ and affects the diffusion substantively.
Numerical simulations \cite{Yamamoto2018} demonstrate that the colloid diffusivity is reduced by an exponential factor with approximately half the activation energy:

\begin{equation}
    D(\epsilon)\approx D(\epsilon=0) \exp (-\epsilon / 2).
    \label{eq:Yamamoto}
\end{equation}

In our model, the energy of a polymer-colloid contact is $\chi_{\text{PC}}/6$, and therefore $\epsilon < 1/3$ within the explored range $0\geq\chi_{\text{PC}}\geq-2$.
Throughout most of the paper, we assume that the surface of the colloid is homogeneous, and neglect the small reduction in the local colloid mobility due to weak polymer attraction.
However, in Section 2.7 we will return to this question in the context of proteins traversing the nuclear pore, where the colloid surface and the polymers may be heterogeneous with affinity localized in a few sticky patches.
We can explore the limiting case where all the (negative) surface free energy $\pi d^2 \gamma(r,z)$ is assigned to a single sticky patch.
The diffusion coefficient is then modified as
\begin{eqnarray}
    D_{\text{sticky patch}} = 
    \begin{cases}
        D \exp(\frac{\pi d^2 \gamma(r,z)}{2}) & \text{if } \gamma(r,z) < 0 \\
        D & \text{otherwise}
    \end{cases}
     \label{eq:Sticky diff}
\end{eqnarray}
to interpolate between non-sticky and sticky particles.
It is safe to assume that realistic cases of attractive polymer-colloid interaction lie between the two extremes of a perfectly homogeneous surface ($D_\text{homo} \approx D$) and a single sticky patch (Eq.~(\ref{eq:Sticky diff})).

Several other theoretical and empirical models have been proposed to describe the diffusion of colloids in polymer meshworks \cite{Schweizer2003,Kohli2012,Holyst2009,Phillies1988}.
Although the predictions of different models differ quantitatively, they all share the same qualitative trend.

%COMMENT RR: Do we need a figure here that shows an illustrative map of D/D0 - perhaps as a separate panel in Figure 3?


%%%%%%%%%%
\subsubsection{Linking local resistivity to global transport}
%%%%%%%%%%

%COMMENT RR: I found this section quite technical and difficult to follow. Can we spell out the main assumptions in simpler terms, understandable for readers not experienced with some of the math and concepts?

Having defined the position-dependent insertion free energy and mobility, we can develop an analytical method to estimate the total resistance of the brush-filled pore to the diffusive flow of colloidal particles.

Diffusive transport in the presence of an external force generated by the insertion free energy $\Delta F$ is described by the Smoluchowski equation,

\begin{equation}
    \frac{\partial c(\bold r)}{\partial t}=-{\bold \nabla}\cdot (D(\bold r)({\bold \nabla}c({\bold r})+c({\bold r}){\bold \nabla}(\Delta F({\bold  r}))),
    \label{eq:Smoluch}
\end{equation}
where $c(\bold r)$ is the colloid concentration. The flux density 
\begin{equation}
    \bold j=- D(\bold r)({\bold \nabla}c({\bold r})+c({\bold r}){\bold \nabla}(\Delta F({\bold  r}))
    \label{eq:j}
\end{equation}
can be expressed in terms of the gradient of the potential function $\psi(\bold r)$
\begin{equation}
    \bold j=- D(\bold r) \exp(-\Delta F({\bold  r}))  {\bold \nabla} \psi(\bold r), \text{ with } \psi(\bold r)=c(\bold r)\exp(\Delta F(\bold r)).
    \label{eq:psi}
\end{equation}
The problem of finding the total resistance of two semi-infinite solution reservoirs separated by the membrane with the brush-filled pore is equivalent to finding the resistance of the medium with position-dependent conductivity possessing axial symmetry:
\begin{equation}
    \rho^{-1} (r,z)= D(r,z)\exp(-\Delta F(r,z)).
    \label{eq:rho}
\end{equation}

%COMMENT RR: In the below sentence, can we calrify what we mean with 'equipotential'?

To this end, we consider a set of approximate equipotential surfaces $\psi(r,z)=\text{const}$ foliating the space available for colloid flow: inside the pore, the surfaces are taken as  discs of radius $r_{\text{p}}$ normal to the pore axis; outside the pore, we use oblate hemispheroids taken from the Rayleigh solution \cite{Strutt1878} (Figure~\ref{fig:resistivity_profile}a).
Analogously to a set of resistors connected in parallel, the total conductivity of a layer between two adjacent equipotential surfaces is obtained by integration of local conductivities over the layer. Within the pore, $|z|\leq L/2$, the result is given by
\begin{equation}
\varrho_{\text{int}}^{-1}(z)= 2\pi\int_{0}^{r_{\text{p}}^{}} \rho^{-1}(r,z) r \, dr
\label{eq:varrho1}
\end{equation}

In the exterior region, $|z| >L/2$, the expression is modified to integrate over the aforementioned hemispheroids:
\begin{equation}
    \begin{gathered}
        \varrho_{\text{ext}}^{-1}(z)= 2\pi\int_{0}^{r_{\text{p}}^{}} \rho^{-1}\left( r'(r,z), z'(r,z) \right)  \tilde{h} (r,z) dr\\
        r'(r,z) = r\sqrt{1 + \frac{(z - L/2)^2}{r_{\text{p}}^2}}\\
        %r'(r,z) \in [0, \sqrt{r_{\text{p}}^2 + (z-L/2)^2}]\\
        z'(r,z) = (|z| - L/2) \frac{\sqrt{r_{\text{p}}^2 - r^2}}{r_{\text{p}}} +  \text{sign}(z) \frac{L}{2}\\
        %z'(r,z) \in [L/2, z]
        \tilde{h} (r,z) = h_r h_{\theta} h_z^{-1} = \dfrac{r}{r_{\text{p}}}\dfrac{r_{\text{p}}^2 + (|z|-L/2)^2}{\sqrt{r_{\text{p}}^2 - r^2}}
    \end{gathered}
\label{eq:varrho2}
\end{equation}
where $r'(r,z) , z'(r,z)$ are the functions parameterizing the equipotential surfaces, and $h_r$, $h_{\theta}$ and $h_z$ are the corresponding Lam\'e coefficients (see Supplementary Methods 7-8). In the case of a homogeneous brush considered here, the function $\varrho_{\text{ext}}^{-1}(z)$ is even.

\begin{figure}
    \centering
    \includegraphics[width=0.48\linewidth]{fig/resistitance_integration_miniature.png}
    \caption{%
    Integration scheme for the layer resistance.  
    The intrinsic orthogonal curvilinear coordinates are defined by equipotential surfaces, $\psi(r,z)=\text{const}$, and the flux lines $\mathbf{j}$.  
    A layer between two adjacent equipotential surfaces with a thickness $\text{d}z$ along the central axis has resistance $\varrho_{\text{int}}\text{d}z$ inside the pore ($|z|<L/2$, blue shading) and $\varrho_{\text{ext}}\text{d}z$ in the exterior region (red shading).  
    The red rectangle illustrates the local conductivity element at $(r',z')$ within the exterior layer.  
    The parametrisation $(r'(r,z),\,z'(r,z))$ traces the integration path and maps the curvilinear coordinates back to the original cylindrical coordinates $(r,z)$, as indicated by the red arrows.%
    }
    \label{fig:integration_scheme}
\end{figure}

On the other hand, since the consecutive layers are connected in series, their total resistance can be found by appropriate integration:
\begin{equation}
    R_{\text{int}} = \int_{-L/2}^{+L/2}\varrho_{\text{int}}(z) dz,
    \label{R_int}
\end{equation}

\begin{equation}
   R_{\text{ext}} =2\int_{+L/2}^{+\infty}\varrho_{\text{ext}}(z)dz
    \label{R_ext}
\end{equation}

For a bare pore without a polymer brush, this method recovers Eq.~\ref{eq:resistance}, as expected.
For a brush entirely contained within the interior of the pore -- which is well justified under poor solvent conditions -- the total resistance is found as $R = R_{\text{int}} + R_{\text{ext}}^{0}$, as the exterior is not modified by the brush.
Conversely, a brush under good or $\theta$-solvent conditions produces swollen fringes (caps) outside the pore (Figure~\ref{fig:phi_hm_grid}), which modify the resistance $R_{\text{ext}}$.


%%%%%%%%%%
\subsection{An attractive polymer filling enhances colloid fluxes through the pore}
%%%%%%%%%%

\begin{figure}
    \centering
    \includegraphics[width = 0.5 \linewidth]{fig/resistitance_components.png}
    \caption{
    \textbf{(a)} Special orthogonal curvilinear coordinate system aligned with the flux density $\bm{j}$ stream surfaces (radial coordinate) and level sets of the potential function $\psi$ (axial coordinate).
    Red lines correspond to constant values of the axial coordinate; gray lines are tangential to the flux density field and correspond to constant values of the radial coordinate.
    The lines define bodies of revolution along the $z$-axis; the angular coordinate is not shown.
    In the exterior of the pore, constant radial coordinates are confocal hyperboloids of revolution, constant axial coordinates are confocal oblate spheroids, and constant angular coordinates are half-planes.
    In the pore interior (green shading), the system reduces to standard cylindrical coordinates. 
    %COMMENT RR: I found the description of panel a quite complex. Can it be simplified, and rendered coherent with the main text?
    \textbf{(b)} Interior ($R_{\text{int}}$), exterior ($R_{\text{ext}}$) and total ($R = R_{\text{int}} + R_{\text{ext}}$) resistance (all normalized by the visocity of the solvent $\eta_\text{S}$, and encoded by line type as indicated) vs polymer-colloid interaction strength $\chi_{\text{PC}}$ for a good solvent ($\chi_{\text{PS}} = 0.3$, in blue) and a poor solvent ($\chi_{\text{PS}} = 0.7$, in grey).
    The exterior $R_{\text{ext}}^{0}$ and total $R_0$ resistances for the empty pore are also shown (red lines of matching type).  
    Pore and brush parameters are as given in Figure~\ref{fig:colloid_transport}; $d = 8$.
    }
    \label{fig:resistivity_profile}
\end{figure}

Figure~\ref{fig:resistivity_profile}b visualizes the relative contributions of the pore interior ($R_{\text{int}}$) and exterior ($R_{\text{ext}}$) to the total resistance as a function of the polymer-colloid attraction $\chi_\text{PC}$,
for a selected colloid size ($d = 8$) in a good solvent ($\chi_\text{PS} = 0.3$) and a poor solvent ($\chi_\text{PS} = 0.7$).
A striking feature is that attracted colloids can achieve diffusive fluxes through a polymer-filled pore that exceed the limit of the empty pore, as indicated by the segments of the $R(\chi_{\text{PC}})$ curves that fall below the solid horizontal red line marking the empty-pore resistance $R_{0}$.
This result may at first appear surprising, given that the polymer medium is expected to slow down the diffusion of colloids.
However, this slowing down is counteracted by the attractive potential of the polymer meshwork, which reduces local resistivity according to the exponential factor in Eq.~(\ref{eq:rho}).

The reduced local resistivity has pronounced consequences for diffusive transport in both the interior and the exterior of the pore.
Figure~\ref{fig:resistivity_profile}b demonstrates that the interior resistance $R_{\text{int}}$ can be driven practically to zero by increasing the polymer-colloid attraction (decreasing $\chi_\text{PC}$) below a certain threshold.
Compared to an empty pore (Eq.~(\ref{eq:resistance})) this short-circuiting effect alone entails a reduction in the resistance, by a factor of up to $R^0_{\text{int}}/R^0_{\text{ext}}+1 \approx 2/\pi \times L / r_{\text{p}} + 1$.
For the pore and colloid considered here, this represents an approximately 3-fold reduction, to a level marked by the dashed horizontal red line in Figure~\ref{fig:resistivity_profile}b which equals the contribution of convergent flow to the empty-pore resistance ($R^0_\text{ext}$).
The reduction would be even stronger for longer pores ($L\gg r_p$), and for larger colloids that increase the effective pore length and decrease the effective pore diameter.

The exterior resistance $R_{\text{ext}}$, on the other hand, always retains a finite contribution from the diffusive fluxes in the semi-infinite reservoir, setting an absolute lower bound on the total resistance.
The reduction of $R_{\text{ext}}$ below $R^0_\text{ext}$ evidenced in Figure~\ref{fig:resistivity_profile}b is due attractive brush fringes that protrude and faciliate diffusive transport outside the pore.
The magnitude of this effect increases with the extension of the polymer cap, and explains why the minimal attainable resistance increases with decreasing solvent quality (Figure~\ref{fig:phi_hm_grid}).
Approximating the brush fringes as a hemisphere cap with radius $r_\text{cap}$, the plateau conditions are equivalent to the doubled resistance of an ideally absorbing hemisphere, $R_\text{ext}^\text{\text{min}} = 1 / (D_0 \pi r_\text{cap})$.
\todo{We draw on the electrostatic analogy of a grounded, perfectly conducting sphere in a uniform field.
In a half-space, a hemisphere has twice the resistance of a full sphere, as it is accessible from only one side.
Since the membrane symmetrically divides space and transport must pass through both hemispheres in series, the total resistance is four times that of a full sphere in an unbounded medium.}
Attractive brush fringes thus entail a reduction in resistance by a factor of up to $R_\text{ext}^0 / R_\text{ext}^\text{\text{min}} = \pi/2 \times r_\text{cap}/r_\text{p}$.
In good solvent ($\chi_\text{PC} = 0.3$), for example, the cap radius (along the pore axis) is comparable to the pore diameter (Figure~\ref{fig:phi_hm_grid}), leading to the 3.3-fold reduction of the external resistance, and a cumulative 10-fold reduction of the total resistance, compared to the empty pore (Figure~\ref{fig:resistivity_profile}b).
As the solvent quality decreases the cap size shrinks (Figure \ref{fig:phi_hm_grid}), with a correspondingly reduced benefit on pore conductivity, as illustrated for $\chi_\text{PC} = 0.7$.
For even poorer solvents, the cap and its benefit disappear entirely ($R_\text{ext} = R_\text{ext}^0$; not shown).


%%%%%%%%%%
\subsection{Polymer-filled mesopores effectively gate colloids by their attraction to the polymer}
%%%%%%%%%%

Figure~\ref{fig:resistivity_profile}b also illustrates how the total resistance of the pore varies with the colloid's affinity to the polymer brush.
As expected, increasing the polymer-colloid attraction strength (i.e., more negative $\chi_{\text{PC}}$) results in a monotonic decrease in the pore's total resistance, since the interfacial term in the insertion free energy becomes more negative, thereby increasing the local conductivity $\rho^{-1}$.

Most notable is a sharp transition from a regime of facilitated permeation ($R < R_0$) to a regime of impeded permeation ($R > R_0$).
The regime of impeded permeation is dominated by the internal resistance.
It exhibits high selectivity with respect to the polymer-colloid interaction strength, and a mostly very high total resistance and thus low colloid flux, both appreciable in Figure~\ref{fig:resistivity_profile}b as a sharp increase in $R$ over a relatively modest $\chi_{\text{PC}}$ range.
In contrast, the region of facilitated permeation is dominated by the external resistance. 
It exhibits high colloid fluxes but rather low (if any) $\chi_{\text{PC}}$ selectivity, as demonstrated by the previously analysed plateau.
Thus, the transition between the two regimes of transport defines the condition for sharp colloid gating, with remarkably efficient transport in the regime limited by external resistance and effective blockage in the regime limited by internal resistance.

In this context, the solvent quality can be seen as a regulator of the polymer-colloid interaction level for gating. 
Lowering the solvent quality (increasing $\chi_{\text{PS}}$) reduces $\chi_{\text{ads}}$ and shifts the entire $R(\chi_{\text{PC}})$ curve toward larger $\chi_{\text{PC}}$ values.
Thus, a poorer solvent extends the range of facilitated permeation towards more weakly interacting with the brush colloids.

The here-presented trends are qualitatively correct also for colloids that are smaller or larger than the $d=8$ colloid considered here.
Naturally, the gating effect will be rather moderate for small colloids, yet even sharper for larger colloids.


%%%%%%%%%%
\subsection{High colloid flux implies colloid enrichment in the pore}
%%%%%%%%%%

Colloid concentration profiles under stationary flux conditions can be found by numerically solving the Smoluchowsky equation (\ref{eq:Smoluch}) with $\frac{\partial c(r,z)}{\partial t} = 0$  (see Supplementary Method 9). 

Figure~\ref{fig:colloid_concentration} maps the steady-state colloid concentration across a polymer-filled pore with colloid size $d = 12$ and polymer-colloid interaction strength $\chi_{\text{PC}} = -1.5$ in a good solvent ($\chi_{\text{PS}} = 0.3$).
This condition corresponds to facilitated transport with a total resistance about 10-fold lower than a bare pore ($R \approx R_0/10$).

\begin{figure}
    \centering
    \includegraphics[width=0.9\linewidth]{fig/streamlines.png}
    \caption{
    Color map of the steady-state colloid concentration, normalized by the bulk concentration in the source compartment $\Delta c$ as a function of $z$ and $r$.
    Isoconcentration surfaces are shown with contours for values from 0.97 to 0.90, and from 0.10 to 0.03, in steps of 0.01.
    The flux is represented by streamlines marked with small arrows, indicating the average colloid trajectory.
    Pore and brush parameters are the same as in Figure~\ref{fig:colloid_transport}; $d = 12$, $\chi_{\text{PC}} = -1.5$ and $\chi_{\text{PS}} = 0.3$.
    }
    \label{fig:colloid_concentration}
\end{figure}

The map illustrates several salient features of the diffusion process.
Outside the region of the pore and polymer fringes, the colloid concentration profile is as expected for plain solution: the concentration rapidly approaches the respective bulk concentrations of the semi-infinite reservoirs, $c(z = -\infty) = \Delta c$ and $c(z = +\infty) = 0$, and the equiconcentration surfaces near the pore entrance ($c/\Delta c$ between 0.92 and 0.97) and exit ($c/\Delta c$ between 0.08 and 0.03) form a symmetric set of oblate hemispheroids.
Inside the pore, the flux lines run almost parallel to the pore axis.

The most notable observation is that the colloid concentrations substantively exceed $\Delta c$ near the pore entrance (by a factor of $\sim30$) and inside the pore (by a factor  of $\sim8$.
This effect is caused by the negative insertion free energy in the space occupied by the polymer brush.
At equilibrium (i.e., with vanishing fluxes), the partitioning would amount to $c_{\text{eq}}/\Delta c = \exp\left( -\Delta F \right)$.
In the steady state (i.e., with non-vanishing fluxes), the colloid concentration is reduced but approaches the equilibrium concentration as the insertion free energy becomes largely negative ($c/\Delta c \to c_{\text{eq}}/\Delta c$).

The presented quantitative results are only valid for sufficiently low bulk concentrations $\Delta c$, as our model disregards any colloid crowding effects.
When this crowding is accounted for, the steady-state colloid concentrations in the brush will be systematically lower.


%%%%%%%%%%
\subsection{Polymer-filled mesopores effectively gate colloids by their size}
%%%%%%%%%%

\begin{figure}
    \centering
    \includegraphics[width = 0.5\linewidth]{fig/permeability_on_d.png}
    \caption{
    \textbf{(a)} 
    Pore resistance $R$, normalized by the viscosity of the solvent $\eta_\text{S}$, as a function of colloid size $d$ for selected polymer-colloid interaction strengths ($\chi_{\text{PC}}$, as indicated with colored lines) at a fixed solvent strength $\chi_{\text{PS}} =0.5$.
    \textbf{(b)}
    Normalized pore resistance $R$ as a function of colloid size $d$ for selected solvent strengths ($\chi_{\text{PS}}$, as indicated with colored lines) at a fixed polymer-colloid interaction strengths $\chi_{\text{PC}} = -1.25$. 
    The normalized resistance of an empty pore $R_{0}$ (black thick lines) serves as a reference.
    Pore and brush parameters are as given in Figure~\ref{fig:colloid_transport}. 
    }
    \label{fig:R_vs_d}
\end{figure}

Figure~\ref{fig:R_vs_d} illustrates how the total resistance varies with colloid size at a fixed solvent strength (panel a, $\chi_{\text{PS}} = 0.5$) or at a fixed polymer-colloid interaction strength (panel b, $\chi_{\text{PC}} = -1.25$.

There are some universal effects that increase the total resistance $R$ of both empty and polymer-filled pores with increasing particle size $d$.

First, according to the Stokes-Einstein relation, $D_0 \sim d^{-1}$, so that $R_0 \sim d$ according to Eq~(\ref{eq:resistance}). 
Figure~\ref{fig:R_vs_d}ab shows all curves converging to the bare-pore permeability with a Stokesian slope $R_0 \sim d$ in the limit of $d \to 0$, as any other contributions that depend on the particle size vanish.
Second, interference constraints between the impermeable membrane and the spherical particle (excluded volume) decrease the effective pore radius and increase the effective membrane thickness.
The effect is seen in the dependences of the bare pore resistance $R_0(d)$ on the particle size:  
as particles become larger the $R_0(d)$ curves in Figure~\ref{fig:R_vs_d}ab curls up.
Naturally, as $d \to 2r_{\text{pore}}$, the pore becomes impermeable to particles, and $R_0 \to \infty$ (the asymptote is not shown in Figure~\ref{fig:R_vs_d}ab).
Finally, as previously mentioned, we underestimate resistance of an empty pore $R_0$ for large particles,
additional drag from the pore walls is neglected, which would otherwise make the dependence $R_0(d)$ even steeper, 
with a significantly larger resistances for large particles ($d \sim r_{\text{pore}}$).

%The same parameters that controls the local conductivity Eq.~(\ref{eq:rho}) also control the total resistance.


Polymer filling substantially alters the dependence  of the pore resistance $R(d)$ on the particle size $d$.
First of all, polymer meshwork lowers local diffusion coefficient $D$ (Eq. (\ref{eq:Rubinstein})), thereby raising the pore's resistance.
Because $D$ can follow scaling $D \sim d^{-3}$, which is stronger than Stokesian $D \sim d^{-1}$, a brush with 
fairly vanishing colloid insertion free energy, $\Delta F\approx 0$,
exhibits resistance that scales as $R(\Delta F=0)\sim d^{3}$.
This behaviour is exemplified in Figures \ref{fig:R_vs_d}a,b for the parameter set $\chi_{\text{PS}}=0.5,\;\chi_{\text{PC}}=-1.0$, 
where $\Delta F$ is slightly negative (or nearly zero) across a wide particle-size range.

Since the pore resistance $R\sim \exp (\Delta F)$ exponentially depends of the net insertion free energy, 
$\Delta F=\Delta F_{\text{osm}}+ \Delta F_{\text{surf}}$,
the dependecne of the pore resistence on the particle size is, in general case, controlled by the interplay between
the osmotic $\Delta F_{\text{osm}} \sim \Pi d^3$, and the surface $\Delta F_{\text{surf}} \sim \gamma d^2$ contribution. 
While the osmotic repulsions due to polymer filling $\Delta F_{\text{osm}} \sim \Pi d^3$ further enhance total resistance,  
the surface term, $\Delta F_{\text{surf}} \sim \gamma d^2$, can either raise the resistance (at $\gamma>0$) or lower it (at $\gamma<0$), 
%$R\sim \exp (\Delta F_{\text{osm}}+ \Delta F_{\text{surf}})$, 
Noticably, sufficiently large negative values of $\gamma$ can give rise to overall permeability can even exceed that of an empty pore.

For inert or weakly attractive particles, $\gamma\geq 0$, and the pore resistance grows monotonously as a function of the particle size,$d$,
due to combined effects of decreasing diffusion coefficient $D(d)$ and increasing isertion free energy  $\Delta F(d)$.
For sufficiently large particles osmotic contribution dominates in the insertion free energy and $R\sim \exp (\Pi d^3)$.

On the contrary, for strongly attractive particles with $\gamma <0$, the dependence of the pore resistance $R(d)$ of the particle size
becomes non-monotonous and exhibits local maximum at $d=d_{\text{max}}$ followd by a minimum at $d=d_{\text{min}}$.
The local maximum corresponds to the particles of intermediate size and arises as an outcome of competition between a decrease in local diffusion coefficient and increasing by the absolute value negative
contribution $\Delta F_{\text{sur}}\sim \gamma d^2$. For larger particules,
the interplay of negative quadratic, $\Delta F_{\text{sur}}\sim \gamma d^2$,
and positive cubic, $\Delta F_{osm}\sim \Pi d^3$, gives rise to a minimum in the insertion free energy $\Delta F$ at $d=d_{\text{min}}\sim (-\gamma/Pi)$, and, 
consequently in the total pore resistance $R\sim \exp(\Delta F)$, as a function of $d$, which is demonstrated by Figure 7 for $\chi_{PC}=-1.25, \chi_{PS}=0.5$. 
For the particles of larger size, $d\gg d_{\text{min}}$, the resistance $R(d)$ sharply grows due to the dominant osmotic contribution, 
$R\sim \exp(\Pi d^3)$.

 

%For the pore and brush parameters of Figure \ref{fig:colloid_transport} at solvent strength $\chi_{\text{PS}}=0.5$, polymer-colloid interactions weaker than $\chi_{\text{PC}}=-0.6$ ensure $\gamma\ge0$ everywhere (see the curves for $\chi_{\text{PC}}=0.0$ and $-0.5$ in Figure \ref{fig:R_vs_d}a).
%Under these conditions both osmotic and surface terms make $\Delta F>0$, so the pore becomes extremely size-selective, with $R\gtrsim\exp(d^{3})$, effectively blocking even moderately sized particles.

%A decrease in $\chi_{\text{PC}}$
%%or increase in $\chi_{\text{PS}}$ (\todo{This maybe not universal}) 
%lowers the interfacial term $\gamma$ throughout the polymer filling.
%When $\chi_{\text{PC}}$ drops below the critical value $\chi_{\text{ads}}^{\text{crit}}$, $\gamma$ becomes negative.
%Due to the cubic dependence of $\Delta F$ on particle size $d$, particles below some size have negative insertion free energy, with $F(d)$ having a maximum $d=0$ followed by a minimum $-\tfrac{4\gamma}{\Pi}$.
%The contributions of diffusion slowdown due to the polymer meshwork to the resistance suppresses the extrema when $\chi_{\text{PC}}$ and $\gamma$ is not sufficiently negative, so non-monotonic behaviour in $R(d)$ appears only for strongly attractive particle surface, when $\chi_{\text{PC}}$ and $\gamma$ is sufficiently negative.

%As long as $\gamma$ is weakly negative the landscape $\Delta F(d)$ is mostly positive—or only slightly negative—and the resistance rises monotonically on particle size $d$, within $d^3 \lesssim R(d) \lesssim \exp(d^3)$ for smaller particles when $\Delta F(d)$ is slightly negative, and $R \gtrsim \exp(d^3)$ for larger particles when $\Delta F(d)$ become positive.
%This $R(d)$-scaling regime appears in Figure \ref{fig:R_vs_d}a for $\chi_{\text{PS}} = 0.5$ when $\chi_{\text{PC}}\approx -1.0$ (and, more generally, for $\chi_{\text{PC}} \lesssim -0.7$, which ensures $\gamma\le 0$ everywhere).
%A similar behaviour is seen in Figure \ref{fig:R_vs_d}b at $\chi_{\text{PC}} = -1.25$ for solvent parameters $\chi_{\text{PS}} = 0.3$ and $0.4$.
%%(and more generally for any $\chi_{\text{PS}}$)


%Further decreasing $\chi_{\text{PC}}$ drives $\gamma$ and $\Delta F(d)$ sufficiently negative such that $R(d)$ is no longer monotonic, showing the expected maximum (labelled \raisebox{0.25em}{\scriptsize$\uparrow$}(a) in Figure \ref{fig:R_vs_d}a) and minimum (\raisebox{0.25em}{\scriptsize$\uparrow$}(b)). 
%For particles larger than the size at this minimum, $\Delta F(d)$ turns positive and the resistance again follows the steep $R!\gtrsim!\exp(d^{3})$ scaling.
%Hence, when the surface attraction exceeds $\chi_{\text{ads}}^{\text{crit}}(\chi_{\text{PS}})$ the pore acts as a gate: it is virtually insensitive to particle size below a critical diameter (as (\raisebox{0.25em}{\scriptsize$\uparrow$} in in Figure~\ref{fig:R_vs_d}a(b))), yet becomes highly restrictive once that diameter is exceeded.
%This gating behaviour is exemplified by the curve for $\chi_{\text{PS}} = 0.5,;\chi_{\text{PC}} = -1.25$ in Figure~\ref{fig:R_vs_d}a,b.

%Increasing the solvent strength parameter $\chi_{\text{PS}}$ raises $\gamma$ Eq.~(\ref{eq:chi_ads}) and lowers the osmotic pressure $\Pi$ Eq.~(\ref{eq:osmotic}).  
%For large particles ($d>4$), decreasing $\chi_{\text{PS}}$ has the same qualitative effect as increasing $\chi_{\text{PC}}$: both drive the insertion-free-energy landscape toward more positive values.
  
%For small particles, hindered diffusion due to polymer meshwork has dominant contribution to $R(d)$ curves that scales within $d \lesssim R \lesssim d^3$.
%When $\chi_{\text{PS}}$ increases, the brush deswells and the polymer concentration in the pore lumen rises, which leads to a slightly lower diffusion coefficient $D(\phi)$ and slightly higher resistance $R$ at higher $\chi_{\text{PS}}$ for the same particle size.
%Accordingly, in Figure~\ref{fig:R_vs_d}b the curves for $d<4$ stack almost vertically, ordered by decreasing $\chi_{\text{PS}}$.

%Further decrease in $\chi_{\text{PC}}$ or increase in $\chi_{\text{PS}}$ will drive 

When the insertion free energy strongly negative, the pore interior is effectively short-circuited, $R_{\text{int}} \to 0$, 
so the total resistance is set by the finite exterior contribution, $R\approx R_{\text{ext}}$ as in Eq.~(\ref{eq:R_tot_tot}); 
see Figure~\ref{fig:resistivity_profile}.  
For very attractive particles the swollen-brush loose fringe can be highly permeable, 
further lowering $R_{\text{ext}} \to 1 / (D_0 \pi r_\text{cap})$ which gives the Stokesian scaling  $R \sim d$.

As particle size increases under sufficiently negative $\gamma$, a local minimum in $R(d)$ is expected (labelled \raisebox{0.25em}{\scriptsize$\uparrow$}(b) in Figure~\ref{fig:R_vs_d}a).  
Because the total resistance is bounded below by $R_{\text{ext}}$, the parabolic section that would otherwise appear is truncated and replaced by an almost linear segment.  
The onset of this linear regime, indicated by \raisebox{0.25em}{\scriptsize$\uparrow$}(c) in Figure~\ref{fig:R_vs_d}a, marks the capped minimum.  
Beyond some particle size, further increases in $d$ amplify the osmotic penalty 
%and particle excluded volume, 
so the curve leaves the linear section and rises along the upward branch of the original parabola.

%%%%%%%%%%
\subsection{Experiments of colloid transport through nuclear pores validate the theoretical predictions}
%%%%%%%%%%

To test how well our theory predicts the experimental reality, we analysed literature data pertinent to colloid transport across nuclear pores.

The estimated distance between nuclear pore complexes (NPCs) in the nuclear envelope is approximately 10 times larger than the pore diameter \cite{Yang2004, Daigle2001, Feldherr1984, Kubitscheck2000}). At such distances, transport across neighbouring NPCs is not mutually interfering \cite{Fabrikant1985}, as can be appreciated from the iso-concentration lines in Figure~\ref{fig:colloid_concentration}. Experimentally measured transport rates, normnalised against the number of pores, therefore can be directly compared with our theoretical predictions.  

It is well-known that colloids with affinity for the disordered nucleoporin FG domains that fill the NPC (such as importins and exportins) are enriched in or near NPCs \cite{Beck2007, Gruenwald2010, Tu2011}, and in microscopic droplets, macroscopic hydrogels and thin films assembled from pure FG domains.
Moreover, high concentrations of transport factors are essential for effective transport through the pore \cite{Lowe2015}.
These observations fully align with our predictions that the accumulation of colloids in the pore is required for facilitated transport (Figure~\ref{fig:colloid_concentration}).
%Need to check and add references.

As the example pore geometry and polymer density in Figure~\ref{fig:colloid_transport} were modelled to represent a NPC (Supporting Method 1), we can directly compare our theoretical predictions with experimental data. To this effect, the effective statistical segment length of disordered polypeptide chains was taken to be $a$ = 0.8 nm. 
%Need to check and add references.

Mohr et al. \cite{Mohr2009} quantified the rates of diffusive transport across NPCs for non-sticky colloids of varying size.
Cytosolic proteins had been washed out in the experiments, and nuclear pores were filled with a plain FG domain brush without transport factors, closely corresponding to the assumptions of our simplified model.
Figure~\ref{fig:NPC_comparison}a compares these experimental results with the theoretical predictions of our model.
To compare theory and experiment, we estimated the effective solvent strength in the NPC to be close to $\theta$-solvent ($\chi_{\text{PS}} = 0.5$), consistent with varying yet generally moderate levels of 'cohesiveness' observed for FG domains.
We indicate the theoretical curves for perfectly inert colloids with $\chi_{\text{PC}} = 0$ and for weakly attractive colloids with $\chi_{\text{PC}} = -0.5$ as reasonable boundaries for non-sticky colloids. 
The only adjustable fitting parameter in our model was the prefactor $\alpha$ in the scaling-based expression for the diffusion coefficient, Eq.~(\ref{eq:Rubinstein}).
It is clear that the theory reproduces the experimental trends for the increase in pore resistance with colloid size remarkably well.
The best fit was obtained with $\alpha = 5.5$ and this value was hence fixed  throughout the paper.
%Need to checkand add references.

\begin{figure}
    \centering
    % \begin{subfigure}[b]{0.45\textwidth}
    %     \includegraphics[width=\textwidth]{fig/experimental_inert.png}
    % \end{subfigure}
    % \begin{subfigure}[b]{0.45\textwidth}
    %     \includegraphics[width=\textwidth]{fig/experimental_permeability_on_partitioning.png}
    % \end{subfigure}
    \includegraphics[width = \linewidth]{fig/validation.png}
    \caption{
    Comparison of theoretical predictions with experimental findings for NPCs
    \textbf{(a)} 
    Gating of non-sticky colloids by size.
    Nuclear import rate $k$ vs. hydrodynamic diameter $d$ (symbols) measured by Mohr et al. for inert colloids of varying sizes on permeabilized Hela cells.
    %Need to add reference.
    Theoretical predictions (lines) are for the pore and brush parameters as given in Figure~\ref{fig:colloid_transport}, $\chi_{\text{PS}} = 0.5$, and $\chi_{\text{PC}} = 0 \text{ and } -0.5$ (as indicated).
    The best fit shown here was obtained with $\alpha = 5.5$.
    The predicted transport rates across a bare pore are also shown, for reference.
    \textbf{(b)} 
    Gating of colloids by their affinity.
    Nuclear import rate $k$ (obtained as in a) vs. partition coefficient $\text{PC}=\left(c_{\text{in}}/c_{\text{out}}\right)_{\text{gel}}$ into phase-separated droplets of pure FG domains (Nup98A - stars; Nup116 - lozenges) measured by Frey et al. for of a range of green fluorescent protein variants.
    Theoretical predictions (lines) represent the limits of a homogeneously attractive colloid surface (blue line) vs. $\Delta F_\text{sur}$ concentrated in a single sticky patch (orange line).
    The pore and brush parameters are as given in (a), except for $\chi_\text{PC}$ which was linked to $\text{PC}$.
    }
    \label{fig:NPC_comparison}
\end{figure}

Using an approach similar to Mohr et al., Frey et al. quantified NPC transport rates for a wide range of green fluorescent proteins (GFPs) with surface amino acids mutated to modulate transport from 'superinert' to 'transport factor like'.
In parallel, the ability of these variants to enrich or deplete in phase-separated droplets of two pure FG domains (Nup98A and Nup116) was quantified (Figure~\ref{fig:NPC_comparison}b).
This set of experiments enabled the effect of colloid affinity to be tested selectively as the colloid size and shape were effectively constant.
The transport rate was observed to correlate strongly with the level of GFP enrichment in FG domain phases.
To recapitulate the correlation between nuclear transport rates $k$ and partition coefficient $PC$ in pure FG domain phases with our theory, we computed $PC = \exp(-\Delta F)$ as a function of $\chi_\text{PC}$ in a homogeneous polymer phase.
To this end the solvent strengths $\chi_\text{PS}$ were estimated independently for each pure FG domain from their respective concentrations in spontaneously phase-separated droplets (Supplementary Methods 10).
Our idealised assumption of colloids being homogeneously interactive (blue line in Figure~\ref{fig:NPC_comparison}b) reproduced the experimental data for non-sticky and weakly attractive colloids well without any adjustable parameter.
For more strongly attractive colloids however, this approach overestimated the transport rated by up to an order of magnitude.
Theoretical prediction assuming the opposite extreme of all surface free energy being concentrated into a single sticky patch (orange line in Figure~\ref{fig:NPC_comparison}b) reproduced the experimental data quite well, suggesting that the presence of localized sticky patches on the colloid surface and the FG domains slows down diffusion and transport.
Interestingly, this comparison suggests that protein transport across the NPC is not optimised for the highest rate.
%Need to check and add references.

Taken together, the quantitative agreement between our theory and a range of experimental data for nuclear pore transport with a very limited number of adjustable parameters provides strong validation for the validity of our theory.


%%%%%%%%%%
\section{DISCUSSION}
%%%%%%%%%%

Our most striking, and counterintuitive, finding is that an attractive polymer brush can enhance the net colloid transport in a concentration gradient across the pore by an order of magnitude and more, compared to a bare pore.
Moreover, we have shown how mesopores with polymer brushes can gate transport with exquisite selectivity with respect to polymer-colloid affinity and colloid size, even for colloids that are substantially smaller than the pore size.

Our findings shed light on possible mechanisms of selective transport through nuclear pore complexes (NPCs) and, at the same time, suggest a molecular design strategy for controlling selective permeability through artificial mesoporous membranes, with potential applications in fields such as targeted drug delivery, biosensing, and filtration systems.

%%%%%%%%%%
\subsection{Towards technological applications of synthetic polymer-filled mesopores}
%%%%%%%%%%

By tuning parameters like particle size, polymer-colloid affinity, and solvent quality, it is possible to modulate transport properties and achieve desired selectivity levels in synthetic membranes.
These insights could pave the way for designing nanoporous materials with enhanced selectivity tailored to specific functional requirements, thereby broadening the scope of applications in nanomedicine, biotechnology, and environmental engineering.

Mixtures of biological colloids such as folded proteins and other biomacromolecular complexes, as well as synthetic colloids such as nanoparticles, may be effectively separated, not only according to their size but also their surface (bio-)chemistry.
Importantly, the here-presented theoretical approach and integration schemes facilitate the rational design of pores with a geometry and polymer filling optimised for the desired separation task.

Individual pores, as we have considered here, are routinely deployed in current nanopore sensing technologies.
These technologies enable detection and characterization of individual macromolecules as they travel across the pore.
Our findings suggest polymer fillings as an attractive tool to optimize the performance of nanopore sensing.
Placing a suitable polymer filling upstream the pore's sensing region would enable pre-selection of target solutes from complex mixtures for a focused analysis by the pore.
Polymer fillings may also be placed in the very sensing region of the pore to enhance both selectivity and sensitivity.
The here-proposed approach is distinct from previous approaches, where responsive polymer coatings along the pore walls were used to open/close a polymer free channel on application of an external stimulus such as a change in temperature, ionic strength or pH. 

Individual pores will though typically be insufficient in applications that focus on separation with high throughput such as filtration systems.
This limitation can be overcome by multiplexing, e.g., with membranes featuring a large array of mesopores.
Our theoretical approach remains valid for such arrays as long as the distance between pores remains sufficiently large for the diffusion trajectories of adjacent pores not to substantially interfere.
Fortunately, this condition can be met with a relatively tight packing of pores, as can be appreciated from the iso-concentration lines in Figure~\ref{fig:colloid_concentration}.
In practice, a distance between pore centres approximately 10-fold greater than the pore diameter should entail minimal interference \cite{Fabrikant1985}.

\bigskip

\noindent{The main design concepts emerging from our theory are:}

\textbf{1.}
For a polymer-filled pore to function as a selective transport channel, high permeation selectivity must be coupled with low resistance to diffusive flux.
We refer to this combination as 'gating' behavior, where a minor change in colloid size or polymer-colloid interaction strength can dramatically shift the permeation rate from facilitated transport to virtually complete blockage.
Both requirements can be achieved near the critical values $d_{\text{crit}}$ and/or $\chi_{\text{PC}}^{\text{crit}}$, which assure that the resistance of the brush-filled pore matches that of the bare pore, $R\simeq R_{0}$.
The gating effect is particularly pronounced for larger particles: in our case, with the diameter of 10 polymer segment lengths or more.

\textbf{2.}
Pore resistance is highly sensitive to parameters that influence insertion free energy.
Strong dependence of the pore resistance on the parameters of the colloid originates from the exponential dependence of the local conductivity on the insertion free energy.
This results in very high selectivity of the polymer brush with respect to colloid size and polymer affinity, as demonstrated in Figures \ref{fig:R_vs_chi_PC} and \ref{fig:R_vs_d}.
The osmotic contribution to the insertion free energy scales as $d^3$ while the interfacial contribution comprises $\chi_{\text{PC}}$ and scales as $d^2$.
Thus, under conditions when colloid transport is limited by the polymer brush (as opposed to plain solvent), a slight change in $d$ and/or $\chi_{\text{PC}}$ translates into a drastic change in permeability (resistance).

\textbf{3.}
The maximal permeability of the polymer-filled pore is limited by the resistance of the exterior region.
Whilst strong  polymer-colloid attraction can make the resistance of the pore interior effectively vanish ($R_{\text{int}} \to 0$), this is not the case for the pore exterior, where mass transport in plain solvent always provides a non-vanishing resistance.
Under poor solvent conditions  polymers are typically confined to the pore interior, and the total resistance is bounded from below by the Rayleigh resistance  $R \geq R_{\text{ext}}^{0} = \frac{1}{2 D_0 r_{\text{p}}}$.
Polymer caps emerge outside the pore under good or $\theta$-solvent conditions decreasing the path through plain solvent, and thereby can reduce the total resistance even further. Approximating the caps as hemispheres with radius $r_{\text{ext}}$, and assuming them highly attractive, leads to $R_{\text{ext}} \to \frac{1}{ \pi D_0 r_{\text{ext}}}$.
The maximum additional reduction in total resistance due to the presence of attractive polymer caps thus scales as $\frac{R_{\text{ext}}^{0}}{R_{\text{ext}}} \to \frac{\pi r_{\text{ext}}}{2 r_{\text{p}}}$.
Hence, a large polymer cap is beneficial for transport rates, although even a moderate cap size can lead to substantial gain (e.g., approximately 3-fold for $r_{\text{ext}}\simeq L = 2r_{\text{p}}$ as suggested by Fig.\ref{fig:phi_hm_grid}).

\bigskip

\noindent{The manufacturing of functional mesoporous membranes is an emerging art, and we hope that our theoretical efforts will both promote and guide future practical developments in this area.}
%COMMENT RR: We ought to provide some references on the manufacturing of mesoporous membranes.
%COMMENT RR: One can expect that transport rates will increase further with a pressure gradient that drives solution flow across the membrane. We could mention this here as an avenue worthy exploring in future work? 


%%%%%%%%%%
\subsection{Implications for nuclear pore permselectivty}
%%%%%%%%%%

A remarkable number of features that we have identified with our theory is also found in the transport of proteins and other biomacromolecules through nuclear pore complexes (NPCs), suggesting that our model is capable of capturing the basic mechanisms of nuclear pore permselectivity in spite of some rather simple assumptions.

The estimated distance between NPCs in the nuclear envelope is approximately 10 times larger than the pore diameter \cite{Yang2004, Daigle2001, Feldherr1984, Kubitscheck2000}), and transport across neighbouring NPCs therefore can be considered mutually non-interfering. 

Single-cargo tracking studies using fluorescence \cite{Musser2016, Lowe2010, Lowe2015, Yang2004, Kubitscheck2000, Ma2010} and tomography \cite{Beck2007} have shown that transported colloids (e.g., importins, exportins and their complexes with cargo) primarily traverse the central region of the NPC and are rarely observed near the pore walls.
Such a behaviour is consistent with the  insertion free energy lanscape in our model (Figure~\ref{fig:DeltaF_map}, bottom).
Importantly, such a path does not require the presence of an empty (i.e., polymer free) channel as had been suggested in some earlier models of NPC transport. Instead, subtle variations in polymer density across the pore's cross-section substantially determine where colloids enrich and translocate.  
%Need to add further references.

On the other hand, these studies also evidenced that transport attempts are frequently aborted, with the transported colloid either dwelling near the pore entrance for a sustained time period or partially traversing into the pore before returning.
This behavior aligns with the predictions of our model that a negative insertion free energy at the periphery of the pore draws colloids into the pore, but the existence of a free energy barrier in the center of the pore (see Figure~\ref{fig:DeltaF_map}, top) would naturally lead to a large number of unsuccessful translocation attempts.

It is also well-known that colloids with affinity for the disordered nucleoporin FG domains that fill the NPC (such as importins and exportins) are enriched in or near NPCs \cite{Beck2007, Gruenwald2010, Tu2011}, and in microscopic droplets, macroscopic hydrogels and thin films assembled from pure FG domains.
Moreover, high concentrations of transport factors are essential for effective transport through the pore \cite{Lowe2015}.
These observations fully align with our predictions that the accumulation of colloids in the pore is required for facilitated transport (Figure~\ref{fig:colloid_concentration}).
%Need to check references.

Solvent strength conditions for nucleoporins can be estimated to be close to $\theta$-solvent, as attested by a certain level of "cohesiveness" observed for FG domains. 
The effective statistical segment length of disordered polypeptide chains is $a$ = 0.8 nm. In the theory, the pore radius and length, as well as the nanoparticle diameter are normalized by it. The pore parameters indicated in Figure~\ref{fig:colloid_transport} were actually taken to represent a NPC.
%An effective size limit of approximately 5 nm has been reported for the passive permeation of 'inert' colloids (i.e., colloids that do not bind or bind only weakly to the FG domains).
%Considering the regime of weak polymer attraction ($\chi_{\text{PC}} > -0.5$), one can see (Figure~\ref{fig:colloid_transport}a) that this value matches the prediction of our model remarkably well.
%Need to add further references.

Mohr et al. quantified the rates of diffusive transport across NPCs for 'inert' colloids of varying size. Cytosolic proteins had been washed out in the experiments, and nuclear pores were filled with a plain FG domain brush without transport factors, closely corresponding to the assumptions of our simplified model. Figure~\ref{fig:NPC_comparison} compares these experimental results with the theoretical predictions of our model. We indicate the theoretical curves for perfectly inert particles with $\chi_{\text{PCS}} = 0$ and for weakly attractive particles with $\chi_{\text{PCS}} = -0.5$. It is clear that the theory reproduces the trend of an increase in pore resistance with colloid size remarkably well. The single fitting parameter is the prefactor $\alpha$ in the scaling-based expression for the diffusion coefficient, see Eq (\ref{eq:Rubinstein}). The best fit value $\alpha=30$ is used throughout the paper, in particular to describe the experimental data in the next Figure \ref{fig:NPC_attr_comparison}.



% \begin{figure}
%     \centering
%     \includegraphics[width = 0.5\linewidth]{fig/experimental.png}
%     \caption{
%     Comparison of theoretical predictions with experimental findings for NPCs.
%     \textbf{(a)} Gating of colloids by size.
%     Pore resistances vs. hyrodynamic diameter (symbols) were estimated from import rates into the nucleus of permeabilised HeLa cells measured by Mohr et al. for inert colloids of varying sizes.
%     Theoretical predictions (lines) are for the pore and brush parameters as given in Figure~\ref{fig:colloid_transport}, $\chi_{\text{PS}} = 0.5$ and $\chi_{\text{PCS}} = 0$.
%     \textbf{(b)} Gating of colloids by affinity.
%     Pore resistances vs. insertion free energy (symbols) were estimated from Frey et al. for a range of green fluorescent proteins with surface amino acids mutated to modulate transport from 'superinert' to 'transport factor like'.
%     Pore resistances were obtained from import rates into the nucleus of permeabilised HeLa cells; insertion free energies were obtained from partition coefficients in phase-separated droplets of Mac98A FG domains.
%     Theoretical predictions (lines) are for the pore and brush parameters as given in Figure~\ref{fig:colloid_transport}, $\chi_{\text{PS}} = 0.5$ and $d = 6$ (i.e. equivalent to the hydrodynamic diameter of GFP).    
%     }
%     \label{fig:NPC_comparison}
% \end{figure}

\begin{figure}
    \centering
    \includegraphics[width = 0.45\linewidth]{fig/experimental_inert.png}
    \caption{
    Comparison of theoretical predictions with experimental findings for NPCs. Gating of colloids by size.
    Pore resistances vs. hyrodynamic diameter (symbols) were estimated from import rates into the nucleus of permeabilised HeLa cells measured by Mohr et al. for inert colloids of varying sizes.
    Theoretical predictions (lines) are for the pore and brush parameters as given in Figure~\ref{fig:colloid_transport}, $\chi_{\text{PS}} = 0.5$ and $\chi_{\text{PC}} = 0, \chi_{\text{PC}} = -0.5$.
    }
    \label{fig:NPC_comparison}
\end{figure}

\begin{figure}
    \centering
    \includegraphics[width = 0.45\linewidth]{fig/experimental_permeability_on_partitioning.png}
    \caption{
    Comparison of theoretical predictions with experimental findings for gating of colloids by affinity in NPCs.
    Nuclear import rate vs. the protein partition coefficient (symbols).
    Theoretical predictions (lines) correspond to two limits: homogeneous surface; a single binding spot. The pore and brush parameters are given in Figure~\ref{fig:colloid_transport}, $\chi_{\text{PS}} = 0.5$.
    }
    \label{fig:NPC_attr_comparison}
\end{figure}

Finally, the NPC features a remarkable rate of facilitated permeation and an exquisite permselectivity with respect to the surface features of relevant proteins (e.g., importins and exportins).
Using an approach similar to Mohr et al., Frey et al. quantified NPC transport rates for a wide range of green fluorescent proteins (GFPs) with surface amino acids mutated to modulate transport from 'superinert' to 'transport factor like'.
In parallel, the ability of these variants to enrich or deplete in phase-separated droplets or hydrogels of pure FG domains was quantified (Figure~\ref{fig:NPC_comparison}b).
This set of experiments enabled the effect of colloid affinity to be tested selectively as the colloid size and shape were effectively constant.
The transport rate was observed to correlate strongly with the level of GFP enrichment in FG domain phases.
Our theory reproduces the salient features of this experimental system.
First, the experimental data provide direct evidence that transport-factor-like proteins can indeed be transported at a rate exceeding the rate of an empty pore.
Second, the experimental data qualitatively demonstrate the expected affinity gating, with the most attractive GFP variants experiencing pore resistances several orders of magnitude smaller then the most inert variants.
Third, transport rates are approaching a plateau for the most attractive GFP variants tested, suggesting that in this regime transport is limited by the diffusion to the pore rather than the pore itself (see Figure~\ref{fig:R_vs_chi_PC}a).
%Need to add references.

Neither the colloids nor the polymers pertinent to the NPC are as regular as assumed in our model.
Importins, exportins and their cargo have complex, non-spherical shapes and display substantial surface heterogeneity.
Similarly, each FG domain type exhibits substantial heterogeneity along the chain contour.
We present two theoretical curves according to two limiting assumptions: homogeneous surface vs. a single binding spot.

Moreover, the NPC features a variety of nucleoporin FG domains, with the body of available structural and biochemical data suggesting that the cohesiveness of nucleoporin FG domains is highest in the centre and decreases towards the periphery of the pore.
Qualitatively, one can envisage that the increased solubility of peripheral FG domains promotes a more extended polymer cap, thus minimising total pore resistance and maximising transport rates for strongly attractive colloids.
The reduced solubility of the central FG domains, on the other hand, would minimize the size threshold for gating of non-adhesive colloids.
Our model can be further extended to incorporate  solubility gradients and to explore such phenomena in more detail.
%Need to add references.

Overall, the agreement between the many experimental observations and the predictions of our theory is striking and strongly suggests that it provides an appropriate description of the basic mechanism of nuclear pore permselectivity.
The agreements are indeed remarkable given that the nuclear pore complex exhibits a much higher chemical complexity than our model.

%COMMENT RR: I have below left some further info about NPCs that Mikhail had gathered. I leave these for further consideration, though I am not sure how useful they are to the discussion. 

% In the ref\cite{MoussaviBaygi2016}, authors proposed that in the transport event locally collapses upon interacting with the NTR-bearing macromolecule, but autonomously reconstructs itself very fast, keeping the pore sealed.
% Ref \cite{Hough2015} also proposed that FG-motives create highly dynamic phase that can extremely quick exchange contacts with transport factors of cargo.
% Ref \cite{Milles2015} anticipates that fast transport requires rapid exchange when engaging FG-motives with the NTR
% Ref \cite{Goodrich2018} also proposed binding-mediated mechanism that changes local structure, destroying local cages.

%%%%
\printbibliography
\end{document}




































%%%%%%%%%%%%%%%%%%%%%%%%%%%%%%%%%%%%%%%%%%%%%%%%%%%%%%%%%%%%%%%%%%%%%%%%%%%%%%%%%%%%%%%%%%%%%%%%%%%%%%%%%%
%RR: In the following, I have left some pieces of text that could be inserted into the main text where desired
%%%%%%%%%%%%%%%%%%%%%%%%%%%%%%%%%%%%%%%%%%%%%%%%%%%%%%%%%%%%%%%%%%%%%%%%%%%%%%%%%%%%%%%%%%%%%%%%%%%%%%%%%%

%Under good (or \theta-) solvent conditions we may consider separately the situations with positive and negative insertion free energies. 
%Negative insertion free energies are rather exceptional under good solvent conditions. As we see below, in this case $R_{caps}\leq R_{convergent}$ and the total resistance
%is lower than that of the empty pore.
%Positive insertion free energies under good solvent conditions are more common. 
%In this case, the resistance of the pore interior is always dominant, 
%$$
%R_{tot}\approx R_{\text{int}}
%$$
%and the accuracy in estimating the resistance contributions from the entrance/exit regions is not of a major concern. 

%In Figure \ref{fig:fe_scf_grid} the insertion free energy profiles $\Delta F(z,r=0)$ calculated by analytical scheme and by SF-SCF method 
%are presented as a function of position of a spherical particle along the pore axis.
%While the SF-SCF method provides the net free energy, the analytical scheme allows decoupling of the free energy into osmotic and surface contributions, 
%which are shown separately in Figure \ref{fig:fe_scf_grid}.
%The numerical coefficients $b_0$ and $b_1$ in eq \ref{} are chosen by the best fit, but appear to be fairly universal and independent of the particle size 
%and interaction parameters $\chi_{PS,PC}$.
%Remarkably, the fit fails when the size $d$ became comparable with the pore diameter or in the case of extreme $\chi_{ads}$ values 
%when analytical scheme is not applicable because of strong perturbation 
%of the brush structure by inserted particle, while SF-SCF method can still be safely used
%for the evaluation of the insertion free energy.

%The 2D insertion free energy $\Delta F(r,z)$ patterns have rather complex shape. However, we can trace their evolution upon changing interaction parameters
%looking at the position-dependent free energy of the particle on the pore axis, $\Delta F(z,r=0)$.
%As one can see from Figure \ref{fig:fe_scf_grid}, the insertion free energy profiles evolve upon changing the interaction parameters $\chi_{PS,PC}$ as follows:
%At $\chi_{ads}\geq \chi_{crit}$ which is the case under good or theta-solvent conditions and weak or absent polymer-particle attraction, $|\chi_{\text{PC}}|\leq 1$, the positive osmotic
%term, $\Delta F_{osm}\geq 0$ dominates in the insertion free energy, which is positive and reach maximal value in the pore center, where polymer concentration is maximal.
%Hence, polymer-particle interaction has overall repulsive character and $\Delta F(r,z)$ has the shape of the free energy barrier preventing penetration and accumulation of particles in the pore.
%By using the insertion free energy $\Delta F(r,z)$ one can calculate the equilibrium partition coefficient 
%$$
%P=\int_{0}^{r_{pore}}2\pi rdr\int_{0}^{L_{0}}dz\exp (-\Delta F(r,z)/k_BT)/\pi r^{2}_{pore}L_{0}
%$$
%is larger than unity, $P\geq 1$. Noticably the repulsive free energy profiles extends beyond the edges of the pore, because of the fringes in the polymer density distribution in swollen brush.

%A decrease in $\chi_{ads}$ triggered by a decrease in  $\chi_{\text{PC}}$ or/and an increase in $\chi_{\text{PS}}$ leads to qualitative changes in the insertion free energy 
%$\Delta F(r,z)$ patterns: At $\chi_{ads}\leq \chi_{crit}$ the particle surface becomes
%adsorbing for the polymer, $\gamma \leq 0$, that gives rise to a negative contribution $\Delta F_{\text{sur}}(r,z)$ to the insertion free energy. 
%When $\chi_{\text{PS}}$ increases (the solvent is getting worse for the polymer)
%the osmotic pressure inside the brush decreases that leads to a decrease in the 
%magnitude of $\Delta F_{osm}(r,z)$ with the concomitant shrinkage of the  protruding outside the pore parts of the brush where  $\Delta F(r,z)\neq 0$.
%As a result, the $\Delta F_{\text{sur}}(r,z)$ aquires two minima with negative values near the endtance and the exit of the pore, separated by a maximum centered in the middle of the pore
%where polymer concentration is larger and the osmotic repulsive term  $\Delta F_{osm}(r,z)$ dominates.
%Finally, at strong polymer-particle attraction $\chi_{ads} < \chi_{crit}$, the negative surface contribution $\Delta F_{\text{sur}}(r,z)\leq 0$ overperform osmotic repulsion everywhere inside the pore
%and the $\Delta F(r,z)$ aquires the shape of the potential well centered in the middle of the pore, which gives rise to preferential accumulation of particles in the pore, $P\geq 1$.
