\documentclass[12pt, a4paper]{article}
\usepackage{graphicx}
\usepackage{amsmath, amssymb, amsfonts, mathtools}
\usepackage{subcaption}
\usepackage[
backend=biber,
natbib=true,
style=numeric,
sorting=none,
doi=false,
isbn=false,
url=false,
eprint=false
]{biblatex}
\usepackage{xcolor}
\usepackage{bm}

%\REMOVE BEFORE SUBMISSION
\usepackage{lineno}
\linenumbers

%\REMOVE BEFORE SUBMISSION
\newcommand\todo[1]{\textcolor{red}{#1}}

\addbibresource{biblio.bib}
\title{A polymer filling enhances the rate and selectivity of colloid permeation across mesopores}

\author{Mikhail Y. Laktionov$^1$, Leonid I. Klushin$^{2,3}$,\\
Ralf P. Richter$^4$, Frans A. M. Leermakers$^5$, Oleg V. Borisov$^1$\\
$^{1}$CNRS, Universit\'e de Pau et des Pays de l'Adour UMR 5254,\\
Institut des Sciences Analytiques et de Physico-Chimie\\
pour l'Environnement et les Mat\'eriaux, 64053 Pau, France\\
$^{2}$Branch of Petersburg Nuclear Physics Institute\\
named by B.P.Konstantinov\\
of National Research Centre "Kurchatov Institute",\\
Institute of Macromolecular Compounds,\\
199004 St.Petersburg, Russia,\\
$^{3}$American University of Beirut, Department of Physics, Lebanon\\
$^{4}$University of Leeds, School of Biomedical Sciences,\\
Faculty of Biological Sciences, 
School of Physics and Astronomy,\\
Faculty of Engineering and Physical Sciences,\\  
Astbury Centre for Structural Molecular Biology,\\ 
and Bragg Center for Materials Research,\\ 
Leeds, LS2 9JT, United Kingdom\\
$^{5}$Physical Chemistry and Soft Matter,\\
Wageningen University, Stippeneng 4, 6708 WE Wageningen, the Netherlands\\
}
\date{}

\begin{document}
\maketitle

\begin{abstract}

Mesoporous membranes are emerging as new materials with potential applications in sensing and separation devices.
The nuclear envelope of eukaryotic cells provides a striking example of a functional mesoporous membrane, where diffusive transport is mediated by nuclear pore complexes (NPCs).
Transport across NPCs is mediated by a pore-filling meshwork of naturally-disordered proteins (FG-domains of nucleoporins) anchored to the pore walls, and is highly selective.
Even colloids much smaller than the inner diameter of the NPC are effectively blocked from transport, but some larger colloids with distinct surface features can readily permeate.
Simplistically, one may expect any polymer meshwork to slow down colloid movement as the steric constraints imposed by the polymer meshwork hinder permeation.
However, we demonstrate how a rationally designed polymer filling can instead increase the permeation rate, by an order of magnitude and more, compared to a bare pore.
Such enhanced permeation is achieved with a polymer phase that attracts the colloid and extends beyond the confines of the mesopore channel itself, thus maximizing colloid capture for diffusive transport across the pore.
We further define how polymer-filled mesopores can be designed to effectively gate colloids according to their size and surface features. 
This combination of feature provides a physical explanation for the basic mechanism of nuclear pore permselectivity, and renders mesopores promising as highly selective separation devices.
\todo{[Can we contrast our results with other separation devices, and spell out what our mesopores enable that was not possible before? This would help spelling out the significance of our findings.]}
\end{abstract}


%%%%%%%%%%
\section{INTRODUCTION}
%%%%%%%%%%

Polymer-modified mesoporous materials and membranes belong to a new class of functional nanostructured materials with great potential in a number of technologies.
\todo{[FL: I do not like this first line and paragraph not saying much... Consider starting with third paragraph and slip in somehow the first and second paragraph in the text. the advantage is that mesoporours have been defined and then you can say that these have potential( first paragraph) and in which direction these can be found (second paragraph).]}

The interaction of (macro)molecules and nanocolloidal particles with porous media, as well as their transport through porous membranes, are important elements of many technological processes (chromatography, heterogeneous catalysis, micro- and ultrafiltration, protein separation and purification, etc.) and, therefore, have been the subject of intensive research for more than sixty years \cite{Watson1959, Rout2003, Huang2023, Uredat2024}.

Advances in macromolecular chemistry have made it possible to significantly improve functional properties of materials with mesopores (i.e., pores with a diameter between a few and many tens of nanometers) by modifying them with polymers of various chemical nature anchored to  the pore walls.
\todo{[FL: Perhaps first/second paragraph inserted here?]}
Thus, a fuzzy meshwork of solvated polymers is formed filling the entire pore volume or just the near-wall regions, depending on the molecular mass and conformational state of the polymer chains.
The interaction of this polymer meshwork with colloids, that is, nanoparticles or (macro)molecules in the solution phase, essentially determines the absorption and separation properties of the polymer-modified mesoporous materials and membranes.
These interactions can be attractive or repulsive, and controlled by a broad spectrum of external stimuli \cite{Jeong2002, Lee2010, Low2019}, 
such as temperature                     \cite{Stetsyshyn2020}, 
pH and/or ionic strength of the medium  \cite{Dai2008, Zhang2005}, 
ion valency and specificity             \cite{Zhulina1999, Robertson2021},
electric fields                         \cite{Lokuge2005}, 
solvent composition                     \cite{Halperin2011, Darabi2016}, 
or complex biological stimuli           \cite{Ikeda2010, Lu2003}.
This opens up a unique opportunity for highly selective and controlled uptake and transport of colloids through polymer-filled mesoscopic channels.

Past experimental and theoretical efforts have focused on systems where an external stimulus triggered the transient opening of a polymer free path, typically along the center of the pore, to gate colloid transport.
However, the polymer phase itself can potentially also provide high selectivity to colloids as a function of their size and attraction by the polymer.
We thus hypothesized that even mesopores that are filled by a polymer meshwork across their entire cross-section can effectively gate colloid transport.
If successful, this approach would enable more robust gating as it does not rely on careful tuning of the diameter of a polymer-free channel, and higher transport rates as the full pore cross-section can potentially participate in colloid transport.

Nature provides a case in point.
Nuclear pore complexes (NPCs) perforate the nuclear envelope of eukaryotic cells and control the bulk transport of proteins and nucleic acids between the nucleus and the cytoplasm.
This process enables the spatial separation of gene transcription (in the nucleus) and translation into proteins (in the cytosol), and thus is critical for the ordered course of gene expression.
Each NPC forms a cylindrical channel, measuring approximately 40-60 nm in diameter and 40-95 nm in length (depending on the species \cite{Yang1998, Beck2004, VonAppen2015, Alberts2015, Hayama2017, Holzer2018}).
The channel is filled with a meshwork of several 100 natively disordered protein domains rich in phenylalainine-glycine dipeptides (FG domains) that are anchored to the channel walls \cite{Holzer2018, Ori2013, Rout2000, Dickmanns2015}.
Collectively, the FG domain meshwork provides remarkable gating function: biocolloids of 5 nm (i.e., just a tenth of the pore diameter) or more in hydrodynamic diameter are effectively blocked, except for some dedicated transport factors (importins and exportins, alone and in complex with cargo) which bind to the FG domains and can undergo rapid permeation. 

%RR: The below section is rather long, and may be shortened? Also need to review reference list.

Several independent strands of evidence indicate that the mechanism of diffusive transport across NPCs is based on rather generic physical principles, whereas the exact chemical makeup of the polymers and colloids is secondary for function.
Firstly, despite significant variations in the molecular building blocks of the NPC and the transport factors across distant eukaryotic taxa, NPCs consistently fulfill the same functional role \cite{DeGrasse2009, Maimon2012, Ori2013, Hayama2017, Yaron2018, Holzer2018}.
Secondly, the NPC can gate diffusive colloid transport similarly well in both directions.
Whilst the native cell is capable of directed transport of cargo against a concentration gradient \cite{Rout2003, Tijana2017}, this function is not an intrinsic part of the NPC itself but relies on energy derived from GTP hydrolysis by soluble intracellular proteins \cite{Lowe2015, Yang2004} and can even be reversed through cell engineering without modifying the NPC structure \cite{Nachury1999, Sakiyama2016}.
Thirdly, the binding behaviour of transport factors to assemblies of purified FG domains could be reproduced by simple physical models that treat FG domains as regular flexible polymers and transport factors as spherical colloids with a homogeneous surface. This approach provided faithful predictions even though it ignored the detailed arrangement of interaction sites along FG domains and on the transport factor surface. 
Fourthly, recent work with a range of mutants of green fluorescent protein (GFP) demonstrated that NPCs exhibit a wide and continuous spectrum of permeabilities as a function of colloid surface properties, and earlier studies with non-interacting colloids similarly evidenced a wide and continuous spectrum of permeabilities as a function of colloid size.
Additionally, certain native proteins with affinity to FG-domains, such as $\beta$-catenin, can translocate through the pore without the need for a transport factor, moving from the cytoplasm to the nucleoplasm along a concentration gradient due to their continuous binding to chromatin \cite{Rout2003}. This suggests that a fine balance of many individually weak physicochemical (e.g., electrostatic, hydrophobic, aromatic stacking, ...) interactions between polymers and biocolloids dictates the gating behaviour, rather than a few highly specific biochemical interactions.

%A similar structural motif was recently found in the internal channels of microtubules (about 15 nm in diameter) decorated with so-called microtubule intrinsic proteins (MEPs), presumably modifying microtubule stability and rigidity \cite{Mukhopadhyay2001}.

However, we currently lack an understanding of the relationship between the molecular architecture of the polymer brush filling the pore and its ability to transport colloids with high selectivity and rate. Here, we develop a theoretical approach to reveal the physical mechanisms of diffusive colloid transport across polymer-filled mesopores.
A meshwork of flexible polymers effectively increases the local viscosity and thereby slows down transport of colloids compared to an open pore. 
On the other hand, an attractive polymer phase recruits colloids into the pore, thus increasing colloid transport, and such recruitment is further enhanced when attractive polymers extend outside the pore lumen. 
Intriguingly, the solvent strength through its influence on the density and compactness of the polymer meshwork impacts all of these effects. 
Using a self-consistent field approach, we define how solvent quality and colloid attraction to the polymer may be tuned to maximize the transport rate (even beyond the rate for an open pore) and to achieve highly selective colloidal transport with respect to colloid size or affinity for the polymer.


%%%%%%%%%%
\section{RESULTS}
%%%%%%%%%%


%%%%%%%%%%
\subsection{Defining the transport scenario}
%%%%%%%%%%

Salient features of our simulated mesopore are illustrated in Figure~\ref{fig:colloid_transport}.
The cylinder-shaped pore perforates a planar membrane, and is the sole conduit for colloids between two semi-infinite solution reservoirs.
\todo{[FL: To my knowledge, in biology there is not a single membrane but a pair of membranes around the nucleus. The pore spans two membranes simultaneously and forms some saddle shaped pore ..... I understand that this issue is not important for you because the details about the membrane as a barrier are not accounted for in your model, but perhaps you can mention that in reality it is a double bilayer, which you choose to represent by an effective membrane...]}
Flexible polymer chains are end-grafted to the inner pore walls, at a density sufficient to form a polymer brush that fills the entire pore cross-section.

We will focus on a pore with a set radius $r_{\text{p}}^0$ and length $L^0$, and polymers with a degree of polymerization $N$ and grafting density $\sigma$ (Figure~\ref{fig:colloid_transport}). \todo{[FL: in units a]}
Whilst the selected values are inspired by the nuclear pore complex \todo{(see Supplementary Methods 1)}, we expect that our findings will be of rather general validity so they can be applied to the performance analysis and rational design of mesopores with other geometries or polymer fillings.

Colloids are taken to be sphere-shaped with diameter $d$.
The interaction strength (contact free energy) between a polymer segment and the colloid surface is represented by the Flory-Huggins parameter $\chi_{\text{PC}}$.

\begin{figure}
    \centering
    \includegraphics[width = 3.5in]{fig/pore_cartoon.png}
    \caption{
        Schematic illustration of colloid diffusive transport through a pore filled with a polymer brush. 
        The brush is formed by linear polymer chains (red strands) with a degree of polymerization $N$, uniformly grafted with grafting density $\sigma$ 
        to the inner surface of a cylindrical pore.
        The pore radius is $r_{\text{p}}^0$ (not indicated) and the thickness of the impermeable membrane is $L^0$.
        Polymer chains are flexible with a statistical segment length $a$ and volume $\sim a^3$. 
        Spherical colloids with diameter $d$ are free to diffuse in the surrounding solvent.
        All length scales are normalized by the statistical segment length $a$.
        As a model pore, we set $L_0 = 2r_{\text{p}}^0 = 56$, $\sigma = 0.02$ and $N = 300$.
        With $a = 0.76 {\text{ nm}}$, these parameters reproduce the basic features of nuclear pore complexes.
        }
    \label{fig:colloid_transport}
\end{figure}

To understand how the polymer brush affects transport, we consider the stationary diffusive flux of colloids through the pore and analyze how it depends on the parameters of the pore, the brush and the colloid.
Considering unidirectional colloid transport driven by a concentration difference across the membrane, we focus on the mechanisms of diffusion mediated by colloid-polymer interactions.
The colloid concentrations are set to zero and $c_0$ far away from the membrane (at $z\rightarrow\pm\infty$, respectively).\todo{[FL: is this clear? .... it is zero at $z \to \infty$ and $c_0$ at $z \to -\infty$]}


%%%%%%%%%%
\subsection{Colloid transport is defined by the sum of resistances of regions outside and inside the pore}
%%%%%%%%%%

%%%%%%%%%%
\subsubsection{Bare pore as a reference case}
%%%%%%%%%%

A natural reference is the diffusive flux through a bare pore, which itself limits the transport of solutes \cite{Deen1987, Sun2024}.
Lord Rayleigh analyzed the flux of point-like solute particles through a circular pore in a planar membrane of negligible thickness \cite{Strutt1878}.
In this simplest case, the equiconcentration surfaces are oblate spheroids, and the streamlines form confocal hyperboloids of revolution \cite{Cooke1966}.
The net flux through the pore is given by
\begin{equation}
    J=2D_0r_{\text{p}}\Delta c,
    \label{eq:flux_Ral}
\end{equation}
where we take $\Delta c = c_0$ without loss of generality, $r_{\text{p}}$ is the effective pore radius and $D_0 = k_{\text{B}}T / (3 \pi \eta_{\text{S}} d)$ is the diffusion coefficient of the colloid in plain solvent with viscosity $\eta_{\text{S}}$ ($k_{\text{B}}T$ is the thermal energy unit).

A membrane of finite effective thickness $L$ allows an approximate analytical solution \cite{Brunn1984}:
\begin{equation}
    J=\frac{2 D_0 r_{\text{p}}}{1+\cfrac{2L}{\pi r_{\text{p}}}}\Delta c.
    \label{eq:flux_finlength}
\end{equation}

Introducing the resistance $R$ to colloid flow $J = \frac{\Delta c}{R}$ provides a natural interpretation of Eq.~(\ref{eq:flux_finlength}) in terms of the total resistance of the pore:
\begin{equation}
    R = \frac{L}{D_0 \pi r_{\text{p}}^{2}} + \frac{1}{2 D_0 r_{\text{p}}} = R_{\text{int}}^{0} + R_{\text{ext}}^{0},
    \label{eq:resistance}
\end{equation}
where the superscript '0' refers to the bare pore.
The first term in Eq.~(\ref{eq:resistance}) represents the resistance of the interior of the bare pore.
The second term is the Rayleigh resistance (Eq.~(\ref{eq:flux_Ral})) and accounts for the effects of convergent flow toward the pore entrance and its symmetric counterpart at the pore exit.
Inside the pore, the flow lines are approximately axial.
In the bare pore scenario, the inverse of the diffusion constant ($\rho_0=D_0^{-1}$) represents the resistivity of the medium both inside and outside the pore.
Naturally, the resistance for a thin membrane is determined by the exterior region ($R \approx R_{\text{ext}}^{0}$ for $L \ll r_{\text{p}}$), whereas for long pores the inner region becomes dominant ($R \approx R_{\text{int}}^{0}$ for $L \gg r_{\text{p}}$).

The finite size of colloids affects the diffusive flux in two ways.
First, the excluded volume reduces the effective pore radius ($r_{\text{p}} = r_{\text{p}}^0 - d/2$) and increases the effective pore length ($L = L_0 + d$) \cite{Renkin1954, Beck1970, Bungay1973, Anderson1974, Brenner1977}.
Second, the presence of the pore walls entails some additional drag \cite{Ladenburg1907, Faxen1922, Haberman1958}.
We neglect the latter as the presence of the polymer brush screens hydrodynamics and thus requires a different kind of drag analysis, as discussed below.


%%%%%%%%%%
\subsubsection{A polymer filling affects the resistance of the pore itself, and also of regions outside the pore}
%%%%%%%%%%

Conformations adopted by overlapping polymer chains grafted to the pore walls are controlled by strong intermolecular interactions that depend on the solvent quality.
The solvent quality is here quantified by the Flory-Huggins parameter $\chi_{\text{PS}}$.
Values of $\chi_{\text{PS}}<0.5$ and $\chi_{\text{PS}}>0.5$ correspond to good and poor solvent, respectively, and $\chi_{\text{PS}}=0.5$ represents the ideal (or $\theta$-)solvent.

The polymer density profile $\phi(z,r)$ in the pore was calculated by the two-gradient self-consistent field numerical method of Scheutjens and Fleer 
\todo{(SF-SCF; see Supplementary Methods 2)}. 
%ML: I removed Supplementary Methods 2-3, as 3 is about extracting free energy contributions
In Figure~\ref{fig:phi_hm_grid}, one can appreciate the expected increase in polymer concentration inside the pore with decreasing solvent quality (increasing $\chi_{\text{PS}}$).
With the selected pore and polymer parameters (Figure~\ref{fig:colloid_transport}), the polymer brush fills the entire pore cross-section within the full range of solvent qualities explored ($\chi_{\text{PS}}\le0.9$), so that colloids need to navigate the polymer meshwork to move across the membrane.
For wider pores, shorter polymers and/or lower grafting densities, an open channel free of polymer may appear in the pore center, as discussed in detail elsewhere~\cite{Ligoure2001,Laktionov2021}.
This scenario would result in a distinct permeation behaviour, as colloids could move through the pore without traversing the polymer brush, and is not considered here.

\begin{figure}
    \centering
    \includegraphics[width = 3.5in]{fig/phi_hm_grid.png}
    \caption{
    Maps of the polymer volume fraction $\phi(r,z)$ for a polymer brush in a cylindrical pore with solvent quality ranging from good (upper left panel) to poor as indicated (lower right panel), as quantified by the Flory-Huggins parameter $\chi_{\text{PS}}$.
    Polymer volume fractions are mapped in cylindrical coordinates (as shown by $rz$-coordinate arrows), color coded as indicated and with selected iso-concentration lines. The blank space corresponds to plain solvent; the membrane is colored green.
    Pore and brush parameters are as given in Figure~\ref{fig:colloid_transport}.
    }
    \label{fig:phi_hm_grid}
\end{figure}

Figure~\ref{fig:phi_hm_grid} further illustrates that whilst the brush remains confined within the pore lumen in poor solvent ($\chi_{\text{PS}}=0.9$) it protrudes substantially into the surrounding space in ideal and good solvents ($\chi_{\text{PS}}\le0.5$), thus forming fringes on either side of the pore.
The polymer brush therefore will impact on the resistance to colloid flow within as well as outside the pore, such that
\begin{equation}
    R=R_{\text{int}}+R_{\text{ext}},
    \label{eq:R_tot_tot}
\end{equation}
with $R_{\text{int}}\rightarrow R_{\text{int}}^{0}$ and $R_{\text{ext}}\rightarrow R_{\text{ext}}^{0}$ in the limit of the bare pore.


%%%%%%%%%%
\subsection{Diffusivity and insertion free energy control diffusive transport}
%%%%%%%%%%

Zooming in, we can analyze how colloids are accumulated or depleted by the polymer meshwork due to attractive or repulsive interactions, respectively, and how the polymer meshwork affects the colloid's local mobility.

%%%%%%%%%%
\subsubsection{Local colloid mobility is determined by polymer mesh size, colloid size, and polymer-colloid attraction strength}
%%%%%%%%%%

The crowded polymers naturally decrease the diffusion of colloids.
A polymer brush is effectively described as an inhomogeneous semi-dilute polymer solution with a concentration-dependent correlation length (mesh size) $\xi(\phi)$.
Colloids of size $d > \xi$ experience additional friction as they are trapped by the polymer meshwork.
As a result, diffusion is slowed compared to pure solvent, leading to a position-dependent diffusion coefficient $D(r,z) < D_0$.

According to a scaling theory by Cai et al. \cite{Cai2011} for diffusion of non-sticky colloids in a semi-dilute polymer solution, the colloid mobility scales as $D\sim D_0 (\xi/d)^2\ll D_0$ for $d\gg \xi$, while small colloids diffuse virtually unimpeded ($D\sim D_0$ for $d\ll \xi$). We use a simple interpolation formula to capture the diffusion coefficient across the full range of relevant colloid sizes relative to the correlation length $d / \xi$:
\begin{equation}
    D\{\phi(r,z)\} = \frac{D_0}{1+[\beta d / \xi\{\phi(r,z)\}]^2} \approx \frac{D_0}{1+[\beta d \phi(r,z)]^2} .
    \label{eq:Rubinstein}
\end{equation}
The correlation length $\xi$ in Eq.~(\ref{eq:Rubinstein}) is controlled by the local polymer concentration $\phi(r,z)$ and also depends on the solvent quality \cite{DeGennes1979}.
In the term on the right hand side we have approximated $\xi \cong \phi^{-1}$, which is valid close to $\theta$-solvent conditions in a mean-field regime.
The coefficient $\beta$ in Eq.~(\ref{eq:Rubinstein}) is a numerical pre-factor.
In Section~\ref{sec:transport_of_non-sticky_colloids}, we estimate $\beta = 5.5$ by comparing our theoretical stationary flux predictions to experimental data on the flux of non-sticky colloids of different sizes through nuclear pore complexes.
In the following sections we consistently use this value for numerical estimates of the pore permeability and permselectivity.
Figures~\ref{fig:D_fe_map}a-b exemplify the colloid diffusivity, $D(\phi)/D_{0}$, varies with the local polymer concentration, $\phi(r,z)$, for a selected  colloid size ($d = 8$) in a $\theta$-solvent ($\chi_{\text{PS}} = 0.5$).

Recent theoretical, experimental and simulation studies extended the scaling approach of Cai et al. \cite{Cai2011} to include the effects of polymer-colloid attraction \cite{Yamamoto2018, Carroll2018}.
In the limit where proper diffusion of polymer chains is negligible (relevant to our case of pore-anchored chains), the characteristic desorption time of a single polymer-colloid contact $\tau_\text{des}(\epsilon)$ appears as an extra timescale.
Here, $\epsilon > 0$ is the absolute value of the activation energy to break a contact.
The polymer-colloid attraction may be considered weak if the desorption time is much smaller than the self-diffusion time.
Typically, this happens when $\epsilon \lesssim k_\text{B} T$, and in this case attractive forces only modify local friction by affecting the local packing of polymers around the colloid.
The magnitude of this effect is estimated to be small, reducing the diffusivity by at most a few 10\% \cite{Yamamoto2011}.
On the other hand, the desorption time grows exponentially with $\epsilon$ and affects the diffusion substantively when $\epsilon$ amounts to several $k_\text{B} T$ units.
Numerical simulations \cite{Yamamoto2018} demonstrate that the colloid diffusivity is reduced by an exponential factor with approximately half the activation energy:

\begin{equation}
    D(\epsilon)\approx D(\epsilon=0) \exp (-\epsilon / 2).
    \label{eq:Yamamoto}
\end{equation}

In our model, the energy of a polymer-colloid contact is $\chi_{\text{PC}}/6$, and therefore $\epsilon < 1/3$ within the explored range $0\geq\chi_{\text{PC}}\geq-2$.
Throughout most of the paper, we assume that the surface of the colloid is homogeneous, and neglect the small reduction in the local colloid mobility due to weak polymer attraction.
However, in Section~\ref{sec:transport_of_sticky_colloids} we will return to this question in the context of proteins traversing the nuclear pore, where the colloid surface and the polymers may be heterogeneous with affinity localized in a few sticky patches.
We can explore the limiting case where all the (negative) surface free energy $\pi d^2 \gamma(r,z)$ (with surface tension $\gamma$, vide infra) is assigned to a single sticky patch.
The diffusion coefficient is then modified as
\begin{eqnarray}
    D_{\text{sticky patch}} = 
    \begin{cases}
        D \exp(\frac{\pi d^2 \gamma(r,z)}{2}) & \text{if } \gamma(r,z) < 0 \\
        D & \text{otherwise}
    \end{cases}
     \label{eq:Sticky diff}
\end{eqnarray}
to interpolate between non-sticky and sticky colloids.
It is safe to assume that realistic cases of attractive polymer-colloid interaction lie between the two extremes of a perfectly homogeneous surface ($D_\text{homo} \approx D$) and a single sticky patch (Eq.~(\ref{eq:Sticky diff})).

Several other theoretical and empirical models have been proposed to describe the diffusion of colloids in polymer meshworks \cite{Schweizer2003,Kohli2012,Holyst2009,Phillies1988}.
Although the predictions of different models differ quantitatively, they all share the same qualitative trend.

%%%%%%%%%%
\subsubsection{Defining leading contributions to the insertion free energy}
%%%%%%%%%%

The position-dependent insertion free energy $\Delta F(r,z)$ is the work required to move a colloid from the exterior solution into the polymer brush.
For colloids that are significantly smaller than the size of the pore, the insertion free energy is determined entirely by the local polymer concentration (i.e., wall effects can be neglected), and comprises two distinct contributions:
\begin{eqnarray}
    \Delta F = \Delta F_{\text{osm}} + \Delta F_{\text{sur}},
    \label{eq:Delta_F}
    \\
    \Delta F_{\text{osm}}(r,z) = \int_{V} \Pi(r',z') dV', \nonumber
    \\
    \Delta F_{\text{sur}}(r,z) = \oint_{S} \gamma (r',z') dS'. \nonumber
\end{eqnarray}
The coordinates $(r,z)$ refer to the center of the colloid, whilst the insertion free energy is obtained by integrating over the volume and surface of the colloid, respectively.
Here and below, all the free energy values are normalized by the thermal energy unit $k_{\text{B}}T$.

The osmotic contribution, $\Delta F_{\text{osm}}$, is proportional to the colloid volume and accounts for the work against excess osmotic pressure upon insertion of the colloid into the brush.
The local osmotic pressure (normalized by $k_\text{B} T$) is calculated from the local polymer concentration as
\begin{equation}
    \begin{aligned}
        \Pi(r,z)=  \phi(r,z)\frac{\partial f\{\phi(r,z)\}}{\partial \phi(r,z)} - f\{\phi(r,z)\}= 
        \\
        [-\ln(1-\phi(r,z)) - \phi(r,z) -\chi_{\text{PS}}\phi^2(r,z)],
    \end{aligned}
    \label{eq:osmotic}
\end{equation}
where 
$$
f\{\phi(r,z)\}=(1-\phi(r,z))\ln(1-\phi(r,z)) +\chi_{\text{PS}}\phi(r,z)(1-\phi(r,z))
$$
is the mean-field Flory expression for the interaction free energy per unit volume of the polymer solution of concentration (volume fraction) $\phi(r,z)$.
As long as the osmotic pressure inside the brush is positive, $\Delta F_{\text{osm}}$ is also positive and provides a dominant contribution for sufficiently large colloids.

The interfacial contribution, $\Delta F_{\text{sur}}$, is proportional to the colloid surface area, with the surface tension $\gamma (r,z)$ approximated as
\begin{gather}
     \gamma (r,z)= \frac{1}{6}(\chi_{\text{ads}} - \chi_{\text{ads}}^{\text{crit}})\phi^{\ast}(r,z),
    \label{eq:chi_ads} 
    \\
    \text{with } \chi_{\text{ads}} = \chi_{\text{PC}} - \chi_{\text{PS}}(1-\phi^{\ast}), \text{ and } \phi^{\ast}(r,z)= (b_{0} + b_{1}\chi_{\text{PC}})\phi(r,z).
    \nonumber
\end{gather}
Here $\gamma$ is the change in the free energy of a unit area upon replacement of a contact of the colloid with solvent by a contact with a polymer solution of concentration $\phi(r,z)$.
The coefficients $b_0 = 0.7$ and $b_1 = -0.3$ are phenomenological parameters that account for depletion or accumulation of the polymer in the vicinity of the colloid surface, and can be treated as constant to a good approximation 
\todo{(Supplementary Methods 3-4)}.
%ML: rmeoved Supplementary Methods 4-5 to reference 3, and 5 is already spherical particle


Depending on the relative strengths of polymer-colloid ($\chi_{\text{PC}}$) and polymer-solvent ($\chi_{\text{PS}}$) interactions, the sign of $\gamma$ can be either positive or negative.
If the colloid surface is repulsive ($\chi_{\text{ads}} \geq 0$) or even weakly attractive for polymers ($\chi_{\text{ads}}^{\text{crit}} \leq \chi_{\text{ads}} < 0$), then, due to steric constraints imposed by the impermeable surface, the available conformations of the polymer are restricted, leading to polymer depletion near the colloid surface and $\gamma > 0$.
At the critical adsorption condition $\chi_{\text{ads}} = \chi_{\text{ads}}^{\text{crit}}$, the losses in conformational entropy caused by the presence of the surface are exactly balanced by the free energy gain from monomer-surface contacts, causing $\gamma$ to vanish \cite{Fleer1993,Birshtein1979,Birshtein1983,Eisenriegler1982}.
Ultimately, $\gamma < 0$ for sufficiently strong attraction ($\chi_{\text{ads}} < \chi_{\text{ads}}^{\text{crit}}$).

We applied an approximate analytical scheme to evaluate the insertion free energy $\Delta F(r,z)$ as $\Delta F\{\phi(r,z)\}$, where $\phi(r,z)$ is the polymer density distribution in a colloid-free brush (Figure~\ref{fig:phi_hm_grid}).
The colloid is thus considered as a 'probe' that does not perturb the global polymer concentration distribution $\phi(r,z)$.
With this scheme, we can evaluate the insertion free energy at any position of the colloid in the brush, including off the pore axis 
%\todo{(Supplementary Method 6)}
\todo{(Supplementary Method 5)}.
Comparison with direct SF-SCF calculations of the insertion free energy for colloids on the pore axis demonstrated good quantitative agreement 
%\todo{(Supplementary Method 5)},
\todo{(Supplementary Method 4)},
justifying the use of the more versatile analytical scheme.

Figure~\ref{fig:D_fe_map}a (colored lines) shows the net insertion free energy $\Delta F(\phi)$ of a colloid in a homogeneous polymer mesh with polymer volume fraction $\phi$.
For sufficiently attractive particles, $\chi_{\text{PC}} < \chi_{\text{ads}}^{\text{crit}} + \chi_{\text{PS}}$, the curve is non-monotonic: it crosses zero at a finite concentration where the interfacial gain, $\Delta F_{\text{sur}}(\phi)$, exactly balances the osmotic penalty, $\Delta F_{\text{osm}}(\phi)$.
Above this concentration the osmotic contribution dominates and $\Delta F(\phi)$ becomes positive, and colloid are repelled; below that concentration polymer meshwork attracts colloids.
Inert or only weakly attractive particles ($\chi_{\text{PC}} = -0.5$) always experience a positive insertion free energy and are therefore repelled by the polymer meshwork.
 
Figure~\ref{fig:D_fe_map}c illustrates how colloids may be either repelled or attracted by the polymer meshwork, depending on the balance of osmotic and interfacial contributions to $\Delta F$.
Since the polymer concentration in the pore is strongly inhomogeneous, the net insertion free energy $\Delta F(r,z)$ may exhibit quite large spatial variations.
For example, the brush shown in Figure~\ref{fig:D_fe_map}c at $\chi_{\text{PC}}=-0.75$ simultaneously exhibits attraction ($\Delta F<0$) at the loose fringes and repulsion ($\Delta F>0$) inside the pore where the polymer concentration is high.

\begin{figure}
    \centering
    %\REMOVE \CENTERLINE BEFORE SUBMISSION
    \centerline{\includegraphics[width = 6.5in]{fig/diffusivity_and_fe.png}}
    \caption{%
        Effect of the polymer meshwork on local diffusivity and insertion free energy.  
        \textbf{(a)}  Comparison of the reduced colloid diffusivity caused by the polymer meshwork ($-\ln(D/D_{0})$; black line) with the insertion free energy (colored lines, labelled by $\chi_{\text{PC}}$) as functions of the local polymer volume fraction~$\phi$.  
        Both quantities share the same vertical scale  
       so that the sum of the black and one of the colored curves,  $\ln(\rho D_{0}) = -\ln(D/D_{0}) + \Delta F$, quantifies the total local resistivity; the horizontal zero line corresponds to unimpeded transport as in pure solvent; positive values of the colored curves indicate resistance enhanced by the free energy contribution, whereas negative values indicate lowered resistance.
       \textbf{(b)}  Spatial map of the position-dependent diffusivity, $-\ln\bigl[D(\phi(r,z))/D_{0}\bigr]$, in cylindrical coordinates.  
        \textbf{(c)}  Maps of the insertion free energy $\Delta F(r,z)$ for polymer-colloid interaction strengths ranging from $\chi_{\text{PC}}=-0.50$ (weakest attraction) to $\chi_{\text{PC}}=-1.25$ (strongest attraction), as indicated.
        Pore and brush parameters are as in Figure~\ref{fig:colloid_transport}; $\chi_{\text{PS}}=0.5$ and $d=8$.
    }
    \label{fig:D_fe_map}
\end{figure}



%%%%%%%%%%
\subsubsection{Linking local resistivity to global transport}
%%%%%%%%%%

Having defined the position-dependent insertion free energy and mobility, we can develop an analytical method to estimate the total resistance of the brush-filled pore to the diffusive flow of colloids.

Diffusive transport in the presence of an external force generated by the insertion free energy $\Delta F$ is described by the Smoluchowski equation,
\begin{equation}
    \frac{\partial c(\bm{r})}{\partial t}=-\nabla \cdot \left[D(\bm{r}) \nabla c({\bm{r}})+D(\bm{r})c({\bm{r}})\nabla(\Delta F(\bm{r}))\right],
    \label{eq:Smoluch}
\end{equation}
where $c(\bm{r})$ is the colloid concentration.
Standard substitution introduces the potential function $\psi(\bm{r})$ and the effective conductivity $\tilde{D}(\bm{r})$ as
\begin{equation}
     \psi(\bm{r})=c(\bm{r})\exp(\Delta F(\bm{r}))
     \label{eq:psi} 
\end{equation}
\begin{equation}
    \tilde{D}(\bm{r})=D(\bm{r}) \exp(-\Delta F(\bm{r})) 
    \label{eq:D_tilde}
\end{equation}
With these substitutions, the flux density is expressed as
\begin{equation}
    \bm{j}=- \tilde{D}(\bm{r})  \nabla \psi(\bm{r})
    \label{eq:flux_psi}
\end{equation}
Under stationary conditions, the original Smoluchowski equation ~(\ref{eq:Smoluch}) is reduced to

\begin{equation}
    \nabla \cdot \left(\tilde{D}(\bm{r})\nabla\psi(\bm{r} \right)=0
    \label{eq:Laplace_modif}
\end{equation}
The boundary conditions for the potential function in our case are $\psi(z\rightarrow -\infty)=\Delta c$ and $\psi(z\rightarrow +\infty)=0$ since the insertion free energy $\Delta F$ vanishes far away from the pore.

One approach to find the total flux (resistance) is by direct numerical solution of Eqs.~(\ref{eq:D_tilde}-\ref{eq:Laplace_modif}) which we did as a check \todo{see Supplementary Method 7}.
%RR: Need to link to a Supporting Figure 
However, a more transparent and instructive approach is to follow an electric analog of the problem which can be reformulated as finding the total resistance of a medium with position-dependent resistivity possessing axial symmetry:

\begin{equation}
    \rho(r,z)= \tilde{D}^{-1}(r,z)
    \label{eq:rho}
\end{equation}
The variation of the local resistivity $\rho(r,z)$ is due to the position-dependent polymer volume fraction $\phi(r,z)$.
Figure~\ref{fig:D_fe_map}a displays separately the two underlying contributions:
(i) the reduction of the colloid diffusivity in a homogeneous polymer mesh of concentration $\phi$ and 
(ii) the net insertion free energy.
The sum of these two contributions gives the net effect on resistance,
$\ln\!\bigl[\rho(\phi)D_{0}\bigr]= -\,\ln\!\bigl[D(\phi)/D_{0}\bigr] + \Delta F(\phi),$
as stated in Eq.~(\ref{eq:rho}).
Only for sufficiently attractive colloids, where $\Delta F(\phi)$ is negative enough, does the sum become negative, $\ln\!\bigl[\rho(\phi)D_{0}\bigr] < 0$, indicating that the brush locally reduces the resistance.
Otherwise the net effect of the polymer brush remains positive and the resistance increases.


We consider a set of approximate equipotential surfaces ($\psi(r,z)=\text{const}$) foliating the space available for the colloid flow: inside the pore, the surfaces are taken as  discs of radius $r_{\text{p}}$ normal to the pore axis; outside the pore, we use oblate hemispheroids taken from the Rayleigh solution \cite{Strutt1878} (Figure~\ref{fig:integration_scheme}).
Analogous to a set of resistors connected in parallel, the total conductivity of a layer between two adjacent equipotential surfaces is obtained by integration of local conductivities over the layer.
Within the pore, $|z|\leq L/2$, the result is given by
\begin{equation}
\varrho_{\text{int}}^{-1}(z)= 2\pi\int_{0}^{r_{\text{p}}^{}} \rho^{-1}(r,z) r \, dr
\label{eq:varrho1}
\end{equation}

In the exterior region, $|z| >L/2$, the expression is modified to integrate over the aforementioned hemispheroids:
\begin{equation}
    \begin{gathered}
        \varrho_{\text{ext}}^{-1}(z)= 2\pi\int_{0}^{r_{\text{p}}^{}} \rho^{-1}\left( r'(r,z), z'(r,z) \right)  \tilde{h} (r,z) dr\\
        r'(r,z) = r\sqrt{1 + \frac{(z - L/2)^2}{r_{\text{p}}^2}}\\
        %r'(r,z) \in [0, \sqrt{r_{\text{p}}^2 + (z-L/2)^2}]\\
        z'(r,z) = (|z| - L/2) \frac{\sqrt{r_{\text{p}}^2 - r^2}}{r_{\text{p}}} +  \text{sign}(z) \frac{L}{2}\\
        %z'(r,z) \in [L/2, z]
        \tilde{h} (r,z) = h_r h_{\theta} h_z^{-1} = \dfrac{r}{r_{\text{p}}}\dfrac{r_{\text{p}}^2 + (|z|-L/2)^2}{\sqrt{r_{\text{p}}^2 - r^2}}
    \end{gathered}
\label{eq:varrho2}
\end{equation}
where $r'(r,z) , z'(r,z)$ are the functions parameterizing the equipotential surfaces, and $h_r$, $h_{\theta}$ and $h_z$ are the corresponding Lam\'e coefficients 
%\todo{(see Supplementary Methods 7-8)}.
\todo{(see Supplementary Methods 6)}.
In the case of a homogeneous brush considered here, the function $\varrho_{\text{ext}}^{-1}(z)$ is even.

\begin{figure}
    \centering
    \includegraphics[width=3in]{fig/resistitance_integration_miniature.png}
    \caption{
    Integration scheme for the layer resistance $\varrho_{\text{int}}\text{d}z, \varrho_{\text{ext}}\text{d}z$.  
    The intrinsic orthogonal curvilinear coordinates are defined by equipotential surfaces, \mbox{$\psi = \text{const}$}, and the flux lines of the flux density field $\bm{j}$.
    A layer between two adjacent equipotential surfaces with a thickness $\text{d}z$ along the central axis has resistance $\varrho_{\text{int}}\text{d}z$ inside the pore ($|z|<L/2$, blue shading) and $\varrho_{\text{ext}}\text{d}z$ in the exterior region (red shading).  
    The red rectangle illustrates the local conductivity element at $(r',z')$ within the exterior layer.  
    The parametrization $(r'(r,z),\,z'(r,z))$ traces the integration path and maps the intrinsic coordinates back to the original cylindrical coordinates $(r,z)$, as indicated by the red arrows.
    }
    \label{fig:integration_scheme}
\end{figure}

On the other hand, since the consecutive layers are connected in series, their total resistance can be found by appropriate integration:
\begin{equation}
    R_{\text{int}} = \int_{-L/2}^{+L/2}\varrho_{\text{int}}(z) dz,
    \label{R_int}
\end{equation}

\begin{equation}
   R_{\text{ext}} =2\int_{+L/2}^{+\infty}\varrho_{\text{ext}}(z)dz
    \label{R_ext}
\end{equation}

For a bare pore without a polymer brush, this method recovers Eq.~\ref{eq:resistance}, as expected.
Under poor solvent conditions, where the brush is entirely contained within the interior of the pore (Figure~\ref{fig:phi_hm_grid}), the total resistance is $R = R_{\text{int}} + R_{\text{ext}}^{0}$ as the exterior is not modified by the brush.
Conversely, a brush under good or $\theta$-solvent conditions produces swollen fringes (caps) outside the pore that modify the resistance $R_{\text{ext}}$.


%%%%%%%%%%
\subsection{An attractive polymer filling enhances colloid fluxes through the pore}
%%%%%%%%%%

\begin{figure}
    \centering
    \includegraphics[width = 3in]{fig/resistitance_components.png}
    \caption{
    Interior ($R_{\text{int}}$), exterior ($R_{\text{ext}}$) and total ($R = R_{\text{int}} + R_{\text{ext}}$) resistance  vs polymer-colloid interaction strength $\chi_{\text{PC}}$ for a good solvent ($\chi_{\text{PS}} = 0.3$, in blue) and a poor solvent ($\chi_{\text{PS}} = 0.7$, in orange).
    The exterior $R_{\text{ext}}^{0}$ and total $R_0$ resistances for the bare pore are also shown (red lines of matching type).  
    The resistances are presented in dimensionless units $R\tfrac{k_{B}T}{\eta_{\text{S}}}$; pore and brush parameters are as given in Figure~\ref{fig:colloid_transport}; particle diameter $d = 12$.
    The results of the direct numerical solution of the Smoluchowski equation are shown with markers \todo{RR: I think these data are better shown as a Supporting Figure, that comprises just $R$ computed through the two different methods.}.
    }
    \label{fig:R_vs_chi_PC}
\end{figure}

Figure~\ref{fig:R_vs_chi_PC} visualizes the relative contributions of the pore interior ($R_{\text{int}}$) and exterior ($R_{\text{ext}}$) to the total resistance as a function of the polymer-colloid attraction $\chi_\text{PC}$,
for a selected colloid size ($d = 8$) in a good solvent ($\chi_\text{PS} = 0.3$) and a poor solvent ($\chi_\text{PS} = 0.7$).
A striking feature is that attracted colloids can achieve diffusive fluxes that exceed the limit of the bare pore, as indicated by the segments of the $R(\chi_{\text{PC}})$ curves that fall below the solid horizontal red line marking the bare-pore resistance $R_{0}$.
This result may at first appear surprising, given that the polymer medium is expected to slow down the diffusion of colloids.
However, this slowing down is counteracted by the attractive potential of the polymer meshwork, which reduces local resistivity according to the exponential factor in Eq.~(\ref{eq:rho}).

The reduced local resistivity has pronounced consequences for diffusive transport in both the interior and the exterior of the pore.
Figure~\ref{fig:R_vs_chi_PC} illustrates that the interior resistance $R_{\text{int}}$ can be driven practically to zero by increasing the polymer-colloid attraction (decreasing $\chi_\text{PC}$) below a certain threshold.
Compared to a bare pore (Eq.~(\ref{eq:resistance})) such a short-circuiting effect entails a reduction in the resistance by a factor of up to $R^0_{\text{int}}/R^0_{\text{ext}}+1 \approx 2/\pi \times L / r_{\text{p}} + 1$.
For the pore and colloid considered here, this represents an approximately 3-fold reduction, to a level marked by the dashed horizontal red line in Figure~\ref{fig:R_vs_chi_PC} which equals the contribution of convergent flow to the bare-pore resistance ($R^0_\text{ext}$).
The reduction would be even stronger for longer pores ($L\gg r_p$), and for larger colloids that increase the effective pore length and decrease the effective pore diameter.

The exterior resistance $R_{\text{ext}}$, on the other hand, always retains a finite contribution from the diffusive fluxes in the semi-infinite reservoir, setting an absolute lower bound to the total resistance.
The reduction of $R_{\text{ext}}$ below $R^0_\text{ext}$ evidenced in Figure~\ref{fig:R_vs_chi_PC} is due attractive brush fringes that protrude and facilitate diffusive transport outside the pore.
The magnitude of this effect increases with the extension of the polymer cap, and explains why the minimal attainable resistance increases with decreasing solvent quality (Figure~\ref{fig:phi_hm_grid}).
Approximating the brush fringes on either end of the pore as hemispherical caps with radius $r_\text{cap}$, the plateau conditions are equivalent to twice the resistance of an ideally absorbing hemisphere \cite{Crank1980},
\begin{equation}
    R_\text{ext}^\text{\text{min}} = 1 / (D_0 \pi r_\text{cap}).
    \label{eq:R_ext_min}
\end{equation}

Attractive brush fringes thus entail a reduction in resistance by a factor of up to $R_\text{ext}^0 / R_\text{ext}^\text{\text{min}} = \pi/2 \times r_\text{cap}/r_\text{p}$.
In good solvent ($\chi_\text{PC} = 0.3$), for example, the cap radius (along the pore axis) is comparable to the pore diameter (Figure~\ref{fig:phi_hm_grid}), leading to a 3.3-fold reduction of the external resistance, and a cumulative 10-fold reduction of the total resistance, compared to the bare pore (Figure~\ref{fig:R_vs_chi_PC}).
As the solvent quality decreases the cap size shrinks (Figure~\ref{fig:phi_hm_grid}), with a correspondingly reduced benefit on pore conductivity, as illustrated for $\chi_\text{PC} = 0.7$.
For even poorer solvents, the cap and its benefit disappear entirely ($R_\text{ext} = R_\text{ext}^0$; not shown).


%%%%%%%%%%
\subsection{Polymer-filled mesopores effectively gate colloids by their attraction to the polymer}
%%%%%%%%%%

Figure~\ref{fig:R_vs_chi_PC} also illustrates how the total resistance of the pore varies with the colloid's affinity to the polymer brush.
As expected, increasing the polymer-colloid attraction strength (i.e., more negative $\chi_{\text{PC}}$) results in a monotonic decrease in the pore's total resistance, since the interfacial term in the insertion free energy becomes more negative, thereby increasing the local conductivity $\rho^{-1}$.

Most notable is a sharp transition from a regime of facilitated permeation ($R < R_0$) to a regime of impeded permeation ($R > R_0$).
The regime of impeded permeation is dominated by the internal resistance.
It exhibits high selectivity with respect to the polymer-colloid interaction strength, and a mostly very high total resistance and thus low colloid flux, both appreciable in Figure~\ref{fig:R_vs_chi_PC} as a sharp increase in $R$ over a relatively modest $\chi_{\text{PC}}$ range.
In contrast, the region of facilitated permeation is dominated by the external resistance. 
It exhibits high colloid fluxes but rather low (if any) $\chi_{\text{PC}}$ selectivity, as demonstrated by the previously analyzed plateau.
Thus, the transition between the two regimes of transport defines the condition for sharp colloid gating, with remarkably efficient transport in the regime limited by external resistance and effective blockage in the regime limited by internal resistance.

In this context, the solvent quality can be seen as a regulator of the polymer-colloid interaction level for gating. 
Lowering the solvent quality (increasing $\chi_{\text{PS}}$) reduces $\chi_{\text{ads}}$ and shifts the entire $R(\chi_{\text{PC}})$ curve toward larger $\chi_{\text{PC}}$ values.
Thus, a poorer solvent extends the range of facilitated permeation towards more weakly interacting colloids.

The here-presented trends are qualitatively correct also for colloids with sizes smaller or larger than the $d=8$ considered here.
Naturally, the gating effect will be rather moderate for small colloids, yet even sharper for larger colloids.


%%%%%%%%%%
\subsection{High colloid flux implies colloid enrichment in the pore}
%%%%%%%%%%

Colloid concentration profiles under stationary flux conditions can be found by numerically solving Eq.~(\ref{eq:Smoluch}) with $\frac{\partial c(r,z)}{\partial t} = 0$
%\todo{(Supplementary Method 9)}.
\todo{(Supplementary Method 7)}.

Figure~\ref{fig:colloid_concentration} maps the steady-state colloid concentration across a polymer-filled pore with colloid size $d = 8$ and polymer-colloid interaction strength $\chi_{\text{PC}} = -1.25$ in an ideal solvent ($\chi_{\text{PS}} = 0.5$).
This condition corresponds to transport rates somewhat inferior to the bare pore ($R \approx 5 R_0$).
\todo{(compared to prohibitive to inert colloids $R(\chi_{\text{PC}}=0) \approx 13500 R_0$)}.

\begin{figure}
    \centering
    %\REMOVE \CENTERLINE BEFORE SUBMISSION
    \centerline{\includegraphics[width=5.5in]{fig/streamlines.png}}
    \caption{
    Color map of the steady-state colloid concentration (calculated by direct numerical solution), normalized by the bulk concentration in the source compartment $c_0$.
    Isoconcentration contours are shown as labeled.
    The flux is represented by streamlines marked with small arrows, indicating the average colloid trajectory.
    Pore and brush parameters are the same as in Figure~\ref{fig:colloid_transport}; $d = 8$, $\chi_{\text{PC}} = -1.25$ and $\chi_{\text{PS}} = 0.5$.
    The physical pore radius ($r_\text{p}^0$) and length ($L_0$), along with their effective counterparts due to colloid excluded volume ($r_\text{p}$ and $L$, respectively), are shown with white arrows.
    The inset (top right) shows the solution in the form of potential function $\psi$.
    }
    \label{fig:colloid_concentration}
\end{figure}

The map illustrates several salient features of the diffusion process.
Outside the region of the pore and polymer fringes, the colloid concentration profile is as expected for plain solution: the concentration rapidly approaches the respective bulk concentrations of the semi-infinite reservoirs, $c(z = -\infty) = c_0$ and $c(z = +\infty) = 0$, and the equiconcentration surfaces near the pore entrance ($0.995 \le c_0 < 1.0$) and exit ($0.0 < c_0 \le 0.005$) form a symmetric set of oblate hemispheroids
\todo{[ML: that coincide with the solution in the form potential function $\psi$ in the inset Figure~\ref{fig:colloid_concentration}]}.
Inside the pore, the flux lines run almost parallel to the pore axis.

The most notable observation is that the colloid concentrations substantively exceed $c_0$ near the pore entrance (by a factor of $\sim20$) and inside the pore (by a factor  of $\sim10$).
This effect is caused by the negative insertion free energy in the space occupied by the polymer brush.
At equilibrium (i.e., with vanishing fluxes), the partitioning would amount to $c_{\text{eq}}/c_0 = \exp\left( -\Delta F \right)$.
In the steady state (i.e., with non-vanishing fluxes), the colloid concentration is reduced but approaches the equilibrium concentration as the insertion free energy becomes largely negative ($c/c_0 \to c_{\text{eq}}/c_0$).

The presented quantitative results are only valid for sufficiently low bulk concentrations $c_0$, as our model disregards any colloid crowding effects.
When this crowding is accounted for, the steady-state colloid concentrations in the brush will be systematically lower.


%%%%%%%%%%
\subsection{Polymer-filled mesopores effectively gate colloids by their size}
%%%%%%%%%%

\begin{figure}
    \centering
    \includegraphics[width = 3.5in]{fig/permeability_on_d.png}
    \caption{
    Normalized dimensionless pore resistance $R\tfrac{k_{B}T}{\eta_{\text{S}}} $ as a function of the colloid size $d$ for \textbf{(a)} selected values of the polymer-colloid interaction strength $\chi_{\text{PC}}$ (as indicated in the legend) at a fixed solvent strength $\chi_{\text{PS}} =0.5$, and \textbf{(b)} selected values of $\chi_{\text{PS}}$ (as indicated in the legend) at a fixed $\chi_{\text{PC}} = -1.25$.
    Pore and brush parameters are as given in Figure~\ref{fig:colloid_transport}. 
    The bare-pore resistance $R_{0}$ is shown by the black thick line. Its deviation from simple Stokesian scaling (Eq.~(\ref{eq:resistance}); thin black line) is due to the excluded volume of the colloid.
    }
    \label{fig:R_vs_d}
\end{figure}

Figure~\ref{fig:R_vs_d} compares how the total resistance, $R=R_{\text{int}}+R_{\text{ext}}$, varies with colloid size $d$ for a bare pore (thick black lines) and for polymer-filled pores with selected solvent (Figure~\ref{fig:R_vs_d}a) and polymer-colloid interaction (Figure~\ref{fig:R_vs_d}b) strengths (thin colored lines with symbols).
For small colloids, the bare-pore resistance follows well the $R_0 \sim D_0^{-1} \sim d$ dependence expected according to Eq.~(\ref{eq:resistance}).
Stronger dependence of $R_0$ on $d$ observed for larger colloids is due to decreasing effective pore length $L$ and increasing the effective pore radius $r_{\text{p}}$.

Naturally, the polymer filling affects the transport of the smallest colloids only marginally, as their volume and net interaction strengths (within the considered $\chi_{\text{PC}}$ range) are too small to have any noticeable effect. 
A rich picture emerges for larger colloids, however, with non-monotonic dependencies of the pore resistance on colloid size and strong effects of $\chi_{\text{PC}}$ and $\chi_{\text{PS}}$. 


%%%%%%%%%%
\subsubsection{Impact of colloid diffusivity within the polymer brush on size-selective transport}
%%%%%%%%%%

The curve with $\chi_{\text{PS}}=0.5$ and $\chi_{\text{PC}} = -1.0$ in Figure~\ref{fig:R_vs_d}a corresponds to the condition of $\Delta F$ fairly vanishing across a wide colloid-size range (as demonstrated in the corresponding panel in \todo{Figure~S4}) due to compensation of osmotic and surface contributions to the insertion free energy.
Here, the reduced colloid diffusivity within the polymer meshwork dominates the pore resistance (Eq.~(\ref{eq:rho})). 
This effect alone leads to a monotonic and pronounced increase of $R$ with $d$ with smooth crossover between asymptotic dependencies $R\sim d$ at $d\ll \xi$ and $R\sim d^3$ at $d\gg \xi$. 
Interestingly, for small and intermediate size colloids the pore resistance slightly grows with inferior solvent quality (increase in $\chi_{PS}$) due to a decrease in the mesh size $\xi$ with concomitant decrease in the local diffusivity, as seen in Figure~\ref{fig:R_vs_d}b.


%%%%%%%%%%
\subsubsection{Impact of insertion free energy on size-selective transport}
%%%%%%%%%%

Since the pore resistance scales exponentially with the insertion free energy ($R \sim D^{-1}\exp (\Delta F)$; Eq.~(\ref{eq:rho})), and $\Delta F =\Delta F_{\text{osm}} + \Delta F_{\text{sur}}$, the dependence of the resistance on colloid size is generally controlled by the interplay between
the osmotic $\Delta F_{\text{osm}} \sim \Pi d^3$ and the interfacial $\Delta F_{\text{sur}} \sim \gamma d^2$ contributions. 
While the osmotic repulsion arising due to the polymer filling always enhances the resistance, the surface contribution may either increase (at $\gamma > 0$) or decrease (at $\gamma<0$) it.

For inert or weakly attractive colloids, $\gamma \geq 0$, the resistance grows monotonically with the colloid size due to the combined effect of a decreasing diffusivity $D(d)$ and an increasing insertion free energy  $\Delta F(d)$.
As both these effects are pronounced, their combination leads to a very strong size selectivity, such that the transport of even rather small colloids is effectively impeded, as can be appreciated for $\chi_{\text{PC}} > -1.0$ in Figure~\ref{fig:R_vs_d}a.
For sufficiently large colloids, the osmotic contribution dominates in the insertion free energy such that $R \sim D^{-1} \exp (\Delta F) \sim d^3 \exp (\Pi d^3)$.

In contrast, for attractive colloids with $\gamma <0$ the dependence of the pore resistance $R(d)$ on colloid size can be non-monotonic, with a local maximum (at $d=d_{\text{max}}$) followed by a local minimum at ($d=d_{\text{min}}$).
This is best illustrated in Figure~\ref{fig:R_vs_d} by the orange curves corresponding to $\chi_{\text{PC}} = -1.25$ and $\chi_{\text{PS}}=0.5$.
The local maximum here arises from the net colloid attraction (which decreases resistance) overcoming the decrease in diffusivity (which increases resistance) with increasing colloid size.
The local minimum in turn arises from the positive osmotic contribution to the free energy ($\Delta F_{\text{osm}} \sim \Pi d^3$) overcoming the negative interfacial contribution ($\Delta F_{\text{sur}} \sim \gamma d^2$).

Most notably, the local minimum for attractive colloids is swiftly followed by a sharp increase in resistance (for $d > d_{\text{min}}$) due to the dominant osmotic contribution recovering the strong $R \sim d^3 \exp(\Pi d^3)$ dependence.
The transition between the regime of good to moderate transport (for $d \lesssim d_{\text{min}}$) and the regime of impeded transport (for $d > d_{\text{min}}$) thus defines the condition for sharp gating of attracted colloids by their size.

When the insertion free energy becomes strongly negative, a new regime appears that is characterized by facilitated transport ($R < R_0$) over a rather wide range of colloid sizes, as illustrated in 
Figure~\ref{fig:R_vs_d}a for $\chi_{\text{PC}} \le -1.4$, and in Figure~\ref{fig:R_vs_d}b for $\chi_{\text{PS}} \ge 0.6$.
Here, the pore interior is effectively short-circuited, $R_{\text{int}} \to 0$, and the total resistance is set by the finite exterior contribution, $R \approx R_{\text{ext}}$ (see Eq.~(\ref{eq:R_tot_tot})).
Due to attractive brush fringes at the pore entrance and exit, $R \approx R_{\text{ext}}^{\text{min}}$, leading to a weak size dependence, $R \sim d$ (see Eq.~(\ref{eq:R_ext_min})).

The sharp gating of colloids by their size is preserved, and even enhanced, for strongly attractive colloids.
This is best seen in Figure~\ref{fig:R_vs_d}a for $\chi_{PC}= -1.4$, where the osmotic penalty to the insertion free energy takes over, and entails a sharp increase in resistance, above a certain colloid size ($d \approxeq 24$).


%%%%%%%%%%
\subsection{Experiments of colloid transport through nuclear pore complexes validate the theoretical predictions}
%%%%%%%%%%

To test how well our theory predicts the experimental reality, we analysed literature data pertinent to colloid transport across nuclear pore complexes (NPCs). The estimated distance between NPCs in the nuclear envelope is approximately 10 times larger than the pore diameter \cite{Yang2004, Daigle2001, Feldherr1984, Kubitscheck2000}. At such distances, transport across neighbouring NPCs is not mutually interfering \cite{Fabrikant1985}, as can be appreciated from the iso-concentration lines in Figure~\ref{fig:colloid_concentration}. Experimentally measured transport rates $k$, normalized against the number of pores, therefore can be directly compared with our theoretical predictions.
\todo{[Ralf to check and add references. Do we need to clarify that this applies for yeast as well as cells from higher organisms?]} 

It is well-known that colloids with affinity for the disordered nucleoporin FG domains that fill the NPC (such as importins and exportins) are enriched in or near NPCs \cite{Beck2007, Gruenwald2010, Tu2011}, and in microscopic droplets, macroscopic hydrogels and thin films assembled from pure FG domains.
Moreover, high concentrations of transport factors are essential for effective transport through the pore \cite{Lowe2015}.
Qualitatively, these observations fully align with our predictions that the accumulation of colloids in the pore is required for facilitated transport (Figure~\ref{fig:colloid_concentration}).
We hence tested our predictions quantitatively.
\todo{[Ralf to check and add references.]}


%%%%%%%%%%
\subsubsection{Transport of non-sticky colloids}
\label{sec:transport_of_non-sticky_colloids}
%%%%%%%%%%

Several studies have quantified the rates of diffusive transport across NPCs for non-sticky proteins, collectively covering two orders of magnitude in molecular mass and five orders of magnitude in transport rate \cite{Ribbeck2001, Mohr2009, Popken2015, Timney2016, Frey2018}.
Figure~\ref{fig:NPC_comparison}a compares these experimental results with the theoretical predictions of our model.
As the example pore geometry and polymer density in Figure~\ref{fig:colloid_transport} were modelled to represent a NPC \todo{(Supporting Method 1)}, we can directly compare our theoretical predictions with experimental data.
The effective statistical segment length of disordered polypeptide chains was taken to be $a$ = 0.76 nm \todo{[Ralf and/or Mikhail to add reference.]}.
The effective solvent strength in the NPC was estimated to be close to $\theta$-solvent ($\chi_{\text{PS}} = 0.6$), consistent with varying yet generally moderate levels of 'cohesiveness' observed for FG domains. \todo{[Ralf and/or Mikhail to add references.]}
We approximated the proteins as perfectly inert colloids ($\chi_{\text{PC}} = 0$).
To match the theoretical colloid volumes to protein molecular masses, we considered the effective density of the protein colloids to be bounded by the densities of aqueous solvent ($\rho_{\text{probe}} \geq \text{1 g/cm}^3$) and pure polypeptide ($\rho_{\text{probe}} \lesssim \text{1.4 g/cm}^3$).
This approximation reflects that an unknown (and possibly variable) amount of solvent contributes to the effective volume of the proteins during their transport across the NPC.
The only adjustable fitting parameter in our model was the prefactor $\beta$ in the scaling-based expression for the diffusion coefficient, Eq.~(\ref{eq:Rubinstein}).

Figure~\ref{fig:NPC_comparison}a demonstrates that the theory reproduces the experimental trends for the increase in pore resistance with colloid size very well.
The best fit was obtained with $\beta = 5.5$ and this value was hence fixed  throughout the paper.
The quality of the fit is quite remarkable given the large range of masses and transport rates covered, and also considering the relative simplicity of our theory.
Some scatter in the experimental data is though notable.
This may be due to some proteins not being strictly non-sticky but interacting weakly with FG domains.
Indeed, Frey et al. \cite{Frey2018} reported a three-fold enhanced transport rate of green fluorescent protein over mCherry despite both these proteins being considered inert and of similar molecular mass.
Moreover, whilst some studies had washed out cytosolic proteins in their assay (with HeLa cells), thus leaving behind intact nuclear pores filled with a plain FG domain brush but lacking most transport factors \cite{Ribbeck2001, Mohr2009, Frey2018}, others used intact yeast cells with all transport factors present \cite{Popken2015, Timney2016}.
The satisfactory fit across all datasets suggests that the crowding of the NPC with transport factors has at best a weak effect on the transport of non-sticky proteins across the NPC \todo{[There may be experimental work that has looked at this -- Ralf ought to reference it.]}.  

\begin{figure}
    \centering
    %REMOVE \CENTERLINE BEFORE SUBMISSION
    \centerline{\includegraphics[width = 6in]{fig/validation.png}}
    \caption{
    Comparison of theoretical predictions with experimental findings for colloid transport rates across NPCs.
    \textbf{(a)} 
    Gating of non-sticky colloids (peptides and globular proteins) by size.
    NPC passage rate $k$ per pore (at $c_0$ = 1 $\mu\text{M}$) vs. molecular mass $M_w$ (symbols) extracted from the literature, as indicated \todo{[ML:Update reference numbers before submission]} (see \todo{Table~S2}).
    Theoretical predictions (shaded orange area) are for the pore and brush parameters as given in Figure~\ref{fig:colloid_transport}, with $\chi_{\text{PC}} = 0$, $a = 0.76$ nm, $\chi_{\text{PS}} = 0.6$, and colloid masses converted to volumes using $\text{1 g/cm}^3 \leq \rho_{\text{probe}} \leq \text{1.4 g/cm}^3$.
    The best fit, shown here, was obtained with $\beta = 5.5$ in Eq. (\ref{eq:Rubinstein}).
    The predicted transport rates across a bare pore are shown (bold black line) for comparison.
    \textbf{(b)} 
    Gating of colloids by their affinity to the polymer.
    NPC passage rate $k$ vs. partition coefficient $P$ in phase-separated droplets of pure FG domains (Nup98A - blue lozenges, Nup116 - gray lozenges) measured by Frey et al. \cite{Frey2018} for a range of green fluorescent protein variants (see \todo{Table~S3}).
    Theoretical predictions (dark orange lines) represent the limits of a homogeneously attractive colloid surface (dashed line) and a single sticky patch (solid line).
    The predicted transport rate across a bare pore is also shown (bold black line).
    The pore and brush parameters are as given in (a); $\chi_\text{PC}$ values (indicated at the top of the graph) were matched to the partition coefficient $P$ \todo{[May link to a Supporting Figure or Table that explains how this was done?]}.
    }
    \label{fig:NPC_comparison}
\end{figure}


%%%%%%%%%%
\subsubsection{Transport of sticky colloids}
\label{sec:transport_of_sticky_colloids}
%%%%%%%%%%

Frey et al. \cite{Frey2018} additionally quantified NPC transport rates for a wide range of green fluorescent proteins (GFPs) with surface amino acids mutated to modulate transport from 'superinert' to 'transport factor like'.
In parallel, the ability of these variants to enrich or deplete in phase-separated droplets of two pure FG domains (Nup98A from \textit{T. thermophila}, and Nup116 from \textit{S. cerevisiae}) was quantified.
The transport rate was observed to correlate strongly with the level of GFP enrichment in FG domain phases (Figure~\ref{fig:NPC_comparison}b).
This set of experiments enabled the effect of polymer-colloid interaction to be tested selectively as the colloid size and shape were effectively constant.

To reproduce theoretically the correlation between the experimentally measured NPC transport rates $k$ and the partition coefficient $P$ in pure FG domain phases, we determined the effective solvent strength for each FG domain $\chi_\text{PC}$ from the experimentally measured volume fractions of the FG domain in its polymer-rich droplets formed upon spontaneous phase separation \cite{Frey2018} 
\todo{(see Supplementary Methods 8)}.
\todo{RR: Please check if correct, or justify why vanishing osmotic
pressure is a sensible assumption}
\todo{ML: SI The estimated critical concentration for Nup116 and Nup98A FG-domains is $1 \, \mu\text{g}/\text{ml}$, corresponding to \mbox{$\phi_{\text{gel}}^{\text{out}} \approx 10^{-6}$} \cite{Schmidt2015}.
This extremely low value implies the osmotic pressure is effectively zero in both phases. SI}

Furthermore, the free energy of insertion $\Delta F = -\ln(P)$ is reduced to the surface contribution ($\Delta F = \Delta F_\text{sur}\left(\chi_\text{PS},\chi_\text{PC}\right)$) since the osmotic pressure vanishes on spontaneous phase separation \todo{{Please check if correct, or justify why vanishing osmotic pressure is a sensible assumption}}, and provides the link between $\chi_\text{PC}$ and the partition coefficient. 
\todo{[What colloid size was assumed here? Can we use the density bounds as done for the non-sticky colloids, for consistency?]}

Our idealised assumption of colloids being homogeneously interactive (dashed orange line in Figure~\ref{fig:NPC_comparison}b) reproduced the experimental data for non-sticky and weakly attractive colloids well without any adjustable parameter.
For more strongly attractive colloids however, this approach overestimated the experimentally observed transport rates.
Assuming the opposite extreme of all surface free energy being concentrated into a single sticky patch (solid orange line in Figure~\ref{fig:NPC_comparison}b) reproduced the experimental data quite well, suggesting that the presence of localized sticky patches on the colloid surface and the FG domains slows down diffusion and transport.
\todo{[Ralf to check and add references.]}

Taken together, the quantitative agreement between our theory and a range of experimental data for NPC passage of proteins with a very limited number of adjustable parameters provides strong validation for the validity of our theory.


%%%%%%%%%%
\section{DISCUSSION}
%%%%%%%%%%

We have shown how mesopores filled with polymer brushes can gate transport with exquisite selectivity with respect to polymer-colloid affinity and colloid size, even for colloids that are substantially smaller than the pore diameter.
A striking finding is that an attractive polymer brush can provide colloid transport rates comparable to, or even exceeding, the bare pore.

Our findings shed light on the basic mechanisms of selective nucleo-cytoplasmic transport and suggest a molecular design strategy for controlling selective permeability through artificial mesoporous membranes.


%%%%%%%%%%
\subsection{Implications for nuclear pore permselectivty}
%%%%%%%%%%

%Comment RR: This paragraph seems secondary, and I have thus removed it given our space constratints.
%Single-cargo tracking studies using fluorescence \cite{Musser2016, Lowe2010, Lowe2015, Yang2004, Kubitscheck2000, Ma2010} and tomography \cite{Beck2007} have shown that transported colloids (e.g., importins, exportins and their complexes with cargo) primarily traverse the central region of the NPC and are rarely observed near the pore walls.
%Such a behaviour is consistent with the insertion free energy landscape in our model (Figure~\ref{fig:D_fe_map}c, $\chi_\text{PC} = -1.0$).
%Importantly, such a path does not require the presence of a bare (i.e., polymer free) channel as had been suggested in some earlier models of NPC transport. Instead, subtle variations in polymer density across the pore's cross-section substantially determine where colloids enrich and translocate.  
%\todo{[Ralf to add further references]}

Figure~\ref{fig:NPC_comparison}b indicates that the theoretical limit for the rate of transport of sticky colloids is higher than what may be realised with proteins in nuclear pores.
This finding is intriguing, as it suggests the rate of transport is not the primary performance factor for NPCs.
Arguably, selectivity of transport may be the more important criterion, and the limited biochemical space available for nature to evolve towards high selectivity (whilst maintaining basic properties such as colloidal stability in the cellular milieu) may have come with a tradeoff in terms of rate.
Our model predicts that the highest transport rates are achieved with homogeneously attractive polymers and colloids.
In contrast, each FG domain polymer type exhibits substantive heterogeneity along the chain contour with preferred interaction sites for transport factors.
Similarly, importins, exportins and their cargo display substantive surface heterogeneity and complex, non-spherical shapes.
Whilst NPC transport factors typically feature multiple 'low-affinity' binding sites for FG domains, these sites remain discrete.
Per-site interaction strengths in the lower mM range \cite{Hough2015} \todo{[Add reference to Milles et al.]}, equivalent to unbinding free energies of $\sim 5 k_\text{B}T$, can reduce the diffusivity by an order of magnitude compared to homogeneously attractive colloids  (Eq.~(\ref{eq:Sticky diff})).
The discreteness of interactions, therefore, is a very plausible candidate for the reduced transport rates in NPCs. 

NPCs feature a variety of nucleoporin FG domains, with the body of available structural and biochemical data suggesting that the cohesiveness of nucleoporin FG domains is highest in the centre and decreases towards the periphery of the pore.
Qualitatively, one can envisage that the increased solubility of peripheral FG domains promotes a more extended polymer cap, thus minimising total pore resistance and maximising transport rates for strongly attractive colloids.
The reduced solubility of the central FG domains, on the other hand, would minimize the size threshold for gating of non-adhesive colloids.
These features clearly are not to be essential for permselective transport through polymer-filled mesopores, but may further enhance selectivity or rates. Our model may be further extended to incorporate solubility gradients and to explore such phenomena in more detail.
\todo{[Ralf to add references]}

%%%%%%%%%%
\subsection{Towards technological applications of synthetic polymer-filled mesopores}
%%%%%%%%%%

Our predictive theoretical approach paves the way for the rational design of nanoporous materials with enhanced selectivity tailored to specific functional requirements, sought after for applications in nanomedicine, biotechnology, and environmental engineering.
Mixtures of biological colloids such as folded proteins and other biomacromolecular complexes, as well as synthetic colloids such as nanoparticles, may be effectively separated, not only according to their size but also their surface (bio-)chemistry.

Individual pores, as we have considered here, are routinely deployed in current nanopore sensing technologies.
These technologies enable detection and characterization of individual macromolecules as they travel across the pore.
Our findings suggest polymer fillings as an attractive tool to optimize the performance of nanopore sensing.
Placing a suitable polymer filling upstream the pore's sensing region would enable pre-selection of target solutes from complex mixtures for a focused analysis by the pore.
Polymer fillings may also be placed in the very sensing region of the pore to enhance both selectivity and sensitivity.
The here-proposed approach is distinct from previous approaches, where responsive polymer coatings along the pore walls were used to open/close a polymer free channel on application of an external stimulus such as a change in temperature, ionic strength or pH. 

Individual pores will though typically be insufficient in applications that focus on separation with high throughput such as filtration systems.
This limitation can be overcome by multiplexing, e.g., with membranes featuring a large array of mesopores.
Our theoretical approach remains valid for such arrays as long as the distance between pores remains sufficiently large for the diffusion trajectories of adjacent pores not to substantially interfere.
Fortunately, this condition can be met with a relatively tight packing of pores, as can be appreciated from the iso-concentration lines in Figure~\ref{fig:colloid_concentration} \cite{Fabrikant1985}.

\bigskip

\noindent{The main design concepts emerging from our theory are:}

\textbf{1.}
For a polymer-filled pore to provide selective transport, high permeation selectivity must be coupled with low resistance to diffusive flux.
We refer to this combination as 'gating' behaviour, where a minor change in colloid size (Figure \ref{fig:R_vs_d}) or polymer-colloid interaction strength (Figure \ref{fig:R_vs_chi_PC}) can dramatically shift the permeation rate from facilitated transport to virtually complete blockage.
The thresholds for gating can be tuned by the polymer-solvent interaction strength.
The gating effect is particularly pronounced for larger colloids.

\textbf{2.}
The maximal permeability is limited by the resistance of the exterior region.
Whilst strong  polymer-colloid attraction can make the resistance of the pore interior effectively vanish ($R_{\text{int}} \to 0$), mass transport in plain solvent always provides a non-vanishing resistance of the exterior.
Polymer fringes of radius $r_\text{ext}$ at the pore entrance and exit decrease the path through plain solvent, and can reduce the external resistance by a factor of up to $\frac{\pi r_{\text{ext}}}{2 r_{\text{p}}}$.

\textbf{3.}
Pore resistance is highly sensitive to parameters that influence the insertion free energy.
The osmotic contribution to the insertion free energy scales as $d^3$ while the interfacial contribution comprises $\chi_{\text{PC}}$ and scales as $d^2$.
Thus, a slight change in $d$ and/or $\chi_{\text{PC}}$ translates into a drastic change in permeability (resistance).

\textbf{4.}
A homogeneous polymer-colloid interaction is preferable over one or multiple distinct binding sites to maximise the diffusivity of sticky colloids in the polymer phase, and thus the overall transport rate.

\bigskip

\noindent{The manufacturing of functional mesoporous membranes is an emerging art, and we hope that our theoretical efforts will both promote and guide future practical developments in this area.}
\todo{[We ought to provide some references on the manufacturing of mesoporous membranes.]}
\todo{[One can expect that transport rates will increase further with a pressure gradient that drives solution flow across the membrane. We could mention this here as an avenue worthy exploring in future work?]}


%%%%
\printbibliography
\end{document}