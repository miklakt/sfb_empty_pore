\documentclass[12pt, a4paper]{article}
\usepackage{graphicx}
\usepackage{amsmath, amssymb, amsfonts, mathtools}
\usepackage{subcaption}
\usepackage[
backend=biber,
natbib=true,
style=numeric,
sorting=none,
doi=false,
isbn=false,
url=false,
eprint=false
]{biblatex}
\usepackage{xcolor}
\usepackage{bm}

%\REMOVE BEFORE SUBMISSION
\usepackage{lineno}
\linenumbers

%\REMOVE BEFORE SUBMISSION
\newcommand\todo[1]{\textcolor{red}{#1}}

\addbibresource{biblio.bib}
\title{A polymer filling enhances the rate and selectivity of colloid permeation across mesopores}

\author{Mikhail Y. Laktionov$^1$, Leonid I. Klushin$^{2,3}$,\\
Ralf P. Richter$^4$, Frans A. M. Leermakers$^5$, Oleg V. Borisov$^1$\\
$^{1}$CNRS, Universit\'e de Pau et des Pays de l'Adour UMR 5254,\\
Institut des Sciences Analytiques et de Physico-Chimie\\
pour l'Environnement et les Mat\'eriaux, 64053 Pau, France\\
$^{2}$Branch of Petersburg Nuclear Physics Institute\\
named by B.P.Konstantinov\\
of National Research Centre "Kurchatov Institute",\\
Institute of Macromolecular Compounds,\\
199004 St.Petersburg, Russia,\\
$^{3}$American University of Beirut, Department of Physics, Lebanon\\
$^{4}$University of Leeds, School of Biomedical Sciences,\\
Faculty of Biological Sciences, 
School of Physics and Astronomy,\\
Faculty of Engineering and Physical Sciences,\\  
Astbury Centre for Structural Molecular Biology,\\ 
and Bragg Center for Materials Research,\\ 
Leeds, LS2 9JT, United Kingdom\\
$^{5}$University of Wageningen, The Netherlands\\
}
\date{}

\begin{document}
\maketitle

\begin{abstract}

Mesoporous membranes are emerging as new materials with potential applications in sensing and separation devices.
The nuclear envelope of eukaryotic cells provides a striking example of a functional mesoporous membrane, where diffusive transport is mediated by nuclear pore complexes (NPCs).
Transport across NPCs is mediated by a pore-filling meshwork of naturally-disordered proteins (FG-domains of nucleoporins) anchored to the pore walls, and is highly selective.
Even colloids much smaller than the inner diameter of the NPC are effectively blocked from transport, but some larger colloids with distinct surface features can readily permeate.
Simplistically, one may expect any polymer meshwork to slow down colloid movement as the steric constraints imposed by the polymer meshwork hinder permeation.
However, we demonstrate how a rationally designed polymer filling can instead increase the permeation rate, by an order of magnitude and more, compared to a bare pore.
Such enhanced permeation is achieved with a polymer phase that attracts the colloid and extends beyond the confines of the mesopore channel itself, thus maximizing colloid capture for diffusive transport across the pore.
We further define how polymer-filled mesopores can be designed to effectively gate colloids according to their size and surface features. 
This combination of features renders mesopores promising as highly selective separation devices.
It also provides a physical explanation for the basic mechanism of nuclear pore permselectivity.
\end{abstract}

%Comment by RR on the storyline: Can we contrast our results with other separation devices, and spell out what our mesopores enable that was not possible before? This would help spelling out the significance of our findings.

%%%%%%%%%%
\section{INTRODUCTION}
%%%%%%%%%%

Polymer-modified mesoporous materials and membranes belong to a new class of functional nanostructured materials with great potential in a number of technologies. 

The interaction of (macro)molecules and nanocolloidal particles with porous media, as well as their transport through porous membranes, are important elements of many technological processes (chromatography, heterogeneous catalysis, micro- and ultrafiltration, protein separation and purification, etc.) and, therefore, have been the subject of intensive research for more than sixty years \cite{Watson1959, Rout2003, Huang2023, Uredat2024}.

Advances in macromolecular chemistry have made it possible to significantly improve functional properties of materials with mesopores (i.e., pores with a diameter between a few and many tens of nanometers) by modifying them with polymers of various chemical nature anchored to  the pore walls.
Thus, a fuzzy meshwork of solvated polymers is formed filling the entire pore volume or just the near-wall regions, depending on the molecular mass and conformational state of the polymer chains.
The interaction of this polymer meshwork with colloids, that is, nanoparticles or (macro)molecules in the solution phase, essentially determines the absorption and separation properties of the polymer-modified mesoporous materials and membranes.
These interactions can be attractive or repulsive, and controlled by a broad spectrum of external stimuli \cite{Jeong2002, Lee2010, Low2019}, 
such as temperature                     \cite{Stetsyshyn2020}, 
pH and/or ionic strength of the medium  \cite{Dai2008, Zhang2005}, 
ion valency and specificity             \cite{Zhulina1999, Robertson2021},
electric fields                         \cite{Lokuge2005}, 
solvent composition                     \cite{Halperin2011, Darabi2016}, 
or complex biological stimuli           \cite{Ikeda2010, Lu2003}.
This opens up a unique opportunity for highly selective and controlled uptake and transport of colloids through polymer-filled mesoscopic channels.

Past experimental and theoretical efforts have focused on systems where an external stimulus triggered the transient opening of a polymer free path, typically along the center of the pore, to gate colloid transport.
However, the polymer phase itself can potentially also provide high selectivity to colloids as a function of their size and attraction by the polymer.
We thus hypothesized that even mesopores that are filled by a polymer meshwork across their entire cross-section can effectively gate colloid transport.
If successful, this approach would enable more robust gating as it does not rely on careful tuning of the diameter of a polymer-free channel, and higher transport rates as the full pore cross-section can potentially participate in colloid transport.

Nature provides a case in point.
Nuclear pore complexes (NPCs) perforate the nuclear envelope of eukaryotic cells and control the bulk transport of proteins and nucleic acids between the nucleus and the cytoplasm.
This process enables the spatial separation of gene transcription (in the nucleus) and translation into proteins (in the cytosol), and thus is critical for the ordered course of gene expression.
Each NPC forms a cylindrical channel, measuring approximately 40-60 nm in diameter and 40-95 nm in length (depending on the species \cite{Yang1998, Beck2004, VonAppen2015, Alberts2015, Hayama2017, Holzer2018}).
The channel is filled with a meshwork of several 100 natively disordered protein domains rich in phenylalainine-glycine dipeptides (FG domains) that are anchored to the channel walls \cite{Holzer2018, Ori2013, Rout2000, Dickmanns2015}.
Collectively, the FG domain meshwork provides remarkable gating function: biocolloids of 5 nm (i.e., just a tenth of the pore diameter) or more in hydrodynamic diameter are effectively blocked, except for some dedicated transport factors (importins and exportins, alone and in complex with cargo) which bind to the FG domains and can undergo rapid permeation. 

%RR: The below section is rather long, and may be shortened? Also need to review reference list.

Several independent strands of evidence indicate that the mechanism of diffusive transport across NPCs is based on rather generic physical principles, whereas the exact chemical makeup of the polymers and colloids is secondary for function.
Firstly, despite significant variations in the molecular building blocks of the NPC and the transport factors across distant eukaryotic taxa, NPCs consistently fulfill the same functional role \cite{DeGrasse2009, Maimon2012, Ori2013, Hayama2017, Yaron2018, Holzer2018}.
Secondly, the NPC can gate diffusive colloid transport similarly well in both directions.
Whilst the native cell is capable of directed transport of cargo against a concentration gradient \cite{Rout2003, Tijana2017}, this function is not an intrinsic part of the NPC itself but relies on energy derived from GTP hydrolysis by soluble intracellular proteins \cite{Lowe2015, Yang2004} and can even be reversed through cell engineering without modifying the NPC structure \cite{Nachury1999, Sakiyama2016}.
Thirdly, the binding behaviour of transport factors to assemblies of purified FG domains could be reproduced by simple physical models that treat FG domains as regular flexible polymers and transport factors as spherical colloids with a homogeneous surface. This approach provided faithful predictions even though it ignored the detailed arrangement of interaction sites along FG domains and on the transport factor surface. 
Fourthly, recent work with a range of mutants of green fluorescent protein (GFP) demonstrated that NPCs exhibit a wide and continuous spectrum of permeabilities as a function of colloid surface properties, and earlier studies with non-interacting colloids similarly evidenced a wide and continuous spectrum of permeabilities as a function of colloid size.
Additionally, certain native proteins with affinity to FG-domains, such as $\beta$-catenin, can translocate through the pore without the need for a transport factor, moving from the cytoplasm to the nucleoplasm along a concentration gradient due to their continuous binding to chromatin \cite{Rout2003}. This suggests that a fine balance of many individually weak physicochemical (e.g., electrostatic, hydrophobic, aromatic stacking, ...) interactions between polymers and biocolloids dictates the gating behaviour, rather than a few highly specific biochemical interactions.

%A similar structural motif was recently found in the internal channels of microtubules (about 15 nm in diameter) decorated with so-called microtubule intrinsic proteins (MEPs), presumably modifying microtubule stability and rigidity \cite{Mukhopadhyay2001}.

However, we currently lack an understanding of the relationship between the molecular architecture of the polymer brush filling the pore and its ability to transport colloids with high selectivity and rate. Here, we develop a theoretical approach to reveal the physical mechanisms of diffusive colloid transport across polymer-filled mesopores.
A meshwork of flexible polymers effectively increases the local viscosity and thereby slows down transport of colloids compared to an open pore. 
On the other hand, an attractive polymer phase recruits colloids into the pore, thus increasing colloid transport, and such recruitment is further enhanced when attractive polymers extend outside the pore lumen. 
Intriguingly, the solvent strength through its influence on the density and compactness of the polymer meshwork impacts all of these effects. 
Using a self-consistent field approach, we define how solvent quality and colloid attraction to the polymer may be tuned to maximize the transport rate (even beyond the rate for an open pore) and to achieve highly selective colloidal transport with respect to colloid size or affinity for the polymer.


%%%%%%%%%%
\section{RESULTS}
%%%%%%%%%%


%%%%%%%%%%
\subsection{Defining the transport scenario}
%%%%%%%%%%

Salient features of our simulated mesopore are illustrated in Figure~\ref{fig:colloid_transport}.
The cylinder-shaped pore perforates a planar membrane, and is the sole conduit for colloids between two semi-infinite solution reservoirs.
Flexible polymer chains are end-grafted to the inner pore walls, at a density sufficient to form a polymer brush that fills the entire pore cross-section.

We will focus on a pore with a set radius $r_{\text{p}}^0$ and length $L^0$, and polymers with a degree of polymerization $N$ and grafting density $\sigma$ (Figure~\ref{fig:colloid_transport}).
Whilst the selected values are inspired by the nuclear pore complex (see Supplementary Methods 1), we expect that our findings will be of rather general validity so they can be applied to the performance analysis and rational design of mesopores with other geometries or polymer fillings.

Colloids are taken to be sphere-shaped with diameter $d$.
The interaction strength (contact free energy) between a polymer segment and the colloid surface is represented by the Flory-Huggins parameter $\chi_{\text{PC}}$.

\begin{figure}
    \centering
    \includegraphics[width = 3.5in]{fig/pore_cartoon.png}
    \caption{
        Schematic illustration of colloid diffusive transport through a pore filled with a polymer brush. 
        The brush is formed by linear polymer chains (red strands) with a degree of polymerization $N$, uniformly grafted with grafting density $\sigma$ 
        to the inner surface of a cylindrical pore.
        The pore radius is $r_{\text{p}}^0$ and the thickness of the impermeable membrane is $L^0$.
        Polymer chains are flexible with a statistical segment length $a$ and volume $\sim a^3$. 
        Spherical colloids with diameter $d$ are free to diffuse in the surrounding solvent.
        All length scales are normalized by the statistical segment length $a$.
        As a model pore, we set $L^0 = 2r_{\text{p}}^0 = 56$, $\sigma = 0.02$ and $N = 300$.
        With $a = 0.76 {\text{ nm}}$, these parameters reproduce the basic features of nuclear pore complexes.
        }
    \label{fig:colloid_transport}
\end{figure}

To understand how the polymer brush affects transport, we consider the stationary diffusive flux of colloids through the pore and analyze how it depends on the parameters of the pore, the brush and the colloid.
We consider unidirectional colloid transport driven solely by the concentration difference across the membrane and focus on the fundamental mechanisms of diffusion mediated by colloid-polymer interactions.
The colloid concentrations are set to zero and $c_0$ far away from the membrane (at $z\rightarrow\pm\infty$, respectively).
%We assume axial symmetry.
%Parameters relevant to colloid transport thus depend on the axial coordinate $z$ and the radial coordinate $r$, but not on the azimuthal angle.


%%%%%%%%%%
\subsection{Colloid transport is defined by the sum of resistances of regions outside and inside the pore}
%%%%%%%%%%

%%%%%%%%%%
\subsubsection{Bare pore as a reference case}
%%%%%%%%%%

A natural reference is the diffusive flux through a bare pore, which itself limits the transport of solutes \cite{Deen1987, Sun2024}.
Lord Rayleigh analyzed the flux of point-like solute particles through a circular pore in a planar membrane of negligible thickness \cite{Strutt1878}.
In this simplest case, the equiconcentration surfaces are oblate spheroids, and the streamlines form confocal hyperboloids of revolution \cite{Cooke1966}.
The net flux through the pore is given by
\begin{equation}
    J=2D_0r_{\text{p}}\Delta c,
    \label{eq:flux_Ral}
\end{equation}
where we take $\Delta c = c_0$ without loss of generality, $r_{\text{p}}$ is the effective pore radius and $D_0 = k_{\text{B}}T / (3 \pi \eta_{\text{S}} d)$ is the diffusion coefficient of the colloid in plain solvent with viscosity $\eta_{\text{S}}$ ($k_{\text{B}}T$ is the thermal energy unit).

A membrane of finite effective thickness $L$ allows an approximate analytical solution \cite{Brunn1984}:
\begin{equation}
    J=\frac{2 D_0 r_{\text{p}}}{1+\cfrac{2L}{\pi r_{\text{p}}}}\Delta c.
    \label{eq:flux_finlength}
\end{equation}

Introducing the resistance $R$ to colloid flow $J = \frac{\Delta c}{R}$ provides a natural interpretation of Eq.~(\ref{eq:flux_finlength}) in terms of the total resistance of the pore:
\begin{equation}
    R = \frac{L}{D_0 \pi r_{\text{p}}^{2}} + \frac{1}{2 D_0 r_{\text{p}}} = R_{\text{int}}^{0} + R_{\text{ext}}^{0},
    \label{eq:resistance}
\end{equation}
where the superscript '0' refers to the bare pore.
The first term in Eq.~(\ref{eq:resistance}) represents the resistance of the interior of the bare pore.
The second term is the Rayleigh resistance (Eq.~(\ref{eq:flux_Ral})) and accounts for the effects of convergent flow toward the pore entrance and its symmetric counterpart at the pore exit.
Inside the pore, the flow lines are approximately axial.
In the bare pore scenario, the inverse of the diffusion constant ($\rho_0=D_0^{-1}$) represents the resistivity of the medium both inside and outside the pore.
Naturally, the resistance for a thin membrane is determined by the exterior region ($R \approx R_{\text{ext}}^{0}$ for $L \ll r_{\text{p}}$), whereas for long pores the inner region becomes dominant ($R \approx R_{\text{int}}^{0}$ for $L \gg r_{\text{p}}$).

The finite size of colloids affects the diffusive flux in two ways.
First, the excluded volume reduces the effective pore radius ($r_{\text{p}} = r_{\text{p}}^0 - d/2$) and increases the effective pore length ($L = L^0 + d$) \cite{Renkin1954, Beck1970, Bungay1973, Anderson1974, Brenner1977}.
Second, the presence of the pore walls entails some additional drag \cite{Ladenburg1907, Faxen1922, Haberman1958}.
We neglect the latter as the presence of the polymer brush screens hydrodynamics and thus requires a different kind of drag analysis, as discussed below.


%%%%%%%%%%
\subsubsection{A polymer filling affects the resistance of the pore itself, and also of regions outside the pore}
%%%%%%%%%%

Conformations adopted by overlapping polymer chains grafted to the pore walls are controlled by strong intermolecular interactions that depend on the solvent quality.
The solvent quality is here quantified by the Flory-Huggins parameter $\chi_{\text{PS}}$.
Values of $\chi_{\text{PS}}<0.5$ and $\chi_{\text{PS}}>0.5$ correspond to good and poor solvent, respectively, and $\chi_{\text{PS}}=0.5$ represents the ideal (or $\theta$-)solvent.

The polymer density profile $\phi(z,r)$ in the pore was calculated by the two-gradient self-consistent field numerical method of Scheutjens and Fleer (SF-SCF; see Supplementary Methods 2-3).
In Figure~\ref{fig:phi_hm_grid}, one can appreciate the expected increase in polymer concentration inside the pore with decreasing solvent quality (increasing $\chi_{\text{PS}}$).
With the selected pore and polymer parameters (Figure~\ref{fig:colloid_transport}), the polymer brush fills the entire pore cross-section within the full range of solvent qualities explored ($\chi_{\text{PS}}\le0.9$), so that colloids need to navigate the polymer meshwork to move across the membrane.
For wider pores, shorter polymers and/or lower grafting densities, an open channel free of polymer may appear in the pore center, as discussed in detail elsewhere~\cite{Ligoure2001,Laktionov2021}.
This scenario would result in a distinct permeation behaviour, as colloids could move through the pore without traversing the polymer brush, and is not considered here.

\begin{figure}
    \centering
    \includegraphics[width = 3.5in]{fig/phi_hm_grid.png}
    \caption{
    Maps of the polymer volume fraction $\phi(r,z)$ for a polymer brush in a cylindrical pore with solvent quality ranging from good (upper left panel) to poor (lower right panel), as quantified by the Flory-Huggins parameter $\chi_{\text{PS}}$.
    Polymer volume fractions are mapped in cylindrical coordinates (as shown by $rz$-coordinate arrows), color coded as indicated and with selected iso-concentration lines. The blank space corresponds to pure solvent; the membrane is colored green.
    Pore and brush parameters are as given in Figure~\ref{fig:colloid_transport}.
    }
    \label{fig:phi_hm_grid}
\end{figure}

Figure~\ref{fig:phi_hm_grid} further illustrates that whilst the brush remains confined within the pore lumen in poor solvent ($\chi_{\text{PS}}=0.9$) it protrudes substantially into the surrounding space in ideal and good solvents ($\chi_{\text{PS}}\le0.5$), thus forming fringes on either side of the pore.
The polymer brush therefore will impact on the resistance to colloid flow within as well as outside the pore, such that
\begin{equation}
    R=R_{\text{int}}+R_{\text{ext}},
    \label{eq:R_tot_tot}
\end{equation}
with $R_{\text{int}}\rightarrow R_{\text{int}}^{0}$ and $R_{\text{ext}}\rightarrow R_{\text{ext}}^{0}$ in the limit of the bare pore.


%%%%%%%%%%
\subsection{Diffusivity and insertion free energy control diffusive transport}
%%%%%%%%%%

Zooming in, we can analyze how colloids are accumulated in or depleted out from the polymer meshwork due to attractive or repulsive interactions, respectively, with the polymer meshwork, and how the meshwork affects the colloid's local mobility.

%%%%%%%%%%
\subsubsection{Local colloid mobility is determined by polymer mesh size, colloid size, and polymer-colloid attraction strength}
%%%%%%%%%%

The crowded polymers naturally decrease the diffusion of colloids.
A polymer brush is effectively described as an inhomogeneous semi-dilute polymer solution with a concentration-dependent correlation length (mesh size) $\xi(\phi)$.
Colloids of size $d > \xi$ experience additional friction as they are trapped by the polymer meshwork.
As a result, diffusion is slowed compared to pure solvent, leading to a position-dependent diffusion coefficient $D(r,z) < D_0$.

According to a scaling theory by Cai et al. \cite{Cai2011} for diffusion of non-sticky colloids in a semi-dilute polymer solution, the colloid mobility scales as $D\sim D_0 (\xi/d)^2\ll D_0$ for $d\gg \xi$, while small colloids diffuse virtually unimpeded ($D\sim D_0$ for $d\ll \xi$). We here use a simple interpolation formula to capture the diffusion coefficient across the full range of relevant colloid sizes relative to the correlation length $d / \xi$:
\begin{equation}
    D\{\phi(r,z)\} = \frac{D_0}{1+[\beta d / \xi\{\phi(r,z)\}]^2} \approx \frac{D_0}{1+[\beta d \phi(r,z)]^2} .
    \label{eq:Rubinstein}
\end{equation}
The correlation length $\xi$ in Eq.~(\ref{eq:Rubinstein}) is controlled by the local polymer concentration $\phi(r,z)$ and also depends on the solvent quality \cite{DeGennes1979}.
In the term on the right hand side we have approximated $\xi \cong \phi^{-1}$, which is valid close to $\theta$-solvent conditions in a mean-field regime.
The coefficient $\beta$ in Eq.~(\ref{eq:Rubinstein}) is a numerical pre-factor.
In section 2.7, we estimate $\beta = 5.5$ by comparing our theoretical stationary flux predictions to experimental data on the flux of non-sticky colloids of different sizes through nuclear pore complexes.
In the following sections we consistently use this value for numerical estimates of the pore permeability and permselectivity.

Figure~\ref{fig:D_fe_map}a shows how the colloid diffusivity, $D(\phi/D_{0}$, varies with the local polymer concentration for a fixed colloid size; the corresponding spatial distribution $D(\{\phi(r,z)\})/D_{0}$ in a $\theta$-solvent ($\chi_{\text{PS}} = 0.5$) is presented in Figure~\ref{fig:D_fe_map}b.

Recent theoretical, experimental and simulation studies extended the scaling approach of Cai et al. \cite{Cai2011} to include the effects of polymer-colloid attraction \cite{Yamamoto2018, Carroll2018}.
In the limit where proper diffusion of polymer chains is negligible (relevant to our case of pore-anchored chains), the characteristic desorption time of a single polymer-colloid contact $\tau_\text{des}(\epsilon)$ appears as an extra timescale.
Here, $\epsilon > 0$ is the absolute value of the activation energy to break a contact.
The polymer-colloid attraction may be considered weak if the desorption time is much smaller than the self-diffusion time.
Typically, this happens when $\epsilon \lesssim k_\text{B} T$, and in this case attractive forces only modify local friction by affecting the local packing of polymers around the colloid.
The magnitude of this effect is estimated to be small, reducing the diffusivity by at most a few 10\% \cite{Yamamoto2011}.
On the other hand, the desorption time grows exponentially with $\epsilon$ and affects the diffusion substantively when $\epsilon$ amounts to several $k_\text{B} T$ units.
Numerical simulations \cite{Yamamoto2018} demonstrate that the colloid diffusivity is reduced by an exponential factor with approximately half the activation energy:

\begin{equation}
    D(\epsilon)\approx D(\epsilon=0) \exp (-\epsilon / 2).
    \label{eq:Yamamoto}
\end{equation}

In our model, the energy of a polymer-colloid contact is $\chi_{\text{PC}}/6$, and therefore $\epsilon < 1/3$ within the explored range $0\geq\chi_{\text{PC}}\geq-2$.
Throughout most of the paper, we assume that the surface of the colloid is homogeneous, and neglect the small reduction in the local colloid mobility due to weak polymer attraction.
However, in Section 2.7 we will return to this question in the context of proteins traversing the nuclear pore, where the colloid surface and the polymers may be heterogeneous with affinity localized in a few sticky patches.
We can explore the limiting case where all the (negative) surface free energy $\pi d^2 \gamma(r,z)$ is assigned to a single sticky patch.
The diffusion coefficient is then modified as
\begin{eqnarray}
    D_{\text{sticky patch}} = 
    \begin{cases}
        D \exp(\frac{\pi d^2 \gamma(r,z)}{2}) & \text{if } \gamma(r,z) < 0 \\
        D & \text{otherwise}
    \end{cases}
     \label{eq:Sticky diff}
\end{eqnarray}
to interpolate between non-sticky and sticky colloids.
It is safe to assume that realistic cases of attractive polymer-colloid interaction lie between the two extremes of a perfectly homogeneous surface ($D_\text{homo} \approx D$) and a single sticky patch (Eq.~(\ref{eq:Sticky diff})).

Several other theoretical and empirical models have been proposed to describe the diffusion of colloids in polymer meshworks \cite{Schweizer2003,Kohli2012,Holyst2009,Phillies1988}.
Although the predictions of different models differ quantitatively, they all share the same qualitative trend.

%%%%%%%%%%
\subsubsection{Defining leading contribution to the insertion free energy}
%%%%%%%%%%

The position-dependent insertion free energy $\Delta F(r,z)$ is the work required to move a colloid from the exterior solution into the polymer brush.
For colloids that are significantly smaller than the size of the pore, the insertion free energy is determined entirely by the local polymer concentration (i.e., wall effects can be neglected), and comprises two distinct contributions:
\begin{eqnarray}
    \Delta F = \Delta F_{\text{osm}} + \Delta F_{\text{sur}},
    \label{eq:Delta_F}
    \\
    \Delta F_{\text{osm}}(r,z) = \int_{V} \Pi(r',z') dV', \nonumber
    \\
    \Delta F_{\text{sur}}(r,z) = \oint_{S} \gamma (r',z') dS'. \nonumber
\end{eqnarray}
The coordinates $(r,z)$ refer to the center of the colloid, whilst the insertion free energy is obtained by integrating over the volume and surface of the colloid, respectively.
Here and below, all the free energy values are normalized by the thermal energy unit $k_{\text{B}}T$.

The osmotic contribution, $\Delta F_{\text{osm}}$, is proportional to the colloid volume and accounts for the work against excess osmotic pressure upon insertion of the colloid into the brush.
The local osmotic pressure (normalized by $k_\text{B} T$) is calculated from the local polymer concentration as
\begin{equation}
    \begin{aligned}
        \Pi(r,z)=  \phi(r,z)\frac{\partial f\{\phi(r,z)\}}{\partial \phi(r,z)} - f\{\phi(r,z)\}= 
        \\
        [-\ln(1-\phi(r,z)) - \phi(r,z) -\chi_{\text{PS}}\phi^2(r,z)],
    \end{aligned}
    \label{eq:osmotic}
\end{equation}
where 
$$
f\{\phi(r,z)\}=(1-\phi(r,z))\ln(1-\phi(r,z)) +\chi_{\text{PS}}\phi(r,z)(1-\phi(r,z))
$$
is the mean-field Flory expression for the interaction free energy per unit volume of the polymer solution of concentration (volume fraction) $\phi(r,z)$.
As long as the osmotic pressure inside the brush is positive, $\Delta F_{\text{osm}}$ is also positive and provides a dominant contribution for sufficiently large colloids.

The interfacial contribution, $\Delta F_{\text{sur}}$, is proportional to the colloid surface area, with the surface tension $\gamma (r,z)$ approximated as
\begin{gather}
     \gamma (r,z)= \frac{1}{6}(\chi_{\text{ads}} - \chi_{\text{ads}}^{\text{crit}})\phi^{\ast}(r,z),
    \label{eq:chi_ads} 
    \\
    \text{with } \chi_{\text{ads}} = \chi_{\text{PC}} - \chi_{\text{PS}}(1-\phi^{\ast}), \text{ and } \phi^{\ast}(r,z)= (b_{0} + b_{1}\chi_{\text{PC}})\phi(r,z).
    \nonumber
\end{gather}
Here $\gamma$ is the change in the free energy of a unit area upon replacement of a contact of the colloid with solvent by a contact with a polymer solution of concentration $\phi(r,z)$.
The coefficients $b_0 = 0.7$ and $b_1 = -0.3$ are phenomenological parameters that account for depletion or accumulation of the polymer in the vicinity of the colloid surface, and can be treated as constant to a good approximation (Supplementary Methods 4-5).

Depending on the relative strengths of polymer-colloid ($\chi_{\text{PC}}$) and polymer-solvent ($\chi_{\text{PS}}$) interactions, the sign of $\gamma$ can be either positive or negative.
If the colloid surface is repulsive ($\chi_{\text{ads}} \geq 0$) or even weakly attractive for polymers ($\chi_{\text{ads}}^{\text{crit}} \leq \chi_{\text{ads}} < 0$), then, due to steric constraints imposed by the impermeable surface, the available conformations of the polymer are restricted, leading to polymer depletion near the colloid surface and $\gamma > 0$.
At the critical adsorption condition $\chi_{\text{ads}} = \chi_{\text{ads}}^{\text{crit}}$, the losses in conformational entropy caused by the presence of the surface are exactly balanced by the free energy gain from monomer-surface contacts, causing $\gamma$ to vanish \cite{Fleer1993,Birshtein1979,Birshtein1983,Eisenriegler1982}.
Ultimately, $\gamma < 0$ for sufficiently strong attraction ($\chi_{\text{ads}} < \chi_{\text{ads}}^{\text{crit}}$).

We applied an approximate analytical scheme to evaluate the insertion free energy $\Delta F(r,z)$ as $\Delta F\{\phi(r,z)\}$, where $\phi(r,z)$ is the polymer density distribution in a colloid-free brush calculated with the SF-SCF approach.
The colloid is thus considered as a 'probe' which does not perturb the global polymer concentration distribution $\phi(r,z)$.
With this scheme, we can evaluate the insertion free energy at any position of the colloid in the brush, including off the pore axis (Supplementary Method 6).
Comparison with direct SF-SCF calculations of the insertion free energy for colloids on the pore axis demonstrated good quantitative agreement (Supplementary Method 5), justifying the use of the more versatile analytical scheme.

Figure~\ref{fig:D_fe_map}a shows the net insertion free energy $\Delta F(\phi)$ of a colloid in a homogeneous polymer mesh with polymer volume fraction $\phi$.
For sufficiently attractive particles, $\chi_{\text{PC}} < \chi_{\text{ads}}^{\text{crit}} + \chi_{\text{PS}}$, the curve is non-monotonic: it crosses zero at a finite concentration where the interfacial gain, $\Delta F_{\text{sur}}(\phi)$, exactly balances the osmotic penalty, $\Delta F_{\text{osm}}(\phi)$.
Above this concentration the osmotic contribution dominates and $\Delta F(\phi)$ becomes positive, and colloid are repelled; below that concentration polymer meshwork attracts colloids.
Inert or only weakly attractive particles ($\chi_{\text{PC}} = -0.5$) always experience a positive insertion free energy and are therefore repelled by the polymer meshwork.
 
Figure~\ref{fig:D_fe_map}c illustrates how colloids may be either repelled or attracted by the polymer meshwork, depending on the balance of osmotic and interfacial contributions to $\Delta F$.
Since the polymer concentration in the pore is strongly inhomogeneous, the net insertion free energy $\Delta F(r,z)$ may exhibit quite large spatial variations.
For example, the brush shown in Figure~\ref{fig:D_fe_map}c at $\chi_{\text{PC}}=-0.75$ simultaneously exhibits attraction ($\Delta F<0$) at the loose fringes and repulsion ($\Delta F>0$) inside the pore where the polymer concentration is high.

\begin{figure}
    \centering
    %\REMOVE \CENTERLINE BEFORE SUBMISSION
    \centerline{\includegraphics[width = 6.5in]{fig/diffusivity_and_fe.png}}
    \caption{%
        Effect of the polymer meshwork on local diffusivity and insertion free energy.  
        \textbf{(a)}  Comparison of the reduced colloid diffusivity caused by the polymer meshwork (black curve, $-\ln(D/D_{0})$) with the insertion free energy (colored curves, labelled by $\chi_{\text{PC}}$) as functions of the local polymer volume fraction~$\phi$.  
        Both quantities share the same vertical scale  
       so that the sum of the black and one of the colored curves,  $\ln(\rho D_{0}) = -\ln(D/D_{0}) + \Delta F$, quantifies the total local resistivity; the horizontal zero line corresponds to unimpeded transport as in pure solvent; positive values of the colored curves indicate resistance enhanced by the free energy contribution, whereas negative values indicate lowered resistance.
       \textbf{(b)}  Spatial map of the position-dependent diffusivity, $-\ln\bigl[D(\phi(r,z))/D_{0}\bigr]$, in cylindrical coordinates.  
        \textbf{(c)}  Maps of the insertion free energy $\Delta F(r,z)$ for polymer-colloid interaction strengths ranging from $\chi_{\text{PC}}=-0.50$ (weakest attraction) to $\chi_{\text{PC}}=-1.25$ (strongest attraction), as indicated.
        The pore and brush parameters are those of Figure~\ref{fig:colloid_transport} in all the panels; the solvent strength parameter is $\chi_{\text{PS}}=0.5$ and the particle diameter is $d=8$.
    }
    \label{fig:D_fe_map}
\end{figure}



%%%%%%%%%%
\subsubsection{Linking local resistivity to global transport}
%%%%%%%%%%

%COMMENT RR: I found this section quite technical and difficult to follow. Can we spell out the main assumptions in simpler terms, understandable for readers not experienced with some of the math and concepts?

Having defined the position-dependent insertion free energy and mobility, we can develop an analytical method to estimate the total resistance of the brush-filled pore to the diffusive flow of colloids.

Diffusive transport in the presence of an external force generated by the insertion free energy $\Delta F$ is described by the Smoluchowski equation,
\begin{equation}
    \frac{\partial c(\bm{r})}{\partial t}=-\nabla \cdot \left[D(\bm{r}) \nabla c({\bm{r}})+D(\bm{r})c({\bm{r}})\nabla(\Delta F(\bm{r}))\right],
    \label{eq:Smoluch}
\end{equation}
where $c(\bm{r})$ is the colloid concentration.
Standard substitution introduces the potential function $\psi(\bm{r})$ and the effective conductivity $\tilde{D}(\bm{r})$ as
\begin{equation}
     \psi(\bm{r})=c(\bm{r})\exp(\Delta F(\bm{r}))
     \label{eq:psi} 
\end{equation}
\begin{equation}
    \tilde{D}(\bm{r})=D(\bm{r}) \exp(-\Delta F(\bm{r})) 
    \label{eq:D_tilde}
\end{equation}
With these substitutions, the flux density is expressed as
\begin{equation}
    \bm{j}=- \tilde{D}(\bm{r})  \nabla \psi(\bm{r})
    \label{eq:flux_psi}
\end{equation}
Under stationary conditions, the original Smoluchowski equation ~(\ref{eq:Smoluch}) is reduced to

\begin{equation}
    \nabla \cdot \left(\tilde{D}(\bm{r})\nabla\psi(\bm{r} \right)=0
    \label{eq:Laplace_modif}
\end{equation}
The boundary conditions for the potential function in our case are $\psi(z\rightarrow -\infty)=\Delta c$ and $\psi(z\rightarrow +\infty)=0$ since the insertion free energy $\Delta F$ vanishes far away from the pore.

One approach to find the total flux (resistance) is by direct numerical solution of Eqs.~(\ref{eq:D_tilde}-\ref{eq:Laplace_modif}) which we did as a check. However, a more transparent and instructive approach is to follow an
electric analog of the problem which can be reformulated as finding the total resistance of a medium with position-dependent resistivity possessing axial symmetry:

\begin{equation}
    \rho(r,z)= \tilde{D}^{-1}(r,z)
    \label{eq:rho}
\end{equation}
The variation of the local resistivity $\rho(r,z)$ is due to the position-dependent polymer volume fraction $\phi(r,z)$ 
Figure~\ref{fig:D_fe_map}a displays separately the two underlying contributions:
(i) the reduction of the colloid diffusivity in a homogeneous polymer mesh of concentration $\phi$ and 
(ii) the net insertion free energy.
Adding these curves gives the net effect on resistance,
$\ln\!\bigl[\rho(\phi)D_{0}\bigr]= -\,\ln\!\bigl[D(\phi)/D_{0}\bigr] + \Delta F(\phi),$
as stated in Eq.~(\ref{eq:rho}). 
Only for sufficiently attractive particles, where $\Delta F(\phi)$ is negative enough, does the sum become negative, $\ln\!\bigl[\rho(\phi)D_{0}\bigr] < 0$, indicating that the brush locally reduces the resistance.  
Otherwise the net effect of the polymer brush remains positive and the resistance increases.


We consider a set of approximate equipotential surfaces ($\psi(r,z)=\text{const}$) foliating the space available for the colloid flow: inside the pore, the surfaces are taken as  discs of radius $r_{\text{p}}$ normal to the pore axis; outside the pore, we use oblate hemispheroids taken from the Rayleigh solution \cite{Strutt1878} (Figure~\ref{fig:integration_scheme}).
Analogously to a set of resistors connected in parallel, the total conductivity of a layer between two adjacent equipotential surfaces is obtained by integration of local conductivities over the layer.
Within the pore, $|z|\leq L/2$, the result is given by
\begin{equation}
\varrho_{\text{int}}^{-1}(z)= 2\pi\int_{0}^{r_{\text{p}}^{}} \rho^{-1}(r,z) r \, dr
\label{eq:varrho1}
\end{equation}

In the exterior region, $|z| >L/2$, the expression is modified to integrate over the aforementioned hemispheroids:
\begin{equation}
    \begin{gathered}
        \varrho_{\text{ext}}^{-1}(z)= 2\pi\int_{0}^{r_{\text{p}}^{}} \rho^{-1}\left( r'(r,z), z'(r,z) \right)  \tilde{h} (r,z) dr\\
        r'(r,z) = r\sqrt{1 + \frac{(z - L/2)^2}{r_{\text{p}}^2}}\\
        %r'(r,z) \in [0, \sqrt{r_{\text{p}}^2 + (z-L/2)^2}]\\
        z'(r,z) = (|z| - L/2) \frac{\sqrt{r_{\text{p}}^2 - r^2}}{r_{\text{p}}} +  \text{sign}(z) \frac{L}{2}\\
        %z'(r,z) \in [L/2, z]
        \tilde{h} (r,z) = h_r h_{\theta} h_z^{-1} = \dfrac{r}{r_{\text{p}}}\dfrac{r_{\text{p}}^2 + (|z|-L/2)^2}{\sqrt{r_{\text{p}}^2 - r^2}}
    \end{gathered}
\label{eq:varrho2}
\end{equation}
where $r'(r,z) , z'(r,z)$ are the functions parameterizing the equipotential surfaces, and $h_r$, $h_{\theta}$ and $h_z$ are the corresponding Lam\'e coefficients (see Supplementary Methods 7-8). In the case of a homogeneous brush considered here, the function $\varrho_{\text{ext}}^{-1}(z)$ is even.

\begin{figure}
    \centering
    \includegraphics[width=3in]{fig/resistitance_integration_miniature.png}
    \caption{
    Integration scheme for the layer resistance $\varrho_{\text{int}}\text{d}z, \varrho_{\text{ext}}\text{d}z$.  
    The intrinsic orthogonal curvilinear coordinates are defined by equipotential surfaces, \mbox{$\psi = \text{const}$}, and the flux lines of the flux density field $\bm{j}$.
    A layer between two adjacent equipotential surfaces with a thickness $\text{d}z$ along the central axis has resistance $\varrho_{\text{int}}\text{d}z$ inside the pore ($|z|<L/2$, blue shading) and $\varrho_{\text{ext}}\text{d}z$ in the exterior region (red shading).  
    The red rectangle illustrates the local conductivity element at $(r',z')$ within the exterior layer.  
    The parametrization $(r'(r,z),\,z'(r,z))$ traces the integration path and maps the intrinsic coordinates back to the original cylindrical coordinates $(r,z)$, as indicated by the red arrows.
    }
    \label{fig:integration_scheme}
\end{figure}

On the other hand, since the consecutive layers are connected in series, their total resistance can be found by appropriate integration:
\begin{equation}
    R_{\text{int}} = \int_{-L/2}^{+L/2}\varrho_{\text{int}}(z) dz,
    \label{R_int}
\end{equation}

\begin{equation}
   R_{\text{ext}} =2\int_{+L/2}^{+\infty}\varrho_{\text{ext}}(z)dz
    \label{R_ext}
\end{equation}

For a bare pore without a polymer brush, this method recovers Eq.~\ref{eq:resistance}, as expected.
Under poor solvent conditions, where the brush entirely contained within the interior of the pore (Figure~\ref{fig:phi_hm_grid}), the total resistance is $R = R_{\text{int}} + R_{\text{ext}}^{0}$ as the exterior is not modified by the brush.
Conversely, a brush under good or $\theta$-solvent conditions produces swollen fringes (caps) outside the pore that modify the resistance $R_{\text{ext}}$.


%%%%%%%%%%
\subsection{An attractive polymer filling enhances colloid fluxes through the pore}
%%%%%%%%%%

\begin{figure}
    \centering
    \includegraphics[width = 3in]{fig/resistitance_components.png}
    \caption{
    Interior ($R_{\text{int}}$), exterior ($R_{\text{ext}}$) and total ($R = R_{\text{int}} + R_{\text{ext}}$) resistance  vs polymer-colloid interaction strength $\chi_{\text{PC}}$ for a good solvent ($\chi_{\text{PS}} = 0.3$, in blue) and a poor solvent ($\chi_{\text{PS}} = 0.7$, in orange).
    %(all normalized by the viscosity of the solvent $\eta_\text{S}$, and encoded by line type as indicated)
    The exterior $R_{\text{ext}}^{0}$ and total $R_0$ resistances for the bare pore are also shown (red lines of matching type).  
    The resistances are presented in dimensionless units $R\tfrac{k_{B}T}{\eta_{\text{S}}}$; pore and brush parameters are as given in Figure~\ref{fig:colloid_transport}; particle diameter $d = 12$.\\
    The results of the direct numerical solution of the Smoluchowski equation are shown with markers.
    }
    \label{fig:R_vs_chi_PC}
\end{figure}

Figure~\ref{fig:R_vs_chi_PC} visualizes the relative contributions of the pore interior ($R_{\text{int}}$) and exterior ($R_{\text{ext}}$) to the total resistance as a function of the polymer-colloid attraction $\chi_\text{PC}$,
for a selected colloid size ($d = 8$) in a good solvent ($\chi_\text{PS} = 0.3$) and a poor solvent ($\chi_\text{PS} = 0.7$).
A striking feature is that attracted colloids can achieve diffusive fluxes that exceed the limit of the bare pore, as indicated by the segments of the $R(\chi_{\text{PC}})$ curves that fall below the solid horizontal red line marking the bare-pore resistance $R_{0}$.
This result may at first appear surprising, given that the polymer medium is expected to slow down the diffusion of colloids.
However, this slowing down is counteracted by the attractive potential of the polymer meshwork, which reduces local resistivity according to the exponential factor in Eq.~(\ref{eq:rho}).

The reduced local resistivity has pronounced consequences for diffusive transport in both the interior and the exterior of the pore.
Figure~\ref{fig:R_vs_chi_PC} illustrates that the interior resistance $R_{\text{int}}$ can be driven practically to zero by increasing the polymer-colloid attraction (decreasing $\chi_\text{PC}$) below a certain threshold.
Compared to a bare pore (Eq.~(\ref{eq:resistance})) such a short-circuiting effect entails a reduction in the resistance by a factor of up to $R^0_{\text{int}}/R^0_{\text{ext}}+1 \approx 2/\pi \times L / r_{\text{p}} + 1$.
For the pore and colloid considered here, this represents an approximately 3-fold reduction, to a level marked by the dashed horizontal red line in Figure~\ref{fig:R_vs_chi_PC} which equals the contribution of convergent flow to the bare-pore resistance ($R^0_\text{ext}$).
The reduction would be even stronger for longer pores ($L\gg r_p$), and for larger colloids that increase the effective pore length and decrease the effective pore diameter.

The exterior resistance $R_{\text{ext}}$, on the other hand, always retains a finite contribution from the diffusive fluxes in the semi-infinite reservoir, setting an absolute lower bound on the total resistance.
The reduction of $R_{\text{ext}}$ below $R^0_\text{ext}$ evidenced in Figure~\ref{fig:R_vs_chi_PC} is due attractive brush fringes that protrude and facilitate diffusive transport outside the pore.
The magnitude of this effect increases with the extension of the polymer cap, and explains why the minimal attainable resistance increases with decreasing solvent quality (Figure~\ref{fig:phi_hm_grid}).
Approximating the brush fringes on either end of the pore as hemispherical caps with radius $r_\text{cap}$, the plateau conditions are equivalent to twice the resistance of an ideally absorbing hemisphere \cite{Crank1980},
\begin{equation}
    R_\text{ext}^\text{\text{min}} = 1 / (D_0 \pi r_\text{cap}).
    \label{eq:R_ext_min}
\end{equation}

Attractive brush fringes thus entail a reduction in resistance by a factor of up to $R_\text{ext}^0 / R_\text{ext}^\text{\text{min}} = \pi/2 \times r_\text{cap}/r_\text{p}$.
In good solvent ($\chi_\text{PC} = 0.3$), for example, the cap radius (along the pore axis) is comparable to the pore diameter (Figure~\ref{fig:phi_hm_grid}), leading to a 3.3-fold reduction of the external resistance, and a cumulative 10-fold reduction of the total resistance, compared to the bare pore (Figure~\ref{fig:R_vs_chi_PC}).
As the solvent quality decreases the cap size shrinks (Figure~\ref{fig:phi_hm_grid}), with a correspondingly reduced benefit on pore conductivity, as illustrated for $\chi_\text{PC} = 0.7$.
For even poorer solvents, the cap and its benefit disappear entirely ($R_\text{ext} = R_\text{ext}^0$; not shown).


%%%%%%%%%%
\subsection{Polymer-filled mesopores effectively gate colloids by their attraction to the polymer}
%%%%%%%%%%

Figure~\ref{fig:R_vs_chi_PC} also illustrates how the total resistance of the pore varies with the colloid's affinity to the polymer brush.
As expected, increasing the polymer-colloid attraction strength (i.e., more negative $\chi_{\text{PC}}$) results in a monotonic decrease in the pore's total resistance, since the interfacial term in the insertion free energy becomes more negative, thereby increasing the local conductivity $\rho^{-1}$.

Most notable is a sharp transition from a regime of facilitated permeation ($R < R_0$) to a regime of impeded permeation ($R > R_0$).
The regime of impeded permeation is dominated by the internal resistance.
It exhibits high selectivity with respect to the polymer-colloid interaction strength, and a mostly very high total resistance and thus low colloid flux, both appreciable in Figure~\ref{fig:R_vs_chi_PC} as a sharp increase in $R$ over a relatively modest $\chi_{\text{PC}}$ range.
In contrast, the region of facilitated permeation is dominated by the external resistance. 
It exhibits high colloid fluxes but rather low (if any) $\chi_{\text{PC}}$ selectivity, as demonstrated by the previously analyzed plateau.
Thus, the transition between the two regimes of transport defines the condition for sharp colloid gating, with remarkably efficient transport in the regime limited by external resistance and effective blockage in the regime limited by internal resistance.

In this context, the solvent quality can be seen as a regulator of the polymer-colloid interaction level for gating. 
Lowering the solvent quality (increasing $\chi_{\text{PS}}$) reduces $\chi_{\text{ads}}$ and shifts the entire $R(\chi_{\text{PC}})$ curve toward larger $\chi_{\text{PC}}$ values.
Thus, a poorer solvent extends the range of facilitated permeation towards more weakly interacting colloids.

The here-presented trends are qualitatively correct also for colloids with sizes smaller or larger than the $d=8$ considered here.
Naturally, the gating effect will be rather moderate for small colloids, yet even sharper for larger colloids.


%%%%%%%%%%
\subsection{High colloid flux implies colloid enrichment in the pore}
%%%%%%%%%%

Colloid concentration profiles under stationary flux conditions can be found by numerically solving Eq.~(\ref{eq:Smoluch}) with $\frac{\partial c(r,z)}{\partial t} = 0$  (Supplementary Method 9). 

Figure~\ref{fig:colloid_concentration} maps the steady-state colloid concentration across a polymer-filled pore with colloid size $d = 8$ and polymer-colloid interaction strength $\chi_{\text{PC}} = -1.25$ in a good solvent ($\chi_{\text{PS}} = 0.5$).
This condition corresponds to the non-blocking pore with transport rates close than of a bare pore $R \approx R_0 \cdot 5$ (compared to prohibitive to inert particle $R(\chi_{\text{PC}=0}) \approx R_0 \cdot 13500$).

\begin{figure}
    \centering
    %\REMOVE \CENTERLINE BEFORE SUBMISSION
    \centerline{\includegraphics[width=5.5in]{fig/streamlines.png}}
    \caption{
    Color map of the steady-state colloid concentration found from the direct numerical solution, normalized by the bulk concentration in the source compartment $c_0$ as a function of $z$ and $r$.
    Isoconcentration contours are shown for $c/c_0$ from 0.002 to 20 as labeled on each contour line.
    The flux is represented by streamlines marked with small arrows, indicating the average colloid trajectory.
    Pore and brush parameters are the same as in Figure~\ref{fig:colloid_transport}; $d = 8$, $\chi_{\text{PC}} = -1.25$ and $\chi_{\text{PS}} = 0.5$.
    The effective and physical pore radius $r_\text{p}$ and $r_\text{p}^0$, respectively, and pore length $L$ and $L_0$, respectively, due to excluded volume are shown with white arrows.
    The inset in the upper right corner shows the solution in the form of potential function $\psi$.
    The total resistance $R$ of the pore is somewhat close to the resistance of the bare pore $R_0$, due to the effective of the negative insertion free energy, hence non-equilibrium partitioning to in the polymer-filling.
    }
    \label{fig:colloid_concentration}
\end{figure}

The map illustrates several salient features of the diffusion process.
Outside the region of the pore and polymer fringes, the colloid concentration profile is as expected for plain solution: the concentration rapidly approaches the respective bulk concentrations of the semi-infinite reservoirs, $c(z = -\infty) = c_0$ and $c(z = +\infty) = 0$, and the equiconcentration surfaces near the pore entrance ($0.995 < c/c_0 <1.0$) and exit ($c/c_0<0.05$) form a symmetric set of oblate hemispheroids.
Inside the pore, the flux lines run almost parallel to the pore axis.

The most notable observation is that the colloid concentrations substantively exceed $c_0$ near the pore entrance (by a factor of $\sim20$) and inside the pore (by a factor  of $\sim10$).
This effect is caused by the negative insertion free energy in the space occupied by the polymer brush.
At equilibrium (i.e., with vanishing fluxes), the partitioning would amount to $c_{\text{eq}}/c_0 = \exp\left( -\Delta F \right)$.
In the steady state (i.e., with non-vanishing fluxes), the colloid concentration is reduced but approaches the equilibrium concentration as the insertion free energy becomes largely negative ($c/c_0 \to c_{\text{eq}}/c_0$).

The presented quantitative results are only valid for sufficiently low bulk concentrations $c_0$, as our model disregards any colloid crowding effects.
When this crowding is accounted for, the steady-state colloid concentrations in the brush will be systematically lower.


%%%%%%%%%%
\subsection{Polymer-filled mesopores effectively gate colloids by their size}
%%%%%%%%%%
%2.7
%%%%%%%%%%%

\begin{figure}
    \centering
    \includegraphics[width = 3.5in]{fig/permeability_on_d.png}
    \caption{
    Normalized dimensionless pore resistance $R\tfrac{k_{B}T}{\eta_{\text{S}}} $ as a function of the colloid size $d$ for \textbf{(a)} selected values of the polymer-colloid interaction strength $\chi_{\text{PC}}$ (as indicated in the legend) at a fixed solvent strength $\chi_{\text{PS}} =0.5$, and \textbf{(b)} selected values of $\chi_{\text{PS}}$ (as indicated in the legend) at a fixed $\chi_{\text{PC}} = -1.25$. 
    Pore and brush parameters are as given in Figure~\ref{fig:colloid_transport}. 
    The bare-pore resistance $R_{0}$ is shown by the black thick line. Its deviation from simple Stokesian scaling of Eq. (\ref{eq:resistance}) (thin black line) is due to the excluded volume of the particle.
    }
    \label{fig:R_vs_d}
\end{figure}

Figure~\ref{fig:R_vs_d} compares how the total resistance, $R=R_{\text{int}}+R_{\text{ext}}$, varies with colloid size $d$ for a bare pore (thick black lines) and for polymer-filled pores with selected solvent (Figure~\ref{fig:R_vs_d}a) and polymer-colloid interaction (Figure~\ref{fig:R_vs_d}b) strengths (thin colored lines with symbols).
For small colloids, the bare-pore resistance follows well the $R_0 \sim D_0^{-1} \sim d$ dependence expected according to Eq.~(\ref{eq:resistance}).
Stronger dependence of $R_0$ on $d$ observed for larger colloids is due to decreasing effective pore length $L$ and increasing the effective pore radius $r_{\text{p}}$.
%We deliberately limited our analysis to colloid sizes $d \lesssim r_{p}^0$ as the resistance for even larger colloids ($d \rightarrow 2r_{\text{p}}^0$) would be appreciably enhanced by drag from the pore well, which is not considered in our theory.

Naturally, the polymer filling affects the transport of the smallest colloids only marginally, as their volume and net interaction strengths (within the considered $\chi_{\text{PC}}$ range) are too small to have any noticeable effect. 
A rich picture emerges for larger colloids, however, with non-monotonic dependencies of the pore resistance on colloid size and strong effects of $\chi_{\text{PC}}$ and $\chi_{\text{PS}}$. 


%%%%%%%%%%
\subsubsection{Impact of colloid diffusivity within the polymer brush on size-selective transport}
%%%%%%%%%%
%2.7.1
%%%%%%%%%%

The curve with $\chi_{\text{PS}}=0.5$ and $\chi_{\text{PC}} = -1.0$ in Figure~\ref{fig:R_vs_d}a corresponds to the condition of $\Delta F$ fairly vanishing across a wide colloid-size range (as demonstrated in the corresponding panel in Figure~S5) due to compensation of osmotic and surface contributions to the insertion free energy.
%\todo{[We may show the dependence somewhere in the Supporting Info, as it is otherwise difficult to appreciate quite how close to zero the insertion free energy is?]}.
Here, the reduced colloid diffusivity within the polymer meshwork dominates the pore resistance (Eq.~(\ref{eq:rho})). 
This effect alone leads to a monotonic and pronounced increase of $R$ with $d$ with smooth crossover between asymptotic dependencies $R\sim d$ at $d\ll \xi$ and $R\sim d^3$ at $d\gg \xi$. 
Interestingly, for small and intermediate size particles the pore resistance slightly grows with inferior solvent quality (increase in $\chi_{PS}$) due to a decrease in the mesh size $\xi$ with concomitant decrease in the local diffusivity, as seen in Figure~\ref{fig:R_vs_d}b.
  


%Generally, $R(\Delta F=0) \sim D^{-\alpha} \sim d^{\alpha}$, with $\alpha = 1$ for very small colloids ($d \ll \xi$) and $\alpha = 3$ for large colloids ($d \gg \xi$) (Eq.~(\ref{eq:Rubinstein})).
%Our specific case (Figure~\ref{fig:R_vs_d}) shows that $\alpha \gtrsim 2$ for $d \gtrsim  1$ and thus illustrates that a strong size selectivity is already attained for rather small colloids.
%\todo{[I was left wondering how the data for $d < 1$ were obtained in practice. Would this merit some explanation?]}


%%%%%%%%%%
\subsubsection{Impact of insertion free energy on size-selective transport}
%%%%%%%%%%
%2.7.2
%%%%%%%%%%

Since the pore resistance scales exponentially with the insertion free energy ($R \sim D^{-1}\exp (\Delta F)$; Eq.~(\ref{eq:rho})), and $\Delta F =\Delta F_{\text{osm}} + \Delta F_{\text{sur}}$, the dependence of the resistance on colloid size is generally controlled by the interplay between
the osmotic $\Delta F_{\text{osm}} \sim \Pi d^3$ and the interfacial $\Delta F_{\text{sur}} \sim \gamma d^2$ contributions. 
While the osmotic repulsion arising due to the polymer filling always enhances the resistance, the surface contribution may either increase (at $\gamma > 0$) or decrease (at $\gamma<0$) it.

For inert or weakly attractive colloids, $\gamma \geq 0$, the resistance grows monotonically with the colloid size due to the combined effect of a decreasing diffusivity $D(d)$ and an increasing insertion free energy  $\Delta F(d)$.
As both these effects are pronounced, their combination leads to a very strong size selectivity, such that the transport of even rather small colloids is effectively impeded, as can be appreciated for $\chi_{\text{PC}} > -1.0$ in Figure~\ref{fig:R_vs_d}a.
For sufficiently large colloids, the osmotic contribution dominates in the insertion free energy such that $R \sim D^{-1} \exp (\Delta F) \sim d^3 \exp (\Pi d^3)$.

In contrast, for attractive colloids with $\gamma <0$ the dependence of the pore resistance $R(d)$ on colloid size can be non-monotonic, with a local maximum at $d=d_{\text{max}}$ followed by a local minimum at $d=d_{\text{min}}$.
This is best illustrated in Figure~\ref{fig:R_vs_d}a,b by the orange curves corresponding to $\chi_{\text{PC}} = -1.3, \chi_{PS}=0.5$.
The local maximum here arises from the net colloid attraction (which decreases resistance) overcoming the decrease in diffusivity (which increases resistance) with increasing colloid size.
The local minimum in turn arises from the positive osmotic contribution to the free energy ($\Delta F_{\text{osm}} \sim \Pi d^3$) overcoming the negative interfacial contribution ($\Delta F_{\text{sur}} \sim \gamma d^2$).

Most notably, the local minimum for attractive colloids is swiftly followed by a sharp increase in resistance for $d > d_{\text{min}}$ due to the dominant osmotic contribution recovering the strong $R \sim d^3 \exp(\Pi d^3)$ dependence.
The transition between the regime of good to moderate transport for $d \lesssim d_{\text{min}}$ and the regime of impeded transport for $d > d_{\text{min}}$ thus defines the condition for sharp gating of attracted colloids by their size.

When the insertion free energy becomes strongly negative, a new regime appears that is characterized by facilitated transport ($R < R_0$) over a rather wide range of colloid sizes, as illustrated in 
Figure~\ref{fig:R_vs_d}a for $\chi_{\text{PC}} = -1.4; - 1.8$ and Figure~\ref{fig:R_vs_d}b for $\chi_{\text{PS}} = 0.6;0.7$.
Here, the pore interior is effectively short-circuited, $R_{\text{int}} \to 0$, and the total resistance is set by the finite exterior contribution, $R \approx R_{\text{ext}}$ (see Eq.~(\ref{eq:R_tot_tot})).
Due to attractive brush fringes at the pore entrance and exit, $R \approx R_{\text{ext}}^{\text{min}}$, leading to a weak size dependence, $R \sim d$ (see Eq.~(\ref{eq:R_ext_min})).

Because the total resistance is bounded from below by $R_{\text{ext}}^{\text{min}}$, the parabolic section of the curve with a defined $d_{\text{min}}$ that would otherwise appear is truncated, and replaced by an almost linear segment.
As it is seen in Figure~\ref{fig:R_vs_d}a for $\chi_{PC}= -1.4$, the osmotic penalty to the insertion free energy takes over and entail a sharp increase in resistance above a certain colloid size $d$.
The sharp gating of colloids by their size is hence preserved for strongly attractive colloids.


%%%%%%%%%%
\subsection{Experiments of colloid transport through nuclear pore complexes validate the theoretical predictions}
%%%%%%%%%%

To test how well our theory predicts the experimental reality, we analysed literature data pertinent to colloid transport across nuclear pore complexes (NPCs).

The estimated distance between NPCs in the nuclear envelope is approximately 10 times larger than the pore diameter \cite{Yang2004, Daigle2001, Feldherr1984, Kubitscheck2000}). At such distances, transport across neighbouring NPCs is not mutually interfering \cite{Fabrikant1985}, as can be appreciated from the iso-concentration lines in Figure~\ref{fig:colloid_concentration}. Experimentally measured transport rates $k$, normalized against the number of pores, therefore can be directly compared with our theoretical predictions.
\todo{[Ralf to check and add references. Do we need to clarify that this applies for yeast as well as cells from higher organisms?]} 

It is well-known that colloids with affinity for the disordered nucleoporin FG domains that fill the NPC (such as importins and exportins) are enriched in or near NPCs \cite{Beck2007, Gruenwald2010, Tu2011}, and in microscopic droplets, macroscopic hydrogels and thin films assembled from pure FG domains.
Moreover, high concentrations of transport factors are essential for effective transport through the pore \cite{Lowe2015}.
These observations fully align with our predictions that the accumulation of colloids in the pore is required for facilitated transport (Figure~\ref{fig:colloid_concentration}).
\todo{[Ralf to check and add references.]}


%%%%%%%%%%
\subsubsection{Transport of non-sticky colloids}
%%%%%%%%%%

Several studies \cite{Ribbeck2001, Mohr2009, Popken2015, Timney2016, Frey2018} have quantified the rates of diffusive transport across NPCs for non-sticky proteins.
Together, these studies cover two orders of magnitude in molecular mass, and five orders of magnitude in transport rate. 
Figure~\ref{fig:NPC_comparison}a compares these experimental results with the theoretical predictions of our model.
As the example pore geometry and polymer density in Figure~\ref{fig:colloid_transport} were modelled to represent a NPC (Supporting Method 1), we can directly compare our theoretical predictions with experimental data.
The effective statistical segment length of disordered polypeptide chains was taken to be $a$ = 0.76 nm \todo{[Ralf and/or Mikhail to add reference.]}.
The effective solvent strength in the NPC was estimated to be close to $\theta$-solvent ($\chi_{\text{PS}} = 0.5$), consistent with varying yet generally moderate levels of 'cohesiveness' observed for FG domains. \todo{[Ralf and/or Mikhail to add references.]}
We approximated the proteins as perfectly inert colloids ($\chi_{\text{PC}} = 0$).
To match the theoretical colloid volumes to protein molecular masses, we considered the effective density of the protein colloids to be bounded by the densities of aqueous solvent ($\rho_{\text{colloid}} \geq \text{1 g/cm}^3$) and pure polypeptide ($\rho_{\text{colloid}} \lesssim \text{1.4 g/cm}^3$).
This approximation reflects that an unknown (and possibly variable) amount of solvent contributes to the effective volume of the proteins during their transport across the NPC.
The only adjustable fitting parameter in our model was the prefactor $\beta$ in the scaling-based expression for the diffusion coefficient, Eq.~(\ref{eq:Rubinstein}).

Figure~\ref{fig:NPC_comparison}a demonstrates that the theory reproduces the experimental trends for the increase in pore resistance with colloid size very well.
The best fit was obtained with $\beta = 5.5$ and this value was hence fixed  throughout the paper.
The quality of the fit is quite remarkable given the large range of masses and transport rates covered, and also considering the relative simplicity of our theory.
Some scatter in the experimental data is though notable.
This may be due to some proteins not being strictly non-sticky but interacting weakly with FG domains.
Indeed, Frey et al. \cite{Frey2018} reported a three-fold enahnced transport rate of green fluorescent protein over mCherry despite both these proteins being considered inert and of similar molecular mass.
Moreover, whilst some studies had washed out cytosolic proteins in their assay (with HeLa cells), thus leaving behind intact nuclear pores filled with a plain FG domain brush but lacking most transport factors \cite{Ribbeck2001, Mohr2009, Frey2018}, others used intact yeast cells with all transport factors present \cite{Popken2015, Timney2016}.
The satisfactory fit across all datasets suggests that the crowding of the NPC with transport factors has at best a weak effect on the transport of non-sticky proteins across the NPC \todo{[There may be experimental work that has looked at this -- we (Ralf!!) ought to reference it.]}.  

\begin{figure}
    \centering
    %REMOVE \CENTERLINE BEFORE SUBMISSION
    \centerline{\includegraphics[width = 6in]{fig/validation.png}}
    \caption{
    Comparison of theoretical predictions with experimental findings for colloid transport rates through NPCs.
    \textbf{(a)} 
    Gating of non-sticky colloids (peptides and globular proteins) by size.
    Nuclear import rate $k$ per one pore (at $c_0$ = 1 $\mu\text{M}$) vs. molecular mass $M_w$ (symbols) extracted from the literature (\cite{Ribbeck2001, Mohr2009, Popken2015, Timney2016, Frey2018}, as indicated).
    Theoretical predictions (shaded orange area) are for the pore and brush parameters as given in Figure~\ref{fig:colloid_transport}, with $\chi_{\text{PS}} = 0.6$, $\chi_{\text{PC}} = 0$, $a = 0.76$ nm, and colloid masses converted to volumes using $\text{1 g/cm}^3 \leq \rho_{\text{colloid}} \leq \text{1.4 g/cm}^3$.
    The best fit shown here was obtained with $\beta = 5.5$ in Eq. (\ref{eq:Rubinstein}).
    The predicted transport rates across a bare pore are also shown by the bold black line.
    \textbf{(b)} 
    Gating of colloids by their affinity to polymer.
    Nuclear import rate vs. partition coefficient $\text{P} \equiv \left(c_{\text{in}}/c_{\text{out}}\right)_{\text{gel}}$ into phase-separated droplets of pure FG domains (Mac98A - blue lozenges, Nup116 - gray lozenges) measured by Frey et al. \cite{Frey2018} for a range of green fluorescent protein variants.
    Theoretical predictions (dark orange lines) represent the limits of a homogeneously attractive colloid surface (dashed line) and a single sticky patch (solid line).
    The predicted transport rate across a bare pore is also shown for comparison by the bold black line.
    The pore and brush parameters are as given in (a); $\chi_\text{PC}$ values (indicated at the top of the graph)  were matched to to the partition coefficient $\text{P}$.
    }
    \label{fig:NPC_comparison}
\end{figure}


%%%%%%%%%%
\subsubsection{Transport of sticky colloids}
%%%%%%%%%%


Frey et al. \cite{Frey2018} additionally quantified NPC transport rates for a wide range of green fluorescent proteins (GFPs) with surface amino acids mutated to modulate transport from 'superinert' to 'transport factor like'.
In parallel, the ability of these variants to enrich or deplete in phase-separated droplets of two pure FG domains (Mac98A and Nup116) was quantified.
The transport rate was observed to correlate strongly with the level of GFP enrichment in FG domain phases (Figure~\ref{fig:NPC_comparison}b).
This set of experiments enabled the effect of polymer-colloid interaction to be tested selectively as the colloid size and shape were effectively constant.

To reproduce theoretically the correlation between the experimentally measured NPC transport rates $k$ and the partition coefficient $P$ in pure FG domain phases we need to identify two interaction parameters $\chi_\text{PS}$ and $\chi_\text{PC}$. The solvent strength, $\chi_\text{PS}$, can be extracted from the experimentally measured volume fractions of the FG domains in their polymer-rich droplets formed upon spontaneous phase separation in aqueous solution (see Supplementary Methods 10).   Assuming vanishing osmotic pressure, the free energy of insertion $\Delta F = -\ln(P)$ is reduced to the surface contribution $\Delta F_s\left(\chi_\text{PS},\chi_\text{PC}\right)$ that provides the link between $\chi_\text{PC}$ and the partition coefficient. 
\todo{[What colloid size was assumed here? Can we use the density bounds as done for the non-sticky particles, for consistency?]}

Our idealised assumption of colloids being homogeneously interactive (dashed orange line in Figure~\ref{fig:NPC_comparison}b) reproduced the experimental data for non-sticky and weakly attractive colloids well without any adjustable parameter.
For more strongly attractive colloids however, this approach overestimated the experimentally observed transport rates.
Theoretical prediction assuming the opposite extreme of all surface free energy being concentrated into a single sticky patch (solid orange line in Figure~\ref{fig:NPC_comparison}b) reproduced the experimental data quite well, suggesting that the presence of localized sticky patches on the colloid surface and the FG domains slows down diffusion and transport.
Interestingly, this comparison suggests that protein transport across the NPC is not optimised for the highest rate.
\todo{[Ralf to check and add references.]}

Taken together, the quantitative agreement between our theory and a range of experimental data for nuclear pore transport with a very limited number of adjustable parameters provides strong validation for the validity of our theory.


%%%%%%%%%%
\section{DISCUSSION}
%%%%%%%%%%

We have shown how mesopores filled with polymer brushes can gate transport with exquisite selectivity with respect to polymer-colloid affinity and colloid size, even for colloids that are substantially smaller than the pore diameter.
A striking finding is that an attractive polymer brush can provide colloid transport rates comparable to, or even exceeding, the bare pore.

Our findings shed light on the basic mechanisms of selective nucleo-cytoplasmic transport. They also suggest a molecular design strategy for controlling selective permeability through artificial mesoporous membranes, with potential applications in fields such as targeted drug delivery, biosensing, and filtration systems.

\todo{[What follows has not been revised yet -- Ralf to do in the next round.]}

%%%%%%%%%%
\subsection{Towards technological applications of synthetic polymer-filled mesopores}
%%%%%%%%%%

By tuning parameters like colloid size, polymer-colloid affinity, and solvent quality, it is possible to modulate transport properties and achieve desired selectivity levels in synthetic membranes.
These insights can pave the way for designing nanoporous materials with enhanced selectivity tailored to specific functional requirements, thereby broadening the scope of applications in nanomedicine, biotechnology, and environmental engineering.

Mixtures of biological colloids such as folded proteins and other biomacromolecular complexes, as well as synthetic colloids such as nanoparticles, may be effectively separated, not only according to their size but also their surface (bio-)chemistry.
Importantly, the theoretical approach presented here and the integration schemes facilitate the rational design of pores with a geometry and polymer filling optimized for the desired separation task.

Individual pores, as we have considered here, are routinely deployed in current nanopore sensing technologies.
These technologies enable detection and characterization of individual macromolecules as they travel across the pore.
Our findings suggest polymer fillings as an attractive tool to optimize the performance of nanopore sensing.
Placing a suitable polymer filling upstream the pore's sensing region would enable pre-selection of target solutes from complex mixtures for a focused analysis by the pore.
Polymer fillings may also be placed in the very sensing region of the pore to enhance both selectivity and sensitivity.
The here-proposed approach is distinct from previous approaches, where responsive polymer coatings along the pore walls were used to open/close a polymer free channel on application of an external stimulus such as a change in temperature, ionic strength or pH. 

Individual pores will though typically be insufficient in applications that focus on separation with high throughput such as filtration systems.
This limitation can be overcome by multiplexing, e.g., with membranes featuring a large array of mesopores.
Our theoretical approach remains valid for such arrays as long as the distance between pores remains sufficiently large for the diffusion trajectories of adjacent pores not to substantially interfere.
Fortunately, this condition can be met with a relatively tight packing of pores, as can be appreciated from the iso-concentration lines in Figure~\ref{fig:colloid_concentration}.
In practice, a distance between pore centres approximately 10-fold greater than the pore diameter should entail minimal interference \cite{Fabrikant1985}.

\bigskip

\noindent{The main design concepts emerging from our theory are:}

\textbf{1.}
For a polymer-filled pore to function as a selective transport channel, high permeation selectivity must be coupled with low resistance to diffusive flux.
We refer to this combination as 'gating' behaviour, where a minor change in colloid size or polymer-colloid interaction strength can dramatically shift the permeation rate from facilitated transport to virtually complete blockage.
Both requirements can be achieved near the critical values $d_{\text{crit}}$ and/or $\chi_{\text{PC}}^{\text{crit}}$, which assure that the resistance of the brush-filled pore matches that of the bare pore, $R\simeq R_{0}$.
The gating effect is particularly pronounced for larger colloids: in our case, with the diameter of 10 polymer segment lengths or more.

\textbf{2.}
Pore resistance is highly sensitive to parameters that influence insertion free energy.
Strong dependence of the pore resistance on the parameters of the colloid originates from the exponential dependence of the local conductivity on the insertion free energy.
This results in very high selectivity of the polymer brush with respect to colloid size and polymer affinity, as demonstrated in Figures \ref{fig:R_vs_chi_PC} and \ref{fig:R_vs_d}.
The osmotic contribution to the insertion free energy scales as $d^3$ while the interfacial contribution comprises $\chi_{\text{PC}}$ and scales as $d^2$.
Thus, under conditions when colloid transport is limited by the polymer brush (as opposed to plain solvent), a slight change in $d$ and/or $\chi_{\text{PC}}$ translates into a drastic change in permeability (resistance).

\textbf{3.}
The maximal permeability of the polymer-filled pore is limited by the resistance of the exterior region.
Whilst strong  polymer-colloid attraction can make the resistance of the pore interior effectively vanish ($R_{\text{int}} \to 0$), this is not the case for the pore exterior, where mass transport in plain solvent always provides a non-vanishing resistance.
Under poor solvent conditions  polymers are typically confined to the pore interior, and the total resistance is bounded from below by the Rayleigh resistance  $R \geq R_{\text{ext}}^{0} = \frac{1}{2 D_0 r_{\text{p}}}$.
Polymer caps emerge outside the pore under good or $\theta$-solvent conditions decreasing the path through plain solvent, and thereby can reduce the total resistance even further. Approximating the caps as hemispheres with radius $r_{\text{ext}}$, and assuming them highly attractive, leads to $R_{\text{ext}} \to \frac{1}{ \pi D_0 r_{\text{ext}}}$.
The maximum additional reduction in total resistance due to the presence of attractive polymer caps thus scales as $\frac{R_{\text{ext}}^{0}}{R_{\text{ext}}} \to \frac{\pi r_{\text{ext}}}{2 r_{\text{p}}}$.
Hence, a large polymer cap is beneficial for transport rates, although even a moderate cap size can lead to substantial gain (e.g., approximately 3-fold for $r_{\text{ext}}\simeq L = 2r_{\text{p}}$ as suggested by Fig.\ref{fig:phi_hm_grid}).

\bigskip

\noindent{The manufacturing of functional mesoporous membranes is an emerging art, and we hope that our theoretical efforts will both promote and guide future practical developments in this area.}
%COMMENT RR: We ought to provide some references on the manufacturing of mesoporous membranes.
%COMMENT RR: One can expect that transport rates will increase further with a pressure gradient that drives solution flow across the membrane. We could mention this here as an avenue worthy exploring in future work? 


%%%%%%%%%%
\subsection{Implications for nuclear pore permselectivty}
%%%%%%%%%%

A remarkable number of features that we have identified with our theory is also found in the transport of proteins and other biomacromolecules through nuclear pore complexes (NPCs), suggesting that our model is capable of capturing the basic mechanisms of nuclear pore permselectivity in spite of some rather simple assumptions.

The estimated distance between NPCs in the nuclear envelope is approximately 10 times larger than the pore diameter \cite{Yang2004, Daigle2001, Feldherr1984, Kubitscheck2000}, and transport across neighbouring NPCs therefore can be considered mutually non-interfering. 

Single-cargo tracking studies using fluorescence \cite{Musser2016, Lowe2010, Lowe2015, Yang2004, Kubitscheck2000, Ma2010} and tomography \cite{Beck2007} have shown that transported colloids (e.g., importins, exportins and their complexes with cargo) primarily traverse the central region of the NPC and are rarely observed near the pore walls.
Such a behaviour is consistent with the  insertion free energy lanscape in our model (Figure~\ref{fig:D_fe_map}c, bottom).
Importantly, such a path does not require the presence of a bare (i.e., polymer free) channel as had been suggested in some earlier models of NPC transport. Instead, subtle variations in polymer density across the pore's cross-section substantially determine where colloids enrich and translocate.  
%Need to add further references.

On the other hand, these studies also evidenced that transport attempts are frequently aborted, with the transported colloid either dwelling near the pore entrance for a sustained time period or partially traversing into the pore before returning.
This behaviour aligns with the predictions of our model that a negative insertion free energy at the periphery of the pore draws colloids into the pore, but the existence of a free energy barrier in the center of the pore (see Figure~\ref{fig:D_fe_map}c, top) would naturally lead to a large number of unsuccessful translocation attempts.

It is also well-known that colloids with affinity for the disordered nucleoporin FG domains that fill the NPC (such as importins and exportins) are enriched in or near NPCs \cite{Beck2007, Gruenwald2010, Tu2011}, and in microscopic droplets, macroscopic hydrogels and thin films assembled from pure FG domains.
Moreover, high concentrations of transport factors are essential for effective transport through the pore \cite{Lowe2015}.
These observations fully align with our predictions that the accumulation of colloids in the pore is required for facilitated transport (Figure~\ref{fig:colloid_concentration}).
%Need to check references.

Solvent strength conditions for nucleoporins can be estimated to be close to $\theta$-solvent, as attested by a certain level of "cohesiveness" observed for FG domains. 
The effective statistical segment length of disordered polypeptide chains is $a$ = 0.8 nm. In the theory, the pore radius and length, as well as the nanoparticle diameter are normalized by it. The pore parameters indicated in Figure~\ref{fig:colloid_transport} were actually taken to represent a NPC.
%An effective size limit of approximately 5 nm has been reported for the passive permeation of 'inert' colloids (i.e., colloids that do not bind or bind only weakly to the FG domains).
%Considering the regime of weak polymer attraction ($\chi_{\text{PC}} > -0.5$), one can see (Figure~\ref{fig:colloid_transport}a) that this value matches the prediction of our model remarkably well.
%Need to add further references.

Mohr et al. quantified the rates of diffusive transport across NPCs for 'inert' colloids of varying size.
Cytosolic proteins had been washed out in the experiments, and nuclear pores were filled with a plain FG domain brush without transport factors, closely corresponding to the assumptions of our simplified model. Figure~\ref{fig:NPC_comparison}a compares these experimental results with the theoretical predictions of our model. We indicate the theoretical curves for perfectly inert colloids with $\chi_{\text{PCS}} = 0$ and for weakly attractive colloids with $\chi_{\text{PCS}} = -0.5$. It is clear that the theory reproduces the trend of an increase in pore resistance with colloid size remarkably well. The single fitting parameter is the prefactor $\beta$ in the scaling-based expression for the diffusion coefficient, see Eq.~(\ref{eq:Rubinstein}). The best fit value $\beta=5.5$ is used throughout the paper, in particular to describe the experimental data in the next Figure~\ref{fig:NPC_comparison}a.



% \begin{figure}
%     \centering
%     \includegraphics[width = 0.5\linewidth]{fig/experimental.png}
%     \caption{
%     Comparison of theoretical predictions with experimental findings for NPCs.
%     \textbf{(a)} Gating of colloids by size.
%     Pore resistances vs. hyrodynamic diameter (symbols) were estimated from import rates into the nucleus of permeabilised HeLa cells measured by Mohr et al. for inert colloids of varying sizes.
%     Theoretical predictions (lines) are for the pore and brush parameters as given in Figure~\ref{fig:colloid_transport}, $\chi_{\text{PS}} = 0.5$ and $\chi_{\text{PCS}} = 0$.
%     \textbf{(b)} Gating of colloids by affinity.
%     Pore resistances vs. insertion free energy (symbols) were estimated from Frey et al. for a range of green fluorescent proteins with surface amino acids mutated to modulate transport from 'superinert' to 'transport factor like'.
%     Pore resistances were obtained from import rates into the nucleus of permeabilised HeLa cells; insertion free energies were obtained from partition coefficients in phase-separated droplets of Mac98A FG domains.
%     Theoretical predictions (lines) are for the pore and brush parameters as given in Figure~\ref{fig:colloid_transport}, $\chi_{\text{PS}} = 0.5$ and $d = 6$ (i.e. equivalent to the hydrodynamic diameter of GFP).    
%     }
%     \label{fig:NPC_comparison}
% \end{figure}

Finally, the NPC features a remarkable rate of facilitated permeation and an exquisite permselectivity with respect to the surface features of relevant proteins (e.g., importins and exportins).
Using an approach similar to Mohr et al., Frey et al. quantified NPC transport rates for a wide range of green fluorescent proteins (GFPs) with surface amino acids mutated to modulate transport from 'superinert' to 'transport factor like'.
In parallel, the ability of these variants to enrich or deplete in phase-separated droplets or hydrogels of pure FG domains was quantified (Figure~\ref{fig:NPC_comparison}b).
This set of experiments enabled the effect of colloid affinity to be tested selectively as the colloid size and shape were effectively constant.
The transport rate was observed to correlate strongly with the level of GFP enrichment in FG domain phases.
Our theory reproduces the salient features of this experimental system.
First, the experimental data provide direct evidence that transport-factor-like proteins can indeed be transported at a rate exceeding the rate of a bare pore.
Second, the experimental data qualitatively demonstrate the expected affinity gating, with the most attractive GFP variants experiencing pore resistances several orders of magnitude smaller then the most inert variants.
Third, transport rates are approaching a plateau for the most attractive GFP variants tested, suggesting that in this regime transport is limited by the diffusion to the pore rather than the pore itself (see Figure~\ref{fig:R_vs_chi_PC}a).
%Need to add references.

Neither the colloids nor the polymers pertinent to the NPC are as regular as assumed in our model.
Importins, exportins and their cargo have complex, non-spherical shapes and display substantial surface heterogeneity.
Similarly, each FG domain type exhibits substantial heterogeneity along the chain contour.
We present two theoretical curves according to two limiting assumptions: homogeneous surface vs. a single binding spot.

Moreover, the NPC features a variety of nucleoporin FG domains, with the body of available structural and biochemical data suggesting that the cohesiveness of nucleoporin FG domains is highest in the centre and decreases towards the periphery of the pore.
Qualitatively, one can envisage that the increased solubility of peripheral FG domains promotes a more extended polymer cap, thus minimising total pore resistance and maximising transport rates for strongly attractive colloids.
The reduced solubility of the central FG domains, on the other hand, would minimize the size threshold for gating of non-adhesive colloids.
Our model can be further extended to incorporate  solubility gradients and to explore such phenomena in more detail.
%Need to add references.

Overall, the agreement between the many experimental observations and the predictions of our theory is striking and strongly suggests that it provides an appropriate description of the basic mechanism of nuclear pore permselectivity.
The agreements are indeed remarkable given that the nuclear pore complex exhibits a much higher chemical complexity than our model.

%COMMENT RR: I have below left some further info about NPCs that Mikhail had gathered. I leave these for further consideration, though I am not sure how useful they are to the discussion. 

% In the ref\cite{MoussaviBaygi2016}, authors proposed that in the transport event locally collapses upon interacting with the NTR-bearing macromolecule, but autonomously reconstructs itself very fast, keeping the pore sealed.
% Ref \cite{Hough2015} also proposed that FG-motives create highly dynamic phase that can extremely quick exchange contacts with transport factors of cargo.
% Ref \cite{Milles2015} anticipates that fast transport requires rapid exchange when engaging FG-motives with the NTR
% Ref \cite{Goodrich2018} also proposed binding-mediated mechanism that changes local structure, destroying local cages.

%%%%
\printbibliography
\end{document}




































%%%%%%%%%%%%%%%%%%%%%%%%%%%%%%%%%%%%%%%%%%%%%%%%%%%%%%%%%%%%%%%%%%%%%%%%%%%%%%%%%%%%%%%%%%%%%%%%%%%%%%%%%%
%RR: In the following, I have left some pieces of text that could be inserted into the main text where desired
%%%%%%%%%%%%%%%%%%%%%%%%%%%%%%%%%%%%%%%%%%%%%%%%%%%%%%%%%%%%%%%%%%%%%%%%%%%%%%%%%%%%%%%%%%%%%%%%%%%%%%%%%%

%Under good (or \theta-) solvent conditions we may consider separately the situations with positive and negative insertion free energies. 
%Negative insertion free energies are rather exceptional under good solvent conditions. As we see below, in this case $R_{caps}\leq R_{convergent}$ and the total resistance
%is lower than that of the empty pore.
%Positive insertion free energies under good solvent conditions are more common. 
%In this case, the resistance of the pore interior is always dominant, 
%$$
%R_{tot}\approx R_{\text{int}}
%$$
%and the accuracy in estimating the resistance contributions from the entrance/exit regions is not of a major concern. 

%In Figure \ref{fig:fe_scf_grid} the insertion free energy profiles $\Delta F(z,r=0)$ calculated by analytical scheme and by SF-SCF method 
%are presented as a function of position of a spherical particle along the pore axis.
%While the SF-SCF method provides the net free energy, the analytical scheme allows decoupling of the free energy into osmotic and surface contributions, 
%which are shown separately in Figure \ref{fig:fe_scf_grid}.
%The numerical coefficients $b_0$ and $b_1$ in eq \ref{} are chosen by the best fit, but appear to be fairly universal and independent of the particle size 
%and interaction parameters $\chi_{PS,PC}$.
%Remarkably, the fit fails when the size $d$ became comparable with the pore diameter or in the case of extreme $\chi_{ads}$ values 
%when analytical scheme is not applicable because of strong perturbation 
%of the brush structure by inserted particle, while SF-SCF method can still be safely used
%for the evaluation of the insertion free energy.

%The 2D insertion free energy $\Delta F(r,z)$ patterns have rather complex shape. However, we can trace their evolution upon changing interaction parameters
%looking at the position-dependent free energy of the particle on the pore axis, $\Delta F(z,r=0)$.
%As one can see from Figure \ref{fig:fe_scf_grid}, the insertion free energy profiles evolve upon changing the interaction parameters $\chi_{PS,PC}$ as follows:
%At $\chi_{ads}\geq \chi_{crit}$ which is the case under good or theta-solvent conditions and weak or absent polymer-particle attraction, $|\chi_{\text{PC}}|\leq 1$, the positive osmotic
%term, $\Delta F_{osm}\geq 0$ dominates in the insertion free energy, which is positive and reach maximal value in the pore center, where polymer concentration is maximal.
%Hence, polymer-particle interaction has overall repulsive character and $\Delta F(r,z)$ has the shape of the free energy barrier preventing penetration and accumulation of particles in the pore.
%By using the insertion free energy $\Delta F(r,z)$ one can calculate the equilibrium partition coefficient 
%$$
%P=\int_{0}^{r_{pore}}2\pi rdr\int_{0}^{L_{0}}dz\exp (-\Delta F(r,z)/k_BT)/\pi r^{2}_{pore}L_{0}
%$$
%is larger than unity, $P\geq 1$. Noticably the repulsive free energy profiles extends beyond the edges of the pore, because of the fringes in the polymer density distribution in swollen brush.

%A decrease in $\chi_{ads}$ triggered by a decrease in  $\chi_{\text{PC}}$ or/and an increase in $\chi_{\text{PS}}$ leads to qualitative changes in the insertion free energy 
%$\Delta F(r,z)$ patterns: At $\chi_{ads}\leq \chi_{crit}$ the particle surface becomes
%adsorbing for the polymer, $\gamma \leq 0$, that gives rise to a negative contribution $\Delta F_{\text{sur}}(r,z)$ to the insertion free energy. 
%When $\chi_{\text{PS}}$ increases (the solvent is getting worse for the polymer)
%the osmotic pressure inside the brush decreases that leads to a decrease in the 
%magnitude of $\Delta F_{osm}(r,z)$ with the concomitant shrinkage of the  protruding outside the pore parts of the brush where  $\Delta F(r,z)\neq 0$.
%As a result, the $\Delta F_{\text{sur}}(r,z)$ aquires two minima with negative values near the endtance and the exit of the pore, separated by a maximum centered in the middle of the pore
%where polymer concentration is larger and the osmotic repulsive term  $\Delta F_{osm}(r,z)$ dominates.
%Finally, at strong polymer-particle attraction $\chi_{ads} < \chi_{crit}$, the negative surface contribution $\Delta F_{\text{sur}}(r,z)\leq 0$ overperform osmotic repulsion everywhere inside the pore
%and the $\Delta F(r,z)$ aquires the shape of the potential well centered in the middle of the pore, which gives rise to preferential accumulation of particles in the pore, $P\geq 1$.
