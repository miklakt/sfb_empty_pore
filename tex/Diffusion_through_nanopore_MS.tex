\documentclass[12pt, a4paper]{article}
\usepackage{graphicx}
\usepackage{amsmath, amssymb, amsfonts, mathtools}
\usepackage{subcaption}
\usepackage[
backend=biber,
natbib=true,
style=numeric,
sorting=none,
doi=false,
isbn=false,
url=false,
eprint=false
]{biblatex}
\usepackage{xcolor}
\usepackage{bm}

\newcommand\todo[1]{\textcolor{red}{#1}}

\addbibresource{biblio.bib}
\title{A polymer filling enhances the rate and selectivity of colloid permeation across mesopores}

\author{Mikhail Y. Laktionov$^1$, Leonid I.Klushin$^{2,5}$,\\Ralf P.Richter$^3$, France A.M. Leermakers$^4$, Oleg V.Borisov$^1$\\
$^{1}$CNRS, Universit\'e de Pau et des Pays de l'Adour UMR 5254,\\
Institut des Sciences Analytiques et de Physico-Chimie\\
pour l'Environnement et les Mat\'eriaux, 64053 Pau, France \\
$^{2}$Institute of Macromolecular Compounds \\
of the Russian Academy of Sciences, \\
199004 St.Petersburg, Russia,\\
$^{3}$University of Leeds, School of Biomedical Sciences, \\
Faculty of Biological Sciences, 
School of Physics and Astronomy, \\
Faculty of Engineering and Physical Sciences,\\  
Astbury Centre for Structural Molecular Biology,\\ 
and Bragg Center for Materials Research,\\ 
Leeds, LS2 9JT, United Kingdom\\
$^{4}$ University of Wageningen, The Netherlands\\
$^{5}$ American University of Beirut, Department of Physics, Lebanon
}
\date{}

\begin{document}
\maketitle

\begin{abstract}

Mesoporous membranes are emerging as new materials with potential applications in sensing and separation devices.
The nuclear envelope of eukaryotic cells provides a striking example of a functional mesoporous membrane, where diffusive transport is mediated by nuclear pore complexes (NPCs).
Transport across NPCs is mediated by a pore-filling meshwork of polymers anchored to the pore walls, and highly selective.
Even colloids much smaller than the inner diameter of the NPC are effectively blocked from transport, but some laeger colloids with distinct surface features can readily permeate.
Simplistically, one may expect any polymer meshwork to slow down colloid movement as the steric constraints imposed by the polymer meshwork hinder permeation.
However, we demonstrate how a rationally designed polymer filling can instead increase the permeation rate, by an order of magnitude and more, compared to an empty pore.
Such enhanced permeation is achieved with a polymer phase that attracts the colloid and extends beyond the confines of the mesopore channel itself, thus maximising colloid capture for diffusive transport across the pore.
We further define how polymer-filled mesopores can be designed to effectively gate colloids according to their size and surface features. 
This combination of features renders mesopores promising as highly selective separation devices.
It also provides a full physical explanation for the basic mechanism of nuclear pore permselectivity.
\end{abstract}

%Comment by RR on the storyline: Can we contrast our results with other separation devices, and spell what our mesopores enable that was not possible before? This would help spelling out the significance of our findings.

%%%%%%%%%%
\section{INTRODUCTION}
%%%%%%%%%%

Polymer-modified mesoporous materials and membranes belong to a new class of functional nanostructured materials with great potential in a number of technologies. 

The interaction of (macro)molecules and nanocolloidal particles with porous media, as well as their transport through porous membranes, are important elements of many technological processes (chromatography, heterogeneous catalysis, micro- and ultrafiltration, protein separation and purification, etc.) and, therefore, have been the subject of intensive research for more than sixty years \cite{Watson1959, Rout2003, Huang2023, Uredat2024}.

Advances in macromolecular chemistry have made it possible to significantly improve functional properties of materials with mesopores (i.e., pores with a diameter between a few and many tens of nanometers) by modifying them with polymers of various chemical nature anchored to  the pore walls.
Thus, a fuzzy meshwork of solvated polymers is formed filling the entire pore volume or just the near-wall regions, depending on the molecular mass and conformational state of the polymer chains.
The interaction of this polymer meshwork with colloids, that is,nanoparticles or (macro)molecules in the solution phase, essentially determines the absorption and separation properties of the polymer-modified mesoporous materials and membranes.
These interactions can be attractive or repulsive, and controlled by a broad spectrum of external stimuli \cite{Jeong2002, Lee2010, Low2019}, 
such as temperature                     \cite{Stetsyshyn2020}, 
pH and/or ionic strength of the medium  \cite{Dai2008, Zhang2005}, 
ion valency and specificity             \cite{Zhulina1999, Robertson2021},
electric fields                         \cite{Lokuge2005}, 
solvent composition                     \cite{Halperin2011, Darabi2016}, 
or complex biological stimuli           \cite{Ikeda2010, Lu2003}.
This opens up a unique opportunity for highly selective and controlled uptake and transport of colloids through polymer-filled mesoscopic channels.

Past experimental and theoretical efforts have focused on systems where an external stimulus triggered the transient opening of a polymer free path, typically along the center of the pore, to gate colloid transport.
However, the polymer phase itself can potentially also provide high selectivity to colloids as a function of their size and attraction by the polymer.
We thus hypothesized that even mesopores that are filled by a polymer meshwork across their entire cross-section can effectively gate colloid transport.
If successful, this approach would enable more robust gating as it does not rely on careful tuning of the diameter of a polymer-free channel, and higher transport rates as the full pore cross-section can potentially participate in colloid transport.

Nature provides a case in point.
Nuclear pore complexes (NPCs) perforate the nuclear envelope of eukaryotic cells and control the bulk transport of proteins and nucleic acids between the nucleus and the cytoplasm.
This process enables the spatial separation of gene transcription (in the nucleus) and translation into proteins (in the cytosol), and thus is critical for the ordered course of gene expression.
Each NPC forms a cylindrical channel, measuring approximately 40-60 nm in diameter and 40-95 nm in length (depending on the species \cite{Yang1998, Beck2004, VonAppen2015, Alberts2015, Hayama2017, Holzer2018}).
The channel is filled with a meshwork of several 100 natively disordered protein domains rich in phenylalainine-glycine dipeptides (FG domains) that are anchored to the channel walls \cite{Holzer2018, Ori2013, Rout2000, Dickmanns2015}.
Collectively, the FG domain meshwork provides remarkable gating function: biocolloids of 5 nm (i.e., just a tenth of the pore diameter) or more in hydrodynamic diameter are effectively blocked, except for some dedicated transport factors (importins and exportins, alone and in complex with cargo) which bind to the FG domains and can undergo rapid permeation. 

%RR: The below section is rather long, and may be shortened? Also need to review reference list.

Several independent strands of evidence indicate that the mechanism of diffusive transport across NPCs is based on rather generic physical principles, whereas the exact chemical makeup of the polymers and colloids is secondary for function.
Firstly, despite significant variations in the molecular building blocks of the NPC and the transport factors across distant eukaryotic taxa, NPCs consistently fulfill the same functional role \cite{DeGrasse2009, Maimon2012, Ori2013, Hayama2017, Yaron2018, Holzer2018}.
Secondly, the NPC can gate diffusive colloid transport similarly well in both directions.
Whilst the native cell is capable of directed transport of cargo against a concentration gradient \cite{Rout2003, Tijana2017}, this function is not an intrinsic part of the NPC itself but relies on energy derived from GTP hydrolysis by soluble intracellular proteins \cite{Lowe2015, Yang2004} and can even be reversed through cell engineering without modifying the NPC structure \cite{Nachury1999, Sakiyama2016}.
Thirdly, the binding behaviour of transport factors to assemblies of purified FG domains could be reproduced by simple physical models that treat FG domains as regular flexible polymers and transport factors as spherical colloids with a homogeneous surface. This approach provided faithful predictions even though it ignored the detailed arrangement of interaction sites along FG domains and on the transport factor surface. 
Fourthly, recent work with a range of mutants of green fluorescent protein demonstrated that NPCs exhibit a wide and continuous spectrum of permeabilities as a function of colloid surface properties, and earlier studies with non-interacting colloids similarly evidenced a wide and continuous spectrum of permeabilities as a function of colloid size.
Additionally, certain native proteins with affinity to FG-domains, such as $\beta$-catenin, can translocate through the pore without the need for a transport factor, moving from the cytoplasm to the nucleoplasm along a concentration gradient due to their continuous binding to chromatin \cite{Rout2003}. This suggests that a fine balance of many individually weak physicochemical (e.g., electrostatic, hydrophobic, aromatic stacking, ...) interactions between polymers and biocolloids dictates the gating behaviour, rather than a few highly specific biochemical interactions.

%A similar structural motif was recently found in the internal channels of microtubules (about 15 nm in diameter) decorated with so-called microtubule intrinsic proteins (MEPs), presumably modifying microtubule stability and rigidity \cite{Mukhopadhyay2001}.

However, we currently lack an understanding of the relationship between the molecular architecture of the polymer brush filling the pore and its ability to transport colloids with high selectivity and rate. Here, we develop a theoretical approach to reveal the physical mechanisms of diffusive colloid transport across polymer-filled mesopores.
A meshwork of flexible polymers effectively increases the local viscosity and thereby slows down transport of colloids compared to an open pore. 
On the other hand, an attractive polymer phase recruits colloids into the pore, thus increasing colloid transport, and such recruitment is further enhanced when attractive polymers extend outside the pore lumen. 
Intriguingly, the solvent strength through its influence on the density and compactness of the polymer meshwork impacts all of these effects. 
Using a self-consistent field approach, we define how solvent quality and colloid attraction to the polymer may be tuned to maximize the transport rate (even beyond the rate for an open pore) and to achieve highly selective colloidal transport with respect to particle size or affinity for the polymer.


%%%%%%%%%%
\section{RESULTS}
%%%%%%%%%%


%%%%%%%%%%
\subsection{Defining the transport scenario}
%%%%%%%%%%

The salient features of our simulated mesopore are illustrated in Figure~\ref{fig:colloid_transport}.
The pore has a cylindrical shape with radius $r_{\text{p}}$.
It perforates an otherwise impermeable, planar membrane of thickness $L$, and thus is the sole conduit for colloids between two semi-infinite solution reservoirs.
Flexible polymer chains are end-grafted to the inner pore walls, at a density sufficient to form a polymer brush that fills the entire pore cross-section.

We will focus on a pore with a set radius and length, and polymers with a set length and grafting density (Figure~\ref{fig:colloid_transport}).
Whilst the selected values are inspired by the nuclear pore complex, we expect that our findings will be of rather general validity so they can be applied to the performance analysis and rational design of mesopores with other geometries or polymer fillings.

Our aim is to understand how the colloid size, and the affinity of the colloid for the polymer, define the transport of colloids across the pore.
Colloids are taken to be spherical in shape, with diameter $d$.
The interaction strength between a polymer segment and the surface of the colloid is represented by the Flory-Huggins parameter $\chi_{\text{PC}}$.

\begin{figure}
    \centering
    \includegraphics[width = 0.7\linewidth]{fig/pore_cartoon.png}
    \caption{
        Schematic illustration of colloid diffusive transport through a pore filled with a polymer brush through its entire cross-section. 
        The brush is formed by linear polymer chains (red strands) with a degree of polymerization $N$, uniformly grafted with grafting density $\sigma$ 
        to the inner surface of a cylindrical pore.
        The pore radius is $r_{\text{p}}$ and the thickness of the impermeable membrane is $L$.
        Polymer chains are flexible with a statistical segment length $a$ and volume $\sim a^3$. 
        Spherical colloids with diameter $d$ are free to diffuse in the surrounding solvent.
        All length scales are normalized by the statistical segment length $a$.
        As a model pore, we set $L = 2r_{\text{p}} = 56$, $\sigma = 0.02$ and $N = 300$.
        With $a = 0.8 {\text{ nm}}$, these parameters reproduce the basic features of nuclear pore complexes.
          }
    \label{fig:colloid_transport}
\end{figure}

To understand how the polymer brush affects the transport of colloids, we consider the stationary diffusive flux of colloids through the pore and analyze how it depends on the parameters of the pore, the brush, and the colloid.
We consider unidirectional transport of colloid particles driven solely by the concentration difference across the membrane and focus on the fundamental mechanisms of diffusion mediated by particle-polymer interactions.
The colloid concentration is set to zero and $\Delta c$ far away from the membrane (at $z\rightarrow\pm\infty$, respectively).
We assume axial (cylindrical) symmetry of the flow in the pore.
Together with the stationary conditions, this implies that parameters relevant to colloid transport depend on the axial coordinate $z$ and the radial coordinate $r$, but not on the azimuthal angle.


%%%%%%%%%%
\subsection{Colloid transport is defined by the sum of resistances of regions outside and inside the pore}
%%%%%%%%%%

%%%%%%%%%%
\subsubsection{Empty pore as a reference case}
%%%%%%%%%%

A natural reference is the diffusive flux through a bare pore, which itself limits the transport of solutes \cite{Deen1987, Sun2024}.
The first approach to this problem dates back to Lord Rayleigh, who analyzed the flux of point-like solute particles through a circular pore in a planar membrane of negligible thickness \cite{Strutt1878}.
In this case, the equiconcentration surfaces are oblate spheroids, and the streamlines form confocal hyperboloids of revolution \cite{Cooke1966}.
The net flux through the pore is given by
\begin{equation}
    J=2D_0r_{\text{p}}\Delta c,
    \label{eq:flux_Ral}
\end{equation}
where $D_0$ is the diffusion coefficient of the colloid in plain solvent.

A membrane of finite thickness $L$ allows an approximate analytical solution (with an error of less than 6\% in the full range of the $\frac{L}{r_{\text{p}}}$ ratio) \cite{Brunn1984}:
\begin{equation}
    J=\frac{2 D_0 r_{\text{p}}}{1+\cfrac{2L}{\pi r_{\text{p}}}}\Delta c.
    \label{eq:flux_finlength}
\end{equation}

Introducing the resistance $R$ to colloid flow according to $J = \frac{\Delta c}{R}$ provides a natural interpretation of Eq.~(\ref{eq:flux_finlength}) in terms of the total resistance of the pore:
\begin{equation}
    R = \frac{L}{D_0 \pi r_{\text{p}}^{2}} + \frac{1}{2 D_0 r_{\text{p}}} = R_{\text{int}}^{0} + R_{\text{ext}}^{0},
    \label{eq:resistance}
\end{equation}
where the superscript '0' refers to the bare pore.
The first term in Eq.~(\ref{eq:resistance}) represents the resistance of the interior of the empty pore.
The second term corresponds to the Rayleigh resistance for a pore of infinitesimal thickness (Eq.~(\ref{eq:flux_Ral})), which accounts for the effects of convergent flow toward the pore entrance (in the region exterior to the pore) and its symmetric counterpart at the pore exit.
Inside the cylindrical pore, the flow lines are approximately axial.
In the empty pore scenario, the inverse of the diffusion constant ($\rho_0=D_0^{-1}$) represents the resistivity of the medium both inside and outside the pore.
Naturally,  the resistance of a pore in a thin membrane $L \ll r_{\text{p}}$ is determined by the resistance of the exterior region, $R \approx R_{\text{ext}}^{0}$.
In contrast, for long pores ($L \gg r_{\text{p}}$) the resistance of the inner region becomes dominant, such that $R \approx R_{\text{int}}^{0}$.

The finite size of colloids affects the diffusive flux in two ways.
First, the excluded volume of diffusing colloids is accounted for through the effective pore radius $r_{\text{p}} - d/2\rightarrow r_{\text{p}} $ and pore length $L + d \rightarrow L$ \cite{Renkin1954, Beck1970, Bungay1973, Anderson1974, Brenner1977}.
Second, the presence of the pore walls entails some additional drag \cite{Ladenburg1907, Faxen1922, Haberman1958}.
This effect is neglected here as the presence of the polymer brush screens hydrodynamics and thus requires a different kind of drag analysis, as discussed below.


%%%%%%%%%%
\subsubsection{A polymer filling affects the resistance of the pore itself, and also of regions outside the pore}
%%%%%%%%%%

Conformations adopted by overlapping polymer chains grafted to the pore walls are controlled by strong intermolecular interactions that depend on the solvent quality.
The solvent quality is here quantified by the Flory-Huggins parameter $\chi_{\text{PS}}$.
Values of $\chi_{\text{PS}}<0.5$ and $\chi_{\text{PS}}>0.5$ correspond to good and poor solvent, respectively, whereas $\chi_{\text{PS}}=0.5$ represents the ideal (or $\theta$-)solvent.

The polymer density profile $\phi(z,r)$ in the pore was calculated by the two-gradient self-consistent field numerical method of Scheutens and Fleer (SF-SCF; see Supporting Information, Section 2).
In Figure \ref{fig:phi_hm_grid}, one can appreciate the expected increase in polymer concentration with decreasing solvent quality (increasing $\chi_{\text{PS}}$).
With the selected pore and polymer parameters (Figure~\ref{fig:colloid_transport}), the polymer brush fills the entire pore cross-section within the full range of solvent qualities explored ($0.1\le\chi_{\text{PS}}\le1.1$), so that colloids need to navigate the polymer meshwork to move across the membrane.
For wider pores, shorter polymers and/or lower grafting densities, an open channel free of polymer may appear in the pore center, as discussed in detail in Ref.~\cite{Laktionov2021}.
This scenario would result in a distinct permeation behavior, as colloids could move through the pore without traversing the polymer brush.
This case is not considered here.

\begin{figure}
    \centering
    \includegraphics[width = 0.7\linewidth]{fig/phi_hm_grid.png}
    \caption{
    Maps of the polymer volume fraction $\phi(r,z)$ for a polymer brush in a cylindrical pore under solvent quality conditions ranging from good (upper left panel) to poor solvent (lower right panel) as quantified by the Flory-Huggins parameter $\chi_{\text{PS}}$.
    Polymer volume fractions are mapped in cylindrical coordinates (as shown by $rz$-coordinate arrows), color coded according to the legend below with selected iso-concentration lines. The blank space corresponds to pure solvent, the membrane is colored green.
    Pore and brush parameters are as given in Figure~\ref{fig:colloid_transport}.
    }
    \label{fig:phi_hm_grid}
\end{figure}

Figure \ref{fig:phi_hm_grid} further illustrates that whilst the brush remains confined within the pore lumen in poor solvent ($\chi_{\text{PS}}=0.9$ and $\chi_{\text{PS}}=1.1$) it protrudes substantially into the surrounding space in ideal and good solvent ($\chi_{\text{PS}}\le0.5$), thus forming fringes on either side of the pore.
The polymer brush therefore will impact on the resistance to the colloid flow within as well as outside the pore, such that
\begin{equation}
    R=R_{\text{int}}+R_{\text{ext}},
    \label{eq:R_tot_tot}
\end{equation}
with $R_{\text{int}}\rightarrow R_{\text{int}}^{0}$ and $R_{\text{ext}}\rightarrow R_{\text{ext}}^{0}$ in the limit of the empty pore.


%%%%%%%%%%
\subsection{Insertion free energy and diffusivity control diffusive transport}
%%%%%%%%%%

Zooming in on the local scale, we can analyze how colloids are accumulated or depleted due to attractive or repulsive interactions, respectively, with the polymer meshwork, and how the meshwork affects the colloid's local mobility.


%%%%%%%%%%
\subsubsection{Volume vs. surface contributions to the insertion free energy}
%%%%%%%%%%

The position-dependent insertion free energy $\Delta F(r,z)$ is the isothermal work required to move a colloid from the exterior solution into the polymer brush.
For colloids that are significantly smaller than the size of the pore, the insertion free energy is determined entirely by the local polymer concentration (i.e., wall effects can be neglected), and comprises two distinct contributions:
\begin{equation}
    \Delta F (r,z)= \Delta F_{\text{osm}}(r,z) + \Delta F_{\text{sur}}(r,z),
    \label{eq:Delta_F}
\end{equation}
$$
\Delta F_{\text{osm}}(r,z) = \int_{V} \Pi(r,z) dV,
$$
$$
\Delta F_{\text{sur}}(r,z) = \oint_{S} \gamma (r,z) dS.
$$
The coordinates $(r,z)$ refer to the center of the colloid, whilst the insertion free energy is obtained by integrating over the volume and surface of the colloid, respectively. All the free energy values are normalized by the thermal energy unit $k_{\text{B}}T$.

The osmotic contribution, $\Delta F_{\text{osm}}(r,z)$, is proportional to the colloid volume and accounts for the work against excess osmotic pressure upon insertion of the particle into the brush.
The local osmotic pressure is calculated from the local polymer concentration as
$$
\Pi(r,z)=  \phi(r,z)\frac{\partial f\{\phi(r,z)\}}{\partial \phi(r,z)} - f\{\phi(r,z)\}= 
$$
\begin{equation}
	k_{\text{B}}T[-\ln(1-\phi(r,z)) - \phi(r,z) -\chi_{\text{PS}}\phi^2(r,z)],
\end{equation}
where
$$
f\{\phi(r,z)\}=(1-\phi(r,z))\ln(1-\phi(r,z)) +\chi_{\text{PS}}\phi(r,z)(1-\phi(r,z))
$$
is the mean-field Flory expression for the interaction free energy per unit volume of the polymer solution of concentration (volume fraction) $\phi(r,z)$.
As long as the osmotic pressure inside the brush is positive, $\Delta F_{\text{osm}}$ is also positive and provides a dominant contribution for sufficiently large particles.

The interfacial contribution, $\Delta F_{\text{sur}}(r,z)$, is proportional to the colloid surface area, with the surface tension $\gamma (r,z)$ approximated as
\begin{equation}
    \gamma (r,z)= \frac{1}{6}(\chi_{\text{ads}} - \chi_{\text{ads}}^{\text{crit}})\phi^{\ast}(r,z),
    \label{eq:chi_ads} 
\end{equation}
$$
\chi_{\text{ads}} = \chi_{\text{PC}} - \chi_{\text{PS}}(1-\phi^{\ast}),
$$
$$
\phi^{\ast}(r,z)= (b_{0} + b_{1}\chi_{\text{PC}})\phi(r,z).
$$
Here $\gamma$ is the change in the free energy of a unit area upon replacement of a contact of the colloid with solvent by a contact with a polymer solution of concentration $\phi(r,z)$.
The coefficients $b_0$ and $b_1$ are parameters that account for depletion or accumulation of the polymer in the vicinity of the colloid surface (see Supporting Information, Sections 3-4, for details).

Depending on the relative strengths of polymer-colloid ($\chi_{\text{PC}}$) and polymer-solvent ($\chi_{\text{PS}}$) interactions, the sign of $\gamma$ can be either positive or negative.
If the particle surface is repulsive ($\chi_{\text{ads}} \geq 0$) or even weakly attractive for polymers ($\chi_{\text{ads}}^{\text{crit}} \leq \chi_{\text{ads}} < 0$), then, due to steric constraints imposed by the impermeable surface, the available conformations of the polymer are restricted, leading to polymer depletion near the particle surface and $\gamma > 0$.
At the critical adsorption condition $\chi_{\text{ads}} = \chi_{\text{ads}}^{\text{crit}}$, the losses in conformational entropy caused by the presence of the surface are exactly balanced by the free energy gain from monomer-surface contacts, causing $\gamma$ to vanish \cite{Fleer1993,Birshtein1979,Birshtein1983,Eisenriegler1982}.

In what follows, we applied an approximate analytical scheme to evaluate the insertion free energy $\Delta F(r,z)$ as $\Delta F\{\phi(r,z)\}$, where $\phi(r,z)$ is the polymer density distribution in a colloid-free brush calculated using the SF-SCF approach.
The colloid is thus considered as a 'probe' which does not perturb the global concentration distribution $\phi(r,z)$ in the brush.
With this scheme, we can evaluate the insertion free energy at any position of the colloid in the brush, including off the pore axis (Supporting Information, Section 5).
Comparison of the approximate analytical approach with direct SF-SCF calculations of the insertion free energy for colloids on the pore axis demonstrated good quantitative agreement (Supporting Information, Section 4), justifying the use of the more versatile analytical approach.

Figure~\ref{fig:DeltaF_map} illustrates how colloids may be either repelled or attracted by the polymer meshwork, depending on the balance of osmotic and interfacial contributions to $\Delta F(r,z)$.
Since the polymer concentration in the pore is strongly inhomogeneous, the net insertion free energy $\Delta F(r,z)$ may exhibit quite large spatial variations.
For example, the brush shown in Figure~\ref{fig:DeltaF_map} at $\chi_{\text{PC}}=-0.75$ simultaneously exhibits attraction ($\Delta F<0$) at the protruding fringes and repulsion ($\Delta F>0$) inside the pore.

\begin{figure}
    \centering
    \includegraphics[width = 0.7\linewidth]{fig/free_energy_hm.png}
    \caption{
    Maps of the particle insertion free energy $\Delta F(r,z)$ for a range of polymer-colloid interaction strengths.
    The polymer-colloid interaction strength is quantified by the Flory-Huggins parameter $\chi_{\text{PC}}$ ranging from -0.50 (least attractive) to -1.25 (most attractive), as indicated.
    Insertion free energies are displayed in cylindrical coordinates (as in Figure \ref{fig:phi_hm_grid}), and color coded as indicated.
    Pore and brush parameters are as given in Figure~\ref{fig:colloid_transport}, $\chi_{\text{PS}}=0.5$ and $d=8$.
    }
    \label{fig:DeltaF_map}
\end{figure}


%%%%%%%%%%
\subsubsection{Local colloid mobility is determined by the diameter of the colloid and the polymer mesh size}
%%%%%%%%%%

Crowded polymer chains naturally decrease both rotational \cite{Fu2017} and translational \cite{Stewart1998} diffusion of colloids.
Brush-forming chains are effectively described as an inhomogeneous semi-dilute solution with a concentration-dependent correlation length (mesh size) $\xi(\phi)$.
Colloids of size $d > \xi$ experience additional friction as they are trapped by the polymer meshwork.
As a result, diffusion is slowed compared to pure solvent, leading to a position-dependent diffusion coefficient ($D(r,z) < D_0$).

We follow the theory by Cai et al. \cite{Cai2011} to describe the diffusion coefficient of colloids as a function of their size relative to the correlation length $d / \xi$:
\begin{equation}
    D\{\phi(r,z)\} = \frac{D_0}{1+d^2/\xi^{2}\{\phi(r,z)\}}.
    \label{eq:Rubinstein}
\end{equation}
Colloids smaller than the mesh size diffuse virtually unimpeded ($D \approx D_0$ for $d\leq \xi$), whilst larger colloids are significantly slowed by the polymer medium compared to the pure solvent ($D\cong D_0 (\xi/d)^2\ll D_0$ for $d\gg \xi$).
In Eq.~(\ref{eq:Rubinstein}), the correlation length $\xi$ is controlled by the local polymer concentration $\phi(r,z)$.
The scaling relation between the correlation length and the polymer concentration depends on the solvent quality \cite{DeGennes1979}.
We approximate the dependence of the mesh size on polymer concentration by the power-law dependence valid close to $\theta$-solvent conditions in a mean-field regime, $\xi\cong \phi^{-1}$.

Several other theoretical and empirical models have been proposed to describe the diffusion of colloids in polymer meshworks \cite{Kohli2012,Holyst2009,Phillies1988}.
Although the predictions of different models differ quantitatively, they all share the same qualitative trend. We anticipate that our conclusions are quite insensitive to the specific choice of the model, as supported by the analysis presented in Section 9 of the Supporting Information.

%COMMENT RR: We need to make sure that all SI sections are mentioned in the main text, and in proper order. This is not currently the case.
%COMMENT RR: Do we need a figure here that shows an illustrative map of D/D0 - perhaps as a separate panel in Figure 3?


%%%%%%%%%%
\subsubsection{Linking local resistivity to global transport}
%%%%%%%%%%

%COMMENT RR: I found this section quite technical and difficult to follow. Can we spell out the main assumptions in simpler terms, understandable for readers not experienced with some of the math and concepts?

Having defined the local insertion free energy and mobility, we can develop an analytical method to estimate the total resistance of the brush-filled pore to the diffusive flow of colloidal particles.

Diffusive transport in the presence of an external force generated by the insertion free energy $\Delta F$ is described by the Smoluchowski equation.

\begin{equation}
    \frac{\partial c(\bold r)}{\partial t}=-{\bold \nabla}\cdot (D(\bold r)({\bold \nabla}c({\bold r})+c({\bold r}){\bold \nabla}(\Delta F({\bold  r}))),
    \label{eq:Smoluch}
\end{equation}
where $c(\bold r)$ is the solution concentration. The flux density 
\begin{equation}
    \bold j=- D(\bold r)({\bold \nabla}c({\bold r})+c({\bold r}){\bold \nabla}(\Delta F({\bold  r}))
    \label{eq:j}
\end{equation}
can be expressed in terms of the gradient of the potential function $\psi(\bold r)$
\begin{equation}
    \bold j=- D(\bold r) \exp(-\Delta F({\bold  r}))  {\bold \nabla} \psi(\bold r),
    \label{eq:psi}
\end{equation}
where
\begin{equation}
    \psi(\bold r)=c(\bold r)\exp(\Delta F(\bold r)).
    \label{eq:psi1}
\end{equation}
Under stationary conditions, the flux density is divergence free. In the case of a position-independent diffusion coefficient and vanishing insertion free energy the potential function is a solution of the Laplace equation $\nabla^2 \psi(\bold r)=0$.
In our case the boundary conditions for the potential function are $\psi(z\rightarrow -\infty)=\Delta c$ and $\psi(z\rightarrow +\infty)=0$ since the insertion free energy $\Delta F$ vanishes far away from the pore.
Hence, the problem of finding the total resistance of two semi-infinite solution reservoirs separated by the membrane with the brush-filled pore is equivalent to finding the resistance of the medium with position-dependent conductivity possessing axial symmetry:
\begin{equation}
    \rho^{-1} (r,z)= D(r,z)\exp(-\Delta F(r,z)).
    \label{eq:rho}
\end{equation}

%COMMENT RR: In the below sentence, can we calrify what we mean with 'equipotential'?

To this end, we consider a set of approximate equipotential surfaces $\psi(r,z)=\text{const}$ foliating the space available for colloid flow: inside the pore, the surfaces are discs of radius $r_{\text{p}}$ normal to the pore axis; outside the pore, we use oblate half-spheroids taken from the Rayleigh solution \cite{Strutt1878} (Figure~\ref{fig:resistivity_profile}a).
Analogously to a set of resistors connected in parallel, the total conductivity of any layer between two adjacent equipotential surfaces is obtained by integration of local conductivities over the layer. Within the pore, $|z|\leq L/2$,the result is given by
\begin{equation}
\varrho_{\text{int}}^{-1}(z)= 2\pi\int_{0}^{r_{\text{p}}^{}} \rho^{-1}(r,z) r \, dr
\label{eq:varrho1}
\end{equation}

In the exterior region $|z| >L/2$, the expression is modified to integrate over aforementioned half-spheroids:
\begin{equation}
    \begin{gathered}
        \varrho_{\text{ext}}^{-1}(z)= 2\pi\int_{0}^{r_{\text{p}}^{}} \rho^{-1}\left( r'(r,z), z'(r,z) \right)  \tilde{h} (r,z) dr\\
        r'(r,z) = r\sqrt{1 + \frac{(z - L/2)^2}{r_{\text{p}}^2}}\\
        r'(r,z) \in [0, \sqrt{r_{\text{p}}^2 + (z-L/2)^2}]\\
        z'(r,z) = (|z| - L/2) \frac{\sqrt{r_{\text{p}}^2 - r^2}}{r_{\text{p}}} +  \text{sign}(z) \frac{L}{2}\\
        \tilde{h} (r,z) = h_r h_{\theta} h_z^{-1} = \dfrac{r}{r_{\text{p}}}\dfrac{r_{\text{p}}^2 + (|z|-L/2)^2}{\sqrt{r_{\text{p}}^2 - r^2}}
    \end{gathered}
\label{eq:varrho2}
\end{equation}
where the equipotential surface defined parametrically with $r'(r,z) , z'(r,z)$; and $h_r$, $h_{\theta}$ and $h_z$ are the corresponding Lam\'e coefficients (see Supporting Information, Section 7). In the case of a homogeneous brush considered here, the function $\varrho_{\text{ext}}^{-1}(z)$ is even.%, which defines it for $z<-L/2$.

On the other hand, since the consecutive layers are connected in series, their total resistance can be found by appropriate integration:
\begin{equation}
    R_{\text{int}} = \int_{-L/2}^{+L/2}\varrho_{\text{int}}(z) dz,
    \label{R_int}
\end{equation}

\begin{equation}
   R_{\text{ext}} =2\int_{+L/2}^{+\infty}\varrho_{\text{ext}}(z)dz
    \label{R_ext}
\end{equation}

For a bare pore without a polymer brush, this method recovers Eq.~\ref{eq:resistance}, as expected.
For a brush entirely contained within the interior of the pore -- which is well justified under poor solvent conditions -- the total resistance is found as $R = R_{\text{int}} + R_{\text{ext}}^{0}$, as the exterior is not modified by the brush.
Conversely, a brush under good or $\theta$-solvent conditions produces swollen fringes (caps) outside the pore (Figure~\ref{fig:phi_hm_grid}), which modify the resistance $R_{\text{ext}}$ of the external regions.


%%%%%%%%%%
\subsection{An attractive polymer filling dramatically enhances colloid fluxes through the pore}
%%%%%%%%%%

As mentioned in the previous section, the resistance $\varrho(z)$ of a single layer (per unit thickness as measured along the pore axis), is obtained by inverting the integral of the conductivity.
In the brush-free region, $\varrho_{0}(z)$ is inversely proportional to the layer's surface area.
Figure~\ref{fig:resistivity_profile}b shows $\varrho(z)$ profiles for a few selected polymer-colloid interaction strengths $\chi_{\text{PC}}$.
For comparison, we also show the case of an empty pore, and the hypothetical case when the polymer brush affects only the particle mobility, but not the insertion free energy ($\Delta F(r,z) = 0$).
In all cases, the area under the curve represents the total resistance of the pore, as follows from Eqs.~(\ref{R_int},~\ref{R_ext}).

\begin{figure}
    \centering
    \includegraphics[width = 5cm]{fig/resistitance_components.png}
    \caption{
    %\textbf{(a)} 
    Special orthogonal curvilinear coordinate system aligned with the flux density $\bm{j}$ stream surfaces (radial coordinate) and level sets of the potential function $\psi$ (axial coordinate).
    Red lines correspond to constant values of the axial coordinate; gray lines are tangential to the flux density field and correspond to constant values of the radial coordinate.
    The lines define bodies of revolution along the $z$-axis; the angular coordinate is not shown.
    In the exterior of the pore, constant radial coordinates are confocal hyperboloids of revolution, constant axial coordinates are confocal oblate spheroids, and constant angular coordinates are half-planes.
    In the interior of the pore, the coordinate system is equivalent to the cylindrical coordinate system.
    % %COMMENT RR: I found the description of panel a quite complex. Can it be simplified, and rendered coherent with the main text?
    % \textbf{(b)} Resistance per unit length $\varrho(z)$ (normalized to the pore cross-section $\pi r_{\text{p}}^2$ and $D_{0}$) along the axial coordinate of the curvilinear coordinate system for selected polymer-colloid interaction strengths $\chi_{\text{PC}} \in \{ -0.9, -1.0, -1.1, -1.2, -1.3\}$, as indicated with colored lines.
    % The resistance per unit length of a plain pore without polymers $\varrho_{0}(z)$ (black thick line), a pore with only the diffusion coefficient modulated by the polymer brush ($\Delta F = 0$, dashed black line), and the location of the membrane (green background) are shown for comparison.
    % Pore and brush parameters are as given in Figure~\ref{fig:colloid_transport}, $\chi_{\text{PS}}=0.5$ and $d=8$.
    }
    \label{fig:resistivity_profile}
\end{figure}

Several features are notable. Firstly, an attractive pore interior can enhance the diffusive fluxes through the polymer-filled pore beyond the limit of the empty pore.
This can be appreciated for $\chi_{\text{PC}} = -1.2$ and $-1.3$, where the resistance per unit length within the membrane width ($-28 \leq z \leq 28$) is consistently lower than the resistance per unit length of the empty pore.
This result may at first appear surprising, given that the polymer medium is expected to slow down the diffusion of colloids.
However, this slowing down is counteracted by the attractive potential of the polymer meshwork, which reduces local resistivity according to the exponential factor in Eq~\ref{eq:rho}.
As the colloid attraction increases further $(\chi_{PC}\ll -1)$, the pore interior is effectively short-circuited.
Compared to an empty pore (Eq.~(\ref{eq:resistance})) this effect alone entails a reduction in the resistance by a factor of up to $R^0_{\text{int}}/R^0_{\text{ext}}+1 \approx 2L/(\pi r_{\text{p}})+1$, amounting to a value of $4/\pi+1 \approx 2.3$ for a pore of equal length and diameter ($L \approx 2r_{\text{p}}$) and small colloids ($d\ll r_{\text{p}}$). For longer, $L\gg r_p$ pores or larger particles the conductance enhancement  effect can be even stronger.

Secondly, attractive brush fringes can further enhance the pore permeability by reducing local resistivity in the external regions occupied by the brush fringes, as seen in the curves corresponding to $\chi_{\text{PC}} = -1.2$ and $-1.3$.
The magnitude of this effect increases with the extension of the polymer cap and can be substantial. In good solvent, for example, the cap size (along the pore axis) is comparable to the pore diameter (Figure~\ref{fig:phi_hm_grid}). Under such conditions, a further 5-fold reduction in total pore resistance can be achieved under strong colloid attraction (Supporting Information, Section 7).
Taken together, a strongly attractive polymer meshwork thus can increase colloid fluxes across the pore by an order of magnitude and more compared to an empty pore.

%COMMENT RR: The number of 5 is an estimate. Can we quantify what the effect would be for the conditions described? This number is quite helpful to illustrate the importance of the fringes.


%%%%%%%%%%
\subsection{High colloid flux implies colloid enrichment in the pore}
%%%%%%%%%%

Colloid concentration profiles under stationary flux conditions can be found by numerically solving the Smoluchowsky equation (\ref{eq:Smoluch}) with $\frac{\partial c(r,z)}{\partial t} = 0$  (see Supplementary Information, Section 8). 

Figure~\ref{fig:colloid_concentration} maps the steady-state colloid concentration across a polymer-filled pore with colloid size $d = 12$ and polymer-colloid interaction strength $\chi_{\text{PC}} = -1.5$ in a good solvent ($\chi_{\text{PS}} = 0.3$).
This condition corresponds to facilitated transport with a total resistance about 10-fold lower than a bare pore ($R \approx R_0/10$).

\begin{figure}
    \centering
    \includegraphics[width=0.9\linewidth]{fig/streamlines.png}
    \caption{
    Color map of the steady-state colloid concentration, normalized by the bulk concentration in the source compartment $\Delta c$ as a function of $z$ and $r$.
    Isoconcentration surfaces are shown with contours for values from 0.99 to 0.90 and from 0.10 to 0.00 in steps of 0.01.
    The flux is represented by streamlines marked with small arrows, indicating the average colloid trajectory.
    Pore and brush parameters are the same as in Figure~\ref{fig:colloid_transport}, $d = 12$, $\chi_{\text{PC}} = -1.5$ and $\chi_{\text{PS}} = 0.3$.
    }
    \label{fig:colloid_concentration}
\end{figure}

The map illustrates several salient features of the diffusion process.
Outside the region of the pore and polymer fringes, the colloid concentration profile is as expected for plain solution: the concentration rapidly approaches the respective bulk concentrations of the semi-infinite reservoirs, $c(z = -\infty) = \Delta c$ and $c(z = +\infty) = 0$, and the equiconcentration surfaces near the pore entrance ($c/\Delta c$ between 0.93 and 0.97) and exit ($c/\Delta c$ between 0.07 and 0.03) form a symmetric set of oblate half-spheroids.
Inside the pore, the flux lines run almost parallel to the pore axis.

The most notable observation is that the colloid concentrations strongly exceeds $\Delta c$ near the pore entrance (by a factor of $\sim30$) and inside the pore (by a factor  of $\sim8$.
This effect is caused by the negative insertion free energy in the space occupied by the polymer brush.
At equilibrium (i.e., with vanishing fluxes), the partitioning would amount to $c_{\text{eq}}/\Delta c = \exp\left( -\Delta F \right)$.
In the steady state (i.e., with non-vanishing fluxes), the colloid concentration is reduced but approaches the equilibrium concentration as the insertion free energy becomes largely negative ($c/\Delta c \to c_{\text{eq}}/\Delta c$).

The presented quantitative results are only valid for sufficiently low bulk concentrations $\Delta c$, as our model disregards any colloid crowding effects.
When this crowding is accounted for, the steady-state colloid concentration will be systematically lower.


%%%%%%%%%%
\subsection{Spatial variations in resistivity can entail a preferred colloid diffusion path, and abortive translocation}
%%%%%%%%%%

The inhomogeneous distribution of the polymer density in the pore can entail rather strong spatial variations in resistivity, with some regions facilitating ($\rho D_0 < 1$) and others impeding ($\rho D_0 > 1$) colloid transport.

% \begin{figure}
%     \centering
%     \includegraphics[width = 5cm]{fig/resistivity_map.png}
%     \caption{
%     Maps of normalized resistivity $\rho D_0$ exemplifying a transition between hindered and enhanced permeability upon a subtle increase in the polymer-colloid interaction strength.
%     $\chi_{\text{PC}}$ was varied from $-1.0$ (top) to $-1.1$ (bottom); resistivities are color-coded as indicated.
%     The arrow in the lower frame marks the width of the bottleneck $r_{\text{bn}}$ defining the path of reduced resistivity.
%     Pore and brush parameters are as given in Figure~\ref{fig:colloid_transport}, $\chi_{\text{PS}}=0.5$ and $d=8$.
%     }
%     \label{fig:R_map}
% \end{figure}

This may lead to the formation of a preferred path for colloid diffusion, e.g., visible as a region of reduced resistivity along the pore axis for $\chi_{\text{PC}} = -1.1$, as illustrated in Figure~\ref{fig:R_map}.
The region of reduced resistivity of smallest width $r_{\text{bn}}$ near the pore midpoint ($z=0$) represents a "bottleneck" for colloid transport across the pore.

As the polymer-colloid attraction strength decreases (i.e., $\chi_{\text{PC}}$ increases), the reduced-resistivity path narrows down and ultimately disappears, as seen for $\chi_{\text{PC}} = -1.0$.
In this particular case colloids can penetrate the polymer caps with ease, as $\rho D_0 < 1 $ in these regions, but they cannot easily traverse the pore, as $\rho D_0 > 1 $ throughout the pore. 
Under steady-state conditions, this would manifest as low or negligible flux despite enhanced partitioning near the pore entrance. In the context of single-particle dynamics, this would be reflected in extended dwell times within the pore, with numerous abortive translocation attempts.


%%%%%%%%%%
\subsection{Polymer-filled mesopores effectively gate colloids by their attraction to the polymer}
%%%%%%%%%%

Figure~\ref{fig:R_vs_chi_PC}a illustrates how the total resistance of the pore varies with the colloid's affinity to the polymer brush, characterized by $\chi_{\text{PC}}$.
As expected, increasing the polymer-colloid attraction strength (i.e., more negative $\chi_{\text{PC}}$) results in a decrease in the pore's total resistance, since the interfacial term in the insertion free energy becomes more negative, thereby increasing the local conductivity $\rho^{-1}$.
Naturally, this effect is only moderate for small particles (irrespective of solvent strength), as can be appreciated for particles up to about $d = 4$ in Figure~\ref{fig:R_vs_chi_PC}a.
In contrast, for larger particles, a pore with a polymer brush can exhibit high selectivity based on the polymer-colloid interaction strength, observed as curves with steep slopes in Figure~\ref{fig:R_vs_chi_PC}a.

% \begin{figure}
%     \centering
%     \begin{subfigure}[b]{0.4\textwidth}
%         \includegraphics[width=\textwidth]{fig/resistivity_on_chi_PC.png}
%     \end{subfigure}
%     \hspace{0.03\textwidth}
%     \begin{subfigure}[b]{0.52\textwidth}
%         \includegraphics[width=\textwidth]{fig/chi_PC_crit_on_d.png}
%     \end{subfigure}%
%     \caption{
%         \textbf{(a)} Total pore resistance $R$, normalized by the viscosity of the solvent $\eta_\text{S}$, as a function of the polymer-colloid interaction strength $\chi_{\text{PC}}$ (thin black lines) for selected particle sizes $d $, (as indicated), and solvent strength $\chi_{\text{PS}}=0.5$.
%         The thick brown curve separates the parameter spaces of facilitated ($R < R_{0}$, left and below) and impeded ($R > R_{0}$, right and above) permeation, as hihglighted with the contoured and solid arrows, respectively.
%         Each intersection point of the brown curve with a black line corresponds to the critical polymer-colloid interaction strength $\chi_{\text{PC}}^{\text{crit}}$ for a given particle size $d$, where the pore resistance is equal to the resistance of a bare pore ($R = R_{0}$).
%         \textbf{(b)} Critical polymer-colloid interaction strength $\chi_{\text{PC}}^{\text{crit}}$ as a function of particle size $d$ at different solvent strengths ($\chi_{\text{PS}}$), ranging from good to $\theta$-solvents, as indicated.
%         The region below each curve corresponds to facilitated permeation ($R > R_{0}$), while the region above corresponds to impeded permeation ($R < R_{0}$) for a given particle size $d$ and solvent strength $\chi_{\text{PS}}$.
%         Pore and brush parameters are the same as in Figure~\ref{fig:colloid_transport}.
%     }
%     \label{fig:R_vs_chi_PC}
% \end{figure}

It is notable that, for a given colloid size, the total resistances tend to plateau towards strong polymer-colloid attractions (negative $\chi_{\text{PC}}$ values).
In this regime, the resistance of the pore interior and the brush fringes is minimal, and the total resistance is dominated by the flow in the bulk solution.

To analyze the pore resistance we introduce the critical condition
\\$R(\{\chi_{\text{PC}}, \chi_{\text{PS}}, d\}_{\text{crit}}) = R_{0}$ corresponding to a vanishing combined effect of the polymer mesh. At a fixed value of the solvent quality parameter $\chi_{\text{PS}}$, this defines the critical value of the polymer-colloid affinity as a function of the colloid diameter $\chi_{\text{PC}}^{\text{crit}}(d)$, and vice versa.
In Figure~\ref{fig:R_vs_chi_PC}a, we trace $\chi_{\text{PC}}^{\text{crit}}(d)$ (brown curve), which bisects the lines of total resistance so as to separate the region of faciliated permeation ($R < R_{0}$, where $\chi_{\text{PC}} < \chi_{\text{PC}}^{\text{crit}}$) from the region of impeded permeation ($R > R_{0}$, where $\chi_{\text{PC}} > \chi_{\text{PC}}^{\text{crit}}$).
Although the region of impeded permeation exhibits high selectivity with respect to the polymer-colloid interaction strength $\chi_{\text{PC}}$, it is also characterized by rather high total resistance and low colloid flux.
The region of facilitated permeation, in contrast, exhibits high colloid fluxes but rather low (if any) $\chi_{\text{PC}}$ selectivity. 
Thus, the traced $\chi_{\text{PC}}^{\text{crit}}(d)$ line of unperturbed transport (compared to the bare pore) defines the condition for sharp colloid gating, with remarkably efficient transport for all colloids obeying $\chi_{\text{PC}} < \chi_{\text{PC}}^{\text{crit}}$ and effective blockage for $\chi_{\text{PC}} < \chi_{\text{PC}}^{\text{crit}}$.

The critical polymer-colloid interaction strength $\chi_{\text{PC}}^{\text{crit}}$ as a function of colloid size $d$ and solvent strength $\chi_{\text{PS}}$ is shown in Figure~\ref{fig:R_vs_chi_PC}b.
Typically, the critical interaction parameter $\chi_{\text{PC}}^{\text{crit}}$ decreases monotonically with colloid size $d$. 
This trend is explained by the osmotic insertion free energy penalty ($\Delta F_{\text{osm}}$) proportional to the particle volume, combined with a reduction in the diffusion coefficient $D$ (Eq.~(\ref{eq:Rubinstein})), and a reduction in the effective pore radius due to volume exclusion, all these factors to be compensated for by stronger polymer-colloid attraction. 

For smaller particles, some non-monotonic behavior of $\chi_{\text{PC}}^{\text{crit}}(d)$ may appear due the interplay of attractive and repulsive terms (quadratic and cubic in $d$, respectively) in $\Delta F$, as seen in the upper $\chi_{\text{PC}}^{\text{crit}}(d)$  curve corresponding to $\theta$-solvent. Interestingly, this leads to a relatively weak dependence of $\chi_{\text{PC}}^{\text{crit}}$ on the colloid size, implying that the sharp gating of colloids by their surface property is largely independent of the colloid size in the vicinity of the $\theta$-solvent conditions.

It is clear that an increase in solvent strength (decrease in $\chi_{\text{PS}}$) entails a decrease in $\chi_{\text{PC}}^{\text{crit}}$ irrespective of the colloid size. This trend is primarily due to an increased osmotic barrier within the pore.


%%%%%%%%%%
\subsection{Polymer-filled mesopores effectively gate colloids by their size}
%%%%%%%%%%

Figure~\ref{fig:R_vs_d} illustrates how the total resistance varies with the colloid size. 
Two distinct trends are generally observed.

\begin{figure}
    \centering
    \includegraphics[width = 0.95\linewidth]{fig/permeability_on_d.png}
    \caption{
    Total pore resistance $R$, normalized by the viscosity of the solvent $\eta_\text{S}$, as a function of colloid size $d$ for selected polymer-colloid interaction strengths ($\chi_{\text{PC}}$, as indicated with colored lines) and solvent qualities ($\chi_{\text{PS}}$, as indicated above each panel). 
    The resistance of a bare pore $R_{0}$ without polymers (black thick line) separates the parameter spaces of facilitated ($R < R_{0}$, below) and impeded ($R > R_{0}$, above) permeation.
    The intersection of the colored curves with the $R_{0}$ curve defines the critical particle size $d_{\text{crit}}$.
    Pore and brush parameters are as given in Figure~\ref{fig:colloid_transport}. 
    }
    \label{fig:R_vs_d}
\end{figure}

At sufficiently weak polymer-colloid attraction ($\chi_{\text{PC}} \gtrsim -1.0$), the polymer filled pore tends to be more resistant to colloid transport than the bare pore (black thick line in Figure~\ref{fig:R_vs_d}) across all colloid sizes irrespective of the solvent quality.
In this regime, the resistance increases gradually yet substantially with colloid size, owing to a combination of enhanced osmotic repulsion and reduced diffusivity.

As the polymer-colloid attraction gets stronger ($\chi_{\text{PC}}$ decreases), permeability for small colloids is enhanced compared to an empty pore. 
Interestingly, the resistance remains approximately constant over a range of colloid sizes, until a critical colloid size $d_{\text{crit}}$ is reached above which the resistance grows very sharply effectively impeding permeation. 
In this regime, the polymer-filled pore thus acts as a gate that enhances the transport of all colloids with a size below $d_{\text{crit}}$ and effectively blocks all larger colloids.
It can be seen that for lower solvent strength, the level of attraction required for effective gating by size decreases. 
Moreover, the threshold $d_{\text{crit}}$ for impeded permeation (at any given $\chi_{\text{PC}}$) is pushed towards larger sizes.


%%%%%%%%%%
\section{DISCUSSION}
%%%%%%%%%%

Our most striking, and counterintuitive, finding is that an attractive polymer brush can enhance the net colloid transport in a concentration gradient across the pore by an order of magnitude and more, compared to a bare pore.
Moreover, we have shown how mesopores with polymer brushes can gate transport with exquisite selectivity with respect to polymer-colloid affinity and colloid size, even for colloids that are substantially smaller than the pore size.

Our findings shed light on possible mechanisms of selective transport through nuclear pore complexes (NPCs) and, at the same time, suggest a molecular design strategy for controlling selective permeability through artificial mesoporous membranes, with potential applications in fields such as targeted drug delivery, biosensing, and filtration systems.

%%%%%%%%%%
\subsection{Towards technological applications of synthetic polymer-filled mesopores}
%%%%%%%%%%

By tuning parameters like particle size, polymer-colloid affinity, and solvent quality, it is possible to modulate transport properties and achieve desired selectivity levels in synthetic membranes.
These insights could pave the way for designing nanoporous materials with enhanced selectivity tailored to specific functional requirements, thereby broadening the scope of applications in nanomedicine, biotechnology, and environmental engineering.

Mixtures of biological colloids such as folded proteins and other biomacromolecular complexes, as well as synthetic colloids such as nanoparticles, may be effectively separated, not only according to their size but also their surface (bio-)chemistry.
Importantly, the here-presented theoretical approach and integration schemes facilitate the rational design of pores with a geometry and polymer filling optimised for the desired separation task.

Individual pores, as we have considered here, are routinely deployed in current nanopore sensing technologies.
These technologies enable detection and characterization of individual macromolecules as they travel across the pore.
Our findings suggest polymer fillings as an attractive tool to optimize the performance of nanopore sensing.
Placing a suitable polymer filling upstream the pore's sensing region would enable pre-selection of target solutes from complex mixtures for a focused analysis by the pore.
Polymer fillings may also be placed in the very sensing region of the pore to enhance both selectivity and sensitivity.
The here-proposed approach is distinct from previous approaches, where responsive polymer coatings along the pore walls were used to open/close a polymer free channel on application of an external stimulus such as a change in temperature, ionic strength or pH. 

Individual pores will though typically be insufficient in applications that focus on separation with high throughput such as filtration systems.
This limitation can be overcome by multiplexing, e.g., with membranes featuring a large array of mesopores.
Our theoretical approach remains valid for such arrays as long as the distance between pores remains sufficiently large for the diffusion trajectories of adjacent pores not to substantially interfere.
Fortunately, this condition can be met with a relatively tight packing of pores, as can be appreciated from the flux lines in Figure~\ref{fig:colloid_concentration}.
In practice, a distance between pore centres approximately 10-fold greater than the pore diameter should entail minimal interference \cite{Fabrikant1985}.

\bigskip

\noindent{The main design concepts emerging from our theory are:}

\textbf{1.}
For a polymer-filled pore to function as a selective transport channel, high permeation selectivity must be coupled with low resistance to diffusive flux.
We refer to this combination as 'gating' behavior, where a minor change in colloid size or polymer-colloid interaction strength can dramatically shift the permeation rate from facilitated transport to virtually complete blockage.
Both requirements can be achieved near the critical values $d_{\text{crit}}$ and/or $\chi_{\text{PC}}^{\text{crit}}$, which assure that the resistance of the brush-filled pore matches that of the bare pore, $R\simeq R_{0}$.
The gating effect is particularly pronounced for larger particles: in our case, with the diameter of 10 polymer segment lengths or more.

\textbf{2.}
Pore resistance is highly sensitive to parameters that influence insertion free energy.
Strong dependence of the pore resistance on the parameters of the colloid originates from the exponential dependence of the local conductivity on the insertion free energy.
This results in very high selectivity of the polymer brush with respect to colloid size and polymer affinity, as demonstrated in Figures \ref{fig:R_vs_chi_PC} and \ref{fig:R_vs_d}.
The osmotic contribution to the insertion free energy scales as $d^3$ while the interfacial contribution comprises $\chi_{\text{PC}}$ and scales as $d^2$.
Thus, under conditions when colloid transport is limited by the polymer brush (as opposed to plain solvent), a slight change in $d$ and/or $\chi_{\text{PC}}$ translates into a drastic change in permeability (resistance).

\textbf{3.}
The maximal permeability of the polymer-filled pore is limited by the resistance of the exterior region.
Whilst strong  polymer-colloid attraction can make the resistance of the pore interior effectively vanish ($R_{\text{int}} \to 0$), this is not the case for the pore exterior, where mass transport in plain solvent always provides a non-vanishing resistance.
Under poor solvent conditions  polymers are typically confined to the pore interior, and the total resistance is bounded from below by the Rayleigh resistance  $R \geq R_{\text{ext}}^{0} = \frac{1}{2 D_0 r_{\text{p}}}$.
Polymer caps emerge outside the pore under good or $\theta$-solvent conditions decreasing the path through plain solvent, and thereby can reduce the total resistance even further. Approximating the caps as half-spheres with radius $r_{\text{ext}}$, and assuming them highly attractive, leads to $R_{\text{ext}} \to \frac{1}{ \pi D_0 r_{\text{ext}}}$.
The maximum additional reduction in total resistance due to the presence of attractive polymer caps thus scales as $\frac{R_{\text{ext}}^{0}}{R_{\text{ext}}} \to \frac{\pi r_{\text{ext}}}{2 r_{\text{p}}}$.
Hence, a large polymer cap is beneficial for transport rates, although even a moderate cap size can lead to substantial gain (e.g., approximately 3-fold for $r_{\text{ext}}\simeq L = 2r_{\text{p}}$ as suggested by Fig.\ref{fig:phi_hm_grid}).

\bigskip

\noindent{The manufacturing of functional mesoporous membranes is an emerging art, and we hope that our theoretical efforts will both promote and guide future practical developments in this area.}
%COMMENT RR: We ought to provide some references on the manufacturing of mesoporous membranes.
%COMMENT RR: One can expect that transport rates will increase further with a pressure gradient that drives solution flow across the membrane. We could mention this here as an avenue worthy exploring in future work? 


%%%%%%%%%%
\subsection{Implications for nuclear pore permselectivty}
%%%%%%%%%%

A remarkable number of features that we have identified with our theory is also found in the transport of proteins and other biomacromolecules through nuclear pore complexes (NPCs), suggesting that our model is capable of capturing the basic mechanisms of nuclear pore permselectivity in spite of some rather simple assumptions.

The estimated distance between NPCs in the nuclear envelope is approximately 10 times larger than the pore diameter \cite{Yang2004, Daigle2001, Feldherr1984, Kubitscheck2000}), and transport across neighbouring NPCs therefore can be considered mutually non-interfering. 

Single-cargo tracking studies using fluorescence \cite{Musser2016, Lowe2010, Lowe2015, Yang2004, Kubitscheck2000, Ma2010} and tomography \cite{Beck2007} have shown that transported colloids (e.g., importins, exportins and their complexes with cargo) primarily traverse the central region of the NPC and are rarely observed near the pore walls.
Such a behaviour is consistent with the preferred colloid diffusion path arising as a consequence of spatial variations in resistivity in our model (Figure~\ref{fig:R_map}, bottom).
Importantly such a path does not require the presence of an empty (i.e., polymer free) channel as had been suggested in some earlier models of NPC transport. Instead, subtle variations in polymer density across the pore's cross-section substantially determine where colloids enrich and translocate.  
%Need to add further references.

On the other hand, these studies also evidenced that transport attempts are frequently aborted, with the transported colloid either dwelling near the pore entrance for a sustained time period or partially traversing into the pore before returning.
This behavior aligns with the predictions of our model that a negative insertion free energy at th pore periphery draws colloids into the pore, but the existence of the free energy barrier in the center of the pore (see Figure~\ref{fig:R_map}, top) would naturally lead to a large number of unsucsessful translocation attempts.

It is also well-known that colloids with affinity for the disordered nucleoporin FG domains that fill the NPC (such as importins and exportins) are enriched in or near NPCs \cite{Beck2007, Gruenwald2010, Tu2011}, and in microscopic droplets, macroscopic hydrogels and thin films assembled from pure FG domains.
Moreover, high concentrations of transport factors are essential for effective transport through the pore \cite{Lowe2015}.
These observations fully align with our predictions that accumulation of colloids in the pore is required for facilitated transport (Figure~\ref{fig:colloid_concentration}).
%Need to check references.

Solvent strength conditions for nucleoporins can be estimated to be close to $\theta$-solvent as attested by a certain level of 'cohesiveness' observed for FG domains. 
The parameters taken in Figure~\ref{fig:colloid_transport}a thus appear particularly pertinent for the NPC.
An effective size limit of approximately 5 nm has been reported for the passive permeation of 'inert' colloids (i.e., colloids that do not or only weakly bind to FG domains).
Normalised by $a$ = 0.8 nm, the effective statistical segment length of disordered polypeptide chains, this corresponds to $d/a\approx 6$. Considering the regime of weak polymer attraction ($\chi_{\text{PC}} > -0.5$), one can see (Figure~\ref{fig:colloid_transport}a) that this value matches the prediction of our model remarkably well.
%Need to add further references.

Mohr et al. quantified the rates of diffusive tranport across NPCs for 'inert' colloids of varying size. Figure~\ref{fig:NPC_comparison}a compares these experimental results with predictions of our model for colloids.
On can see that the model (representing inert colloids with $\chi_{\text{PCS}} = 0$) reproduces the trend of an increase in pore resistance with colloid size remarkably well.
Of note, cytosolic proteins had been washed out in the experiments, and nuclear pores filled with a plain FG domain brush (i.e., without transport factors) were hence probed, a scenario that is effectively recpitulated by our theory. 

% \begin{figure}
%     \centering
%     \includegraphics[width = 0.5\linewidth]{fig/experimental.png}
%     \caption{
%     Comparison of theoretical predictions with experimental findings for NPCs.
%     \textbf{(a)} Gating of colloids by size.
%     Pore resistances vs. hyrodynamic diameter (symbols) were estimated from import rates into the nucleus of permeabilised HeLa cells measured by Mohr et al. for inert colloids of varying sizes.
%     Theoretical predictions (lines) are for the pore and brush parameters as given in Figure~\ref{fig:colloid_transport}, $\chi_{\text{PS}} = 0.5$ and $\chi_{\text{PCS}} = 0$.
%     \textbf{(b)} Gating of colloids by affinity.
%     Pore resistances vs. insertion free energy (symbols) were estimated from Frey et al. for a range of green fluorescent proteins with surface amino acids mutated to modulate transport from 'superinert' to 'transport factor like'.
%     Pore resistances were obtained from import rates into the nucleus of permeabilised HeLa cells; insertion free energies were obtained from partition coefficients in phase-separated droplets of Mac98A FG domains.
%     Theoretical predictions (lines) are for the pore and brush parameters as given in Figure~\ref{fig:colloid_transport}, $\chi_{\text{PS}} = 0.5$ and $d = 6$ (i.e. equivalent to the hydrodynamic diameter of GFP).    
%     }
%     \label{fig:NPC_comparison}
% \end{figure}

\begin{figure}
    \centering
    \includegraphics[width = 0.45\linewidth]{fig/experimental_inert.png}
    \caption{
    Comparison of theoretical predictions with experimental findings for NPCs. Gating of colloids by size.
    Pore resistances vs. hyrodynamic diameter (symbols) were estimated from import rates into the nucleus of permeabilised HeLa cells measured by Mohr et al. for inert colloids of varying sizes.
    Theoretical predictions (lines) are for the pore and brush parameters as given in Figure~\ref{fig:colloid_transport}, $\chi_{\text{PS}} = 0.5$ and $\chi_{\text{PCS}} = 0$.
    }
    \label{fig:NPC_comparison}
\end{figure}

\begin{figure}
    \centering
    \includegraphics[width = 0.45\linewidth]{fig/experimental_permeability_on_partitioning.png}
    \caption{
    Comparison of theoretical predictions with experimental findings for NPCs. Gating of colloids by size.
    Pore resistances vs. hyrodynamic diameter (symbols) were estimated from import rates into the nucleus of permeabilised HeLa cells measured by Mohr et al. for inert colloids of varying sizes.
    Theoretical predictions (lines) are for the pore and brush parameters as given in Figure~\ref{fig:colloid_transport}, $\chi_{\text{PS}} = 0.5$ and $\chi_{\text{PCS}} = 0$.
    }
    \label{fig:NPC_attr_comparison}
\end{figure}

Lastly, the NPC features a remarkable rate of facilitated permeation and an exquisite permselectivity with respect to the surface features of relevant proteins (e.g., importins and exportins).
Using an approach similar to Mohr et al., Frey et al. quantified NPC transport rates for a wide range of green fluorescent proteins (GFPs) with surface amino acids mutated to modulate transport from 'superinert' to 'transport factor like'.
In parallel, the ability of these variants to enrich or deplete in phase-separated droplets or hydrogels of pure FG domains was quantified (Figure~\ref{fig:NPC_comparison}b).
This set of experiments enabled the effect of colloid affinity to be tested selectively as the colloid size and shape were effectively constant.
The transport rate was observed to correlate strongly with the level of GFP enrichment in FG domain phases.
Our theory reproduces the salient features of this experimental system.
First, the experimental data provide direct evidence that transport-factor-like proteins can indeed be transported at a rate exceeding the rate of an empty pore.
Second, the experimental data qualitatively demonstrate the expected affinity gating, with the most attractive GFP variants experiencing pore resistances several orders of magnitude smaller then the most inert variants.
Third, transport rates are approaching a plateau for the most attractive GFP variants tested, suggesting that in this regime transport is limited by the diffusion to the pore rather than the pore itself (see Figure~\ref{fig:R_vs_chi_PC}a).
%Need to add references.

Neither the colloids nor the polymers pertinent to the NPC are as regular as assumed in our model.
Importins, exportins and their cargo have complex, non-spherical shapes and display substantial surface heterogeneity.
Similarly, each FG domain type exhibits substantial heterogeneity along the chain contour.
A well-established salient feature of both these interaction partners is that they display multiple, and typically weak, binding sites. In this context, the assumption of homogeneous colloid surface and polymer chain properties appears rather well justified. 

Moreover, the NPC features a variety of nucleoporin FG domains, with the body of available structural and biochemical data suggesting that the cohesiveness of nucleoporin FG domains is highest in the centre and decreases towards the periphery of the pore.
Qualitatively, one can envisage that the increased solubility of peripheral FG domains promotes a more extended polymer cap, thus minimising total pore resistance and maximising transport rates for strongly attractive colloids.
The reduced solubility of the central FG domains, on the other hand, would minimize the size threshold for gating of non-adhesive colloids.
Our model can be further extended to incorporate  solubility gradients and to explore such phenomena in more detail.
%Need to add references.

Overall, the agreement between the many experimental observations and the predictions of our theory is striking and strongly suggests that it provides an appropriate description of the basic mechanism of nuclear pore permselectivity.
The agreements are indeed remarkable given that the nuclear pore complex exhibits a much higher chemical complexity than our model.

%COMMENT RR: I have below left some further info about NPCs that Mikhail had gathered. I leave these for further consideration, though I am not sure how useful they are to the discussion. 

% In the ref\cite{MoussaviBaygi2016}, authors proposed that in the transport event locally collapses upon interacting with the NTR-bearing macromolecule, but autonomously reconstructs itself very fast, keeping the pore sealed.
% Ref \cite{Hough2015} also proposed that FG-motives create highly dynamic phase that can extremely quick exchange contacts with transport factors of cargo.
% Ref \cite{Milles2015} anticipates that fast transport requires rapid exchange when engaging FG-motives with the NTR
% Ref \cite{Goodrich2018} also proposed binding-mediated mechanism that changes local structure, destroying local cages.

%%%%
\printbibliography
\end{document}




































%%%%%%%%%%%%%%%%%%%%%%%%%%%%%%%%%%%%%%%%%%%%%%%%%%%%%%%%%%%%%%%%%%%%%%%%%%%%%%%%%%%%%%%%%%%%%%%%%%%%%%%%%%
%RR: In the following, I have left some pieces of text that could be inserted into the main text where desired
%%%%%%%%%%%%%%%%%%%%%%%%%%%%%%%%%%%%%%%%%%%%%%%%%%%%%%%%%%%%%%%%%%%%%%%%%%%%%%%%%%%%%%%%%%%%%%%%%%%%%%%%%%

%Under good (or \theta-) solvent conditions we may consider separately the situations with positive and negative insertion free energies. 
%Negative insertion free energies are rather exceptional under good solvent conditions. As we see below, in this case $R_{caps}\leq R_{convergent}$ and the total resistance
%is lower than that of the empty pore.
%Positive insertion free energies under good solvent conditions are more common. 
%In this case, the resistance of the pore interior is always dominant, 
%$$
%R_{tot}\approx R_{\text{int}}
%$$
%and the accuracy in estimating the resistance contributions from the entrance/exit regions is not of a major concern. 

%In Figure \ref{fig:fe_scf_grid} the insertion free energy profiles $\Delta F(z,r=0)$ calculated by analytical scheme and by SF-SCF method 
%are presented as a function of position of a spherical particle along the pore axis.
%While the SF-SCF method provides the net free energy, the analytical scheme allows decoupling of the free energy into osmotic and surface contributions, 
%which are shown separately in Figure \ref{fig:fe_scf_grid}.
%The numerical coefficients $b_0$ and $b_1$ in eq \ref{} are chosen by the best fit, but appear to be fairly universal and independent of the particle size 
%and interaction parameters $\chi_{PS,PC}$.
%Remarkably, the fit fails when the size $d$ became comparable with the pore diameter or in the case of extreme $\chi_{ads}$ values 
%when analytical scheme is not applicable because of strong perturbation 
%of the brush structure by inserted particle, while SF-SCF method can still be safely used
%for the evaluation of the insertion free energy.

%The 2D insertion free energy $\Delta F(r,z)$ patterns have rather complex shape. However, we can trace their evolution upon changing interaction parameters
%looking at the position-dependent free energy of the particle on the pore axis, $\Delta F(z,r=0)$.
%As one can see from Figure \ref{fig:fe_scf_grid}, the insertion free energy profiles evolve upon changing the interaction parameters $\chi_{PS,PC}$ as follows:
%At $\chi_{ads}\geq \chi_{crit}$ which is the case under good or theta-solvent conditions and weak or absent polymer-particle attraction, $|\chi_{\text{PC}}|\leq 1$, the positive osmotic
%term, $\Delta F_{osm}\geq 0$ dominates in the insertion free energy, which is positive and reach maximal value in the pore center, where polymer concentration is maximal.
%Hence, polymer-particle interaction has overall repulsive character and $\Delta F(r,z)$ has the shape of the free energy barrier preventing penetration and accumulation of particles in the pore.
%By using the insertion free energy $\Delta F(r,z)$ one can calculate the equilibrium partition coefficient 
%$$
%P=\int_{0}^{r_{pore}}2\pi rdr\int_{0}^{L_{0}}dz\exp (-\Delta F(r,z)/k_BT)/\pi r^{2}_{pore}L_{0}
%$$
%is larger than unity, $P\geq 1$. Noticably the repulsive free energy profiles extends beyond the edges of the pore, because of the fringes in the polymer density distribution in swollen brush.

%A decrease in $\chi_{ads}$ triggered by a decrease in  $\chi_{\text{PC}}$ or/and an increase in $\chi_{\text{PS}}$ leads to qualitative changes in the insertion free energy 
%$\Delta F(r,z)$ patterns: At $\chi_{ads}\leq \chi_{crit}$ the particle surface becomes
%adsorbing for the polymer, $\gamma \leq 0$, that gives rise to a negative contribution $\Delta F_{surf}(r,z)$ to the insertion free energy. 
%When $\chi_{\text{PS}}$ increases (the solvent is getting worse for the polymer)
%the osmotic pressure inside the brush decreases that leads to a decrease in the 
%magnitude of $\Delta F_{osm}(r,z)$ with the concomitant shrinkage of the  protruding outside the pore parts of the brush where  $\Delta F(r,z)\neq 0$.
%As a result, the $\Delta F_{surf}(r,z)$ aquires two minima with negative values near the endtance and the exit of the pore, separated by a maximum centered in the middle of the pore
%where polymer concentration is larger and the osmotic repulsive term  $\Delta F_{osm}(r,z)$ dominates.
%Finally, at strong polymer-particle attraction $\chi_{ads} < \chi_{crit}$, the negative surface contribution $\Delta F_{surf}(r,z)\leq 0$ overperform osmotic repulsion everywhere inside the pore
%and the $\Delta F(r,z)$ aquires the shape of the potential well centered in the middle of the pore, which gives rise to preferential accumulation of particles in the pore, $P\geq 1$.
