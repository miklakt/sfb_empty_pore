\documentclass[12pt, a4paper]{article}
\usepackage{graphicx}
\usepackage{amsmath, amssymb, amsfonts, mathtools}
\usepackage{subcaption}
\usepackage{xcolor}
\usepackage{bm}

\usepackage{authblk}
\renewcommand\Authfont{\normalsize}
\renewcommand\Affilfont{\small}

\usepackage[T1]{fontenc}
\usepackage{lmodern}        % better fonts in T1

\usepackage[
  backend=biber,
  style=nature,
  maxnames=12,  % limit names in citations
  doi=false,
  url=false,
  eprint=false,
  isbn=false,
]{biblatex}
\addbibresource{biblio.bib}

% % Set limits programmatically (works even on older versions)
% \AtBeginBibliography{%
%   \setcounter{maxnames}{99}%
%   \setcounter{minnames}{1}%
% }

%\REMOVE BEFORE SUBMISSION
\usepackage[right]{lineno}
\linenumbers

\setcounter{secnumdepth}{2} % SET 2 BEFORE SUBMISSION

%\REMOVE BEFORE SUBMISSION
\usepackage[
  author={},         % no author shown
  subject={},        % no subject
  date={},           % no date/timestamp
  color=yellow,      % note color (viewer-dependent rendering)
  icon=Note,         % classic sticky note icon
  open=false         % closed by default
]{pdfcomment}
\newcommand{\todo}[1]{\pdfcomment{#1}} %\REMOVE ALL \todo{...} BEFORE SUBMISSION


\title{A polymer filling enhances the rate and selectivity of colloid permeation across mesopores}

\author[1]{Mikhail Y. Laktionov}
\author[2]{Frans A. M. Leermakers}
\author[3]{Ralf P. Richter$^{*}$}
\author[4,5]{Leonid I. Klushin$^{*}$}
\author[1]{Oleg V. Borisov$^{*}$}

\affil[1]{CNRS, Université de Pau et des Pays de l'Adour, UMR 5254, 
Institut des Sciences Analytiques et de Physico-Chimie pour l'Environnement et les Matériaux, 64053 Pau, France}

\affil[2]{Physical Chemistry and Soft Matter, Wageningen University, Stippeneng 4, 6708 WE, Wageningen, The Netherlands}

\affil[3]{University of Leeds, School of Biomedical Sciences, Faculty of Biological Sciences, 
School of Physics and Astronomy, Faculty of Engineering and Physical Sciences, 
Astbury Centre for Structural Molecular Biology, 
and Bragg Center for Materials Research, Leeds, LS2 9JT, United Kingdom}

\affil[4]{Branch of Petersburg Nuclear Physics Institute 
named by B.P. Konstantinov of National Research Centre ''Kurchatov Institute'', 
Institute of Macromolecular Compounds, 199004 St. Petersburg, Russia}

\affil[5]{American University of Beirut, Department of Physics, Beirut 1107 2020, Lebanon}

\date{}

\begin{document}
\maketitle

\vspace{-8ex}
\begin{center}
\small{
$^{*}$Corresponding authors: Ralf P. Richter - \href{mailto:r.richter@leeds.ac.uk}{r.richter@leeds.ac.uk}; Leonid I. Klushin - \href{mailto:leo@aub.edu.lb}{leo@aub.edu.lb}; Oleg V. Borisov - \href{mailto:oleg.borisov@univ-pau.frb}{oleg.borisov@univ-pau.fr}
}
\end{center}
\vspace{1ex}

\begin{abstract}
Polymer-functionalised mesopores are an emerging technology for colloid separation, sensing and delivery.
Their potential is strikingly illustrated in living cells, where nuclear pore complexes (NPCs) control biocolloid transport between the nucleus and the cytosol.
Even colloids much smaller than the biopolymer-filled NPC channel are effectively blocked, but some larger colloids with distinct surface features rapidly permeate.
Simplistically, one may expect any polymer filling to obstruct and slow down colloid transport.
We demonstrate how a polymer filling that attracts colloids and extends beyond the mesopore, thus maximizing colloid capture, can instead increase permeation compared to a bare pore.
We also define how polymer-filled mesopores can effectively gate colloids according to their size and surface features.
Our findings provide a basic physical explanation for the exquisite permselectivity of NPCs, and a rational design strategy for novel mesopore-based separation, sensing, catalysis and drug delivery devices with enhanced performance features.
\end{abstract}


%%%%%%%%%%
\section{INTRODUCTION}
%%%%%%%%%%

Advances in macromolecular chemistry have enabled the functionalization of mesopores (i.e., pores with a diameter between a few and many tens of nanometers) by anchoring polymer chains of various chemical nature to the pore wall.
This creates a solvated polymer meshwork that fills the entire pore volume or remains confined to near-wall regions, depending on the polymer's molecular weight and conformational state \cite{Peleg2011, Laktionov2021}.
Polymer-modified mesoporous materials and membranes represent a new class of functional nanostructured systems with significant potential across a wide range of technologies (\cite{Uredat2024, Pardehkhorram2022} and references therein), such as ultrafiltration, sensing, catalysis and drug delivery.

The interaction of the polymer meshwork with colloids, i.e., nanoparticles or (macro)molecules in the solution phase, essentially determines the absorption and separation properties of the polymer-modified mesoporous materials.
Interactions can be attractive or repulsive, and of varying specificity, controlled by a broad spectrum of parameters \cite{Low2019} such as pH and ion strength, valency and specificity \cite{Lee2010}, solvent quality and temperature \cite{Halperin2011, Stetsyshyn2020}, and more specific 'ligand binding' features.
This opens up novel opportunities for highly selective and controlled uptake and transport of colloids through polymer-filled mesoscopic channels \cite{Sirkin2020}.

Past experimental and theoretical efforts have focused on triggering the transient opening of a polymer free path through an external stimulus to gate transport \cite{Alberti2015, Emilsson2018, Caspi2008}.
However, the polymer phase itself can potentially also provide high selectivity to colloids as a function of their size and attraction by the polymer.
We thus hypothesized that even mesopores filled with a polymer meshwork across their entire cross-section can effectively gate transport.
If successful, this approach would enable more robust gating as it does not rely on careful tuning of the diameter of a polymer-free channel, and higher transport rates as the full pore cross-section can participate in colloid transport.

Nature provides a case in point.
Nuclear pore complexes (NPCs) perforate the nuclear envelope of eukaryotic cells and provide for speedy and highly selective nucleo-cytoplasmic transport of proteins and nucleic acids.
This process spatially separates gene transcription (in the nucleus) from translation into proteins (in the cytosol), critical for the ordered course of gene expression.
Each NPC forms an approximately cylindrical channel, measuring 40-80 nm in diameter (depending on cell state and species \cite{Zimmerli2021, Schuller2021}) and similar in length.
The channel is filled with a meshwork of hundreds of natively disordered protein domains rich in phenylalanine-glycine dipeptides (FG domains) that are anchored to the channel walls \cite{Ori2013, Rajoo2018, Kim2018, Zimmerli2021}.
Collectively, the FG domains provide remarkable gating function: biocolloids of just a tenth of the pore diameter or more in hydrodynamic diameter are effectively blocked, except for dedicated transport factors (importins and exportins, alone and in complex with cargo) that bind to the FG domains and permeate rapidly \cite{Hoogenboom2021}. 

Several independent strands of evidence indicate that diffusive transport across NPCs is based on rather universal physical principles, whereas the exact chemical makeup of the polymers and colloids is secondary for function \cite{Hoogenboom2021}.
First, NPCs robustly fulfill their transport functions despite substantial compositional variations of the NPC architecture across distant eukaryotic taxa and cell states \cite{VonAppen2015, Ori2013, Schmidt2015, Kim2018, Rajoo2018, Zimmerli2021}.
Second, NPCs can gate diffusive colloid transport similarly well in both directions.
Whilst the native cell is capable of directed transport of cargo against a concentration gradient, this function is not intrinsic to the NPC and reversible through cell engineering \cite{Nachury1999}.
Third, the binding behaviour of transport factors to assemblies of purified FG domains could be reproduced by simple physical models that treat FG domains as regular flexible polymers and transport factors as spherical colloids with a homogeneous surface \cite{Zahn2016, Vovk2016}, i.e., ignoring the detailed arrangement of interaction sites. 
Fourth, studies with a spectrum of globular proteins as model colloids demonstrated that NPCs exhibit a wide and continuous spectrum of permeabilities as a function of protein surface properties (at constant size \cite{Frey2018}) and size (for weakly or non-interacting proteins \cite{Ribbeck2001, Mohr2009, Popken2015, Timney2016}).
Based on these and other findings, it is thought that a fine balance of many individually weak physicochemical (e.g., electrostatic, hydrophobic, aromatic stacking) interactions between polymers and biocolloids dictates the gating behaviour, rather than a few highly specific biochemical interactions \cite{Hoogenboom2021}.

However, we currently lack an understanding of the relationship between the molecular architecture of the polymer brush filling the pore, and its ability to transport colloids with high selectivity and rate.
Here, we address this question with a self-consistent field theoretical approach.
The theory considers that a meshwork of flexible polymers effectively increases the local viscosity and thereby slows down transport of colloids compared to an open pore.
On the other hand, an attractive polymer phase recruits colloids into the pore, thus increasing colloid transport.
Intriguingly, the solvent strength through its influence on the density and compactness of the polymer meshwork impacts both these effects.
We define how solvent quality and colloid attraction to the polymer may be tuned to maximize the transport rate (even beyond the rate for an open pore) and to achieve highly selective transport with respect to the colloid's size or affinity for the polymer.

%%%%%%%%%%
\section{RESULTS}
%%%%%%%%%%


%%%%%%%%%%
\subsection{Defining the transport scenario}
%%%%%%%%%%

Salient features of our simulated mesopore are illustrated in Figure~\ref{fig:colloid_transport}.
The cylinder-shaped pore perforates a planar membrane (for the NPC this would be space bounded by two lipid bilayers) and is the sole conduit for colloids between two semi-infinite solution reservoirs.
Flexible polymer chains are end-grafted to the inner pore walls, at a density sufficient to form a polymer brush that fills the entire pore cross-section.

We will focus on a pore with a set radius $r_{\text{p}}^0$ and length $L_0$ (in units $a$), and polymers with a degree of polymerization $N$ and grafting density $\sigma$ (Figure~\ref{fig:colloid_transport}).
Whilst the selected values are inspired by the NPC (see Supplementary Note 1), we expect that our findings will be of rather general validity so they can be applied to the performance analysis and rational design of mesopores with other geometries or polymer fillings.

Colloids are taken to be spheres with diameter $d$.
The interaction strength (contact free energy) between a polymer segment and the colloid surface is represented by the Flory-Huggins parameter $\chi_{\text{PC}}$.

\begin{figure}
    \centering
    \includegraphics[width = 3.5in]{fig/pore_cartoon.png}
    \caption{
        Schematic of colloid diffusive transport through a pore filled with a polymer brush. 
        The brush is formed by linear polymer chains (red strands) with a degree of polymerization $N$, uniformly grafted with grafting density $\sigma$ 
        to the inner surface of a cylindrical pore.
        The pore radius is $r_{\text{p}}^0$ (not indicated) and the thickness of the impermeable membrane is $L_0$.
        Polymer chains are flexible with a statistical segment length $a$ and volume $\sim a^3$. 
        Spherical colloids with diameter $d$ are free to diffuse in the surrounding solvent.
        All length scales are normalized by $a$.
        As a model pore, we set $L_0 = 2r_{\text{p}}^0 = 56$, $\sigma = 0.02$ and $N = 300$.
        With $a = 0.76 {\text{ nm}}$, these parameters approximate basic features of a NPC.
        }
    \label{fig:colloid_transport}
\end{figure}

To understand how the polymer brush affects transport, we consider the stationary diffusive flux of colloids through the pore and analyze how it depends on the parameters of the pore, the brush, and the colloid.
We consider unidirectional colloid transport driven solely by the concentration difference across the membrane and focus on the fundamental mechanisms of diffusion mediated by colloid-polymer interactions.
Far away from the membrane, the colloid concentrations are set to $c = c_0$ (at $z \to -\infty$) and $c = 0$ (at $z \to +\infty$).


%%%%%%%%%%
\subsection{Stationary flux}
%%%%%%%%%%

Diffusive transport in the presence of the external mean force generated by the position-dependent insertion free energy $\Delta F(\bm{r})$ (in units of thermal energy $k_{\text{B}}T$) is described by the Smoluchowski equation,
\begin{equation}
    \frac{\partial c(\bm{r})}{\partial t}=-\nabla \cdot \bigl[D(\bm{r}) \nabla c({\bm{r}})+D(\bm{r})c({\bm{r}})\nabla\bigl(\Delta F(\bm{r})\bigr)\bigr],
    \label{eq:Smoluch}
\end{equation}
where $c(\bm{r})$ is the concentration of the colloid, and $D(\bm{r})$ is its position-dependent diffusion coefficient.
Standard substitution introduces the potential function $\psi(\bm{r})$ and the effective conductivity $\tilde{D}(\bm{r})$ as
\begin{equation}
     \psi(\bm{r})=c(\bm{r})\exp(\Delta F(\bm{r}))
     \label{eq:psi} 
\end{equation}
\begin{equation}
    \tilde{D}(\bm{r})=D(\bm{r}) \exp(-\Delta F(\bm{r})) 
    \label{eq:D_tilde}
\end{equation}
With these substitutions, the flux density is expressed as
\begin{equation}
    \bm{j}=- \tilde{D}(\bm{r})  \nabla \psi(\bm{r})
    \label{eq:flux_psi}
\end{equation}
Under stationary conditions, Eq.~\eqref{eq:Smoluch} is reduced to
\begin{equation}
    \nabla \cdot \left(\tilde{D}(\bm{r})\nabla\psi(\bm{r}) \right)=0
    \label{eq:Laplace_modif}
\end{equation}
The boundary conditions for the potential function in our case are $\psi(z\rightarrow -\infty)=c_0$ and $\psi(z\rightarrow +\infty)=0$ since the insertion free energy $\Delta F$ vanishes far away from the pore.
The normal component of the flux density vanishes at the impenetrable membrane walls.
Once the insertion free energy field $\Delta F(\bm{r})$ and the position-dependent diffusion coefficient $D(\bm{r})$  are defined, a natural approach to find the total flux (resistance) is by direct numerical solution of Eqs.~(\ref{eq:D_tilde}-\ref{eq:Laplace_modif}) which we did as a check (see below).
However, a more transparent and instructive approach is to follow an electric analog of the problem which can be reformulated as finding the total resistance of a medium with position-dependent resistivity possessing axial symmetry:

\begin{equation}
    \rho(r,z)= \tilde{D}^{-1}(r,z)
    \label{eq:rho}
\end{equation}
The variation of the local resistivity $\rho(r,z)$ is due to the position-dependent polymer volume fraction $\phi(r,z)$


%%%%%%%%%%
\subsubsection{Bare pore as a reference case}
%%%%%%%%%%

A natural reference is the diffusive flux through a bare pore, which itself limits the transport of solutes \cite{Deen1987, Sun2024}.
The insertion free energy is $\Delta F=0$ everywhere, so that the potential function $\psi(\bm{r})$ coincides with $c(\bm{r})$, and the diffusion coefficient is position independent.
Lord Rayleigh analyzed the flux of solute particles of negligible size through a circular pore in a planar membrane of negligible thickness \cite{Strutt1878}.
In this simplest case, the equiconcentration surfaces are oblate spheroids, and the streamlines form confocal hyperboloids of revolution \cite{Cooke1966}.
The net flux through the pore is
\begin{equation}
    J=2D_0r_{\text{p}}\Delta c,
    \label{eq:flux_Ral}
\end{equation}
where $r_{\text{p}}$ is the pore radius and $D_0 = k_{\text{B}}T / (3 \pi \eta_{\text{S}} d)$ is the diffusion coefficient of the colloid in plain solvent with viscosity $\eta_{\text{S}}$.
Negligible colloid size means $d\ll r_{\text{p}}$.

A membrane of finite thickness $L$ allows the approximate analytical solution \cite{Brunn1984}
\begin{equation}
    J=\frac{2 D_0 r_{\text{p}}}{1+\cfrac{2L}{\pi r_{\text{p}}}}\Delta c.
    \label{eq:flux_finlength}
\end{equation}

Introducing the resistance $R$ to colloid flow $J = \frac{\Delta c}{R}$ provides a natural interpretation of Eq.~(\ref{eq:flux_finlength}) in terms of the total resistance of the pore:
\begin{equation}
    R = \frac{L}{D_0 \pi r_{\text{p}}^{2}} + \frac{1}{2 D_0 r_{\text{p}}} = R_{\text{int}}^{0} + R_{\text{ext}}^{0},
    \label{eq:resistance}
\end{equation}
where the superscript '0' refers to the bare pore.
The first term in Eq.~(\ref{eq:resistance}) represents the resistance of the pore interior.
The second term is the Rayleigh resistance (Eq.~(\ref{eq:flux_Ral})) and accounts for the effects of convergent transport toward the pore entrance and its symmetric counterpart at the pore exit.
Inside the pore, the flux lines are approximately axial.
In the bare pore scenario, the inverse of the diffusion constant ($\rho_0=D_0^{-1}$) represents the resistivity of the medium both inside and outside the pore.
Naturally, the resistance for a thin membrane is determined by the exterior region ($R \approx R_{\text{ext}}^{0}$ for $L \ll r_{\text{p}}$), whereas for long pores the inner region becomes dominant ($R \approx R_{\text{int}}^{0}$ for $L \gg r_{\text{p}}$).

The finite size of colloids affects the diffusive flux in two ways.
First, the excluded volume reduces the effective pore radius ($r_{\text{p}} = r_{\text{p}}^0 - d/2$) and increases the effective pore length ($L \approx L_0 + d$) \cite{Brenner1977}.
Second, when the distance to the pore wall is comparable to the size of the colloid, some additional drag appears\cite{Faxen1922, Haberman1958}.
We neglect the latter as the presence of the polymer brush screens hydrodynamics and thus requires a different kind of drag analysis, as discussed below.


%%%%%%%%%%%%%%%%%%
\subsubsection{Internal vs. external resistances}
%%%%%%%%%%%%%%%%

In the general case of position-dependent resistivity $\tilde{D}(r,z)$, we consider a set of approximate equipotential surfaces ($\psi(r,z)=\text{const}$).
Analogously to a set of resistors connected in parallel, the total conductivity of a layer between two adjacent equipotential surfaces is obtained by integrating local conductivity over the layer.

Inside the pore, $|z| \leq L/2$, the surfaces are taken as discs of radius $r_{\text{p}}$ normal to the pore axis, and the conductivity is
\begin{equation}
\varrho_{\text{int}}^{-1}(z)= 2\pi\int_{0}^{r_{\text{p}}^{}} \rho^{-1}(r,z) r \, dr
\label{eq:varrho1}
\end{equation}

Outside the pore, $|z| > L/2$, we use oblate hemispheroids taken from the Rayleigh solution \cite{Strutt1878} (Figure~\ref{fig:integration_scheme}), and
\begin{equation}
    \begin{gathered}
        \varrho_{\text{ext}}^{-1}(z)= 2\pi\int_{0}^{r_{\text{p}}^{}} \rho^{-1}\left( r'(r,z), z'(r,z) \right)  \tilde{h} (r,z) dr\\
        r'(r,z) = r\sqrt{1 + \frac{(z - L/2)^2}{r_{\text{p}}^2}}\\
        z'(r,z) = (|z| - L/2) \frac{\sqrt{r_{\text{p}}^2 - r^2}}{r_{\text{p}}} +  \text{sign}(z) \frac{L}{2}\\
        \tilde{h} (r,z) = h_r h_{\theta} h_z^{-1} = \dfrac{r}{r_{\text{p}}}\dfrac{r_{\text{p}}^2 + (|z|-L/2)^2}{\sqrt{r_{\text{p}}^2 - r^2}}
    \end{gathered}
\label{eq:varrho2}
\end{equation}
where the functions $r'(r,z)$ and $z'(r,z)$ parameterize the equipotential surfaces, and $h_r$, $h_{\theta}$ and $h_z$ are corresponding Lam\'e coefficients (Supplementary Note 2). 
For the homogeneous brush considered here, the function $\varrho_{\text{ext}}^{-1}(z)$ is even.

Since the consecutive layers are connected in series, their total resistance is found by integration:
\begin{equation}
    R_{\text{int}} = \int_{-L/2}^{+L/2}\varrho_{\text{int}}(z) dz,
    \label{R_int}
\end{equation}
\begin{equation}
   R_{\text{ext}} =2\int_{+L/2}^{+\infty}\varrho_{\text{ext}}(z)dz
    \label{R_ext}
\end{equation}
For a bare pore without a polymer brush ($\tilde{D}(r,z)=D_0$), the integrations recover Eq.~\ref{eq:resistance}, as expected.

\begin{figure}
    \centering
    \includegraphics[width=3in]{fig/resistitance_integration_miniature.png}
    \caption{
    Integration scheme for the layer resistance $\varrho_{\text{int}}\text{d}z, \varrho_{\text{ext}}\text{d}z$.  
    The intrinsic orthogonal curvilinear coordinates are defined by equipotential surfaces, \mbox{$\psi = \text{const}$}, and the flux lines of the flux density field $\bm{j}$.
    A layer between two adjacent equipotential surfaces with a thickness $\text{d}z$ along the central axis has resistance $\varrho_{\text{int}}\text{d}z$ inside the pore ($|z| \leq L/2$, blue shading) and $\varrho_{\text{ext}}\text{d}z$ outside the pore (red shading).  
    The red rectangle illustrates the local conductivity element at $(r',z')$ within an outside layer.  
    The parametrization $(r'(r,z),\,z'(r,z))$ traces the integration path and maps the intrinsic coordinates back to the original cylindrical coordinates $(r,z)$, as indicated by the red arrows.
    }
    \label{fig:integration_scheme}
\end{figure}


%%%%%%%%%%
\subsection{Diffusivity and insertion free energy control diffusive transport}
%%%%%%%%%%

Equations \ref{eq:D_tilde} and \ref{eq:rho} indicate that the flux is determined by the local diffusivity $D(r,z)$ and the insertion free energy $\Delta F(r,z)$ of the colloid, which in turn are linked to the spatial distribution of the polymer density.


%%%%%%%%%%
\subsubsection{Polymer density profiles}
%%%%%%%%%%

Conformations adopted by overlapping polymer chains grafted to the pore walls are controlled by the solvent quality, quantified by the Flory-Huggins parameter $\chi_{\text{PS}}$.
Values of $\chi_{\text{PS}}<0.5$ and $\chi_{\text{PS}}>0.5$ correspond to good and poor solvent, respectively, and $\chi_{\text{PS}}=0.5$ represents the ideal (or $\theta$-)solvent.

The polymer density profile $\phi(z,r)$ in the pore was calculated by the two-gradient self-consistent field numerical method of Scheutjens and Fleer (SF-SCF; Supplementary Note 3).
In Figure~\ref{fig:phi_hm_grid}, one can appreciate the expected increase in polymer concentration inside the pore with decreasing solvent quality (increasing $\chi_{\text{PS}}$).
With the selected pore and polymer parameters (Figure~\ref{fig:colloid_transport}), the polymer brush fills the entire pore cross-section within the full range of solvent qualities explored, and colloids need to navigate the polymer meshwork to traverse the pore.
For wider pores, shorter polymers and/or lower grafting densities, an open channel free of polymer may appear in the pore center, as detailed elsewhere~\cite{Ligoure2001,Laktionov2021}; this scenario would result in a different permeation behavior and is not considered here.

\begin{figure}
    \centering
    \includegraphics[width = 3.5in]{fig/phi_hm_grid.png}
    \caption{
    Maps of the polymer volume fraction $\phi(r,z)$ for a polymer brush in a cylindrical pore with solvent quality ranging from good (upper left panel) to poor (lower right panel), quantified by the Flory-Huggins parameter $\chi_{\text{PS}}$ as indicated.
    Polymer volume fractions are mapped in cylindrical coordinates (as shown by $rz$-coordinate arrows), color coded as indicated and with selected iso-concentration lines. The blank space corresponds to pure solvent; the membrane is colored green.
    Pore and brush parameters are as given in Figure~\ref{fig:colloid_transport}.
    }
    \label{fig:phi_hm_grid}
\end{figure}

Figure~\ref{fig:phi_hm_grid} further illustrates that while the brush remains confined within the pore lumen in poor solvent ($\chi_{\text{PS}}=0.9$) it protrudes substantially into the surrounding space in ideal and good solvents ($\chi_{\text{PS}}\le0.5$).
The polymer brush will therefore affect the resistance to colloid flow outside the pore as well:
\begin{equation}
    R=R_{\text{int}}+R_{\text{ext}},
    \label{eq:R_tot_tot}
\end{equation}
with $R_{\text{int}}\rightarrow R_{\text{int}}^{0}$ and $R_{\text{ext}}\rightarrow R_{\text{ext}}^{0}$ in the limit of the bare pore.


%%%%%%%%%%
\subsubsection{Leading contributions to the insertion free energy}
%%%%%%%%%%

The position-dependent insertion free energy $\Delta F(r,z)$ is the work required to move a colloid from the exterior solution into the polymer brush.
For colloids that are significantly smaller than the pore, wall effects can be neglected, and the insertion free energy is determined entirely by the local polymer concentration:
\begin{eqnarray}
    \Delta F = \Delta F_{\text{osm}} + \Delta F_{\text{sur}},
    \label{eq:Delta_F}
    \\
    \Delta F_{\text{osm}}(r,z) = \int_{V} \Pi(r',z') dV', \nonumber
    \\
    \Delta F_{\text{sur}}(r,z) = \oint_{S} \gamma (r',z') dS'. \nonumber
\end{eqnarray}
The coordinates $(r,z)$ refer to the colloid center, and the two contributions are obtained by integrating over the colloid volume $V$ and surface $S$, respectively.

The osmotic contribution, $\Delta F_{\text{osm}}$, accounts for the work against excess osmotic pressure upon insertion of the colloid into the brush.
The local osmotic pressure (normalized by $k_\text{B} T$) is calculated from the local polymer concentration as
\begin{equation}
    \begin{aligned}
        \Pi(r,z)=  \phi(r,z)\frac{\partial f\{\phi(r,z)\}}{\partial \phi(r,z)} - f\{\phi(r,z)\}= 
        \\
        [-\ln(1-\phi(r,z)) - \phi(r,z) -\chi_{\text{PS}}\phi^2(r,z)],
    \end{aligned}
    \label{eq:osmotic}
\end{equation}
where 
$$
f\{\phi(r,z)\}=(1-\phi(r,z))\ln(1-\phi(r,z)) +\chi_{\text{PS}}\phi(r,z)(1-\phi(r,z))
$$
is the mean-field Flory expression for the interaction free energy per unit volume of the polymer solution of concentration (volume fraction) $\phi(r,z)$.

The interfacial contribution, $\Delta F_{\text{sur}}$ comprises the surface tension $\gamma (r,z)$ as the change in the free energy of a unit area of the colloid upon replacement of a contact with solvent by a polymer solution of concentration $\phi(r,z)$. $\gamma$ is approximated as
\begin{gather}
     \gamma (r,z)= \frac{1}{6}(\chi_{\text{ads}} - \chi_{\text{ads}}^{\text{crit}})\phi^{\ast}(r,z),
    \label{eq:chi_ads} 
    \\
    \text{with } \chi_{\text{ads}} = \chi_{\text{PC}} - \chi_{\text{PS}}(1-\phi^{\ast}), \text{ and } \phi^{\ast}(r,z)= (b_{0} + b_{1}\chi_{\text{PC}})\phi(r,z).
    \nonumber
\end{gather}
The coefficients $b_0 = 0.7$ and $b_1 = -0.3$ are phenomenological parameters accounting for polymer depletion or accumulation near the colloid, and can be treated as constant to a good approximation (Supplementary Note 4).

We applied an approximate analytical scheme (Supplementary Note 5) to evaluate the insertion free energy $\Delta F(r,z)$ as $\Delta F\{\phi(r,z)\}$, where $\phi(r,z)$ is the polymer density distribution in a colloid-free brush calculated with the SF-SCF approach (Supplemenatary Note 3).
The colloid is thus considered as a 'probe' which does not perturb the global polymer concentration distribution $\phi(r,z)$.
This scheme provided insertion free energies at any position, including off the pore axis.
Comparison with direct SF-SCF calculations for colloids on the pore axis demonstrated good quantitative agreement (Supplementary Note 4), justifying the use of the more versatile analytical scheme.


%%%%%%%%%%
\subsubsection{Local colloid mobility}
%%%%%%%%%%

The crowded polymers naturally decrease the diffusion of colloids.
A polymer brush is effectively described as an inhomogeneous semi-dilute polymer solution with a concentration-dependent correlation length (mesh size) $\xi(\phi)$.
Colloids of size $d > \xi$ experience additional friction as they are trapped by the polymer meshwork.
As a result, diffusion is slowed compared to pure solvent, leading to a position-dependent diffusion coefficient $D(r,z) < D_0$.

According to a scaling theory by Cai et al. \cite{Cai2011} for diffusion of non-sticky colloids in a semi-dilute polymer solution, the colloid mobility scales as $D\sim D_0 (\xi/d)^2\ll D_0$ for $d\gg \xi$, while small colloids diffuse virtually unimpeded ($D\sim D_0$ for $d\ll \xi$). We here use a simple interpolation formula to capture the diffusion coefficient across the full range of relevant colloid sizes relative to the correlation length $d / \xi$:
\begin{equation}
    D\{\phi(r,z)\} = \frac{D_0}{1+[\beta d / \xi\{\phi(r,z)\}]^2} \approx \frac{D_0}{1+[\beta d \phi(r,z)]^2} .
    \label{eq:Rubinstein}
\end{equation}
The correlation length $\xi$ in Eq.~(\ref{eq:Rubinstein}) is controlled by the local polymer concentration $\phi(r,z)$ and also depends on the solvent quality \cite{DeGennes1979}.
On the right hand side we approximated $\xi \cong \phi^{-1}$, valid close to ideal solvent conditions in a mean-field regime.
The coefficient $\beta$ in Eq.~(\ref{eq:Rubinstein}) is a numerical pre-factor.
By comparing our theoretical stationary flux predictions to experimental data on the flux of non-sticky colloids of different sizes through nuclear pore complexes (see below), we estimate $\beta = 5.5$.
In the following sections we consistently use this value for numerical estimates of the pore permeability and permselectivity.

Recent theoretical, experimental and simulation studies extended the scaling approach of Cai et al. \cite{Cai2011} to include the effects of polymer-colloid attraction \cite{Yamamoto2018, Carroll2018}.
In the limit where proper diffusion of polymer chains is negligible (relevant to our case of pore-anchored chains), the characteristic desorption time of a single polymer-colloid contact $\tau_\text{des}(\epsilon)$ appears as an extra timescale.
Here, $\epsilon$ is the absolute value of the activation energy to break a contact.
The polymer-colloid attraction may be considered weak if the desorption time is much smaller than the self-diffusion time.
Typically, this happens when $\epsilon \lesssim k_\text{B} T$, and in this case attractive forces only modify local friction by affecting the local packing of polymers around the colloid.
The magnitude of this effect is estimated to be small, reducing the diffusivity by up to a few 10\% \cite{Yamamoto2011}.
On the other hand, the desorption time grows exponentially with $\epsilon$ and affects the diffusion substantively when $\epsilon$ amounts to several $k_\text{B} T$ units.
Numerical simulations demonstrated that the colloid diffusivity is reduced by \cite{Yamamoto2018}

\begin{equation}
    D(\epsilon)\approx D(\epsilon=0) \exp (-\epsilon / 2).
    \label{eq:Yamamoto}
\end{equation}

In our model, the energy of a polymer-colloid contact is $\chi_{\text{PC}}/6$, and therefore $\epsilon < 1/3$ within the explored range $0\geq\chi_{\text{PC}}\geq-2$.
Throughout most of the paper, we assume that the surface of the colloid is homogeneous, and neglect the small reduction in the local colloid mobility due to weak polymer attraction.
However, we will return to this question in the context of proteins traversing the nuclear pore, where the colloid surface and the polymers may be heterogeneous with affinity localized in a few sticky patches.
We can explore the limiting case where all the (negative) surface free energy $\pi d^2 \gamma(r,z)$ is assigned to a single sticky patch.
The diffusion coefficient is then modified as
\begin{eqnarray}
    D_{\text{sticky patch}} = 
    \begin{cases}
        D \exp(\frac{\pi d^2 \gamma(r,z)}{2}) & \text{if } \gamma(r,z) < 0 \\
        D & \text{otherwise}
    \end{cases}
     \label{eq:Sticky diff}
\end{eqnarray}
to interpolate between non-sticky and sticky colloids.
It is safe to assume that realistic cases of attractive polymer-colloid interaction lie between the two extremes of a perfectly homogeneous surface ($D_\text{homo} \approx D$) and a single sticky patch (Eq.~(\ref{eq:Sticky diff})).

Several other theoretical and empirical models have been proposed to describe the diffusion of colloids in polymer meshworks \cite{Schweizer2003,Kohli2012,Holyst2009,Phillies1988}.
Although the predictions of different models differ quantitatively, they all share the same qualitative trend.

\begin{figure}
    \centering
    %\REMOVE \CENTERLINE BEFORE SUBMISSION
    \centerline{\includegraphics[width = 6.5in]{fig/diffusivity_and_fe.png}}
    \caption{%
        Effect of the polymer meshwork on local diffusivity and insertion free energy.  
        \textbf{(a)}  Comparison of the reduced colloid diffusivity caused by the polymer meshwork ($-\ln(D/D_{0})$; black line) with the insertion free energy (colored lines, labelled by $\chi_{\text{PC}}$) as functions of the local polymer volume fraction~$\phi$.  
        Both quantities share the same vertical scale  so that the sum of the black and one of the colored curves,  $\ln(\rho D_{0}) = -\ln(D/D_{0}) + \Delta F$, quantifies the total local resistivity; the horizontal zero line corresponds to unimpeded transport as in pure solvent; positive values of the colored curves indicate resistance enhanced by the free energy contribution, whereas negative values indicate lowered resistance.
        \textbf{(b)}  Spatial map of the position-dependent diffusivity, $-\ln\bigl[D(\phi(r,z))/D_{0}\bigr]$, in cylindrical coordinates.  
        \textbf{(c)}  Maps of the insertion free energy $\Delta F(r,z)$ for polymer-colloid interaction strengths ranging from $\chi_{\text{PC}}=-0.50$ (weakest attraction) to $\chi_{\text{PC}}=-1.25$ (strongest attraction), as indicated.
        Pore and brush parameters are as in Figure~\ref{fig:colloid_transport}; $\chi_{\text{PS}}=0.5$ and $d=8$.
    }
    \label{fig:D_fe_map}
\end{figure}


%%%%%%%%%%
\subsubsection{Resistivity maps derived from polymer density}
%%%%%%%%%%

Following Eq.~(\ref{eq:rho}), the local resistivity anywhere in the polymer phase can be expressed as the sum of diffusivity and insertion free energy contributions, $\ln[\rho(\phi)D_{0}] = -\ln[D(\phi)/D_{0}] + \Delta F(\phi)$.
Figure~\ref{fig:D_fe_map}a illustrates the distinct dependencies of these two contributions on the polymer volume fraction: while $-\ln[D(\phi)/D_{0}]$ increases monotonically (black line), $\Delta F(\phi)$ can be non-monotonic (colored lines).
For sufficiently attractive colloids ($\chi_{\text{PC}} < \chi_{\text{ads}}^{\text{crit}} + \chi_{\text{PS}}$), $\Delta F(\phi)$ transits from an attractive ($\Delta F<0$) to a repulsive ($\Delta F>0$) regime at a finite $\phi$ where the interfacial gain $\Delta F_{\text{sur}}$ balances the osmotic penalty $\Delta F_{\text{osm}}$.
Inert or weakly attractive colloids ($\chi_{\text{PC}} \geq -0.5$) always have $\Delta F>0$ and are thus repelled.  

Figure~\ref{fig:D_fe_map}b-c illustrates the spatial distribution of the diffusivity and free energy contributions, respectively, for a colloid of size $d=8$ in an in an ideal solvent ($\chi_{\text{PS}}=0.5$) and a set of polymer-colloid interaction strengths ($\chi_{\text{PC}}$).
Because the polymer concentration is highly inhomogeneous, $\Delta F(r,z)$ varies strongly: for example, at $\chi_{\text{PC}}=-0.75$ the pore fringes are attractive while the dense interior is repulsive.  
Thus, diffusivity and free energy, and consequently the local resistivity, are spatially resolved functions of the polymer volume fraction.


%%%%%%%%%%
\subsection{An attractive polymer filling enhances colloid fluxes through the pore}
%%%%%%%%%%

Figure~\ref{fig:R_vs_chi_PC} visualizes the relative contributions of the pore interior ($R_{\text{int}}$) and exterior ($R_{\text{ext}}$) to the total resistance as a function of the polymer-colloid attraction $\chi_\text{PC}$, for a selected colloid size ($d = 12$) in a good ($\chi_\text{PS} = 0.3$) and a poor ($\chi_\text{PS} = 0.7$) solvent.
A comparison of the total resistance obtained by direction numerical solution of the Smoluchowski equation (diamond symbols; Supplementary Note 6) here validates the accuracy of the analytical scheme (solid lines).

A striking feature is that attracted colloids can achieve diffusive fluxes that exceed the limit of the bare pore, as indicated by the segments of the $R(\chi_{\text{PC}})$ curves that fall below the solid horizontal red line marking the bare-pore resistance $R_{0}$.
This result may at first appear surprising, given that the polymer medium is expected to slow down the diffusion of colloids.
However, this slowing down is counteracted by the attractive potential of the polymer meshwork, which reduces local resistivity according to the exponential factor in Eq.~(\ref{eq:rho}).

The reduced local resistivity has pronounced consequences for diffusive transport in both the interior and the exterior of the pore.
Figure~\ref{fig:R_vs_chi_PC} illustrates that the interior resistance $R_{\text{int}}$ can be driven practically to zero by increasing the polymer-colloid attraction (decreasing $\chi_\text{PC}$) below a certain threshold.
Compared to a bare pore (Eq.~(\ref{eq:resistance})) such a short-circuiting effect entails a reduction in the resistance by a factor of up to $R^0_{\text{int}}/R^0_{\text{ext}}+1 \approx 2/\pi \times L / r_{\text{p}} + 1$.
For the pore and colloid considered here, this represents an approximately 3-fold reduction, to a level marked by the dashed horizontal red line in Figure~\ref{fig:R_vs_chi_PC} which equals the contribution of convergent flux to the bare-pore resistance ($R^0_\text{ext}$).
The reduction would be even stronger for longer pores ($L\gg r_p$), and for larger colloids that increase the effective pore length and decrease the effective pore diameter.

\begin{figure}
    \centering
    \includegraphics[width = 3in]{fig/resistitance_components.png}
    \caption{
    Interior ($R_{\text{int}}$), exterior ($R_{\text{ext}}$) and total ($R = R_{\text{int}} + R_{\text{ext}}$) resistance vs. polymer-colloid interaction strength $\chi_{\text{PC}}$ for a good solvent ($\chi_{\text{PS}} = 0.3$, in blue) and a poor solvent ($\chi_{\text{PS}} = 0.7$, in orange), obtained throught the analytical scheme.
    The exterior $R_{\text{ext}}^{0}$ and total $R_0$ resistances for the bare pore are also shown (red lines of matching type).  
    The resistances are presented in dimensionless units $R\tfrac{k_{B}T}{\eta_{\text{S}}}$; pore and brush parameters are as given in Figure~\ref{fig:colloid_transport}; $d = 12$.
    The results of direct numerical solution of the Smoluchowski equation are shown with diamond symbols.
    }
    \label{fig:R_vs_chi_PC}
\end{figure}

The exterior resistance $R_{\text{ext}}$, on the other hand, always retains a finite contribution from the diffusive fluxes in the semi-infinite reservoir, setting an absolute lower bound to the total resistance.
The reduction of $R_{\text{ext}}$ below $R^0_\text{ext}$ evidenced in Figure~\ref{fig:R_vs_chi_PC} is due to attractive brush fringes that protrude and facilitate diffusive transport outside the pore.
Approximating the brush fringes on either end of the pore as hemispherical caps with radius $r_\text{cap}$, the plateau conditions are equivalent to the resistance of an ideally absorbing sphere \cite{Crank1980},
\begin{equation}
    R_\text{ext}^\text{\text{min}} = 1 / (D_0 \pi r_\text{cap}).
    \label{eq:R_ext_min}
\end{equation}

Attractive brush fringes thus entail a reduction in resistance by a factor of up to $R_\text{ext}^0 / R_\text{ext}^\text{\text{min}} = \pi/2 \times r_\text{cap}/r_\text{p}$.
In good solvent ($\chi_\text{PC} = 0.3$), for example, the cap radius (along the pore axis) is comparable to the pore diameter (Figure~\ref{fig:phi_hm_grid}), leading to a 3.3-fold reduction of the external resistance, and a cumulative 10-fold reduction of the total resistance, compared to the bare pore (Figure~\ref{fig:R_vs_chi_PC}).
The cap size shrinks as the solvent quality decreases (Figure~\ref{fig:phi_hm_grid}), with a correspondingly reduced benefit on pore conductivity, as illustrated for $\chi_\text{PC} = 0.7$.
For even poorer solvents, the cap and its benefit disappear entirely ($R_\text{ext} = R_\text{ext}^0$; not shown).


%%%%%%%%%%
\subsection{Polymer-filled mesopores effectively gate colloids by their attraction to the polymer}
%%%%%%%%%%

Figure~\ref{fig:R_vs_chi_PC} also illustrates how the total resistance of the pore varies with the colloid's affinity to the polymer brush.
As expected, increasing the polymer-colloid attraction strength (i.e., more negative $\chi_{\text{PC}}$) results in a monotonic decrease in the pore's total resistance, since the interfacial term in the insertion free energy becomes more negative, thereby increasing the local conductivity $\rho^{-1}$.

Most notable is a sharp transition from a regime of facilitated permeation ($R < R_0$) to a regime of impeded permeation ($R > R_0$).
The regime of impeded permeation is dominated by the internal resistance.
It exhibits high selectivity with respect to the polymer-colloid interaction strength, and a mostly very high total resistance and thus low colloid flux, both appreciable in Figure~\ref{fig:R_vs_chi_PC} as a sharp increase in $R$ over a relatively modest $\chi_{\text{PC}}$ range.
In contrast, the region of facilitated permeation is dominated by the external resistance. 
It exhibits high colloid fluxes but rather low (if any) $\chi_{\text{PC}}$ selectivity, as demonstrated by the previously analyzed plateau.
Thus, the transition between the two regimes of transport defines the condition for sharp colloid gating, with remarkably efficient transport in the regime limited by external resistance and effective blockage in the regime limited by internal resistance.

In this context, the solvent quality can be seen as a regulator of the polymer-colloid interaction level for gating. 
Lowering the solvent quality (increasing $\chi_{\text{PS}}$) reduces $\chi_{\text{ads}}$ and shifts the entire $R(\chi_{\text{PC}})$ curve toward larger $\chi_{\text{PC}}$ values.
Thus, a poorer solvent extends the range of facilitated permeation towards more weakly interacting colloids.

The here-presented trends are qualitatively correct also for colloids with sizes smaller or larger than the $d=12$ considered here.
Naturally, the gating effect will be rather moderate for small colloids, yet even sharper for larger colloids.


%%%%%%%%%%
\subsection{High colloid flux implies colloid enrichment in the pore}
%%%%%%%%%%

Colloid concentration profiles under stationary flux conditions can be found by numerically solving Eq.~(\ref{eq:Smoluch}) with $\frac{\partial c(r,z)}{\partial t} = 0$ (Supplementary Note 6).
Figure~\ref{fig:colloid_concentration} maps the steady-state colloid concentration across a polymer-filled pore for a colloid of size $d = 8$ and polymer-colloid interaction strength $\chi_{\text{PC}} = -1.25$ in an ideal solvent ($\chi_{\text{PS}} = 0.5$).
This condition correspnds to transport rates somewhat inferior to the bare pore ($R \approx 5 R_0$), compared to same-size but inert colloids where the resistance increases by orders of magnitude ($R(\chi_\text{PC}=0) \approx 10^4 R_0$).

\begin{figure}
    \centering
    %\REMOVE \CENTERLINE BEFORE SUBMISSION
    \centerline{\includegraphics[width=5.5in]{fig/streamlines.png}}
    \caption{
    Color map of the steady-state colloid concentration, normalized by the bulk concentration in the source compartment $c_0$.
    Isoconcentration contours are shown as labeled.
    The flux is represented by streamlines marked with small arrows, indicating the average colloid trajectory.
    Pore and brush parameters are the same as in Figure~\ref{fig:colloid_transport}; $d = 8$, $\chi_{\text{PC}} = -1.25$ and $\chi_{\text{PS}} = 0.5$.
    The physical pore radius ($r_\text{p}^0$) and length ($L_0$), along with their effective counterparts due to colloid excluded volume ($r_\text{p}$ and $L$, respectively), are shown with white arrows.
    The inset (top right) shows a part of the corresponding $\psi$ potential map.
    }
    \label{fig:colloid_concentration}
\end{figure}

The map illustrates several salient features of the diffusion process.
Inside the pore, the flux lines remain nearly parallel to the pore axis.
Outside the pore and polymer fringes, the concentration rapidly approaches the bulk value of each semi-infinite reservoir.
Also, the corresponding map of the potential $\psi$ (Figure~\ref{fig:colloid_concentration}, inset) reveals the same shape of equipotential surfaces as the bare pore (e.g., oblate hemispheroids), further validating the analytical scheme used earlier to evaluate the total resistance.

The most notable observation is that the colloid concentrations substantively exceed $c_0$ near the pore entrance (by a factor of $\sim20$) and inside the pore (by a factor  of $\sim10$).
This effect is caused by the negative insertion free energy in the space occupied by the polymer brush.
At equilibrium (i.e., with vanishing fluxes), the partitioning would amount to $c_{\text{eq}}/c_0 = \exp\left( -\Delta F \right)$.
In the steady state (i.e., with non-vanishing fluxes), the colloid concentration is reduced but approaches the equilibrium concentration as the insertion free energy becomes very negative ($c/c_0 \to c_{\text{eq}}/c_0$).

The presented quantitative results are only valid for sufficiently low bulk concentrations $c_0$, as our model disregards any colloid crowding effects.
Such interactions are expected to be predominantly of excluded-volume (repulsive) nature, which would systematically lower the steady-state colloid concentration inside the brush.


%%%%%%%%%%
\subsection{Polymer-filled mesopores effectively gate colloids by their size}
%%%%%%%%%%

Figure~\ref{fig:R_vs_d} compares how the total resistance, $R=R_\text{int}+R_\text{ext}$, varies with colloid size $d$ for a bare pore (thick black lines) and for polymer-filled pores with selected solvent (Figure~\ref{fig:R_vs_d}a) and polymer-colloid interaction (Figure~\ref{fig:R_vs_d}b) strengths (thin colored lines with symbols).
For small colloids, the bare-pore resistance follows well the $R_0 \sim D_0^{-1} \sim d$ dependence expected according to Eq.~(\ref{eq:resistance}).
A stronger dependence of $R_0$ on $d$ observed for larger colloids is due to decreasing effective pore length $L$ and increasing effective pore radius $r_\text{p}$.

Naturally, the polymer filling affects the transport of the smallest colloids only marginally, as their volume and net interaction strengths (within the considered $\chi_{\text{PC}}$ range) are too small to have any noticeable effect. 
A rich picture emerges for larger colloids, however, with non-monotonic dependencies of the pore resistance on colloid size and strong effects of $\chi_{\text{PC}}$ and $\chi_{\text{PS}}$. 

\begin{figure}
    \centering
    \includegraphics[width = 3.5in]{fig/permeability_on_d.png}
    \caption{
    Normalized dimensionless pore resistance $R\tfrac{k_{B}T}{\eta_{\text{S}}} $ as a function of the colloid size $d$ for \textbf{(a)} selected values of the polymer-colloid interaction strength $\chi_{\text{PC}}$ (as indicated in the legend) at a fixed solvent strength $\chi_{\text{PS}} =0.5$, and \textbf{(b)} selected values of $\chi_{\text{PS}}$ (as indicated in the legend) at a fixed $\chi_{\text{PC}} = -1.25$.
    Pore and brush parameters are as given in Figure~\ref{fig:colloid_transport}. 
    The bare-pore resistance $R_{0}$ is shown by the thick black line.
    Its deviation from simple Stokesian scaling (Eq.~(\ref{eq:resistance}); thin black line) is due to the excluded volume of the colloid.
    }
    \label{fig:R_vs_d}
\end{figure}


%%%%%%%%%%
\subsubsection{Impact of colloid diffusivity within the polymer brush on size-selective transport}
%%%%%%%%%%

The curve with $\chi_{\text{PS}}=0.5$ and $\chi_{\text{PC}} = -1.0$ in Figure~\ref{fig:R_vs_d}a corresponds to the condition of $\Delta F$ fairly vanishing across a wide colloid-size range (Supplementary Note 4 - Figure S4) due to compensation of osmotic and surface contributions to the insertion free energy.
Here, the reduced colloid diffusivity within the polymer meshwork dominates the pore resistance (Eq.~(\ref{eq:rho})). 
This effect alone leads to a monotonic and pronounced increase of $R$ with $d$ with smooth crossover between asymptotic dependencies $R\sim d$ at $d\ll \xi$ and $R\sim d^3$ at $d\gg \xi$. 
Notably, for small and intermediate size colloids the pore resistance slightly grows with inferior solvent quality (increase in $\chi_{PS}$) due to a decrease in the mesh size $\xi$ with concomitant decrease in the local diffusivity, as seen in Figure~\ref{fig:R_vs_d}b.


%%%%%%%%%%
\subsubsection{Impact of insertion free energy on size-selective transport}
%%%%%%%%%%

Since the pore resistance scales exponentially with the insertion free energy ($R \sim D^{-1}\exp (\Delta F)$; Eq.~(\ref{eq:rho})), and $\Delta F =\Delta F_{\text{osm}} + \Delta F_{\text{sur}}$, the dependence of the resistance on colloid size is generally controlled by the interplay between
the osmotic $\Delta F_{\text{osm}} \sim \Pi d^3$ and the interfacial $\Delta F_{\text{sur}} \sim \gamma d^2$ contributions. 
While the osmotic repulsion arising due to the polymer filling always enhances the resistance, the surface contribution may either increase (at $\gamma > 0$) or decrease (at $\gamma<0$) it.

For inert or weakly attractive colloids, $\gamma \geq 0$, the resistance grows monotonically with the colloid size due to the combined effect of a decreasing diffusivity $D(d)$ and an increasing insertion free energy  $\Delta F(d)$.
As both these effects are pronounced, their combination leads to a very strong size selectivity, such that the transport of even rather small colloids is effectively impeded, as can be appreciated for $\chi_{\text{PC}} > -1.0$ in Figure~\ref{fig:R_vs_d}a.
For sufficiently large colloids, the osmotic contribution dominates in the insertion free energy such that $R \sim D^{-1} \exp (\Delta F) \sim d^3 \exp (\Pi d^3)$.

In contrast, for attractive colloids with $\gamma <0$ the dependence of the pore resistance $R(d)$ on colloid size can be non-monotonic, with a local maximum (at $d=d_\text{max}$) followed by a local minimum (at $d=d_\text{min}$).
This is best illustrated in Figure~\ref{fig:R_vs_d} by the orange curves corresponding to $\chi_{\text{PC}} = -1.25$ and $\chi_{\text{PS}}=0.5$.
The local maximum here arises from the net colloid attraction (which decreases resistance) overcoming the decrease in diffusivity (which increases resistance) with increasing colloid size.
The local minimum in turn arises from the positive osmotic contribution to the free energy ($\Delta F_{\text{osm}} \sim \Pi d^3$) overcoming the negative interfacial contribution ($\Delta F_{\text{sur}} \sim \gamma d^2$).

Most notably, the local minimum for attractive colloids is swiftly followed by a sharp increase in resistance (for $d > d_{\text{min}}$) due to the dominant osmotic contribution recovering the strong $R \sim d^3 \exp(\Pi d^3)$ dependence.
The transition between the regime of good to moderate transport (for $d \lesssim d_{\text{min}}$) and the regime of impeded transport (for $d > d_{\text{min}}$) thus defines the condition for sharp gating of attracted colloids by their size.

When the insertion free energy becomes strongly negative, a new regime appears that is characterized by facilitated transport ($R < R_0$) over a rather wide range of colloid sizes, as illustrated in 
Figure~\ref{fig:R_vs_d}a for $\chi_{\text{PC}} \le -1.4$, and in Figure~\ref{fig:R_vs_d}b for $\chi_{\text{PS}} \ge 0.6$.
Here, the pore interior is effectively short-circuited, $R_{\text{int}} \to 0$, and the total resistance is set by the finite exterior contribution, $R \approx R_{\text{ext}}$ (see Eq.~(\ref{eq:R_tot_tot})).
Due to attractive brush fringes at the pore entrance and exit, $R \approx R_{\text{ext}}^{\text{min}}$, leading to a weak size dependence, $R \sim d$ (see Eq.~(\ref{eq:R_ext_min})).

The sharp gating of colloids by their size is preserved, and even enhanced, for strongly attractive colloids.
This is best seen in Figure~\ref{fig:R_vs_d}a for $\chi_\text{PC}= -1.4$, where the osmotic penalty to the insertion free energy takes over, and entails a sharp increase in resistance, above a certain colloid size ($d \approx 24$).


%%%%%%%%%%
\subsection{Experiments of colloid transport through NPCs validate the theoretical predictions}
%%%%%%%%%%

To test our theory, we analyzed literature pertinent to colloid transport across NPCs.
The average distance between NPCs in the nuclear envelope exceeds the pore diameter by almost an order of magnitude \cite{Winey1997, Varberg2022, Maeshima2006}.
Transport across neighbouring NPCs thus is not mutually interfering \cite{Fabrikant1985} (see iso-concentration lines in Figure~\ref{fig:colloid_concentration}). Experimentally measured transport rates $k$, normalized against the number of pores (Supplementary Note 7), therefore can be directly compared with our theoretical predictions.

It is well-known that colloids with affinity for the disordered nucleoporin FG domains that fill the NPC (such as importins and exportins) are enriched in or near NPCs \cite{Beck2007, Lowe2015, Kim2018}, and in microscopic droplets, macroscopic hydrogels and thin films assembled from pure FG domains \cite{Schmidt2015, Zahn2016, Vovk2016, Frey2018}.
Qualitatively, these observations fully align with our predictions that the accumulation of colloids in the pore is required for facilitated transport (Figure~\ref{fig:colloid_concentration}).
We hence tested our predictions quantitatively.


%%%%%%%%%%
\subsubsection{Transport of non-sticky colloids}
%%%%%%%%%%

Literature on the rates of diffusive transport across NPCs for non-sticky proteins \cite{Ribbeck2001, Mohr2009, Popken2015, Timney2016, Frey2018} collectively coveres two orders of magnitude in molecular mass and five orders of magnitude in transport rate.
Figure~\ref{fig:NPC_comparison}a compares these experimental data with the theoretical predictions of our model.
As the example pore geometry and polymer density in Figure~\ref{fig:colloid_transport} were modelled to approximate NPCs (Supplementary Note 1), we can directly compare our theoretical predictions with experimental data.
The effective statistical segment length of disordered polypeptide chains was taken to be $a$ = 0.76 nm \cite{Hoogenboom2021}.
The effective solvent strength in the NPC was estimated to be close to ideal ($\chi_{\text{PS}} = 0.6$), consistent with varying yet generally moderate levels of 'cohesiveness' observed for FG domains \cite{Eisele2013, Zahn2016, Vovk2016, Fuertes2017, Hoogenboom2021}.
We approximated the proteins as perfectly inert colloids ($\chi_{\text{PC}} = 0$).
To match the theoretical colloid volumes to protein molecular masses, we considered the effective density of the protein colloids to be bounded by the densities of aqueous solvent ($\rho_{\text{probe}} \geq \text{1 g/cm}^3$) and pure polypeptide ($\rho_{\text{probe}} \lesssim \text{1.4 g/cm}^3$).
This reflects that an unknown (and possibly variable) amount of solvent contributes to the effective volume of the proteins during their transport across the NPC.
The only adjustable fitting parameter in our model was the prefactor $\beta$ in the scaling-based expression for the diffusion coefficient, Eq.~(\ref{eq:Rubinstein}).

Figure~\ref{fig:NPC_comparison}a demonstrates that the theory reproduces the experimentally observed increase in pore resistance with colloid size very well.
The best fit was obtained with $\beta = 5.5$ and this value was hence fixed  throughout the paper.
The quality of the fit is quite remarkable given the large range of masses and transport rates covered, and the relative simplicity of our theory.
Some scatter in the experimental data is though notable.
This may be due to some proteins not being strictly non-sticky but interacting weakly with FG domains.
Indeed, Frey et al. \cite{Frey2018} reported a three-fold enhanced transport rate of green fluorescent protein over mCherry despite these proteins being of similar molecular mass and considered inert.
Moreover, whilst some studies had washed out cytosolic proteins in their assay (with HeLa cells), thus leaving behind intact nuclear pores filled with a plain FG domain brush but lacking most transport factors \cite{Ribbeck2001, Mohr2009, Frey2018}, others used intact yeast cells with all transport factors present \cite{Popken2015, Timney2016}.
The satisfactory fit across all datasets suggests that the crowding of the NPC with transport factors has a comparatively weak effect on the transport of non-sticky proteins across the NPC, consistent with a moderate effect of transport factor depletion observed in vivo \cite{Kalita2022}.  

\begin{figure}
    \centering
    %REMOVE \CENTERLINE BEFORE SUBMISSION
    \centerline{\includegraphics[width = 6in]{fig/validation.png}}
    \caption{
    Comparison of theoretical predictions with experimental findings for colloid transport rates across NPCs.
    \textbf{(a)} 
    Gating of non-sticky colloids (peptides and globular proteins) by size.
    NPC passage rate $k$ per pore (at $c_0$ = 1 $\mu\text{M}$) vs. molecular mass $M_w$ (symbols) extracted from the literature, as indicated (Table~S1).
    Theoretical predictions (shaded orange area) are for the pore and brush parameters as given in Figure~\ref{fig:colloid_transport}, with $\chi_\text{PC} = 0$, $a = 0.76$ nm, $\chi_\text{PS} = 0.6$, and colloid volumes converted to masses using $\text{1 g/cm}^3 \leq \rho_\text{probe} \leq \text{1.4 g/cm}^3$.
    The best fit, shown here, was obtained with $\beta = 5.5$ in Eq. (\ref{eq:Rubinstein}).
    The predicted transport rates across a bare pore are shown (black line) for comparison.
    \textbf{(b)} 
    Gating of colloids by their affinity to the polymer.
    NPC passage rate $k$ vs. partition coefficient $P$ in phase-separated droplets of pure FG domains (Nup98A - blue lozenges, Nup116 - gray lozenges) measured by Frey et al. \cite{Frey2018} for a range of green fluorescent protein variants and mCherry (Table~S2).
    Theoretical predictions (orange lines) represent the limits of a homogeneously attractive colloid surface (dashed line) and a surface with a single sticky patch (solid line).
    The predicted transport rate across a bare pore is also shown (black line).
    The pore and brush parameters are as given in (a); $\chi_\text{PS} = 0.6$ for both pore and pure FG domain droplets; $\chi_\text{PC}$ values (indicated at the top of the graph) were matched to the partition coefficient $P$, and $d$ = 6 (Supplementary Note 7).
    }
    \label{fig:NPC_comparison}
\end{figure}


%%%%%%%%%%
\subsubsection{Transport of sticky colloids}
%%%%%%%%%%

Frey et al. \cite{Frey2018} additionally quantified NPC transport rates for a wide range of green fluorescent proteins (GFPs) with surface amino acids mutated to modulate transport from 'superinert' to 'transport factor like'.
In parallel, the ability of these variants to enrich or deplete in phase-separated droplets of two pure FG domains (Nup98A from \textit{T. thermophila}, and Nup116 from \textit{S. cerevisiae}) was quantified.
The transport rate was observed to correlate strongly with the level of GFP enrichment in FG domain phases (Figure~\ref{fig:NPC_comparison}b).
This set of experiments enabled the effect of polymer-colloid interaction to be tested selectively as the colloid size and shape were effectively constant.

To reproduce theoretically the correlation between the experimentally measured NPC transport rates $k$ and the partition coefficient $P$ in pure FG domain phases, we assumed an effective solvent strength $\chi_\text{PS} = 0.6$ for the two pure FG domain phases and the NPC. This simplified assumption has only a moderate impact on the predictions (Supplementary Note 7).
Furthermore, the free energy of insertion $\Delta F = -\ln(P)$ is reduced to the surface contribution ($\Delta F = \Delta F_\text{sur}\left(\chi_\text{PS},\chi_\text{PC}\right)$) since the osmotic pressure vanishes on spontaneous phase separation at low polymer concentration (Supplementary Note 7), and provides the link between $\chi_\text{PC}$ and the partition coefficient. 

Our idealised assumption of colloids being homogeneously interactive (Figure~\ref{fig:NPC_comparison}b, dashed orange line) reproduced the experimental data for non-sticky and weakly attractive colloids well without any adjustable parameter.
For more strongly attractive colloids however, this approach overestimated the experimentally observed transport rates.
Assuming the opposite extreme of all surface free energy being concentrated into a single sticky patch (Figure~\ref{fig:NPC_comparison}b, solid orange line) reproduced the experimental data quite well, suggesting that the presence of localized sticky patches on the colloid surface and the FG domains slows down diffusion and transport.

Taken together, the quantitative agreement between our theory and a range of experimental data for NPC passage of proteins with a very limited number of adjustable parameters provides strong validation for the validity of our theory.


%%%%%%%%%%
\section{DISCUSSION}
%%%%%%%%%%

We have shown how mesopores filled with polymer brushes can gate transport with exquisite selectivity with respect to polymer-colloid affinity and colloid size, even for colloids that are substantially smaller than the pore diameter.
A striking finding is that an attractive polymer brush can provide colloid transport rates comparable to, or even exceeding, the bare pore.
Our findings shed light on the basic mechanisms of selective nucleo-cytoplasmic transport and suggest a molecular design strategy for controlling selective permeability through artificial mesoporous membranes.


%%%%%%%%%%
\subsection{Implications for nuclear pore permselectivty}
%%%%%%%%%%

Figure~\ref{fig:NPC_comparison}b indicates that the theoretical limit for the rate of transport of sticky colloids is higher than what may be realised with proteins in nuclear pores.
This finding is intriguing, as it suggests the rate of transport is not the primary performance factor for NPCs.
Arguably, selectivity of transport may be the more important criterion, and the limited biochemical space available for nature to evolve towards high selectivity (whilst maintaining basic properties such as colloidal stability in the cellular milieu) may have come with a tradeoff in terms of rate.

Our model predicts that the highest transport rates are achieved with homogeneously attractive polymers and colloids.
In contrast, each FG domain polymer type exhibits substantive heterogeneity along the chain contour with preferred interaction sites for transport factors.
Similarly, importins, exportins and their cargo display substantive surface heterogeneity and complex, non-spherical shapes.
Whilst NPC transport factors typically feature multiple 'low-affinity' binding sites for FG domains, these sites remain discrete.
Per-site interaction strengths in the lower mM range \cite{Hough2015, Milles2015, Hoogenboom2021}, equivalent to unbinding free energies of $\sim 5 k_\text{B}T$, can reduce the diffusivity by an order of magnitude compared to homogeneously attractive colloids (Eq.~(\ref{eq:Sticky diff})).
The discreteness of interactions thus is a plausible candidate for the reduced transport rates in NPCs. 

NPCs feature a variety of nucleoporin FG domains, with the body of available structural and biochemical data suggesting that the cohesiveness of nucleoporin FG domains is highest in the centre and decreases towards the periphery of the pore \cite{Hoogenboom2021, Ng2023}.
Qualitatively, one can envisage that the increased solubility of peripheral FG domains promotes a more extended polymer cap, thus minimising total pore resistance and maximising transport rates for strongly attractive colloids.
The reduced solubility of the central FG domains, on the other hand, would minimize the size threshold for gating of non-adhesive colloids.
Moreover, the accumulation of attractive colloids in the pore may also modulate their transport rates \cite{Zheng2023}.
These features clearly are not essential for permselective transport through polymer-filled mesopores, but may further enhance selectivity or rates.
Our model may be further extended to incorporate solubility gradients and to explore such phenomena in more detail.

%%%%%%%%%%
\subsection[Towards technological applications of synthetic polymer-filled mesopores]{Towards technological applications of synthetic \\ polymer-filled mesopores}
%%%%%%%%%%

Our predictive theoretical approach paves the way for the rational design of nanoporous materials with enhanced selectivity tailored to specific functional requirements, sought after for applications in nanomedicine, biotechnology, and environmental engineering.
Mixtures of biological colloids such as folded proteins and other biomacromolecular complexes, as well as synthetic colloids such as nanoparticles, may be effectively separated, not only according to their size but also their surface (bio-)chemistry.

Individual pores, as considered here, are routinely deployed in current nanopore sensing technologies.
These technologies enable detection and characterization of individual macromolecules as they travel across the pore.
Our findings suggest polymer fillings as an attractive tool to optimize the performance of nanopore sensing.
Placing a suitable polymer filling upstream the pore's sensing region would enable pre-selection of target solutes from complex mixtures for a focused analysis by the pore.
Polymer fillings may also be placed in the very sensing region of the pore to enhance both selectivity and sensitivity.

Individual pores will though typically be insufficient in applications that focus on high-throughput separation (such as filtration systems) or delivery (such as porous particles).
This limitation can be overcome by multiplexing, e.g., with materials featuring a large array of mesopores.
Our theoretical approach remains valid for such arrays as long as the distance between pores remains sufficiently large for the diffusion trajectories of adjacent pores not to substantially interfere.
Fortunately, this condition can be met with a relatively tight packing of pores (see iso-concentration lines in Figure~\ref{fig:colloid_concentration}) \cite{Fabrikant1985}.
An avenue not considered here but worthy exploring to increase transport rates further is a pressure gradient that drives solution flow across the mesoporous membrane.

\bigskip

\noindent{The main design concepts emerging from our theory are:}

\textbf{1.}
To provide selective transport, high permeation selectivity must be coupled with low resistance to diffusive flux.
We refer to this combination as 'gating' behaviour, where a minor change in colloid size (Figure \ref{fig:R_vs_d}) or polymer-colloid interaction strength (Figure \ref{fig:R_vs_chi_PC}) dramatically shifts the permeation rate from facilitated transport to virtually complete blockage.
The thresholds for gating can be tuned by the polymer-solvent interaction strength, and gating is more pronounced for larger colloids.

\textbf{2.}
The maximal permeability is limited by the resistance of the exterior region.
Whilst strong  polymer-colloid attraction can make the resistance of the pore interior effectively vanish ($R_{\text{int}} \to 0$), mass transport in plain solvent always provides a non-vanishing resistance of the exterior.
Polymer fringes of radius $r_\text{ext}$ at the pore entrance and exit decrease the path through plain solvent, and can reduce the external resistance by a factor of up to $\frac{\pi r_{\text{ext}}}{2 r_{\text{p}}}$.

\textbf{3.}
Pore resistance is highly sensitive to parameters that influence the insertion free energy.
The osmotic contribution to the insertion free energy scales as $d^3$ while the interfacial contribution comprises $\chi_{\text{PC}}$ and scales as $d^2$.
Thus, a slight change in $d$ and/or $\chi_{\text{PC}}$ translates into a drastic change in permeability.

\textbf{4.}
A homogeneous polymer-colloid interaction is preferable over one or multiple distinct binding sites to maximise the diffusivity of sticky colloids in the polymer phase, and thus the overall transport rate.

\bigskip

\noindent{The manufacturing of functional mesoporous membranes is an emerging art \cite{Uredat2024, Pardehkhorram2022}, and we hope that our theoretical efforts will both promote and guide future practical developments in this area.}


%%%%%%%%%%
\subsection*{Data availability}
%%%%%%%%%%

Source data are provided with this paper.
\todo{RR: We shall need to supply the data presented in figures in numerical form. Can we do this as a single Excel file, with one worksheet per figure or panel?}


%%%%%%%%%%
\subsection*{Code availability}
%%%%%%%%%%

\todo{RR: Here is what Nat Commun requires. Would it perhaps be sensible to upload all required code on GitHub?

Authors must make available upon request, to editors and reviewers, any previously unreported custom computer code or algorithm used to generate results that are reported in the paper and central to its main claims. Any reason that would preclude the need for code or algorithm sharing will be evaluated by the editors who reserve the right to decline the paper if important code is unavailable.

For all studies using custom code or mathematical algorithm that is deemed central to the conclusions, a statement must be included under the heading "Code availability", indicating whether and how the code or algorithm can be accessed, including any restrictions to access. Code availability statements should be provided as a separate section after the data availability statement but before the references.

Upon publication, Nature Portfolio journals consider it best practice to release custom computer code in a way that allows readers to repeat the published results. Code should be deposited in a DOI-minting repository such as Zenodo or Code Ocean and cited in the reference list following the guidelines described here. Authors are encouraged to manage subsequent code versions and to use a license approved by the open source initiative. Full details about how the code can be accessed and any restrictions must be described in the Code Availability statement.}


%%%%%%%%
\todo{RR: Mikhail, please can you check the following references for correct formatting:
Ref. 5 is a book chapter - need to add 'In' before the book title, and include publisher.
Your extra code below does not seem to do this properly.
Also,pPlease display all authors for all references. I think there is no paper with too many authors.
Thanks!}
% --- Force "In:" for incollection entries by default printed is in nature style biblatex---
\renewbibmacro*{in:}{%
  \printtext{In:\space}%
}
\printbibliography


%%%%%%%%%%
\subsection*{Acknowledgements}
%%%%%%%%%%

This work was financially supported by the Royal Society (International Exchanges
Award IEC/R2/202035 to R.P.R.), the UK Biotechnology and Biological
Sciences Research Council (grant BB/X00158X/1 to R.P.R.), and the Russian Science Foundation (grant 23-13-00174).

The authors thank Charley Schaefer and Paolo Actis (both University of Leeds) for helpful discussions.

%%%%%%%%%%
\subsection*{Author contributions}
%%%%%%%%%%

R.P.R., L.K. and O.V.B. conceived the study.
All authors developed the theoretical model.
M.Y.L. and F.A.M.L. performed the computer simulations.
All authors analyzed the data, and contributed to manuscript review and editing.


%%%%%%%%%%
\subsection*{Competing interests}
%%%%%%%%%%

The authors declare no competing interests.

\end{document}