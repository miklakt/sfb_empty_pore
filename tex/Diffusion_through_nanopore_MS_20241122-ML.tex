\documentclass[12pt, a4paper]{article}
\usepackage{graphicx}
\usepackage{amsmath, amssymb, amsfonts, mathtools}
\usepackage{subcaption}
\usepackage[
backend=biber,
natbib=true,
style=numeric,
sorting=none
]{biblatex}
\usepackage{xcolor}
\usepackage{bm}

\newcommand\todo[1]{\textcolor{red}{#1}}

\addbibresource{biblio.bib}
\title{Physical principles of selective colloid permeation through polymer-filled mesopores}

\author{Mikhail Y. Laktionov$^1$, Leonid I.Klushin$^{2}$,\\Ralf P.Richter$^3$, France A.M. Leermakers$^4$, Oleg V.Borisov$^1$\\
$^{1}$CNRS, Universit\'e de Pau et des Pays de l'Adour UMR 5254,\\
Institut des Sciences Analytiques et de Physico-Chimie\\
pour l'Environnement et les Mat\'eriaux, 64053 Pau, France \\
$^{2}$Institute of Macromolecular Compounds \\
of the Russian Academy of Sciences, \\
199004 St.Petersburg, Russia,\\
$^{3}$University of Leeds, School of Biomedical Sciences, \\
Faculty of Biological Sciences, 
School of Physics and Astronomy, \\
Faculty of Engineering and Physical Sciences,\\  
Astbury Centre for Structural Molecular Biology,\\ 
and Bragg Center for Materials Research,\\ 
Leeds, LS2 9JT, United Kingdom\\
$^{4}$ University of Wageningen, the Netherlands
}
\date{}

\begin{document}
\maketitle

\begin{abstract}

% RR: I have note worked on the abstract yet.

Physical mechanisms of selective facilitated permeation of nanocolloidal particles 
through polymer-grafted mesopores are unravelled on the basis of self-consistent field theoretical modelling.
We predict that diffusive transport of particle can be accelerated compared to that through a bare pore due to
cohesive polymer-particle interactions, while penetration of inert with respect to the polymer particles of even smaller size can be 
efficiently impeded. We formulate thermodynamic criteria for unrestricted gating threshold through the pore and anticipate, that underlying
physical mechanisms may apply for facilitated permeation of biologically active molecules in complex with NTR through NPC.   
\end{abstract}

%%%%%%%%%%
\section{INTRODUCTION}
%%%%%%%%%%

% RR: I have not worked on this part of the introduction yet.

Polymer-modified mesoporous materials and membranes belong to a new class of functional nanostructured materials with great potential in a number of key technologies. 
The interaction and absorption of (macro)molecules and nanocolloidal particles by porous media, as well as their transport through macro- and mesoporous membranes, 
are important elements of many technological processes (chromatography, heterogeneous catalysis, micro- and ultrafiltration, protein separation and purification etc.) 
and, therefore, have been the subject of intensive research for decades. 

Advances in macromolecular chemistry have made it possible to significantly improve functional properties of mesoporous (with a pore diameter within 100 nm) 
materials by modifying them with covalently (or strongly non-covalently) bound to the pore walls macromolecules of various chemical nature,  
forming a “soft” , solvated physical polymer meshwork that fills the pore volume or only the near-wall regions, 
depending on molecular mass and conformational state of the polymer chains. 
The interaction of this polymer meshwork with guest molecules/nanoparticles 
%and, in particular, the presence or absence of a hollow (polymer-free) path in the center of the pore, 
essentially determine the absorption and separation properties of the polymer-modified mesoporous materials and membranes, respectively. 
These interactions can be attractive or repulsive, short- or long-range (in the presence of charges on the chains and on guest molecules/particles), 
and most importantly, they can be controlled by a complex of external stimuli, such as temperature, pH and/or ionic strength of the medium, valence of ions , 
solvent composition, etc. This opens up a unique opportunity for highly selective and controlled uptake and transport of guest molecules/nanoparticles 
through polymer-filled mesoscopic channels. 
%For example, mesopores modified with ionic polymer molecules can potentially be used to separate molecules/nanocolloids that are almost identical in size and shape, 
%but differ in a small number of charged groups. 

Nature uses the principle of controlling the selective transport of biological molecules 
between the nucleus and cytoplasm of eukaryotic cells through the so-called nucleopores (cylindrical channels of about 40 nm in diameter), 
which perforate the nuclear envelope and are filled with a swollen meshwork consisting of natively denatured proteins attached to the pore walls. 
A similar structural motif was recently found in the internal channels of microtubules (about 15 nm in diameter) 
decorated with so-called microtubule intrinsic proteins (MEPs), presumably modifying microtubule stability and rigidity.
A popular nowadays paradigm suggests that the accuracy and efficiency of many processes in nature are ensured 
not so much by specific (molecular recognition) interactions, but due to a fine balance of fundamental (electrostatic, hydrophobic...) 
interactions between biomacromolecules and (bio) nanocolloids.  

However, up to date, 
%the theoretical knowledge and systematic 
understanding of the relationship between molecular architecture of the brush 
decorating the pore walls and the spatial structure, 
cohesive and rheological properties of the resulting "soft" meshwork and its ability to selectively absorb in the 
volume of pores or modulate the diffusion transport of nanocolloidal particles through the pores is lacking. 

% RR: I have prepared the following final paragraph.

Our analysis focuses on pores that are pervaded by a dynamic meshwork of flexible polymers. 
This meshwork effectively increases the local viscosity and thereby slows down transport of colloids compared to an open pore. 
On the other hand, an attractive polymer phase recruits colloids into the pore, thus increasing colloid transport, 
and such recruitment is further enhanced when attractive polymers extend outside the pore. 
Intriguingly, the solvent strength through its influence on the density and compactness of the polymer plug impacts all these effects. 
Here, using a self-consistent field approach, we define how solvent quality and colloid attraction to the polymer may be tuned 
to maximise the transport rate (including beyond the rate for an open pore) or for the highly selective transport of colloids of a desired size or attraction to the polymers.


%%%%%%%%%%
\section{RESULTS}
%%%%%%%%%%


%%%%%%%%%%
\subsection{Defining the transport scenario}
%%%%%%%%%%

\begin{figure}
    \centering
    \includegraphics[width = 0.7\linewidth]{fig/pore_cartoon.png}
    \caption{
        Schematic illustration of colloid diffusive transport through a pore filled with polymer brush. 
        The brush is formed by linear polymer chains (red strands) with a degree of polymerization $N$, uniformly grafted with grafting density $\sigma$ 
        to the inner surface of a cylindrical pore in an impermeable membrane. The pore radius is $r_{\text{p}}^{0}$ and the thickness of the membrane is $L_{0}$.
        Polymer chains are flexible with a statistical segment length $a$ and volume $\sim a^3$. 
        Spherical colloids with diameter $d$ are free to diffuse in the surrounding solvent.
        All length scales are normalized by the statistical segment length $a$.
        \\
        The pore is permselective to larger particles; larger magenta-colored particles are drawn only on the left side of the membrane to symbolize this effect.
        }
    \label{fig:colloid_transport}
\end{figure}

The salient features of our simulated mesopore (Figure \ref{fig:colloid_transport}) are inspired by the nuclear pore complex.
Pores with such features can though also be realized synthetically.
The pore has a cylindrical shape with diameter $r_{\text{p}}^{0}$.
The pore perforates an otherwise impermeable, planar membrane of thickness $L_{0}$, and thus is the sole conduit for colloids between two semi-infinite solution reservoirs.
Flexible polymer chains are end-grafted to the inner pore walls, at a density sufficient to form a polymer brush that penetrates the entire pore cross-section.
We aim to define how the diffusive transport of colloids across the pore is modulated by the polymer brush. 
The interaction strength between a polymer segment and a unit surface area of the colloid is represented by the Flory-Huggins parameter $\chi_{\text{PC}}$. 


%%%%%%%%%%
\subsection{Colloid transport is defined by the sum of resistances outside and inside the pore}
%%%%%%%%%%

To understand how the polymer brush affects the transport of colloids, we consider the stationary diffusive flux of colloids through the pore 
and analyze how it is affected by the parameters of the pore, the brush, and the colloid.
To this end, the colloid concentration is fixed to be zero and $c_{\text{b}}$ far away from the membrane (at $z\rightarrow\mp\infty$, respectively). 
We assume axial (cylindrical) symmetry of the flow in the pore. Together with the stationary conditions, 
this implies that parameters relevant to colloid transport depend on the axial coordinate $z$ and the radial coordinate $r$, but not on the azimuthal angle.

%%%%%%%%%%
\subsubsection{Empty pore as a reference case}
%%%%%%%%%%

A natural reference is the diffusive flux through an empty pore without any polymer. The earliest approach to that problem goes back to Lord Rayleigh 
who analyzed the flux through a circular pore in a planar membrane of negligible thickness \cite{Strutt1878}. 
The equiconcentration surfaces in this case are oblate spheroids and the streamlines form confocal hyperboloids of revolution \cite{Cooke1966}.
The net flux through the pore is given by

\begin{equation}
    J=2D_0r_{\text{p}}\Delta c
    \label{eq:flux_Ral}
\end{equation}

\noindent where $D_0$ is the diffusion coefficient of the colloid in plain solvent. 

Diffusion through a pore in a membrane of finite thickness $L_{0}$ allows an approximate analytical solution (with an error of less than 6\% in the full range of the $\frac{L_{0}}{r_{\text{p}}}$ ratio) \cite{Brunn1984}: 

\begin{equation}
    J=\frac{2D_0r_{\text{p}}}{1+\frac{2L}{\pi r_{\text{p}}}}\Delta c
    \label{eq:flux_finlength}
\end{equation}

\noindent To account for the excluded volume of the diffusing colloids, assuming them being spheres of diameter $d$, we here use the effective pore radius $r_{\text{p}}=r_{\text{p}}^{0}-d/2$ and pore thickness $L = L_{0} + d$.
%UPDATE MANUALLY%
We illustrate the effect of volume exclusion in Figure~9a of the Supplementary Information.

Introducing the resistance $R$ to colloid flow as $J=\frac{\Delta c}{R}$ admits a most natural interpretation of Eq. (\ref{eq:flux_finlength}) in terms of the total resistance of the pore

\begin{equation}
    R_{0}=\frac{L}{D_0\pi r_{\text{p}}^{2}}+\frac{1}{2D_0r_{\text{p}}}=R_{\text{int}}^{0}+R_{\text{ext}}^{0}
    \label{eq:resistance}
\end{equation}

\noindent The first term in Eq. (\ref{eq:resistance}) is the resistance of the interior of the empty pore, while the second term is the Rayleigh resistance of a pore of infinitesimal thickness (Eq. (\ref{eq:flux_Ral})). 
The latter represents the effects of the convergent flow towards the pore entrance (in the region exterior to the pore) and its symmetric counterpart at the pore exit, while the flow lines inside the cylindrical pore turn out to be approximately axial.

Naturally, for small particles $d \ll r_{\text{p}}^{0}$ the resistance of a pore in a thin membrane $L_{0} \ll r_{\text{p}}^{0}$ is determined by the resistance of the exterior region, $R_{0} \approx R_{\text{ext}}^{0}$.
In contrast, for long pores (or for large particles that make the pores effectively long), where $L \gg r_{\text{p}}$, the resistance of the inner region becomes dominant, such that $R_{0} \approx R_{\text{int}}^{0}$.


In the empty pore scenario, the inverse of the diffusion constant ($\rho_0=D_0^{-1}$) represents the resistivity of the medium both inside and outside the pore. 

%%%%%%%%%%
\subsubsection{A polymer filling affects the resistance of the pore itself, and also of regions outside the pore}
%%%%%%%%%%

\begin{figure}
    \centering
    \includegraphics[width = 0.7\linewidth]{fig/phi_hm_grid.png}
    \caption{
    Maps of the polymer volume fraction $\phi(r,z)$ for a polymer brush in a cylindrical pore at a range of solvent qualities, as predicted by self-consistent field theory. 
    The solvent quality is quantified by the Flory-Huggins parameter $\chi_{\text{PS}}$ ranging from 0.1 (good solvent) to 1.1 (poor solvent).
    Polymer volume fractions are mapped in cylindrical coordinates (as shown $rz$-coordinate arrows), color coded as indicated in the legend below and with selected iso-concentration lines shown, with the blanc space indicating pure solvent.
    For illustrative purposes the colormaps are mirrored along the $z$ axis, and the membrane is indicated in green.
    Conditions: $L_{0}=2r_{\text{p}}^{0}=56$, $\sigma=0.02$, $N=300$.
    }
    \label{fig:phi_hm_grid}
\end{figure}

The conformations adopted by polymer chains grafted to the pore walls are controlled by strong (under overlapping conditions) intermolecular interactions, and depend on the solvent quality. 
The solvent quality is quantified by the Flory-Huggins solubility parameter $\chi_{\text{PS}}$. 
Values of $\chi_{\text{PS}}<0.5$ and $\chi_{\text{PS}}>0.5$ correspond to good and poor solvent, respectively, whereas $\chi_{\text{PS}}=0.5$ represents the ideal (or $\theta-$)solvent.

The polymer density profile $\phi(r,z)$ in the pore was calculated by two-gradient Scheutjens-Fleer self-consistent field method.
It is worth noting that in the case of sufficiently wide pore and small polymerization degree/grafting density, 
an open channel free of polymer may appear under poor solvent conditions in the pore center, as discussed in details in~\cite{Laktionov2021}. 
This scenario would lead to a distinct permeation behaviour as the colloids can move through the pore without traversing the brush, and is not further considered here. 

Figure \ref{fig:phi_hm_grid} further illustrates that whilst the brush remains confined within the pore lumen in poor solvent ($\chi_{\text{PS}}=0.9$ and $\chi_{\text{PS}}=1.1$) 
it protrudes substantially into the surrounding space in ideal and good solvent ($\chi_{\text{PS}}\le0.5$), thus forming 'caps' on either side of the pore.  
The polymer brush therefore will impact on colloid flow within as well as outside the pore, such that

\begin{equation}
    R=R_{\text{int}}+R_{\text{ext}}
    \label{eq:R_tot_tot}
\end{equation}
    
\noindent with $R_{\text{int}}\rightarrow R_{\text{int}}^{0}$ and $R_{\text{ext}}\rightarrow R_{\text{ext}}^{0}$ in the limit of the empty pore. 


%%%%%%%%%%
\subsection{Insertion free energy and diffusivity define transport locally}
%%%%%%%%%%

Zooming in on the local scale, we can define how colloids are accumulated or depleted due to attractive or repulsive interactions, respectively, by the presence of the polymer meshwork, 
and how the polymer meshwork affects the rate of diffusion.

%%%%%%%%%%
\subsubsection{Insertion free energy is a balance of volume and surface effects}
%%%%%%%%%%

The insertion free energy $\Delta F(r,z)$ defines the energy penalty upon moving a colloid from plain solvent into the polymer meshwork.
A positive $\Delta F$ thus implies that the brush repels the particle, and vice versa.

For colloids that are significantly smaller than the size of the pore, the insertion free energy is determined entirely by the local polymer concentration (i.e., wall effects can be neglected), 
and made up of two distinct contributions, one osmotic and the other interfacial.

\begin{eqnarray}
    \Delta F (r,z)= \Delta F_{\text{osm}}(r,z) + \Delta F_{\text{sur}}(r,z)
    \\
    \Delta F_{\text{osm}}(r,z) = \int_{V} \Pi(r,z) dV
    \\
    \Delta F_{\text{sur}}(r,z) = \oint_{S} \gamma (r,z) dS
    \label{eq:Delta_F}
\end{eqnarray}

\noindent The coordinates $(r,z)$ here refer to the center of the colloid, whilst the insertion free energy is obtained by integrating over the colloid volume and surface, respectively.

The osmotic contribution, $\Delta F_{\text{osm}}(r,z)$, is proportional to the colloid volume 
and accounts for the work performed against excess osmotic pressure upon insertion of the particle into the brush. 
The local osmotic pressure is calculated from the local polymer concentration using a Flory mean field approach 

$$
\Pi(r,z)=  \phi(r,z)\frac{\partial f\{\phi(r,z)\}}{\partial \phi(r,z)} - f\{\phi(r,z)\}= 
$$
\begin{equation}
	k_{\text{B}}T[-\ln(1-\phi(r,z)) - \phi(r,z) -\chi_{\text{PS}}\phi^2(r,z)]
\end{equation}

\noindent where
$$
f\{\phi(r,z)\}=(1-\phi(r,z))\ln(1-\phi(r,z)) +\chi_{\text{PS}}\phi(r,z)(1-\phi(r,z))
$$

\noindent is the mean-field Flory expression for the interaction free energy per unit volume of the polymer solution of concentration (volume fraction) $\phi(r,z)$.
As long as the osmotic pressure inside the brush is positive, $\Delta F_{\text{osm}}$ is positive as well and dominates for sufficiently large colloids. 

The interfacial (surface) contribution, $\Delta F_{\text{sur}}(r,z)$, is proportional to the colloid surface, 
with the surface tension $\gamma (r,z)$ approximated as

\begin{eqnarray}
    \gamma (r,z)= \frac{1}{6}(\chi_{\text{ads}} - \chi_{\text{ads}}^{\text{crit}})\phi^{\ast}(r,z)
    \\
    \label{eq:chi_ads}
    \chi_{\text{ads}} = \chi_{\text{PC}} - \chi_{\text{PS}}(1-\phi^{\ast})
    \\
    \phi^{\ast}(r,z)= (b_{0} + b_{1}\chi_{\text{PC}})\phi(r,z)
\end{eqnarray}

\noindent Here $\gamma$ is a free energy change upon replacement of a contact of the unit surface area of the colloid with plain solvent by a contact with a polymer solution of concentration $\phi(r,z)$.
Coefficients $b_0$ and $b_1$ are adjustable parameters to account for depletion/accumulation of polymer in the proximity of the colloid surface, 
thus adjusting the local polymer concentration in a plain brush to the effective concentration $\phi^{\ast}(r,z)$.

Depending on the relative strengths of polymer-colloid ($\chi_{\text{PC}}$) and polymer-solvent ($\chi_{\text{PS}}$) interactions, the sign of $\gamma \sim (\chi_{\text{ads}} - \chi_{\text{ads}}^{\text{crit}}) \phi^{\ast}$ can be either positive or negative.
If the particle surface is repulsive ($\chi_{\text{ads}} \geq 0$) or even weakly attractive for polymers ($\chi_{\text{ads}}^{\text{crit}} \leq \chi_{\text{ads}} < 0$), then, due to steric constraints imposed by the impermeable surface, the available conformations of the polymer are restricted, leading to polymer depletion near the particle surface and resulting in $\gamma \geq 0$.
At the critical adsorption value $\chi_{\text{ads}} = \chi_{\text{ads}}^{\text{crit}}$, the losses in conformational entropy caused by the presence of the surface are exactly balanced by the free energy gain from monomer-surface contacts, causing $\gamma$ to vanish \cite{Fleer1993,Birshtein1979,Birshtein1983,Eisenriegler1982}.

SF-SCF results in a more accurate calculation of the insertion free energy with account of the actual perturbation of the brush structure by the inserted colloid is possible with SF-SCF simulations, but only for cylindrical colloids positioned along the pore axis ($r=0$).

As the maps of the polymer volume fraction $\phi(r,z)$ in an unperturbed brush (Figure \ref{fig:phi_hm_grid}) were calculated using a lattice based method, a special discretization scheme was employed for the integration of volumes and surfaces for cylindrical particles in
%UPDATE MANUALLY%%%%%%%%%%%%%%%%%%%%%%%%%%%%%%%%%%%%%%%%%%%%%%%%%%%%%%%%%%%%%%%%%%%%%%%%%%%%%%%%%%%%%%%%%%%%%%%%%%%%%%%%
Section 3 of the Supplementary Information, Eqs.~(18,19).
%%%%%%%%%%%%%%%%%%%%%%%%%%%%%%%%%%%%%%%%%%%%%%%%%%%%%%%%%%%%%%%%%%%%%%%%%%%%%%%%%%%%%%%%%%%%%%%%%%%%%%%%%%%%%%%%%%%%%%%%
Coefficients $b_0$ and $b_1$ were tuned to fit approximate analytical scheme to cylindrical particles with SF-SCF results.
%UPDATEMANUALLY%%%%%%%%%%%%%%%%%%%%%%%%%%%%%%%%%%%%%%%%%%%%%%%%%%%%%%%%%%%%%%%%%%%%%%%%%%%%%%%%%%%%%%%%%%%%%%%%%%%%%%%%
The details on the fitting of $b_0$ and $b_1$ coefficients are disclosed in Section 4 of the Supplementary Information.
%%%%%%%%%%%%%%%%%%%%%%%%%%%%%%%%%%%%%%%%%%%%%%%%%%%%%%%%%%%%%%%%%%%%%%%%%%%%%%%%%%%%%%%%%%%%%%%%%%%%%%%%%%%%%%%%%%%%%%%%

In the next step using the $b_0$ and $b_1$ coefficients we generalize an approximate analytical scheme to arbitrary placed spherical particles, with yet another discretization scheme for the integration of volumes and surfaces for for spherical particles.
%UPDATE MANUALLY%%%%%%%%%%%%%%%%%%%%%%%%%%%%%%%%%%%%%%%%%%%%%%%%%%%%%%%%%%%%%%%%%%%%%%%%%%%%%%%%%%%%%%%%%%%%%%%%%%%%%%%%
Further details is in Section 5 of the Supplementary Information.
%%%%%%%%%%%%%%%%%%%%%%%%%%%%%%%%%%%%%%%%%%%%%%%%%%%%%%%%%%%%%%%%%%%%%%%%%%%%%%%%%%%%%%%%%%%%%%%%%%%%%%%%%%%%%%%%%%%%%%%%

In what follows, we apply an approximate analytical scheme for calculating the insertion free energy $\Delta F(r,z)$ as $\Delta F\{\phi(r,z)\}$, 
where $\phi(r,z)$ is the polymer density distribution in a colloid-free brush. 
The colloid is thus considered within this analytical approach as a 'probe' which does not perturb the global concentration distribution $\phi(r,z)$ in the brush. 
The advantage of this scheme is that it enables evaluating the insertion free energy at any position of the colloid in the brush.

Comparison of the approximate analytical approach for cylindrical colloid with the SF-SCF simulations for colloids on the pore axis demonstrated good quantitative agreement (see Section 4 of the Supplementary Information, Figure~6), 
thus justifying the use of the more versatile analytical approach.

\begin{figure}
    \centering
    \includegraphics[width = 0.7\linewidth]{fig/free_energy_hm.png}
    %\includegraphics[scale = 1.0]{fig/DeltaF_map.png}
    \caption{
    Maps of the insertion free energy $\Delta F(r,z)$ for a polymer brush in a cylindrical pore for a range of polymer-colloid interaction strengths. 
    The polymer-colloid interaction strength is quantified by the Flory-Huggins parameter $\chi_{\text{PC}}$ ranging from -0.50 (least attractive) to -1.25 (most attractive), as indicated. 
    Insertion free energies are mapped in cylindrical coordinates (as in Figure \ref{fig:phi_hm_grid}), and colour coded as indicated. 
    For illustrative purposes the colormaps are mirrored along the $z$ axis, and the membrane is indicated in green. 
    Conditions: pore and polymer brush as in Figure \ref{fig:phi_hm_grid}, $\chi_{\text{PS}}=0.5$, $d=8$.
    }
    \label{fig:DeltaF_map}
\end{figure}

Figure \ref{fig:DeltaF_map} illustrates how colloids may be either repelled or attracted by the polymer meshwork, 
depending on the balance of osmotic and interfacial contributions to $\Delta F(r,z)$. 
Since both $\Delta F_{\text{osm}}(r,z)$ and $\Delta F_{\text{sur}}(r,z)$ depend on the on local polymer concentration $\phi(r,z)$, 
the net insertion free energy $\Delta F(r,z)$ is position-dependent, and may exhibit quite large spatial variations. 
For example, the brush shown in Figure \ref{fig:DeltaF_map} switches from attraction at the caps to repulsion inside the pore at $\chi_{\text{PC}}=-0.75$.


%%%%%%%%%%
\subsubsection{Diffusivity is determined by the ratio of polymer mesh size and colloid size}
%%%%%%%%%%
The local diffusion coefficient depends on the polymer volume fraction $\phi$ and the particle size, environment crowded by polymer chains naturally decrease both rotational \cite{Fu2017} and translational \cite{Stewart1998} diffusion coefficients.

The polymer brush forms a polymer semi-dilute solution with a concentration-dependent correlation length $\xi$.
Particles with a size $d > \xi$ experience additional friction as they are trapped by the network of the polymer mesh.
As a result, diffusion is slowed down compared to that in the pure solvent leading to position-dependent diffusion coefficient($D(r,z)/D_0 < 1$).
The effect have been studied, and several theoretical and empirical models have been proposed~\cite{Kohli2012,Cai2011,Holyst2009,Phillies1988}.
These models show the same qualitative trends in the diffusion coefficient as a function of particle size relative to  correlation length $d / \xi$.

% In the previous paper \cite{Laktionov2023} we also accounted for the fact that a semi-dilute polymer meshwork formed by the polymer brush slows the rate of colloid movement compared to plain solution, leading to a position-dependent diffusion coefficient $D\{\phi(r,z)\}$.

In this paper we chose the results of the model proposed by Cai \emph{et al.} \cite{Cai2011} to estimate position-dependent diffusion coefficient as a function of particle size $d$ and polymer concentration $\phi$.

\begin{equation}
    D\{\phi(r,z)\} = \frac{D_0}{1+d^2/\xi^{2}\{\phi(r,z)\}}
    \label{eq:Rubinstein}
\end{equation}

\noindent where the correlation length (or 'mesh size') $\xi$ is controlled by the local polymer concentration $\phi(r,z)$. 
As follows from Eq. (\ref{eq:Rubinstein}), particles smaller than the mesh size diffuse virtually unhindered ($D\{\phi(r,z)\}\approx D_0$ for $d\leq \xi$). 
In contrast, large colloids are significantly slowed down by the polymer medium compared to plain solvent ($D\cong D_0 (\xi/d)^2\ll D_0$ for $d\gg \xi$). 
We approximate the dependence of the mesh size on polymer concentration by the power-law dependence valid close to $\theta$-solvent conditions in a mean-field regime, $\xi\cong \phi^{-1}$. 

We anticipate that even significant discrepancies between the employed model and experimental data will not alter the qualitative conclusions of the following results. 
%UPDATE_MANUALLY%
This is partially demonstrated in Section 9 of the Supplementary Information, where we analyze the impact of the diffusion coefficient on pore resistance.


%Do we need a figure here that shows illustrative map of D/D0? 

\subsubsection{Linking Local Resistivity to Global Transport}

We develop an analytical method to estimate the resistance of a cylindrical pore with a grafted polymer brush to the diffusion of colloidal particles. 
This approach builds upon the classic solution for diffusion through an empty pore and incorporates the effects of the polymer brush by modifying the local diffusion coefficient and introducing a free energy landscape.

In an empty pore, the steady-state concentration profile of diffusing particles features iso-concentration lines that form oblate spheroids, with the pore acting as a focal circle.
The flux density is given by $\mathbf{j} = -D_0 \nabla c$, where $D_0$ is the constant diffusion coefficient.
When a polymer brush is grafted inside the pore, the local diffusion coefficient $D(r,z)$ becomes position-dependent, depending on the local polymer concentration and accounting for slower diffusion through a semi-dilute polymer mesh~\cite{Cai2011}.
Particles also experience a mean force, with an insertion free energy $\Delta F(r,z)$ playing the role of a potential. To account for these effects, we introduce a scalar potential function $\psi = c\exp(\Delta F / k_B T)$ in the form of a modified Boltzmann distribution, which satisfies $\mathbf{j} = -D \nabla \psi$ and maintains a structure similar to the empty pore solution.
As we noted in our previous paper~\cite{Laktionov2023}, for the solution in the form of a modified Boltzmann distribution, the product
\begin{equation}
    D(r,z) \exp\left( \frac{-\Delta F(r,z)}{k_B T} \right) \equiv \rho^{-1}(r,z)
    \label{eq:local_conuctivity}
\end{equation}
has the meaning of local conductivity and encapsulates the local effects of the polymer meshwork on diffusivity and insertion free energy.

To simplify the analysis, we utilize an orthogonal curvilinear coordinate system aligned with the iso-surfaces of $\psi$ and stream surfaces of $\bm{j}$.
Inside the pore, we employ cylindrical coordinates, while outside the pore, we approximate the iso-surfaces using a variant of oblate spheroidal coordinates, as seen in Figure~\ref{fig:R_map}B. The chosen coordinate system has axial, radial, and azimuthal axes.
This curvilinear coordinate system shares axial coordinate with the axial coordinate $z$ of the cylindrical coordinate system we used to define position-dependent properties.

By integrating the local conductivity $\rho^{-1}(r,z)$ (Eq. (\ref{eq:local_conuctivity})) over the radial and azimuthal coordinates of the curvilinear coordinate system at each axial position $z$, we calculate the differential conductivity of infinitely thin layer $\varrho_{z}^{-1}$ that intersects .
The total resistance $R$ is then obtained by integrating $\varrho_{z}$ along the axial coordinate $z$:
\begin{eqnarray}
    \label{eq:R_z_inv}
    \varrho_{z}^{-1} = \int_{0}^{r_{\text{p}}^{0}}\int_{0}^{2\pi}\rho^{-1} \, \tilde{h} \, d\theta \, dr\\
    \label{eq:R_z_pore}
    R_{\text{int}} = \int_{-L/2}^{+L/2} \varrho_{z} \, dz\\
    \label{eq:R_z_conv}
    R_{\text{ext}} = \int_{-\infty}^{-L/2} \varrho_{z} \, dz + \int_{+L/2}^{+\infty} \varrho_{z} \, dz\\
    \label{eq:R_z_tot}
    R = R_{\text{int}} + R_{\text{ext}}
\end{eqnarray}
where $\tilde{h}$ depends on the coordinate system metric. Here, $R_{\text{int}}$ is the resistance of the pore channel, and $R_{\text{ext}}$ is the resistance due to convergent flow in the exterior regions.

% The resistances of the channel $R_{\text{int}}$ and the pore's exterior $R_{\text{ext}}$ can be translated to respective dwell-times for permeating species.
% When permeation is enhanced the dwell time inside the pore lumen is shorter that time spend to reach the pore entrance.
% It was reported in \cite{Gruenwald2010} that 'docking' dwell time for mRNA transport is about an order of magnitude longer than the time spend in a nucleopore itself.
% Conversely, when the permeation is hindered ($R_{\text{int}} > R_{\text{int}}^{0}$) the permeating species spend the most time traversing through the pore channel.

For the interior region, the integration is straightforward:
\begin{eqnarray}
    R_{\text{int}} = 2\pi\int_{-L_{0}/2}^{+L_{0}/2}\left(\int_{0}^{r_{\text{p}}^{0}} r \, dr \, \rho^{-1}(r, z')\right)^{-1} dz'
    \label{eq:R_pore}
\end{eqnarray}
The resistance due to the convergent flow, $R_{\text{ext}}$, in the exterior region requires integration over the surfaces of a series of oblate spheroids.
%UPDATE MANUALLY% 
See the details in Section~7 of the Supplementary Information.

For an empty pore without a polymer brush, this method recovers the classic expression for the total resistance (Eq.~\ref{eq:resistance}).

For the brush entirely contained within the interior of the pore -- which is well justified under poor solvent conditions -- the total resistance is found as $R = R_{\text{int}} + R_{\text{ext}}^{0}$, as the exterior is not modified by the brush.
Conversely, a brush in good or $\Theta$-solvent conditions produces caps outside the pore (Figure~\ref{fig:phi_hm_grid}), requiring a more elaborate calculation of the resistance due to modulated convergent flow, $R_{\text{ext}}$.

Because of the discrete nature of the numerical data, we perform the integration on a cylindrical lattice, approximating the iso-potential surfaces as disks in the interior of the pore (as in Eq.~\ref{eq:R_pore}) and half-cylinders in the exterior regions.
We introduce a correction factor to account for differences between this approximation and the actual oblate spheroidal geometry.
%UPDATE MANUALLY%
See the details in Section~7 of the Supplementary Information.

This analytical method provides a tool to analyze and compare the resistance experienced by particles during axial transport with that of an empty pore.
The $\varrho_{z}$ profiles can provide insight into the pore resistance structure in a compact form, as demonstrated in Figure~\ref{fig:R_map}.


\subsection{An Attractive Polymer Filling Dramatically Enhances Colloid Fluxes Through a Mesopore}

\begin{figure}
    \centering
    %\includegraphics[scale = 0.7]{fig/R.png}
    \includegraphics[width = 10cm]{fig/resistivity_on_z_and_hm.png}
    \caption{
    \textbf{(a)} Resistance per unit length $\varrho_{z}$ along the axial coordinate of the curvilinear coordinate system for selected polymer-colloid interaction strengths $\chi_{\text{PC}} \in \{ -0.9, -1.0, -1.1, -1.2, -1.3\}$, as indicated with colored lines.
    The resistance per unit length of a plain pore without polymers $\varrho_{z}^{0}$ (black thick line), a pore with only the diffusion coefficient modulated by the polymer brush ($\Delta F = 0$, dashed black line), and the location of the membrane (green background) are shown for comparison.
    \\
    \textbf{(b)} Special orthogonal curvilinear coordinate system aligned with the flux density $\bm{j}$ stream surfaces (radial coordinate) and level sets of the potential function $\psi$ (axial coordinate).
    Red lines correspond to constant values of the axial coordinate; gray lines are tangential to the flux density field and correspond to constant values of the radial coordinate.
    The lines define bodies of revolution along the $z$-axis; the angular coordinate is not shown.
    In the exterior of the pore, constant radial coordinates are confocal hyperboloids of revolution, constant axial coordinates are confocal oblate spheroids, and constant angular coordinates are half-planes.
    In the interior of the pore, the coordinate system is equivalent to the cylindrical coordinate system.
    \\
    \textbf{(c)} Maps of normalized resistivity $\rho D_0$ exemplifying a transition between hindered and enhanced permeability upon a subtle increase in the polymer-colloid interaction strength.
    $\chi_{\text{PC}}$ was varied from $-1.0$ (left) to $-1.1$ (right); resistivities are color-coded as indicated.
    An arrow on the lower frame marks the bottleneck radius $r_{\text{bn}}$ for a region with reduced resistivity.\\
    Conditions: pore and polymer brush as in Figure~\ref{fig:phi_hm_grid}, $\chi_{\text{PS}}=0.5$, $d=8$.
    }
    \label{fig:R_map}
\end{figure}

As mentioned in the previous section, we analyze the resistance of a pore filled with a polymer brush by dividing the space into an infinite set of layers with infinitesimally small thickness with a curvilinear coordinate system.
The surfaces of these layers are oriented normal to the colloid particle flux density.
This means that the resistance of a single layer is $\varrho_{z} \, dz$, where $dz$ is the layer thickness.
For a region free of polymer brush, the layer resistance $\varrho_{z} \equiv \varrho_{z}^{0}$ is inversely proportional to the layer's surface area.

Figure~\ref{fig:R_map}A provides example resistance per unit length $\varrho_{z}$ profiles along axial coordinate $z$ for a colloid particle with size $d=8$ at a few selected values polymer-colloid interaction parameters $\chi_{\text{PC}}$ and for an empty pore.
For comparison, we also show the case where the polymer brush has an effect only on the diffusion coefficient of the medium, as if the colloid particles have no interactions with the polymer brush but only experience increased viscosity of the medium.
The area under the curves represents the total resistance of the pore, as follows from Eq.~(\ref{eq:R_z_pore}, \ref{eq:R_z_conv}).
Several features are notable. 

Firstly, an attractive pore interior can enhance the permeability such that the diffusive fluxes through the polymer-filled pore exceed the limit of the empty pore.
This can be clearly appreciated for $\chi_{\text{PC}} = -1.2$ and $-1.3$, where the resistance per unit length within the membrane width ($-28 \leq z \leq 28$) is consistently lower than the resistance per unit length of the empty pore.
This result may at first appear surprising, given that the polymer medium is expected to slow down the diffusion of colloids.
However, this effect is counteracted by the attractive potential of the pore, which facilitates a higher colloid flux and thus reduces the pore resistance.

Secondly, attractive caps can further enhance permeability.
This can again be clearly appreciated for $\chi_{\text{PC}} = -1.2$ and $-1.3$ (and to a lesser extent for $\chi_{\text{PC}} = -1.1$), where the resistance per unit length outside the membrane ($-50 \lesssim z < -28$ and $28 < z \lesssim 50$) is consistently lower than the resistance per unit length for the empty pore. 
Clearly, this effect stems from the attractive potential of the caps, which in this case extend to $z \approx \pm 50$ and again facilitate higher colloid flux. 

As the colloid attraction increases further (not shown here), the resistance per unit length across the $z$ range spanning the pore interior and the caps becomes negligible, such that the convergent flow outside the polymer-filled volume dominates the resistance against colloid transport.
Without caps, the colloid flux may thus increase by up to approximately 3-fold. 
Attractive caps can increase the flux substantially further.
When reaching $z = \pm 50$, an extra 5-fold enhancement is possible, for example, and even larger caps would facilitate further enhancement.

Thirdly, enhanced permeability requires a sufficiently wide path with high conductivity, $D\exp\left(\frac{-\Delta F}{k_B T}\right) > D_0$.
The criterion for enhanced permeability is the existence of a path with resistance lower than that of an empty pore, $R_{0}$.

Consider a pore with a polymer brush that facilitates the permeation of a particle of size $d$ and interaction strength parameter $\chi_{\text{PC}}$.
As the polymer-colloid interaction strength decreases (i.e., $\chi_{\text{PC}}$ increases), making the particle less attractive, the region of space with high conductivity, $D\exp\left(\frac{-\Delta F}{k_B T}\right) > D_0$, shrinks.
This forms a narrowing path with high conductivity and negative insertion free energy, bottlenecking the transport with an effective radius $r_{\text{bn}}$ as indicated by the arrow in Figure~\ref{fig:R_map}C.

We can estimate the upper bound for the required surface tension coefficient $\gamma$ to overcome osmotic pressure and the reduction in diffusion due to the polymer mesh:
\begin{equation}
    \label{eq:gamma_crit}
    \gamma \lesssim \frac{1}{\pi d^2} \left( \ln\frac{\min\{D(r,z)\}}{D_0} + 2\ln\left( \frac{r_{\text{bn}}}{r_{\text{p}}} \right) \right) - \frac{d}{6}\Pi\left[\max\{\phi(r,z)\}\right]
\end{equation}
where $\min\{D(r,z)\}$ is the minimal value of the position-dependent diffusion coefficient and $\Pi\left[\max\{\phi(r,z)\}\right]$ is maximal osmotic pressure in the brush.
An important conclusion from the equation above is that the decrease in the diffusion coefficient or the narrowing of the path with enhanced conductivity are logarithmic terms.
This confirms that the exact model used to account for diffusion slowdown will not significantly alter the observed trends.

For the details on the upper bound for the required surface tension for facilitated permeation coefficient see
%UPDATE_MANUALLY% 
Section 10 of the Supplementary Information.

It is important to note the existence of a path with negative insertion free energy is not sufficient to ensure enhanced permeability.
As mentioned earlier, the path must also have high conductivity, $\rho^{-1} > D_0$.
For example, although $\chi_{\text{PC}} = -1.0$ at $\chi_{\text{PS}} = 0.5$ ensures the presence of a negative insertion free energy path (see Figure~\ref{fig:phi_hm_grid}), it is insufficient to overcome the reduction in diffusion caused by the polymer mesh.
When $\chi_{\text{PC}} = -0.9$, the insertion free energy becomes only slightly negative, and the negative insertion free energy path narrows, with a positive insertion free energy penalty near the interior walls. 
This leads to a resistance per unit length $\varrho_{z}$ in the interior region $z \in [-L_{0}/2, L_{0}/2]$ that exceeds even the case when only the increased viscosity of the medium is considered, as shown in Figure~\ref{fig:R_map}.

Furthermore, a further decrease in polymer-colloid interaction strength (not shown here) may completely inhibit permeation, as every possible path a colloid particle could take would involve encountering a free energy barrier.

In conclusion, polymer brushes can readily enhance the permeability of the pore by one or more orders of magnitude.

\subsection{Polymer-Filled Mesopores Sharply Select Colloids Based on Their Attraction to the Polymer}

\begin{figure}
    \centering
    \begin{subfigure}[b]{0.52\textwidth}
        \includegraphics[width=\textwidth]{fig/chi_PC_crit_on_d.png}
        \caption{}
    \end{subfigure}%
    \hspace{0.03\textwidth}
    \begin{subfigure}[b]{0.4\textwidth}
        \includegraphics[width=\textwidth]{fig/resistivity_on_chi_PC.png}
        \caption{}
    \end{subfigure}
    \caption{
        \textbf{(a)} Critical polymer-colloid interaction strength $\chi_{\text{PC}}^{\text{crit}}$ as a function of particle size $d$ at different solvent strengths, ranging from good to $\theta$-solvents ($\chi_{\text{PS}}$), as indicated by color and labeled on the right side.
        Each curve separates the parameter space into two regions: the region above a curve corresponds to impeded permeation ($R > R_{0}$), while the region below corresponds to facilitated permeation ($R < R_{0}$) for a given particle size $d$ and solvent strength $\chi_{\text{PS}}$.
        This distinction is indicated for one particular curve ($\chi_{\text{PS}}=0.5$) with solid arrow for the impeded permeation and contoured arrow for the facilitated permeation regions.
        \\
        \textbf{(b)} Total pore resistance $R$, normalized by the viscosity of the solvent $\eta_\text{S}$, as a function of polymer-colloid interaction strength $\chi_{\text{PC}}$ (thin black lines) for selected particle sizes $d \in \{2, 4, 6, \dots, 24\}$, as indicated with inline labels on the black lines, and solvent strength $\chi_{\text{PS}}=0.5$ as indicated above the panel.
        The thick brown curve separates the parameter space of enhanced permeability ($R < R_{0}$, left and below) and hindered permeability ($R > R_{0}$, right and above) as shown with the contoured and solid arrow, respectively.
        Each intersection point of the brown curve with the black lines corresponds to the critical polymer-colloid interaction strength $\chi_{\text{PC}}^{\text{crit}}$ for a given particle size $d$, where the pore resistance is equal to the resistance of an empty pore ($R = R_{0}$).
        \\
        Conditions: pore and polymer brush as in Figure~\ref{fig:phi_hm_grid}.
    }
    \label{fig:R_vs_chi_PC}
\end{figure}

Figure~\ref{fig:R_vs_chi_PC}B illustrates how the total resistance of the pore varies with the particle's affinity to the polymer brush, characterized by $\chi_{\text{PC}}$.
As expected, increasing the polymer-colloid interaction strength (i.e., decreasing $\chi_{\text{PC}}$) results in a decrease in the pore's total resistance, since the interfacial term in the insertion free energy becomes more negative, thereby increasing the local conductivity $\rho^{-1}$.
Naturally, this effect is minor for small particles at every solvent strength, as can be seen from Figure~\ref{fig:R_vs_chi_PC}B for particles up to about $d = 4$.
In contrast, for larger particles, a pore with a polymer brush can exhibit high selectivity based on the polymer-colloid interaction strength, observed as curves with steep slopes in Figure~\ref{fig:R_vs_chi_PC}B.

As the insertion free energy becomes more negative with decreasing $\chi_{\text{PC}}$, the resistance of the pore channel $R_{\text{int}}$ decreases, and the resistance due to the convergent flow in the exterior, $R_{\text{ext}}$, becomes the major contribution to the total pore resistance ($R \approx R_{\text{ext}}$).
As we have already discussed, protruding attractive polymer brushes can reduce the resistance even below that of an infinitely thin empty pore ($R < R_{\text{ext}}^{0}$).
In such cases, the total resistance is defined by the resistance of the semi-infinite reservoir, which is not modulated by the pore.
Consequently, the black lines in Figure~\ref{fig:R_vs_chi_PC}B plateau at sufficiently negative $\chi_{\text{PC}}$, signifying non-selective permeation with respect to particle affinity.

For larger particles, upon decrease in particle affinity (increase in $\chi_{\text{PC}}$) these plateaus in Figure~\ref{fig:R_vs_chi_PC}B are followed by steep slopes, indicating that even a tiny change in the polymer-colloid interaction $\chi_{\text{PC}}$ results in a large change in permeability, i.e. permeation selectivity.
% This transition from plateau to steep slope signifies a gating behavior of the pore with respect to the polymer-colloid interaction $\chi_{\text{PC}}$.
Moreover, the larger the particle size, the higher the value of $\chi_{\text{PC}}$ at which this transition from plateau (or non-selective behaviour) to high permeation selectivity occurs.

To analyze the pore resistance in Figure~\ref{fig:R_vs_chi_PC}B, we trace $\chi_{\text{PC}}^{\text{crit}}(d)$ (brown curve), which bisects the lines of total resistance into regions where $R < R_{0}$ when $\chi_{\text{PC}} < \chi_{\text{PC}}^{\text{crit}}$, and regions with impeded permeation ($R > R_{0}$).
Although the region to the right of the brown curve exhibits high selectivity with respect to the polymer-colloid interaction $\chi_{\text{PC}}$, it is also characterized by high pore resistance and low colloid flux.
Thus, the traced $\chi_{\text{PC}}^{\text{crit}}(d)$ effectively defines the pore's operative range.
The critical value $\chi_{\text{PC}}^{\text{crit}}$ as a function of particle size $d$ and solvent strength parameter $\chi_{\text{PS}}$ is shown in Figure~\ref{fig:R_vs_chi_PC}A.
An increase in solvent strength (a decrease in $\chi_{\text{PS}}$) leads to an increase in the osmotic barrier within the pore ($\Delta F_{\text{osm}}$) and a reduction in adhesion strength (an increase in $\chi_{\text{ads}}$, as per Eq.~(\ref{eq:chi_ads})).
To counteract these effects, more attractive particles are required, shifting the critical value curves to lower values of $\chi_{\text{PC}}^{\text{crit}}$.

The critical interaction parameter $\chi_{\text{PC}}^{\text{crit}}$ decreases monotonically with particle size $d$. 
This trend is explained by the increasing contribution of the osmotic term to the insertion free energy, the reduction in the diffusion coefficient $D$ with increasing particle size (Eq.~(\ref{eq:Rubinstein})), and the reduction in effective pore radius due to volume exclusion.
Among these, the increase in the osmotic term is the dominant contributor, as captured by Eq.~(\ref{eq:gamma_crit}) increased osmotic pressure $\Pi$ requires more attractive particles (lower $\gamma$ and $\chi_{\text{PC}}$).

The non-monotonic behavior observed for small particles in a $\theta$-solvent in Figure~\ref{fig:R_vs_chi_PC}A is better explained using the approximate expression for insertion free energy $\Delta F \approx \frac{\pi d^3}{6} \Pi + \pi d^2 \gamma$.
Since the dependency of insertion free energy on particle size is cubic, for low positive osmotic pressure $\Pi$ in poorer solvents and sufficiently large negative surface tension coefficients $\gamma$, the quadratic term $\gamma d^2$ overcomes the cubic term $\Pi d^3$ and an initial increase in particle size results in a decrease in insertion free energy, leading to higher conductivity $\rho^{-1}$.
% This is captured by:
% $
% \frac{\partial \Delta F}{\partial d} \approx \frac{\pi d^2}{2} \Pi + 2\pi d \gamma.
% $
% Notably, as seen from Figure~\ref{fig:R_vs_chi_PC}B, the gating with respect to $\chi_{\text{PC}}$ occurs at values of $\chi_{\text{PC}}$ that correspond to enhanced colloid transport ($R < R_{0}$), which manifests as the plateaus found on the left side to the brown curve.

%%%%%%%%%%
\subsection{Polymer filled mesopores can effectively gate colloids by their size}
%%%%%%%%%%

\begin{figure}
    \centering
    %\includegraphics[scale = 0.5]{fig/R_vs_d.png}
    \includegraphics[width = 0.95\linewidth]{fig/permeability_on_d.png}
    \caption{
    Total pore resistance $R$, normalized by the viscosity of the solvent $\eta_\text{S}$, as a function of colloid size $d$ for selected polymer-colloid interaction strengths ($\chi_{\text{PC}}$, as indicated with coloured lines) and solvent qualities ($\chi_{\text{PS}}$, as indicate above each panel). 
    The resistance of a bare pore $R_{0}$ without polymers (black thick line) separate the parameter space of enhanced permeability ($R < R_{0}$, below) and hindered permeability ($R > R_{0}$, above).
    The intersection of colored curves with the reference is critical value of particle size such that  $d_{\text{crit}} = \{d: R(\chi_{\text{PC}}, d) = R_{0}(d)\}$.
    Conditions: pore and polymer brush as in Figure \ref{fig:phi_hm_grid}. 
    }
    \label{fig:R_vs_d}
\end{figure}

Figure \ref{fig:R_vs_d} illustrates how the total resistance varies with colloid size for a range of polymer-colloid interaction strengths and solvent qualities. 
Two distinct trends are generally observed.

At sufficiently low polymer-colloid interaction strengths ($\chi_{\text{PC}} \gtrsim -1.0$), the polymer filled pore tends to be more resistant to colloid transport than the empty pore across all colloid sizes.
In this regime, the resistance increases gradually yet substantially with colloid size, owing to a combination of enhanced osmotic repulsion and reduced diffusivity.

As the polymer-colloid interaction strength increases ($\chi_{\text{PC}}$ decreases), permeability is enhanced compared to an empty pore for small colloids. 
Interestingly, the resistance remains approximately constant over a range of colloid size, until a critical colloid size $d_{\text{crit}}$ emerges above which the resistance becomes very high, effectively impeding permeation. 
In this regime, the polymer filled pore thus acts as a gate that sharply selects colloids below from colloids above a certain threshold size, as large selectivity with respect to particle size (indicated with high slope of the resistance od n particle size curves) is coupled with permeability close to a bare pore (manifested as cross-section with the bare pore resistance curve at $d = d_{\text{crit}}$).

It can be seen that with decreasing solvent strength, the level of attraction required for sharp size selectivity decreases. 
Moreover, the threshold $d_{\text{crit}}$ for impeded permeation (at any given $\chi_{\text{PC}}$) is pushed towards larger sizes.


\subsection{High Colloid Flux Implies Crowding}

\begin{figure}
    \centering
    \includegraphics[width=0.9\linewidth]{fig/streamlines.png}
    \caption{
    Stationary solution of the Smoluchowski diffusion equation for colloidal particles diffusing in a potential field defined by the position-dependent insertion free energy $\Delta F$, with a modulated diffusion coefficient, through a cylindrical pore in an impermeable membrane.
    The concentration of colloidal particles is defined in the bulk infinitely far from the membrane as $c(z = -\infty) = c_{\text{b}}$ on one side and $c(z = +\infty) = 0$ on the other.
    The normalized colloidal particle concentration $c / c_{\text{b}}$ is presented as a colormap, where white corresponds to the absence of colloidal particles, yellow to red indicates concentrations below $c_{\text{b}}$, and violet to black indicates concentrations above $c_{\text{b}}$.
    % \\
    % In the presence of polymer chains, the diffusion coefficient $D$ in and near the pore decreases compared to the diffusion coefficient in the bulk $D_0$.
    % Additionally, short-range polymer-colloid interactions create a positive or negative insertion free energy landscape.
    % \\
    Isoconcentration surfaces are shown with contours for values from 0.99 to 0.90 and from 0.10 to 0.00 in steps of 0.01.
    Most contours are labeled.
    The flux is represented by streamlines marked with small arrows, indicating the average trajectory of a massless particle.
    % \\
    % Far from the pore, the solution to the equation is symmetrical; streamlines are perpendicular to isoconcentration contours, which are oblate spheroids, similar to the analytical solution for an empty pore~\cite{Brunn1984}.
    % \\
    Conditions: spherical colloidal particle with diameter $d = 12$, polymer-colloid interaction parameter $\chi_{\text{PC}} = -1.5$ (high affinity), in a good solvent with $\chi_{\text{PS}} = 0.3$.
    The pore has a radius $r_{\text{pore}} = 26$ and membrane thickness $L_{0} = 52$.
    Brush-forming chains have a length $N = 300$ with a grafting density $\sigma = 0.02$ chains per unit area.
    }
    \label{fig:colloid_concentration}
\end{figure}

It is well-known that high concentrations of biomacromolecules that have affinity for FG motifs (such as importins, exportins, and cargo complexes) are found in the vicinity of nuclear pore complexes (NPCs)~\cite{Beck2007, Gruenwald2010, Tu2011}.
% \todo{refs to be checked}.
Moreover, it is speculated that such high concentrations are required for effective transport through the pore.% [\dots].

In Figure~\ref{fig:colloid_concentration}, we present a map of the steady-state concentration of diffusing colloidal particles for a representative pore, as in Figure~\ref{fig:phi_hm_grid}, in a good solvent ($\chi_{\text{PS}} = 0.3$).
The particle size is $d = 12$ with a polymer-colloid interaction parameter $\chi_{\text{PC}} = -1.5$.
This condition corresponds to facilitated transport with a total resistance to transport about 10-fold lower compared to an empty pore.

The colloid concentration decays rapidly to the respective bulk concentrations of the semi-infinite reservoirs, $c(z = -\infty) = c_{\text{b}}$ and $c(z = +\infty) = 0$, as seen from the contour lines.
In the absence of a polymer brush, the diffusion coefficient is not modulated, and there is no insertion free energy penalty; thus, the potential function $\psi = c$, and the isoconcentration surfaces become oblate spheroids, as can be seen from Figure~\ref{fig:colloid_concentration}.
The solution is symmetric in the sense that the isoconcentration surfaces are mirrored with respect to the pore's center, and the mirrored isoconcentration values sum up to $c_{\text{b}} = 1$.
This is demonstrated in Figure~\ref{fig:colloid_concentration}, where the contour line labels are paired accordingly.

The average colloid trajectories are shown with streamlines (contours with arrows) in Figure~\ref{fig:colloid_concentration}.

The colloid concentration is up to 30 times larger near the pore entrance at $z \approx -L_{0}/2 = -26$ and about 8 times larger in the pore channel and proximal space ($|z| \lesssim 50$), which is caused by the negative insertion free energy in the space occupied by the polymer brush.
We remind that we disregard any colloid-colloid interactions, so the presented partitioning is valid only for low concentrations in the bulk.

Notably, an increase in polymer-colloid interaction (lower $\chi_{\text{PC}}$) increases the colloid concentration in the polymer brush but also shifts the maximum concentration toward the pore center, as the relative osmotic contribution to the insertion free energy drops.
At the same time, an increase in particle size with no change in polymer-colloid interaction $\chi_{\text{PC}}$ also increases the colloid concentration near the pore entrance but inhibits colloid transport because of the higher osmotic barrier in the pore interior.

A higher magnitude of the insertion free energy also brings the steady-state solution closer to equilibrium partitioning with no flux, $c = c_{\text{b}} \exp\left( \frac{-\Delta F}{k_B T} \right)$, as the drift fluxes caused by the free energy gradient dominate diffusion fluxes driven by the colloid concentration gradient.

One can imagine a case where a transported species has a sufficiently negative surface tension coefficient, such that in regions with lower polymer concentration in the protruding part of the brush, the insertion free energy is negative and dominated by the interfacial term, while there is an osmotic barrier in the middle (as in the upper right panel in Figure~\ref{fig:DeltaF_map}).
Such species would still dock to the polymer brush but would not permeate the pore.

% \todo{This could explain}
% It has been reported that the cargo-imp$\beta$ complex strongly stains the nuclear envelope but does not efficiently enter the nuclear interior; however, when RanGDP was added, the cargos then efficiently exited the NPC and accumulated in the nucleus~\cite{Lowe2015}.
% It was also reported that imp$\beta$ is seldom found in the pore channel until CAS is present~[\dots].

Presented colloid concentration map was calculated with computational fluid dynamics, see the details in 
%UPDATE_MANUALLY%
Section 8 of Supplementary Information.


The effect of solute crowding and the corresponding penalty to insertion free energy can be accounted for using a simple model based on virial expansion, which captures this penalty at moderately low concentrations.
The insertion free energy for particles that repel each other can then be expressed as:
\begin{equation}
    \Delta F = \Delta F_{c \to 0} + \omega c,
\end{equation}
where $\Delta F_{c \to 0}$ is the insertion free energy of a single particle, as derived from Eq.~(\ref{eq:Delta_F}), and $\omega c$ represents the crowding penalty, with $\omega > 0$ being the second virial coefficient.

This effect becomes significant at high bulk concentrations of colloid particles or when there is high particle partitioning in the brush ($\Delta F_{c \to 0} \ll 0$). 
When colloid particle's crowding is accounted for, the steady-state concentration of colloid particles will be systematically lower than that predicted without considering the crowding penalty, as the resulting steady-state insertion free energy profile will be systematically higher, i.e., $\Delta F \ge \Delta F_{c \to 0}$.

For moderately low colloid concentrations, this effect does not alter the qualitative picture and is therefore outside the scope of this paper.

\section{DISCUSSION ON AND CONCLUSIONS}
Here we present a theory aiming to explain the physical principles that govern the diffusive transport of colloids through mesopores with a polymer brush grafted to the inner surface of the pore.
Typically, particles are expelled by the polymer brush because compression of the polymer chains creates an osmotic penalty, leading to a repulsive force that drives the colloid out.
Moreover, the polymer brush creates a dynamic meshwork that obstructs colloid diffusion through the brush.
These two effects hinder colloid transport.
We show that non-specific adhesive interactions of colloid particles with polymer chains can overcome osmotic expulsion and facilitate colloid transport through the pore.
Notably, the presence of a polymer brush may result in lower resistance to diffusive transport $R$ than a bare pore with no brush $R_{0}$, potentially even bringing the resistance of the pore below that of a pore in an infinitely thin membrane $R_{\text{int}}^{0}$.
We demonstrate that mesopores with polymer brushes are selective with respect particle affinity to the polymer and exhibit higher selectivity to particle size compared to a bare pore.

We explore how parameters such as solvent quality, particle size, and particle adhesion strength to the polymer control flux through the pore (or resistance) and pore selectivity.
We provide an analysis of the pore resistance $R$ as the sum of the resistances of the pore channel $R_{\text{int}}$ and the exterior regions $R_{\text{ext}}$, which is associated with resistance caused by convergent flow.
We show that pore filling modulates the resistance of both the pore channel $R_{\text{int}}$ and the regions outside the pore.
To better assess the results, we compared the resistance of the polymer brush-filled pore with that of a bare pore, $R_{0}$, as a reference. 
% We identified critical values $\chi^{\text{crit}}_{\text{PC}}$ and $d_{\text{crit}}$, which correspond to the points at which the resistance of the brush-filled pore equals that of the bare pore, i.e., $R = R_{0}$.
% The results may provide insights into the facilitated permeation of biomacromolecules from the cytoplasm to the nucleoplasm through nucleopores in the nuclear envelope.

In this paper, we study a scenario where a single cylindrical pore perforates a finite-thickness membrane separating two semi-infinite reservoirs.
The pore contains freely-jointed polymer chains grafted to the inner surface.
Colloid fluxes are analyzed for a setup in which the colloid concentration is fixed to some finite value far from the pore on one side and is zero on the other.
The question of how multiple pores in the same membrane interfere to affect permeability was first posed by Rayleigh \cite{Strutt1878}.
Fabrikant proposed a quantitative theory for a negligibly thin membrane with multiple circular apertures of different radii and arbitrary mutual positions \cite{Fabrikant1985}.
The resultant effect of pore interference is an increase in pore permeability since the exterior region's resistance is partially shared by neighboring pores.
However, this effect is relatively small (a few percent) when the distance between pore centers is greater than their diameters by an order of magnitude or more.
It is intuitively clear that when resistance due to finite pore length and the polymer brush is non-negligible, the mutual interference effect becomes even smaller.
Hence, we do not address this aspect of the problem.

The polymer concentration distribution $\phi$ (the concentration of segments of the brush-forming polymer chains) was studied using Scheutjens-Fleer self-consistent field theory for different solvent qualities.
Solvent strength was quantified using the Flory-Huggins polymer-solvent interaction parameter $\chi_{\text{PS}}$.
Upon swelling, the polymer brush occupies space outside the pore, modulating the resistance of the exterior regions.

Particle adhesion to the polymer is controlled by the Flory-Huggins interaction parameter $\chi_{\text{PC}}$.
Together with particle size $d$ and local polymer concentration $\phi$, it governs the insertion free energy $\Delta F$, which is the free energy penalty associated with embedding a particle in the brush.
In this paper, we analyze the insertion free energy landscape and show how it can shift, influenced by solvent strength, particle affinity, or particle size, 
from a free energy well dominated by favorable adhesive interactions to a free energy barrier created by osmotic repulsion.

The research employed the results of theoretical diffusion model developed by \cite{Cai2011} \emph{et al.} to assess the localized slowdown of particle diffusion due to polymer entanglement within the brush.
For particles smaller than the polymer mesh’s correlation length $\xi$, diffusion was largely unhindered, whereas larger particles exhibited hopping diffusion, becoming temporarily trapped within the mesh and awaiting relaxation of polymer chains before moving to the next position.
This dynamic resulted in a slower diffusion rate for larger particles, with diffusion coefficient decreased compared to that of a pure solvent $D/D_0 < 1$.

The local insertion free energy and local diffusion coefficient are associated with local conductivity $\rho^{-1}\equiv D \exp\left(\frac{-\Delta F}{k_B T}\right)$.
We developed an approximate integration scheme to analytically determine the resistance of the polymer brush-filled pore based on these local conductivities.

In the space defined by the parameters controlling resistance to diffusive flux, one can identify a set of conditions under which the resistance of the brush-filled pore equals that of the bare pore, i.e., $R = R_{0}$.
We identified critical permeation values $\chi^{\text{crit}}_{\text{PC}}$ and $d_{\text{crit}}$, corresponding to the polymer-colloid interaction and particle size, respectively, such that the total resistance matches that of a bare pore.
These should not be confused with the critical adsorption (denoted with 'ca') value of the polymer-colloid interaction, $\chi^{\text{ca}}_{\text{PC}} = \chi_{\text{ads}}^{\text{crit}} + \chi_{\text{PS}}(1 + \phi)$, which corresponds to the vanishing interfacial term in the insertion free energy ($\gamma = 0$, $\Delta F_{\text{sur}} = 0$ as follows from Eq.~(\ref{eq:chi_ads})), nor with the condition of vanishing insertion free energy (denoted with 'cfe'), where osmotic repulsion is exactly offset by adhesive interactions, $\chi^{\text{cfe}}_{\text{PC}} = \{\Delta F (\chi_{\text{PC}}) = 0\}$.

The order of these critical values, each fulfilling different conditions, is naturally as follows:
$\chi^{\text{crit}}_{\text{PC}} \lesssim \chi^{\text{cfe}}_{\text{PC}} \lesssim \chi^{\text{ca}}_{\text{PC}}$
% \begin{equation*}
%     \chi^{\text{crit}}_{\text{PC}} \lesssim \chi^{\text{cfe}}_{\text{PC}} \lesssim \chi^{\text{ca}}_{\text{PC}}
% \end{equation*}
 
To validate our approximate analytical solution, we also performed a full numerical solution of the Smoluchowski diffusion equation, which is illustrated in 
%%%%%%UPDATE_MANUALLY%%%%%%%%%%%%%%%%%%%%%%%%%%%%%%%%%%%%%%%%%%%%%%%%%%%%%%%%%%%
Figure 14 of the Supplementary Information.
%%%%%%%%%%%%%%%%%%%%%%%%%%%%%%%%%%%%%%%%%%%%%%%%%%%%%%%%%%%%%%%%%%%%%%%%%%%%%%%%
The corresponding data points are presented alongside analytical results in Figure \ref{fig:R_vs_d}.
Good quantitative agreement demonstrates the accuracy of our analytical theory within the limit of particle size being sufficiently smaller than the pore radius.

Here are some key findings:

\textbf{The polymer brush slows down diffusion of inert or slightly adsorbing colloid particles.}
As shown in Figures \ref{fig:R_vs_chi_PC}B and \ref{fig:R_vs_d}, particles with an interaction parameter $\chi_{\text{PC}} \gtrsim -0.75$ experience higher resistance compared to a bare pore for particles of any size.
For $\chi_{\text{PC}} \gtrsim -0.75$, the interfacial term becomes mostly positive ($\gamma \gtrsim 0$), and the insertion free energy is dominated by the osmotic repulsion term, resulting in $\Delta F > 0$.
As seen in Figure \ref{fig:R_vs_d}, larger particles are effectively blocked, with resistance values systematically exceeding those of a bare pore and growing exponentially with particle volume, $R \approx e^{\Delta F} \approx e^{\Pi V}$.

\textbf{Facilitated permeation} (i.e., resistance lower than that of a bare pore) \textbf{occurs for sufficiently attractive particles when a path with negative insertion free energy exists} in the free energy landscape such that for any cross-section perpendicular to $z$-axis exist a region with negative insertion free energy,
%$\forall z \exists r, \Delta F(r,z) < 0$,
as shown in Figure \ref{fig:DeltaF_map} for $\chi_{\text{PC}} \le -1.00$.

However, \textbf{the existence of negative insertion free energy alone does not necessarily result in facilitated permeation}.
Two effects must be overcome at sufficiently high adhesive interaction ($\chi_{\text{PC}} < \chi^{\text{crit}}_{\text{PC}}$): (i) the slowdown due to steric hindrance from the polymer meshwork, and (ii) path narrowing due to either volume exclusion or the shape of the energy landscape.
This was analyzed and illustrated in Figure \ref{fig:R_map}, where $\chi_{\text{PC}} = -1.00$ at $\theta$-solvent conditions ($\chi_{\text{PS}} = 0.5$) provides a path with negative insertion free energy (as seen in Figure \ref{fig:DeltaF_map}) but fails to facilitate transport ($R > R_{0}$).
Notably, a slight increase in particle attractiveness ($\chi_{\text{PC}}$) ensures facilitated permeation, resulting in $R < R_{0}$.

\textbf{Larger particles require higher affinity to the polymer for facilitated permeation} ($\chi_{\text{PC}} < \chi^{\text{crit}}_{\text{PC}}$) as the positive osmotic term grows with particle volume, $\Delta F_{\text{osm}} \sim d^3$, whereas the negative interfacial term scales with the particle surface area, $\Delta F_{\text{osm}} \sim d^2$.
As demonstrated in Figure \ref{fig:R_vs_chi_PC}, the $\chi^{\text{crit}}_{\text{PC}}$ curve intersects with the resistance–particle affinity curves, shifting leftward as particle size increases (to lower $\chi_{\text{PC}}$).
Similarly, in Figure \ref{fig:R_vs_d}, the intersection of resistance–particle size curves with the bare pore reference curve shifts to larger particle sizes for higher values of $\chi_{\text{PC}}$.

\textbf{The permeability of the pore with grafted polymer brush is limited by the exterior region}, which sets a lower bound on the total resistance of the pore $R_{\text{int}} \to 0 , R \approx R_{\text{ext}}$.
While this depends on boundary conditions, this phenomenon is observed for any system where the diffusing particle must travel a sufficiently long path through pure solvent via diffusion, as is the case for cargo transport through nucleopores.% [\dots].
Notably, as solvent strength changes, causing the polymer brush to swell and occupy more space in the exterior region (see Figure \ref{fig:phi_hm_grid}), the passive path through pure solvent decreases, which modulates the resistance of the exterior region, $R_{\text{ext}}$.
In the case of very large negative insertion free energy, approaching infinitely negative values ($\Delta F \to -\infty$), the total resistance becomes limited by the resistance of the exterior region, with $R_{\text{int}} \to 0$ and $R_{\text{ext}} \to \frac{1}{D_0 2 \pi r_{\text{ext}}}$.
Here, $\frac{1}{D_0 2 \pi r_{\text{ext}}}$ represents twice the resistance to diffusion flow for a particle moving from a semi-infinite solution to an ideally absorbing sphere of radius $r_{\text{ext}}$, which surrounds the exterior region where the polymer brush is still present.

Since local conductivity depends exponentially on local insertion free energy, total resistance is also exponentially dependent on the insertion free energy landscape.
As a consequence, \textbf{pore resistance is highly sensitive to parameters that influence insertion free energy, such as particle size and affinity}.
This results in very high selectivity of the polymer brush with respect to these parameters, as demonstrated in Figures \ref{fig:R_vs_chi_PC} and \ref{fig:R_vs_d}.
The dependence of insertion free energy on particle size is cubic and proportional to polymer-colloid interaction parameter $\chi_{\text{PC}}$, for larger particles, when permeability is not limited by the exterior regions a slight change in particle size $d$ or polymer-colloid interaction parameter $\chi_{\text{PC}}$ translates into a drastic change in permeability (resistance).
The selectivity of permeation both with respect to particle size and polymer-colloid interaction parameter $\chi_{\text{PC}}$ tends to increase with a particle size and adsorption strength $\chi_{\text{ads}}$.

\textbf{For a pore with a polymer brush to function as a selective transport channel for colloid particles, high permeation selectivity must be coupled with low resistance to diffusive flux.}
We refer to this combination as "gating" behavior, where a minor change in particle size can dramatically shift the permeation rate from facilitated transport to complete blockage.
Both conditions can be achieved near the critical permeation values ($d_{\text{crit}}, \chi_{\text{PC}}^{\text{crit}}$), where the resistance of the brush-filled pore equals that of the bare pore, i.e., $R = R_{0}$.

Gating with respect to particle size is illustrated in Figure \ref{fig:R_vs_d}, where a step-like change in resistance is observed for particles with high adhesion strength (low $\chi_{\text{ads}}$ and $\chi_{\text{aPC}}$) near the particle size of critical permeation, $d_{\text{crit}}$.
Here, resistance shifts from below that of a bare pore to a much higher level with only a minor change in particle size.
With decreasing solvent strength (increasing $\chi_{\text{PS}}$), the particle size $d_{\text{crit}}$ associated with gating shifts to larger values, as adsorption strength increases (indicated by a decrease in $\chi_{\text{ads}}$).

For similar reasons, larger particles with affinities close to the critical permeation value, $\chi^{\text{crit}}_{\text{PC}}$, also exhibit gating behavior in response to particle affinity as illustrated in Figure \ref{fig:R_vs_chi_PC}: a slight change in $\chi_{\text{PC}}$ can shift the permeation rate from below that of a bare pore to high resistance, with a minor decrease in interaction polymer-colloid strength (increase in $\chi_{\text{PS}}$).
Again, a decrease in solvent strength (an increase in $\chi_{\text{PS}}$) shifts $\chi^{\text{crit}}_{\text{PC}}$ to higher values, as poorer solvents enhance adsorption strength (further decreasing $\chi_{\text{ads}}$).

Finally, \textbf{facilitated permeability implies an increased concentration of colloid particles within the polymer brush}.
This follows directly from the requirement for facilitated permeation, which necessitates negative insertion free energy that causes colloid partitioning inside the brush.
The resulting colloid concentration in the steady state, $c/c_{\text{b}}$, is not identical to that of the equilibrium state, $c/c_{\text{b}}$, but approaches it as insertion free energy becomes largely negative.
We present a selected case of colloid crowding in a steady-state in Figure \ref{fig:colloid_concentration}.
Interestingly, \textbf{high partitioning of diffusing species within the brush does not necessarily equate to a high permeation rate.}
For example, in a $\theta$-solvent, an interaction parameter of $\chi_{\text{PC}} = -1.00$ ensures negative insertion free energy (as shown in Figure \ref{fig:DeltaF_map}), resulting in an increased concentration of diffusing species near the pore entrance ($c/c_{\text{b}} > 1$) without facilitating permeation ($R < R_{0}$), as seen in Figure \ref{fig:R_map}.
In other words, while high partitioning requires negative insertion free energy ($\Delta F(r,z) < 0$), facilitated permeation requires a pathway with negative insertion free energy, as we mentioned before, such that for any cross-section perpendicular to $z$-axis, there is a region with negative insertion free energy.

% In the case of a nucleopore we can envisage that high partitioning with low permeability can ensure that some macromolecule is guarantied to be found near the pore entrance while not transported. 
% For example a cargo can be transported through nucleopore only when nucleocytoplasmic transport receptors (NTR) such as importins, exportins and other auxilary molecules are attached.



% Altogether, our findings shed the light onto possible mechanisms of selective transport through NPC and, at the same time, suggest a molecular design strategy for controlling selective
% permeability through artificial mesoporous membranes with the eye on applications in...
%continued with language model
Altogether, our findings shed light on possible mechanisms of selective transport through nuclear pore complexes (NPCs) and, at the same time, suggest a molecular design strategy for controlling selective permeability through artificial mesoporous membranes, with potential applications in fields such as targeted drug delivery, biosensing, and filtration systems.
By tuning parameters like particle size, affinity, and solvent quality, it is possible to modulate transport properties and achieve desired selectivity levels in synthetic membranes.
These insights could pave the way for designing nanoporous materials with enhanced selectivity tailored to specific functional requirements, thereby broadening the scope of applications in nanomedicine, biotechnology, and environmental engineering.


\printbibliography
\end{document}




































%%%%%%%%%%%%%%%%%%%%%%%%%%%%%%%%%%%%%%%%%%%%%%%%%%%%%%%%%%%%%%%%%%%%%%%%%%%%%%%%%%%%%%%%%%%%%%%%%%%%%%%%%%
%RR: In the following, I have left some pieces of text that could be inserted into the main text where desired
%%%%%%%%%%%%%%%%%%%%%%%%%%%%%%%%%%%%%%%%%%%%%%%%%%%%%%%%%%%%%%%%%%%%%%%%%%%%%%%%%%%%%%%%%%%%%%%%%%%%%%%%%%

%Under good (or \theta-) solvent conditions we may consider separately the situations with positive and negative insertion free energies. 
%Negative insertion free energies are rather exceptional under good solvent conditions. As we see below, in this case $R_{caps}\leq R_{convergent}$ and the total resistance
%is lower than that of the empty pore.
%Positive insertion free energies under good solvent conditions are more common. 
%In this case, the resistance of the pore interior is always dominant, 
%$$
%R_{tot}\approx R_{\text{int}}
%$$
%and the accuracy in estimating the resistance contributions from the entrance/exit regions is not of a major concern. 

%In Figure \ref{fig:fe_scf_grid} the insertion free energy profiles $\Delta F(z,r=0)$ calculated by analytical scheme and by SF-SCF method 
%are presented as a function of position of a spherical particle along the pore axis.
%While the SF-SCF method provides the net free energy, the analytical scheme allows decoupling of the free energy into osmotic and surface contributions, 
%which are shown separately in Figure \ref{fig:fe_scf_grid}.
%The numerical coefficients $b_0$ and $b_1$ in eq \ref{} are chosen by the best fit, but appear to be fairly universal and independent of the particle size 
%and interaction parameters $\chi_{PS,PC}$.
%Remarkably, the fit fails when the size $d$ became comparable with the pore diameter or in the case of extreme $\chi_{ads}$ values 
%when analytical scheme is not applicable because of strong perturbation 
%of the brush structure by inserted particle, while SF-SCF method can still be safely used
%for the evaluation of the insertion free energy.

%The 2D insertion free energy $\Delta F(r,z)$ patterns have rather complex shape. However, we can trace their evolution upon changing interaction parameters
%looking at the position-dependent free energy of the particle on the pore axis, $\Delta F(z,r=0)$.
%As one can see from Figure \ref{fig:fe_scf_grid}, the insertion free energy profiles evolve upon changing the interaction parameters $\chi_{PS,PC}$ as follows:
%At $\chi_{ads}\geq \chi_{crit}$ which is the case under good or theta-solvent conditions and weak or absent polymer-particle attraction, $|\chi_{\text{PC}}|\leq 1$, the positive osmotic
%term, $\Delta F_{osm}\geq 0$ dominates in the insertion free energy, which is positive and reach maximal value in the pore center, where polymer concentration is maximal.
%Hence, polymer-particle interaction has overall repulsive character and $\Delta F(r,z)$ has the shape of the free energy barrier preventing penetration and accumulation of particles in the pore.
%By using the insertion free energy $\Delta F(r,z)$ one can calculate the equilibrium partition coefficient 
%$$
%P=\int_{0}^{r_{pore}}2\pi rdr\int_{0}^{L_{0}}dz\exp (-\Delta F(r,z)/k_BT)/\pi r^{2}_{pore}L_{0}
%$$
%is larger than unity, $P\geq 1$. Noticably the repulsive free energy profiles extends beyond the edges of the pore, because of the fringes in the polymer density distribution in swollen brush.

%A decrease in $\chi_{ads}$ triggered by a decrease in  $\chi_{\text{PC}}$ or/and an increase in $\chi_{\text{PS}}$ leads to qualitative changes in the insertion free energy 
%$\Delta F(r,z)$ patterns: At $\chi_{ads}\leq \chi_{crit}$ the particle surface becomes
%adsorbing for the polymer, $\gamma \leq 0$, that gives rise to a negative contribution $\Delta F_{surf}(r,z)$ to the insertion free energy. 
%When $\chi_{\text{PS}}$ increases (the solvent is getting worse for the polymer)
%the osmotic pressure inside the brush decreases that leads to a decrease in the 
%magnitude of $\Delta F_{osm}(r,z)$ with the concomitant shrinkage of the  protruding outside the pore parts of the brush where  $\Delta F(r,z)\neq 0$.
%As a result, the $\Delta F_{surf}(r,z)$ aquires two minima with negative values near the endtance and the exit of the pore, separated by a maximum centered in the middle of the pore
%where polymer concentration is larger and the osmotic repulsive term  $\Delta F_{osm}(r,z)$ dominates.
%Finally, at strong polymer-particle attraction $\chi_{ads} < \chi_{crit}$, the negative surface contribution $\Delta F_{surf}(r,z)\leq 0$ overperform osmotic repulsion everywhere inside the pore
%and the $\Delta F(r,z)$ aquires the shape of the potential well centered in the middle of the pore, which gives rise to preferential accumulation of particles in the pore, $P\geq 1$.