\documentclass[12pt, a4paper]{article}
\usepackage{graphicx}
\usepackage{amsmath}
\usepackage[
backend=biber,
natbib=true,
style=numeric,
sorting=none
]{biblatex}
\usepackage{xcolor}

\newcommand\todo[1]{\textcolor{red}{#1}}

\addbibresource{flow.bib}
\title{Physical principles of the colloid selective permeation through polymer-filled mesopores.}

\author{Mikhail Y. Laktionov$^1$, Leonid I.Klushin$^{1,2}$, Ralf P.Richter$^3$, Oleg V.Borisov$^1$\\
$^{1}$CNRS, Universit\'e de Pau et des Pays de l'Adour UMR 5254,\\
Institut des Sciences Analytiques et de Physico-Chimie\\
pour l'Environnement et les Mat\'eriaux, Pau, France \\
$^{2}$Institute of Macromolecular Compounds \\
of the Russian Academy of Sciences, \\
199004 St.Petersburg, Russia,\\
$^{3}$University of Leeds, School of Biomedical Sciences, \\
Faculty of Biological Sciences, 
School of Physics and Astronomy, \\
Faculty of Engineering and Physical Sciences,\\  
Astbury Centre for Structural Molecular Biology,\\ 
and Bragg Center for Materials Research,\\ 
Leeds, LS2 9JT, United Kingdom}

\begin{document}
\maketitle

\begin{abstract}
Physical mechanisms of selective permeation facilitated permeation of nanocolloidal particles 
through polymer-grafted mesopores are unravelled on the basis of self-consistent field theoretical modelling.
We predict that diffusive transport of particle can be accelerated compared to that through a bare pore due to
cohesive polymer-particle interactions, while penetration of inert with respect to the polymer particles of even smaller size can be 
efficiently impeded. We formulate thermodynamic criteria for unrestricted gating threshold through the pore and anticipate, that underlying
physical mechanisms may apply for facilitated permeation of biologically active molecules in complex with NTR through NPC.   
\end{abstract}

%%%%%%%%%%%%%%%%%%%%%%%%%
\section{Introduction}
%%%%%%%%%%%%%%%%%%%%%%%%%

Polymer-modified mesoporous materials and membranes belong to a new class of functional nanostructured materials with great potential in a number of key technologies. 
The interaction and absorption of (macro)molecules and nanocolloidal particles by porous media, as well as their transport through macro- and mesoporous membranes, 
are important elements of many technological processes (chromatography, heterogeneous catalysis, micro- and ultrafiltration, protein separation and purification etc.) 
and, therefore, have been the subject of intensive research for decades. 

Advances in macromolecular chemistry have made it possible to significantly improve functional properties of mesoporous (with a pore diameter within 100 nm) 
materials by modifying them with covalently (or strongly non-covalently) bound to the pore walls macromolecules of various chemical nature,  
forming a “soft” , solvated physical polymer meshwork that fills the pore volume (or only the near-wall regions), 
depending on molecular mass and conformational state of the polymer chains. 
The interaction of this polymer meshwork with guest molecules/nanoparticles 
%and, in particular, the presence or absence of a hollow (polymer-free) channel in the center of the pore, 
essentially determine the absorption and separation properties of the polymer-modified mesoporous materials and membranes. 
These interactions can be attractive or repulsive, short- or long-range (in the presence of charges on the chains and on guest molecules/particles), 
and most importantly, they can be controlled by a complex of external stimuli, such as temperature, pH and/or ionic strength of the medium, valence of ions , 
solvent composition, etc. This opens up a unique opportunity for highly selective and controlled uptake and transport of guest molecules/nanoparticles through mesoscopic channels. 
%For example, mesopores modified with ionic polymer molecules can potentially be used to separate molecules/nanocolloids that are almost identical in size and shape, 
%but differ in a small number of charged groups. 

Nature uses the principle of controlling the selective transport of biological molecules 
between the nucleus and cytoplasm of eukaryotic cells through the so-called nucleopores (cylindrical channels of about 40 nm in diameter), 
which perforate the nuclear envelope and are filled with a swollen meshwork consisting of natively denatured proteins attached to the pore walls. 
A similar structural motif was recently found in the internal channels of microtubules (about 15 nm in diameter) 
decorated with so-called microtubule intrinsic proteins (MEPs), presumably modifying microtubule stability and rigidity.
A popular nowadays paradigm suggests that the accuracy and efficiency of many processes in nature are ensured 
not so much by specific (molecular recognition) interactions, but due to a fine balance of fundamental (electrostatic, hydrophobic...) 
interactions between biomacromolecules and (bio) nanocolloids.  

However, up to date, the theoretical knowledge and systematic understanding of the relationship between the molecular architecture (topology) 
and the physicochemical properties of the macromolecules decorating the pore walls on the one hand and the spatial structure, 
cohesive and rheological properties of the resulting "soft" meshwork and its ability to selectively absorb in the 
volume of pores or modulate the diffusion transport of nanocolloidal particles through the pores, on the other hand,  are poor. 

In this paper we present a self-consistent field theory describing equilibrium absorption and diffusive transport of nanocolloidal particles
through mesoscopic channel in a planar membrane filled with a polymer brush grafted onto the channel walls. We investigate how an interplay of the particle size,
polymer-particle interaction and thermodynamic solvent quality for the brush-forming chains affect the diffusion rate of the particles through the pore. 
The main focus of the research is to predict how competition of repulsive (osmotic) force with polymer cohesion to the particle surface in combination with
enhanced viscosity of the polymer medium do control the  pore permeability threshold as a function of the particle size.


The rest of the paper is organized as follows: 



%%%%%%%%%%%%%%%%%%%%%%%%%%%%%%%%%%%%%%%%%%%%%%%%%%%%%%%%%%%%%%%%%%%%%%%%%%%%%%%%%%%%%%%
\section{Colloid interaction with a polymer brush in a cylindrical pore}
%%%%%%%%%%%%%%%%%%%%%%%%%%%%%%%%%%%%%%%%%%%%%%%%%%%%%%%%%%%%%%%%%%%%%%%%%%%%%%%%%%%%%%


\begin{figure}
    \centering
    \includegraphics[scale = 1.0]{fig/pore_cartoon.png}
    \caption{
        Schematic illustration of colloid particle interaction and diffusive transport through a pore filled with polymer brush. 
        The brush is formed by linear polymer chains (red strands) with a degree of polymerization $N$, uniformly grafted with grafting density $\sigma$  
        to the inner surface of a cylindrical pore in an impermeable membrane. The pore radius is $r_{pore}$ and the thickness of the membrane is $s$.
        Polymer chains are flexible with a statistical segment length $a$ and volume $\sim a^3$.
}
    \label{fig:colloid_transport}
\end{figure}

The goal of this work is to understand the transport of colloidal particles though a cylindrical pore in a membrane, the pore being decorated by a polymer brush grafted to its inner surface. 
We consider a nanoscale cylindrical pore perforating planar membrane and filled by a polymer brush which is formed  
by flexible polymer chains end-grafted to the inner surface of the pore, Figure \ref{fig:colloid_transport}. 
We aim to understand how the diffusion of solute particles between separated by the porous
membrane semi-infinite solution reservoirs is modulated by interaction of the particles with the 
polymer brush filling the pore.

We anticipate, and prove this below, that transport of colloid particles through the pore is affected by the brush in a complex manner: First, local mobility of the particles in a polymer medium
(semidilute solution to which the brush can be assimilated) is slowing down  as compared to that in pure solvent. Second, insertion of the particle into the brush at certain position
(characterized by longitudinal $z$ and radial $r$ coordinates of the particle center) leads to change in the system free energy that can be interpreted as position-dependent differential insertion free energy $\Delta F(z,r)$. 
The latter can be either positive or negative depending on the 2D polymer density distribution $\phi(z,r)$, particle volume, 
thermodynamic solvent quality for polymers and short-range interactions (repulsive or attractive) between polymer segments and particle surface. 
As demonstrated below, the negative landscape of local insertion free facilitates colloid transport, whilst free energy barrier can impede the transport.

Below we apply an approximate analytical scheme for calculating $\Delta F(z,r)$ as $\Delta F\{\phi(z,r)\}$, where $\phi(z,r)$ is polymer density distribution in a particle-free brush
The particle is considered as a "probe" which does not perturbed the density distribution $\phi(z,r)$ in the brush. 
The 2D polymer density profile $\phi(z,r)$ in the particle-free brush can be calculated for given brush architecture at arbitrary solvent quality for polymers using two-gradient SF-SCF theory.
The advantage of this scheme is that it enables evaluating of the insertion free energy $\Delta F\{\phi(z,r)\}$ at arbitrary position of the particle in the brush.
A more accurate calculation of the insertion free energy $\Delta F\{\phi(z,r=0)\}$ with the account of actual perturbation of the brush structure by inserted particle can be performed 
for the particle positioned at the axis of the filled by the brush pore, $r=0$, that is, when the whole system possesses cylindrical symmetry, 
that enables using two-gradient SF-SCF theory for the brush with a particle inside it. An excellent quantitative matching of the results obtained by these two approaches for the free energy $\Delta F\{\phi(z,r=0)\}$
of insertion of the particle along the axis of the pore justifies using of $\Delta F(z,r)=\Delta F\{\phi(z,r)\}$ calculated by approximate analytical scheme for analysis of diffusive transport of particles 
through polymer-filled pore. 

 

%%%%%%%%%%%%%%%%%%%%%%%%%%%%%%%%%%%%%%%%%%%%%%%%%%%%%%%%%%%%%%%%%%%
\subsection{Conformation of the polymer brush in the pore: a self-consistent field theory }
%%%%%%%%%%%%%%%%%%%%%%%%%%%%%%%%%%%%%%%%%%%%%%%%%%%%%%%%%%%%%%%%%%%%%%%%%%%%%%%%%%%%%%




\begin{figure}
    \centering
    \includegraphics[scale = 1.0]{fig/phi_hm_grid.png}
    \caption{
    Polymer volume fraction $\phi(z,r)$ profiles in cylindrical coordinates for a polymer brush grafted inside a cylindrical pore at varied solvent quality calculated using SF-SCF theory. 
    The solvent quality is quantified  by Flory-Huggins parameter $\chi_{PS}$ ranging from 0.1 to 0.9, that is, from good to poor solvent conditions.
    (the value of $\chi_{PS}$ is written on the top of each frame).
    \todo{The specific choice of the brush and pore parameters...}
    Polymer volume fraction profiles $\phi(z,r)$ are presented as a colormaps,
    where white color corresponds to pure solvent, yellow to magenta is low polymer concentration, blue to black correspond to high concentration.
    Horizontal axis corresponds to longitudinal coordinate $z$, vertical axis corresponds to the radial coordinate $r$.
    For illustrative reasons the colormaps are mirrored along $z$ axis, in a cylindrical coordinate system radial coordinate is always positive.
    The cylindrical coordinate system origin is shown in the lower right miniature.
    }
    \label{fig:phi_hm_grid}
\end{figure}

The conformations adopted by polymer chains grafted to the pore walls are controlled by strong (under overlapping conditions) intermolecular interactions and depend on the solvent quality. 
The latter can be quantified by Flory-Huggins solubility parameter $\chi_{PS}$ (here subscript "PS" denotes polymer-solvent interaction). 
The values of $\chi_{PS}<0.5$ and $\chi_{PS}>0.5$ correspond to good or poor solvent, respectively, whereas $\chi_{PS}=0.5$ in $\theta-$solvent.


The swelling/deswelling conformational transition triggered by variation in the solvent quality in the brush grafted inside the pore is illustrated on Figure \ref{fig:phi_hm_grid}, \ref{fig:phi_center}.
As one can see from Figure \ref{fig:phi_hm_grid}, \ref{fig:phi_center},
under good or theta-solvent conditions ($\chi_{PS}\leq 0.5$) the brush is swollen and fills not only interior of the pore but also protrudes outside the pore. 
On the contrary, in poor solvent, polymer brush is predominantly condensed inside the pore though (for chosen set of the pore and brush parameters) at moderately poor solvent strength
a fraction of the collapsed polymer chains form condensed "caps" at both ends of the pore.
It is worth noting that in the case of sufficiently wide pore and small polymerization degree/grafting density, an open channel free of polymer 
may appear under poor solvent strength conditions in the pore center, as discussed in details in ref \cite{Laktionov_Polymers}.


Colloid particles has a bulk concentration $c_0$ in one of the reservoirs and allowed to diffuse into the other reservoir with pure solvent only through the pore.
Colloid particle interaction with solvent and polymer are very complex, we consider approximate model, where the particle has homogenous surface with a given affinity to polymer and solvent.
When a particle come into contact with a polymer, the contact colloid-solvent (CS) and polymer-solvent (PS) are substituted to the contact polymer-colloid (PC).
The particles, approximated with a sphere with diameter $d$, the affinity to the polymer is controlled with Flory-Huggins interaction parameter $\chi_{PC}$, 
where subscript PC denotes polymer-colloid interaction. Without loss in generality $\chi_{CS}=0.0$, where subscript PS denotes colloid-solvent interaction.



%%%%%%%%%%%%%%%%%%%%%%%%%%%%%%%%%%%%%%%%%%%%%
\subsection{Free energy of the colloid insertion into the brush: depletion versus absorption.}
%%%%%%%%%%%%%%%%%%%%%%%%%%%%%%%%%%%%%%%%%%%%%
\subsubsection*{Analytical method}

When a colloidal particle is moved from the bulk solvent into the brush, there is a change in the system free energy $\Delta F(z,r)$, which defines insertion free energy penalty. 
Positive $\Delta F$ thus implies that the brush repels the particle, and vice versa.

As long as the particle size is larger than the correlation length in semidilute polymer solution, to which the brush can be locally assimilated, but significanltly smaller than the characteristic 
pore size (coinciding under pore filling conditions with characteristic dimensions of individual macromolecules), 
the insertion free energy $\Delta F(z,r)$ can be decoupled into two contributions: osmotic $\Delta F_{osm}(z,r)$ and surface $\Delta F_{sur}(z,r)$ free energy,

\begin{eqnarray}
    \Delta F = \Delta F_{osm} + \Delta F_{sur}
    \\
    \Delta F_{osm} = \int_{V} \Pi(z,r) dV
    \\
    \Delta F_{sur} = \oint_{S} \gamma (z,r) dS
\label{Delta_F}
\end{eqnarray}
where the integration is performed over the particle volume or the particle surface, respectively.

The osmotic contribution, $\Delta F_{osm}(z,r)$, is proportional to the particle volume and accounts for the work performed against excess osmotic pressure upon insertion of the partivle into the brush
The osmotic pressure in the brush can be calculated from local polymer density using Flory mean field approach, that leads to 
\begin{equation}
	\Pi(z,r)=  \phi(z,r)\frac{\partial f\{\phi(z,r)\}}{\partial \phi(z,r)} - f\{\phi(z,r)\}= 
	k_BT[-\ln(1-\phi(z,r)) - \phi(z,r) -\chi_{PS}\phi^2(z,r)]
\end{equation}
where
$$
f\{\phi(z,r)\}=(1-\phi(z,r))\ln(1-\phi(z,r)) +\chi_{PS}\phi(z,r)(1-\phi(z,r))
$$
is the mean-field Flory expression for the interaction free energy per unit volume of the polymer solution of concentration (volume fraction) $\phi(z,r)$.

As long as osmotic pressure inside the brush is positive, $\Pi(z,r)$, the $\Delta F_{osm}(z,r)$ term is positive as well and dominating for sufficiently large particle. 

The surface contribution is proportional to the particle surface, with surface tension coefficient $\gamma (z,r)$ approximated as
\begin{eqnarray}
    \gamma = \frac{1}{6}(\chi_{ads} - \chi_{crit})\phi^{\ast}
    \\
    \chi_{ads} = \chi_{PC} - \chi_{PS}(1-\phi^{\ast})
    \\
    \phi^{\ast}= (b_{0} + b_{1}\chi_{PC})\phi
\end{eqnarray}
Here $\gamma$ is a free energy change upon replacement of a contact of the unit surface area of the particle with the pure solvent by a contact with polymer solution of concentration $\phi$.
%When the particle placed into the brush, it creates a region depleted from polymer, if the particle is substantially attractive the region is enriched in polymer.
Coefficients $b_0$ and $b_1$ are introduced to account for  depletion/accumulation of polymer in the proximity of the colloid particle surface, thus correcting local polymer concentration in an empty brush 
to the apparent concentration $\phi^{\ast}$. Coefficients $b_0$ and $b_1$ are subject to fitting.
Depending on the strength of polymer particle interaction parameter $\chi_{PC}$ and solvent strength quantified by  $\chi_{PS}$, the sign of $\gamma \sim (\chi_{ads} - \chi_{crit})\phi^{\ast}$ may be either politive or negative.
In the former case there is a depleated of polymer layer near the particle surface, that gives rise to positive $\gamma$ and, in combination with $\Delta F_{osm}(z,r)\geq 0$, overall positive insertion free energy $\Delta F(z,r)$: 
The particle is repeled from the brush. The latter case of negative $\gamma$ corresponds to polymer adsorption on the particle surface and to competition between $\Delta F_{osm}(z,r)\geq 0$ and $\Delta F_{surf}(z,r)\leq 0$, 
that may result in $\Delta F(z,r) = \Delta F_{osm} + \Delta F_{sur}\leq 0$ and preferential accumulation of particles inside the brush. Since both $\Delta F_{osm}\{\phi(z,r)\}$ and $\Delta F_{surf}\{\phi(z,r)\}$ depend
on local polymer density $\phi(z,r)$ the net insertion free energy $\Delta F(z,r)$ is position-dependent as well, and it may exhibit rough 2D patterns.

Because the 2D distributions of the polymer volume fraction $\phi(z,r)$ in an unperturbed brush are calculated using lattice SF-SCF method, a special discretization scheme was emploied for integration 
of the the volume and the surface of either a spherical particle of diameter $d$ or a cylindrical particle with the base diameter and height both equal to $d$ (See SI for details).
%with $z_c,r_c$ being coordinates ot its center .


%%%%%%%%%%%%%%%%%%%%%%%%%%%%%%%%%%%%%%%%%%%%%%%%%%%%%%%%%%%%%%
%%%%%%%%%%%%%%%%%%%%%%%%%%%%%%%%%%%%%%%%%%%%%%%%%%%%%%
\subsubsection{Free energy calculation by numerical SF-SCF method}
%%%%%%%%%%%%%%%%%%%%%%%%%%%%%%%%%%%%%%%%%%%%%%%%%%%%%%
As follows from \todo{eq.} to construct analytical scheme one has to specify coefficient $b_0$ and $b_1$.
This was done by mapping insertion free energy obtained by analytical method 
to the free energy of the probe particle inserted at arbitrary $z$ position at the pore axis ($r=0)$ 
calculated with Scheutjens-Fleer self-consistent field (SF-SCF) numerical method.
We remind that calculation of the free energy of the pore with a particle in it by SF-SCF method  can be performed only 
for cylindrically symmetrical configuration of the system, that is, when the particle of cylindrical shape is placed at arbitrary position at the axis of the pore, $r=0$ and
the particle axis coincides with the axis of the pore (See Figure in SI).
  

%Similar to the procedure the authors employed in \todo{ref}.
In the SF-SCF scheme the colloid particles is defined as an array of the lattice cells impermeable for the polymer embedded at the pore axis.
%with some surface affinity to the polymer. 
%The particles were moved from bulk solution to the center of the pore, by defining impermeable for the polymer lattice cells.
%Not to break axial symmetry of the cylindrical pore, we consider cylindrical particle embedded in the polymer brush coaxial with the system's main axis.

The particle position is specified by the $z$-coordinate of its center. 
%of the cylinder $z_c$, where $z=0$ corresponds to a cylindrical particle placed in the pore center.
For each particle position the total free energy $F_{SF}(z_c)$ was calculated with SF-SCF scheme. 
The free energy of the system at the particle position far away from the pore (ground state) was subtrated to 
assure reference value $\Delta F_{SF}(z_c \rightarrow \pm infty) = 0$
%These values were ground state corrected such that the system total free energy of a particle in bulk solution equal to zero $F_{SF}(z_c \gg 0) = 0$.

%For fixed pore radius $r$, membrane thickness $s$, polymer grafting density $\sigma$ and polymerization degree $N$ insertion free energy along $z$ 
%were explored for different values of interaction parameters $\chi_{PC}$, $\chi_{PS}$ and particle size $d$.
%Polymer-colloid interaction parameter $\chi_{PC}$ ranged from -1.5 (attractive colloid particle) to 0.0 (inert colloid particle).
%Polymer-solvent interaction parameter $\chi_{PS}$ ranged from 0.0 (good solvent) to 1.0 (poor solvent).
%Cylindrical particles with equal diameter $d$ and height were explored for size ranging from 4 to 24 unit lengths.


\begin{figure}
    \centering
    \includegraphics[width = 4in]{fig/fe_scf_grid2.png}
    \caption{ 
    The insertion free energy calculated by analytical (SS-SCF) and SF-SCF methods for a cylindrical particles with varied diameter and height $d=[8, 12, 16]$ moving along the main axis of the pore with radius $r_{pore} = 26$ and membrane thickness $s=52$.
    Solvent quality is varied near $\theta$-point with $\chi_{PS} = [0.4, 0.5, 0.6]$, ordered from left to right column, respectively.
    Polymer volume fractions along the main axis of the pore $\phi(z,r=0)$ at different solvent qualities are shown in the first row.
    The results are presented for two particle affinities to polymer chains $\chi_{PC} = -0.5$ in the middle row and $\chi_{PC} = -1.0$, i.e. form less to more attractive particles.
    \\
    Horizontal axes corresponds to position of the particle's center $z_c$, vertical axes corresponds to polymer volume fraction for the first row or free energy value in the others.
    \\
    Insertion free energy calculated with SF-SCF scheme $\Delta F_{SF}$ is shown with squares, the results of analytical scheme $\Delta F_{SS}$ are presented with solid line. 
    The colors code particle diameter.
    The light green hatched area marks values of $z$ that corresponds to the volume inside the pore $z\in [-26, 26]$.
    \label{fig:fe_scf_grid}
    }
\end{figure}

......
\begin{figure}
    \centering
    \includegraphics[scale = 1.5]{fig/excluded_volume.png}
    \caption{
        Schematic illustration of excluded volume effect.
        \\
        As a spherical colloid particle can not come closer than $d/2$ to the membrane the excluded volume is defined by the equidistant surface and is shown as greenish area that envelops the membrane.
        Naturally, the cross-section area of the pore that supports transport shrinks to $r_{pore}^{\ast} = r_{pore} - d/2$.
    }
    \label{fig:excluded_volume}
\end{figure}

\begin{figure}
    \centering
    \includegraphics[width =5in]{fig/streamlines.png}
    \caption{
    Stationary solution for Smoluchoswki Diffusion equation for a colloid particles diffusing in a potential field with non-constant diffusion coefficient through a cylindrical pore in an impermeable membrane. Colloid particles concentration is defined in the bulk infinitely far from membrane as $c(z=-\infty) = c_0$ on the one side and $c(z=+\infty) = 0$ on the other. Colloid particles concentration $c/c_0$ is presented as a colormap, where white corresponds to absence of colloid particles, yellow to red is concentration of colloids bellow $c_0$ and violet to black is concentration above $c_0$.
    \\
    In the presence of polymer chains the diffusion coefficient $D$ in the pore and near it is decreased compared to the diffusion coefficient in in the bulk $D_0$. Also short range polymer-colloid interactions create positive or negative insertion free energy landscape.
    \\
    Isoconcentration surfaces are shown with contours for a values from 0.99 to 0.90 and from 0.10 to 0.00 with a 0.01 step. Most contours are labeled.
    The flux is shown with streamlines marked with small arrows which is the average trajectory of a massless particle. 
    Far enough from the pore the solution to the equation is symmetrical, streamlines are perpendicular to isoconcentration countours wich are oblate spheroids; similar to analytical solution for an empty pore \cite{Brunn, 1984}. 
    \\
    The solution is presented for a spherical colloid particle with diameter $d = 12$, polymer-colloid interaction parameter $\chi_{PC} = -1.5$ (high affinity) in a good solvent $\chi_{PS} = 0.3$. 
    The pore has a radius $r_{pore} = 26$ with the thickness of membrane $s = 52$.
    Brush forming chains has a length $N=300$ with a grafting density $\sigma = 0.02$ chains per unit area.
    }
    \label{fig:streamlines}
\end{figure}


\begin{figure}
    \centering
    \includegraphics[scale = 1]{fig/c_at_r0.png}
    \caption{
    Complimentary plot to Figure \ref{fig:streamlines}. 
    The curves trace colloid concentration $c/c_0$ along the main axis of the pore. 
    The blue curves correspond to the solution of Smoluchoswki Diffusion equation and the orange curves are solution to diffusion equation through a pore with no polymer brush, \textit{i.e.} constant diffusion coefficient and uniform potential.
    To refer to an empty pore solution the lower frame zooms vertical axis on the region $c \in [0, c_0]$.
    The light green hatched area marks values of $z$ that corresponds to the volume inside the pore $z\in [-26, 26]$.
    }
    \label{fig:c_at_r0}
\end{figure}


In Figure \ref{fig:fe_scf_grid} the insertion free energy profiles $\Delta F(z,r=0)$ calculated by analytical scheme and by SF-SCF method are presented as a function of position of a cylindrical particle along the pore axis.
While the SF-SCF method provides the net free energy, the analytical scheme allows decoupling of the free energy into osmotic and surface contributions, which are shown separately in Figure \ref{fig:fe_scf_grid}.
The numerical coefficients $b_0$ and $b_1$ in eq \ref{} are chosen by the best fit, but appear to be fairly universal and independent of the particle size and interaction parameters $\chi_{PS,PC}$.
Remarkably, the fit fails when the size $d$ became comparable with the pore diameter or in the case of extreme $\chi_{ads}$ values when analytical scheme is not applicable because of strong perturbation 
of the brush structure by inserted particle, while SF-SCF method can still be safely used
for the evaluation of the insertion free energy.

As one can see from Figure \ref{fig:fe_scf_grid}, the insertion free energy profiles evolve upon changing the interaction parameters $\chi_{PS,PC}$ as follows:....

The effect of the particle size on the insertion free energy is illustrated in Figure???...

by using the insertion free energy $\Delta F(z,r)$ available from analytical scheme, one can calculate equilibrium partition coefficient...


%%%%%%%%%%%%%%%%%%%%%%%%%%%%%%%%%%%%%%%%%%%%%%%%%%%
\section{Diffusive colloid transport through polymer-filled pore}
%%%%%%%%%%%%%%%%%%%%%%%%%%%%%%%%%%%%%%%%%%%%%%%%%%%

% \subsubsection{Analytical solution for empty pore problem}
% \todo{Here text from Leonid}
% The goal of this work is to understand the transport of colloidal particles though a cylindrical pore in a membrane, the pore being decorated by a polymer brush grafted to its inner surface. For that purpose, we find the stationary diffusive flux of colloidal particles through the pore and analyze how it is affected by the parameters of the pore, the brush, and the colloid. A natural starting point is the diffusive flux through an empty pore without any brush. The earliest approach to that problem goes back to Lord Rayleigh who analyzed a potential flow though a circular aperture (pore) in a planar membrane of negligible thickness while recognizing and exploiting its electrostatic and gravitational analogies\cite{Strutt1878}.  In a standard setup, the position of the membrane coincides with the XY plane at  $z=0$, and the pore is a circle of radius a. The concentration of the diffusing species is fixed to be 0 and c far away from the membrane (at   $z\rightarrow\mp\infty$, respectively).  The equipotential surfaces are oblate spheroids and the streamlines form confocal hyperboloids of revolution\cite{Cooke1966}.
The net flux through the pore is given by


\begin{equation}
\Phi=2Dac\label{eq:flux_Ral}
\end{equation}

\noindent where $D$ is the diffusion coefficient. The fact that the flux is proportional to the linear size of the pore rather than its area was a subject of some historical discussion \cite{Cooke1966}.
Diffusion through a cylindrical pore in a membrane of finite thickness $L$ also allows an analytical solution but in this case it involves an implicit infinite series \cite{Brunn1984}. The lowest order approximation turns out to be quite accurate (with an error of less than 6 percent in the full range of the $\frac{L}{a}$ ratio) and reads:

\begin{equation}
    \Phi=\frac{2Dac}{1+\frac{2L}{\pi a}}\label{eq:flux_finlength}
\end{equation}

Eq (\ref{eq:flux_finlength}) admits a most natural interpretation in terms of the total resistance, $R=\frac{c}{\Phi}$:

\begin{equation}
R=\frac{L}{D\pi a^{2}}+\frac{1}{2Da}\label{eq:resistance}
\end{equation}

The first term can be recognized as the resistance of the cylindrical pore itself (the resistivity of the medium being $D^{-1}$ while the second term is the Rayleigh resistance of the pore of infinitesimal thickness as deduced from Eq  (\ref{eq:flux_Ral}) . The latter represents the effects of the convergent flow at the entrance of the pore and its symmetric counterpart on the exit side of the membrane, while the flow lines inside the cylindrical pore turn out to be approximately axial. The relatively small error carried by the approximate solution  (\ref{eq:resistance})  is due to deviations from flow axiality inside the pore and to the corresponding minor modification of the convergent flow at the entrance/exit as compared to the case of a membrane of negligible thickness.. 
Altogether the resistance of the setup with a membrane of finite thickness and the boundary conditions imposed at $z\rightarrow\mp\infty$ coincides with that of an equivalent cylinder of the same radius $a$ and of total length $L_{eq}=L+l_{R}$   where the additional length,  $l_{R}=\frac{\pi}{2}a$ , accounts for the Rayleigh resistance contribution. The boundary conditions of fixed concentration are now imposed at the caps of the equivalent cylinder, see the cartoon illustration in Figure \ref{fig:flow_cartoon}. In what follows, we will refer to the additional cylindrical sections outside the membrane, each of length $\frac{l_{R}}{2}=\frac{\pi}{4}a$ , as the Rayleigh cylinders.
 
 
\begin{figure}
    \centering
    \includegraphics[width=0.9\linewidth]{fig/flowcartoon.pdf}
    \caption{ (a) Cartoon representing the flow lines for a pore in a thick membrane with the boundary conditions imposed far way from the membrane (at $\pm\infty$). (b) An equivalent cylinder with the boundary conditions imposed at the caps leading to strictly axial flow lines; additional cylindrical sections (transparent) represent the Rayleigh resistance and have the length of  $\frac{l_{R}}{2}=\frac{\pi}{4}a$  each, where $a$ is the radius of the pore. The equivalent cylinder accurately approximates the total flux in the situation depicted in panel (a).}
    \label{fig:flow_cartoon}
\end{figure}


The notion of the equivalent cylinder is very helpful for estimating the effects of the brush in the interior of the pore on the diffusive flux. Interaction of the brush with the diffusing particles is described via the insertion free energy profile, which is in turn linked to the profiles of the brush concentration and of the osmotic pressure \cite{Laktionov2023}. On top of that, we introduce the position-dependent diffusion coefficient which depends on the local polymer concentration and accounts for slower diffusion through a semidilute polymer mesh \cite{Laktionov2023}.
Diffusion of colloidal particles in the presence of an effective potential is described by the Smoluchowsky equation which represents a high friction limit of the Fokker-Planck equation \cite{Risken1996}:

\begin{equation}
    \frac{\partial c(\textbf{r},t)}{\partial t}=\nabla\cdotp D(\textbf{r})\left(\nabla c(\textbf{r},t)+c(\textbf{r},t)\nabla\Delta F(\textbf{r})\right)
    \label{eq:smoluchowsky}
\end{equation}
Here $c$ is the concentration of the colloidal particles, $D$ is the local (position-dependent) diffusion coefficient, and $\Delta F$ is the position-dependent free energy of insertion which plays the role of the potential of mean force.
We assume the axial (cylindrical) symmetry of the pore. Together with the stationary conditions, this implies that all the relevant functions,i.e. $c$, $\Delta F$, and $D$ depend on the axial coordinate $z$ and the radial coordinate $r$ but not on the azimuthal angle.
The stationary flux density has two components linked to the corresponding components of the gradients of the particle concentration and the insertion free energy:


\begin{equation}
j_{z}(z,r)=-D(z,r)\left(\frac{\partial c(z,r)}{\partial z}+c(z,r)\frac{\partial\Delta F(z,r)}{\partial z}\right)\label{eq:flux_axial}
\end{equation}

\begin{equation}
j_{r}(z,r)=-D(z,r)\left(\frac{\partial c(z,r)}{\partial r}+c(z,r)\frac{\partial\Delta F(z,r)}{\partial r}\right),
\label{eq:flux_radial}
\end{equation}
\noindent where $c(z,r)$ is the stationary colloid concentration.

A general analytical solution of the stationary equation is not available to our best knowledge. Here we discuss an approximate solution which amounts to neglecting the radial component of the flux density within the pore. This is inspired by the fact that the net transport across the membrane is associated only with the axial component of the flux density, and by the notion of the equivalent cylinder with the boundary conditions imposed at its caps as introduced above. 
We seek the solution for the stationary colloid concentration in the form of a modified Boltzmann distribution, similar to the planar case explored earlier \cite{Laktionov2023}:                                                         

\begin{equation}
c(z,r)=\psi(z)e^{-\Delta F(z,r)}\label{eq:stationary_c_ansatz}
\end{equation}

For the axial flux density, we obtain:

\begin{equation}
j_{z}(z,r)=-D(z,r)\psi'(z)e^{-\Delta F(z,r)}\label{eq:flux_ansatz}
\end{equation}

\noindent Here the prime in  $\psi'(z)$ stands for the derivative with respect to $z$. Stationarity implies that the net flux over any cross-section of the pore is the same, independent of the position  $z$ :

\begin{equation}
\Phi=\int_{0}^{a}2\pi rdrj_{z}(z,r)=\psi'(z)\int_{0}^{a}2\pi rdrD(z,r)e^{-\Delta F(z,r)}=const,\label{eq:fi_const}
\end{equation}

\noindent where $a$ is the radius of the pore as introduced above. Solving for $\psi(z)$  we obtain 

\begin{equation}
\psi(z)=C-\Phi\int_{0}^{z}\left(\int_{0}^{a}2\pi rdrD(z',r)e^{-\Delta F(z',r)}\right)^{-1}dz'\label{eq:psi}
\end{equation}

\noindent Here the origin of the Z-axis,  $z=0$, is placed at one of the caps of the equivalent cylinder  of total length $L_{eq}= L+\frac{\pi}{2}a$  . Assuming the insertion free energy outside the pore is zero, $ C$  can be recognized as the colloid concentration at the boundary with $z=0$ which plays the role of the source.   We impose the zero boundary condition at the opposite cap of the equivalent cylinder (the sink) and obtain for the total resistance, $R=C/\Phi$ : 

\begin{equation}
R=\int_{0}^{L_{eq}}\left(\int_{0}^{a}2\pi rdrD(z',r)e^{-\Delta F(z',r)}\right)^{-1}dz'\label{eq:res_with_brush}
\end{equation}

In our previous paper we noted that the product $D(z',r)e^{-\Delta F(z',r)}$ has the meaning of local conductivity. Then integration over the pore cross-section gives the inverse resistance per unit length (as appropriate for resistors connected in parallel) and the integration over the axial coordinate simply adds contributions from all the slices connected in series. This simple interpretation is of course consistent with neglecting the radial component of the flux density. Naturally, if the brush is absent and the insertion free energy vanishes everywhere, Eq  (\ref{eq:res_with_brush}) reduces to Eq (\ref{eq:resistance}). 
The assumption that the brush is entirely contained in the interior of the pore is well justified under poor solvent conditions. Contrary to that, in a $\Theta$- or good solvent the brush would swell producing a fringe that resides outside the pore, see Figure (NEEDS AN ILLUSTRATION!!). In this case, the flow lines at the entrance to the pore are modified and the Rayleygh resistance may not fairly represent the corresponding contribution. 
In order to produce a reliable approximate scheme for good solvent conditions we consider separately the situations with positive and negative insertion free energies. Negative insertion free energies are rather exceptional under good solvent conditions. We propose that in this case  the resistance of the entrance/exit regions is bounded between the Rayleigh resistance (without any brush effects) and the resistance of the Rayleigh cylinder filled with the actual brush fringe, and use both these estimates in our calculations.
Positive insertion free energies are much more common. In this case, the resistance of the pore interior is always dominant, and the accuracy in estimating the resistance contributions from the entrance/exit regions is not of a major concern. Hence neglect the tentative changes in the picture of the flow lines and evaluate both the pore interior and the brush fringe contributions by applying  Eq (\ref{eq:res_with_brush}) with the insertion free energy profile defined everywhere within the equivalent cylinder.
Another computational aspect that must be addressed in the case when the brush fringe extends not just beyond the membrane but beyond the caps of the additional Ryleigh cylinders as well. Then the insertion free energy is non-zero at the source and the sink boundaries, and Eq (\ref{eq:res_with_brush}) must be modified such that the free energy of insertion $\Delta F(z,r)$ is counted from the reference state that represents the colloid free energy averaged over the different radial positions along the boundary cap of the Ryleigh cylinder. Another way out is to shift the position of the boundary cap away from membrane so that the brush fringe does not touch it. Then the reference state of the colloid in a pure solvent is restored. Our results are rather insensitive to the choice of treatment of the fringe problem for the reasons discussed above.

For spherical colloidal particles of finite size the coordinates $(z,r)$ refer to the position of its center while the insertion free energy is obtained by integrating the volume and the surface contributions (see Eqs. (?)-(SEE TEXT ON FREE ENERGY EVALUATION))  over the volume and the surface of the colloid, respectively.

The question of how several pores in the same membrane interfere affecting their permeability was first posed by Rayleigh himself \cite{Strutt1878}. Fabrikant  proposed a quantitative theory for a negligibly thin membrane with several circular apertures of different radii and arbitrary mutual positions \cite{Fabrikant1985}. The resultant effect of the pore interference is an increase in the pore permeability since the Rayleigh resistance is partially shared by the neighboring pores. However, the effect is quite small (a few percent) whenever the distance between the pore centers is larger than their diameters but an order of magnitude or more. It is intuitively clear that once the resistance due to a finite pore length and due to the brush is non-negligible, the mutual interference effect becomes even smaller. Hence, we are not concerned with this aspect of the problem.

%%%%%%%%%%%%%%%%%%%%%%%%%%%%%%%%%%%%RESULTS%%%%%%%%%%%%%%%%%%%%%%%%%%%%%%%%%%%%%%%%%%%%%%%%%%%%%%%%%%%%%%%%%%%%%%%%%%%%%
\section{Discussion}
\subsection{Resistance as function of size}

\begin{figure}
    \centering
    \includegraphics[width = \textwidth]{fig/permeability_on_d.png}
    \caption{
        Total resistance $R$ of the polymer brush mesopore in a membrane with a finite thickness $s$ in semi-infinite reservoir for a spherical colloid particle as a function of particle size for different solvent quality and particle affinity.
        \\
        Solvent quality is defined with polymer-solvent interaction parameter $\chi_{PS}$ ranging from good to theta-solvent, three frames from left to right corresponds to a set of $\chi_{PS} = {0.1, 0.3, 0.5}$.
        \\
        Colored solid lines corresponds to different values of polymer-colloid interaction parameter $\chi_{PC} = {-2.0, -1.75, -1.50, -1.25}$ from more to less attractive particle. 
        The color code explained in the legend.
        \\
        Dashed black line traces analytical solution to resistance of an empty pore $R^{\textrm{empty}} = R^{\textrm{empty}}_{\textrm{convergent}} + R^{\textrm{empty}}_{\textrm{channel}}$ of of the same shape to a diffusing spherical particle of size $d$.
        Dotted black line traces the convergent flow contribution to the total resistance of the empty pore $R^{\textrm{empty}}_{\textrm{convergent}} = \frac{\pi}{2 (r_{pore} - d/2)} \frac{3 d \eta_0}{k_B T}$.
        Dash-dotted line traces the pore channel contribution to the total resistance of the empty pore
        $R^{\textrm{empty}}_{\textrm{channel}} = \frac{s}{(r_{pore} - d/2)^2} \frac{3  d \eta_0}{k_B T}$.
        \\
        Selected cases were calculated numerically using computational fluid dynamics (CFD) methods, the results shown as square markers that shares the same color as solid lines.
        \\
        Note while the region above the red line might provide better selectivity, the resistance is too high (low flux) to utilize.
        }
        \label{fig:resistivity_on_d}
\end{figure}

\subsection{Size and affinity selectivity}
\begin{figure}
    \centering
    \includegraphics[width = \textwidth]{fig/resistivity_contourplot.png}
    \caption{
        Total resistance $R$ of the polymer brush mesopore in finite thickness membrane in semi-infinite reservoir for a spherical colloid particle as a function of particle size and particle affinity at different solvent strength.
        Solvent quality is defined with polymer-solvent interaction parameter $\chi_{PS}$ ranging from good to theta-solvent, three frames from left to right corresponds to a set of $\chi_{PS} = {0.1, 0.3, 0.5}$.
        \\
        Contour lines trace isovalues of total resistance $R \frac{k_B T}{\eta_0}$ normalized by viscosity of the pure solvent $\eta_0$.
        \\
        % Selectivity is defined as relative change in total flux with respect to change in particle size or affinity (polymer-solvent interaction parameter $\chi_{PC}$).
        Selectivity with respect to size corresponds to to distances between contour lines in horizontal direction, selectivity with respect to particle affinity corresponds to distances between contour lines in vertical direction.
        The region below red lines correspond to the total resistance to lower empty pore of the same geometry for a given particle size $d$.
        While in the regions high above red lines the selectivity is enhanced, it can not be utilized due to high resistance (low flux).
    }
    \label{fig:R_contourplot}
\end{figure}


\subsection{Permeability versus partitioning}

\begin{figure}
    \centering
    \includegraphics[scale = 1.0]{fig/permeability_on_partition.png}
    \caption{
        Mapping of polymer brush pore permeability to partitioning.  for a set of colloid sizes $d = {8, 12, 16}$; as indicated atop the graphs and solvent strengths $\chi_{PS} = {0.1, 0.3, 0.5}$ with symbols and colors as indicated in the legend. 
        In the plots partitioning defined as $\textrm{PC} = \frac{1}{V_{brush}}\int_{V_{brush}} \frac{c_{eq}}{c_0} dV$ and permeability $P$ is normalized by the pure solvent viscosity $\eta_{0}$.
        Vertical red lines corresponds to $\textrm{PC}=1$, when average colloid concentration inside the pore is equal to bulk concentration.
        }
        \label{fig:permeability_on_partition}
    \end{figure}


\printbibliography


\end{document}