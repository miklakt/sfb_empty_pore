\documentclass[12pt, a4paper]{article}
\usepackage{graphicx}
\usepackage{amsmath, amssymb, amsfonts, mathtools}
\usepackage[
backend=biber,
natbib=true,
style=numeric,
sorting=none
]{biblatex}
\usepackage{xcolor}

\newcommand\todo[1]{\textcolor{red}{#1}}

\addbibresource{biblio.bib}
\title{Physical principles of selective colloid permeation through polymer-filled mesopores}

\author{Mikhail Y. Laktionov$^1$, Leonid I.Klushin$^{2}$,\\Ralf P.Richter$^3$, France A.M. Leermakers$^4$, Oleg V.Borisov$^1$\\
$^{1}$CNRS, Universit\'e de Pau et des Pays de l'Adour UMR 5254,\\
Institut des Sciences Analytiques et de Physico-Chimie\\
pour l'Environnement et les Mat\'eriaux, 64053 Pau, France \\
$^{2}$Institute of Macromolecular Compounds \\
of the Russian Academy of Sciences, \\
199004 St.Petersburg, Russia,\\
$^{3}$University of Leeds, School of Biomedical Sciences, \\
Faculty of Biological Sciences, 
School of Physics and Astronomy, \\
Faculty of Engineering and Physical Sciences,\\  
Astbury Centre for Structural Molecular Biology,\\ 
and Bragg Center for Materials Research,\\ 
Leeds, LS2 9JT, United Kingdom\\
$^{4}$ University of Wageningen, the Netherlands
}

\begin{document}
\maketitle

\begin{abstract}

% RR: I have note worked on the abstract yet.

Physical mechanisms of selective facilitated permeation of nanocolloidal particles 
through polymer-grafted mesopores are unravelled on the basis of self-consistent field theoretical modelling.
We predict that diffusive transport of particle can be accelerated compared to that through a bare pore due to
cohesive polymer-particle interactions, while penetration of inert with respect to the polymer particles of even smaller size can be 
efficiently impeded. We formulate thermodynamic criteria for unrestricted gating threshold through the pore and anticipate, that underlying
physical mechanisms may apply for facilitated permeation of biologically active molecules in complex with NTR through NPC.   
\end{abstract}

%%%%%%%%%%
\section{INTRODUCTION}
%%%%%%%%%%

% RR: I have not worked on this part of the introduction yet.

Polymer-modified mesoporous materials and membranes belong to a new class of functional nanostructured materials with great potential in a number of key technologies. 
The interaction and absorption of (macro)molecules and nanocolloidal particles by porous media, as well as their transport through macro- and mesoporous membranes, 
are important elements of many technological processes (chromatography, heterogeneous catalysis, micro- and ultrafiltration, protein separation and purification etc.) 
and, therefore, have been the subject of intensive research for decades. 

Advances in macromolecular chemistry have made it possible to significantly improve functional properties of mesoporous (with a pore diameter within 100 nm) 
materials by modifying them with covalently (or strongly non-covalently) bound to the pore walls macromolecules of various chemical nature,  
forming a “soft” , solvated physical polymer meshwork that fills the pore volume or only the near-wall regions, 
depending on molecular mass and conformational state of the polymer chains. 
The interaction of this polymer meshwork with guest molecules/nanoparticles 
%and, in particular, the presence or absence of a hollow (polymer-free) channel in the center of the pore, 
essentially determine the absorption and separation properties of the polymer-modified mesoporous materials and membranes, respectively. 
These interactions can be attractive or repulsive, short- or long-range (in the presence of charges on the chains and on guest molecules/particles), 
and most importantly, they can be controlled by a complex of external stimuli, such as temperature, pH and/or ionic strength of the medium, valence of ions , 
solvent composition, etc. This opens up a unique opportunity for highly selective and controlled uptake and transport of guest molecules/nanoparticles 
through polymer-filled mesoscopic channels. 
%For example, mesopores modified with ionic polymer molecules can potentially be used to separate molecules/nanocolloids that are almost identical in size and shape, 
%but differ in a small number of charged groups. 

Nature uses the principle of controlling the selective transport of biological molecules 
between the nucleus and cytoplasm of eukaryotic cells through the so-called nucleopores (cylindrical channels of about 40 nm in diameter), 
which perforate the nuclear envelope and are filled with a swollen meshwork consisting of natively denatured proteins attached to the pore walls. 
A similar structural motif was recently found in the internal channels of microtubules (about 15 nm in diameter) 
decorated with so-called microtubule intrinsic proteins (MEPs), presumably modifying microtubule stability and rigidity.
A popular nowadays paradigm suggests that the accuracy and efficiency of many processes in nature are ensured 
not so much by specific (molecular recognition) interactions, but due to a fine balance of fundamental (electrostatic, hydrophobic...) 
interactions between biomacromolecules and (bio) nanocolloids.  

However, up to date, 
%the theoretical knowledge and systematic 
understanding of the relationship between molecular architecture of the brush 
decorating the pore walls and the spatial structure, 
cohesive and rheological properties of the resulting "soft" meshwork and its ability to selectively absorb in the 
volume of pores or modulate the diffusion transport of nanocolloidal particles through the pores is lacking. 

% RR: I have prepared the following final paragraph.

Our analysis focuses on pores that are pervaded by a dynamic meshwork of flexible polymers. 
This meshwork effectively increases the local viscosity and thereby slows down transport of colloids compared to an open pore. 
On the other hand, an attractive polymer phase recruits colloids into the pore, thus increasing colloid transport, 
and such recruitment is further enhanced when attractive polymers extend outside the pore. 
Intriguingly, the solvent strength through its influence on the density and compactness of the polymer plug impacts all these effects. 
Here, using a self-consistent field approach, we define how solvent quality and colloid attraction to the polymer may be tuned 
to maximise the transport rate (including beyond the rate for an open pore) or for the highly selective transport of colloids of a desired size or attraction to the polymers.


%%%%%%%%%%
\section{RESULTS}
%%%%%%%%%%


%%%%%%%%%%
\section{Defining the transport scenario}
%%%%%%%%%%

\begin{figure}
    \centering
    \includegraphics[scale = 0.5]{fig/pore_cartoon.png}
    \caption{
        Schematic illustration of colloid diffusive transport through a pore filled with polymer brush. 
        The brush is formed by linear polymer chains (red strands) with a degree of polymerization $N$, uniformly grafted with grafting density $\sigma$ 
        to the inner surface of a cylindrical pore in an impermeable membrane. The pore radius is $r_{\text{pore}}$ and the thickness of the membrane is $L$.
        Polymer chains are flexible with a statistical segment length $a$ and volume $\sim a^3$. 
        Spherical colloids with diameter $d$ are free to diffuse in the surrounding solvent.
        All length scales are normalized by the statistical segment length $a$.
        }
    \label{fig:colloid_transport}
\end{figure}

The salient features of our simulated mesopore (Figure \ref{fig:colloid_transport}) are inspired by the nuclear pore complex.
Pores with such features can though also be realized synthetically.
The pore has a cylindrical shape with diameter $r_{\text{pore}}$.
The pore perforates an otherwise impermeable, planar membrane of thickness $L$, and thus is the sole conduit for colloids between two semi-infinite solution reservoirs.
Flexible polymer chains are end-grafted to the inner pore walls, at a density sufficient to form a polymer brush that penetrates the entire pore cross-section.
We aim to define how the diffusive transport of colloids across the pore is modulated by the polymer brush. 
The interaction strength between a polymer segment and a unit surface area of the colloid is represented by the Flory-Huggins parameter $\chi_{\text{PC}}$. 


%%%%%%%%%%
\section{Colloid transport is defined by the sum of resistances outside and inside the pore}
%%%%%%%%%%

To understand how the polymer brush affects the transport of colloids, we consider the stationary diffusive flux of colloids through the pore 
and analyze how it is affected by the parameters of the pore, the brush, and the colloid.
To this end, the colloid concentration is fixed to be zero and $c$ far away from the membrane (at $z\rightarrow\mp\infty$, respectively). 
We assume axial (cylindrical) symmetry of the flow in the pore. Together with the stationary conditions, 
this implies that parameters relevant to colloid transport depend on the axial coordinate $z$ and the radial coordinate $r$, but not on the azimuthal angle.

%%%%%%%%%%
\subsection{Empty pore as a reference case}
%%%%%%%%%%

A natural reference is the diffusive flux through an empty pore without any polymer. The earliest approach to that problem goes back to Lord Rayleigh 
who analyzed the flux through a circular pore in a planar membrane of negligible thickness \cite{Strutt1878}. 
The equiconcentration surfaces in this case are oblate spheroids and the streamlines form confocal hyperboloids of revolution \cite{Cooke1966}.
The net flux through the pore is given by

\begin{equation}
    J=2D_0r_{\text{pore}}c
    \label{eq:flux_Ral}
\end{equation}

\noindent where $D_0$ is the diffusion coefficient of the colloid in plain solvent. 

Diffusion through a pore in a membrane of finite thickness $L$ allows an approximate analytical solution (with an error of less than 6\% in the full range of the $\frac{L}{r_{\text{pore}}}$ ratio) \cite{Brunn1984}: 

\begin{equation}
    J=\frac{2D_0r_{\text{pore,eff}}}{1+\frac{2L}{\pi r_{\text{pore,eff}}}}c
    \label{eq:flux_finlength}
\end{equation}

\noindent To account for the excluded volume of the diffusing colloids, assuming them being spheres of diameter $d$, we here use the effective pore radius $r_{\text{pore,eff}}=r_{\text{pore}}-d/2$.

Introducing the resistance $R$ to colloid flow as $J=\frac{c}{R}$ admits a most natural interpretation of Eq. (\ref{eq:flux_finlength}) in terms of the total resistance of the pore

\begin{equation}
    R=\frac{L}{D_0\pi r_{\text{pore,eff}}^{2}}+\frac{1}{2D_0r_{\text{pore,eff}}}=R_{\text{empty}}+R_{\text{convergent}}
    \label{eq:resistance}
\end{equation}

\noindent The first term in Eq. (\ref{eq:resistance}) is the resistance of the interior of the empty pore, 
while the second term is the Rayleigh resistance of a pore of infinitesimal thickness (Eq. (\ref{eq:flux_Ral})). 
The latter represents the effects of the convergent flow towards the pore entrance and its symmetric counterpart at the pore exit, 
while the flow lines inside the cylindrical pore turn out to be approximately axial. 
In the empty pore scenario, the inverse of the diffusion constant ($\rho_0=D_0^{-1}$) represents the resistivity of the medium both inside and outside the pore. 

%%%%%%%%%%
\subsection{A polymer filling affects the resistance of the pore itself, and also of regions outside the pore}
%%%%%%%%%%

\begin{figure}
    \centering
    \includegraphics[scale = 0.5]{fig/phi_hm_grid.png}
    \caption{
    Maps of the polymer volume fraction $\phi(z,r)$ for a polymer brush in a cylindrical pore at a range of solvent qualities, as predicted by self-consistent field theory. 
    The solvent quality is quantified by the Flory-Huggins parameter $\chi_{\text{PS}}$ ranging from 0.1 (good solvent) to 0.9 (poor solvent).
    Polymer volume fractions are mapped in cylindrical coordinates (as shown in the inset), colour coded as indicated and with selected iso-concentration lines shown. 
    For illustrative purposes the colormaps are mirrored along the $z$ axis, and the membrane is indicated in green.
    Conditions: $L=2r_{\text{pore}}=56$, $\sigma=0.02$, $N=300$.
    }
    \label{fig:phi_hm_grid}
\end{figure}

The conformations adopted by polymer chains grafted to the pore walls are controlled by strong (under overlapping conditions) intermolecular interactions, and depend on the solvent quality. 
The solvent quality is quantified by the Flory-Huggins solubility parameter $\chi_{\text{PS}}$. 
Values of $\chi_{PS}<0.5$ and $\chi_{PS}>0.5$ correspond to good and poor solvent, respectively, whereas $\chi_{PS}=0.5$ represents the ideal (or $\theta-$)solvent.

The polymer density profile $\phi(z,r)$ in the pore was calculated by two-gradient self-consistent field theory according to Scheutens and Fleer (SF-SCF; see SI for details).
In Figure \ref{fig:phi_hm_grid}, one can appreciate the expected increase in polymer concentration with decreasing solvent quality (increasing $\chi_{\text{PS}}$). 
It is notable that the variations in polymer concentration as a function of solvent strength are substantial, 
and that the concentration is more or less homogeneous across the polymer-filled space. 
The polymer brush penetrates the entire pore cross-section across the full range of solvent qualities explored ($0.1\le\chi_{\text{PS}}\le0.9$), 
implying that colloids need to navigate the polymer medium to reach across the pore.

We remind that swelling of the brush-forming chains with respect to unperturbed Gaussian dimensions occurs under both good ($\chi_{PS} \leq 0.5$) and $\theta$-solvent ($\chi_{PS}= 0.5$) conditions due to binary or ternary monomer-monomer repulsions, respectively.
In contrast, in poor solvent, the polymer brush is predominantly condensed.
It is worth noting that in the case of sufficiently wide pore and small polymerization degree/grafting density, 
an open channel free of polymer may appear under poor solvent conditions in the pore center, as discussed in details in ref \cite{Laktionov2021}. 
This scenario would lead to a distinct permeation behaviour as the colloids can move through the pore without traversing the brush, and is not further considered here. 

Figure \ref{fig:phi_hm_grid} further illustrates that whilst the brush remains confined within the pore lumen in poor solvent ($\chi_{\text{PS}}=0.9$) 
it protrudes substantially into the surrounding space in ideal and good solvent ($\chi_{\text{PS}}\le0.5$), thus forming 'caps' on either side of the pore.  
The polymer brush therefore will impact on colloid flow within as well as outside the pore, such that

\begin{equation}
    R=R_{\text{pore}}+R_{\text{caps}}
    \label{eq:R_tot_tot}
\end{equation}
    
\noindent with $R_{\text{pore}}\rightarrow R_{\text{empty}}$ and $R_{\text{caps}}\rightarrow R_{\text{convergent}}$ in the limit of the empty pore. 


%%%%%%%%%%
\section{Insertion free energy and diffusivity define transport locally}
%%%%%%%%%%

Zooming in on the local scale, we can define how colloids are accumulated or depleted due to attractive or repulsive interactions, respectively, by the presence of the polymer meshwork, 
and how the polymer meshwork affects the rate of diffusion.

%%%%%%%%%%
\subsection{Insertion free energy is a balance of volume and surface effects}
%%%%%%%%%%

The insertion free energy $\Delta F(z,r)$ defines the energy penalty upon moving a colloid from plain solvent into the polymer meshwork.
A positive $\Delta F$ thus implies that the brush repels the particle, and vice versa.

For colloids that are significantly smaller than the size of the pore, the insertion free energy is determined entirely by the local polymer concentration (i.e., wall effects can be neglected), 
and made up of two distinct contributions, one osmotic and the other interfacial.

\begin{eqnarray}
    \Delta F (z,r)= \Delta F_{\text{osm}}(z,r) + \Delta F_{\text{int}}(z,r)
    \\
    \Delta F_{\text{osm}}(z,r) = \int_{V} \Pi(z,r) dV
    \\
    \Delta F_{\text{int}}(z,r) = \int_{S} \gamma (z,r) dS
    \label{eq:Delta_F}
\end{eqnarray}

\noindent The coordinates $(z,r)$ here refer to the center of the colloid, whilst the insertion free energy is obtained by integrating over the colloid volume and surface, respectively.

The osmotic contribution, $\Delta F_{\text{osm}}(z,r)$, is proportional to the colloid volume 
and accounts for the work performed against excess osmotic pressure upon insertion of the particle into the brush. 
The local osmotic pressure is calculated from the local polymer concentration using a Flory mean field approach 

$$
\Pi(z,r)=  \phi(z,r)\frac{\partial f\{\phi(z,r)\}}{\partial \phi(z,r)} - f\{\phi(z,r)\}= 
$$
\begin{equation}
	k_{\text{B}}T[-\ln(1-\phi(z,r)) - \phi(z,r) -\chi_{\text{PS}}\phi^2(z,r)]
\end{equation}

\noindent where
$$
f\{\phi(z,r)\}=(1-\phi(z,r))\ln(1-\phi(z,r)) +\chi_{\text{PS}}\phi(z,r)(1-\phi(z,r))
$$

\noindent is the mean-field Flory expression for the interaction free energy per unit volume of the polymer solution of concentration (volume fraction) $\phi(z,r)$.
As long as the osmotic pressure inside the brush is positive, $\Delta F_{\text{osm}}$ is positive as well and dominates for sufficiently large colloids. 

The interfacial contribution, $\Delta F_{\text{int}}(z,r)$, is proportional to the colloid surface, 
with the surface tension $\gamma (z,r)$ approximated as

\begin{eqnarray}
    \gamma (z,r)= \frac{1}{6}(\chi_{\text{ads}} - \chi_{\text{crit}})\phi^{\ast}(z,r)
    \\
    \chi_{\text{ads}} = \chi_{\text{PC}} - \chi_{\text{PS}}(1-\phi^{\ast})
    \\
    \phi^{\ast}(z,r)= (b_{0} + b_{1}\chi_{\text{PC}})\phi(z,r)
\end{eqnarray}

\noindent Here $\gamma$ is a free energy change upon replacement of a contact of the unit surface area of the colloid with plain solvent by a contact with a polymer solution of concentration $\phi(z,r)$.
Coefficients $b_0$ and $b_1$ are adjustable parameters to account for depletion/accumulation of polymer in the proximity of the colloid surface, 
thus adjusting the local polymer concentration in a plain brush to the effective concentration $\phi^{\ast}(z,r)$.
%Need to add a reference to the fitting procedure in the supporting info.
Depending on the relative strengths of polymer-colloid ($\chi_{\text{PC}}$) and polymer-solvent ($\chi_{\text{PS}}$) interactions, 
the sign of $\gamma\sim(\chi_{\text{ads}}-\chi_{\text{crit}})\phi^{\ast}$ may be either positive or negative.

As the maps of the polymer volume fraction $\phi(z,r)$ in an unperturbed brush (Figure \ref{fig:phi_hm_grid}) were calculated using a lattice based method, 
a special discretization scheme was employed for the integration of volumes and surfaces in Eqs. \todo{add reference} (see SI for details).

In what follows, we apply an approximate analytical scheme for calculating the insertion free energy $\Delta F(z,r)$ as $\Delta F\{\phi(z,r)\}$, 
where $\phi(z,r)$ is the polymer density distribution in a colloid-free brush. 
The colloid is thus considered within this analytical approach as a 'probe' which does not perturb the global concentration distribution $\phi(z,r)$ in the brush. 
The advantage of this scheme is that it enables evaluating the insertion free energy at any position of the colloid in the brush.
A more accurate calculation of the insertion free energy with account of the actual perturbation of the brush structure by the inserted colloid is possible with SF-SCF simulations,
but only for colloids positioned along the pore axis ($r=0$). 
Comparison of the approximate analytical approach with the SF-SCF simulations for colloids on the pore axis demonstrated good quantitative agreement (see SI), 
thus justifying the use of the more versatile analytical approach.

\begin{figure}
    \centering
    \includegraphics[scale = 0.5]{fig/free_energy_hm.png}
    %\includegraphics[scale = 1.0]{fig/DeltaF_map.png}
    \caption{
    Maps of the insertion free energy $\Delta F(z,r)$ for a polymer brush in a cylindrical pore for a range of polymer-colloid interaction strengths. 
    The polymer-colloid interaction strength is quantified by the Flory-Huggins parameter $\chi_{\text{PC}}$ ranging from -0.50 (least attractive) to 1.25 (most attractive), as indicated. 
    Insertion free energies are mapped in cylindrical coordinates (as in Figure \ref{fig:phi_hm_grid}), and colour coded as indicated. 
    For illustrative purposes the colormaps are mirrored along the $z$ axis, and the membrane is indicated in green. 
    Conditions: pore and polymer brush as in Figure \ref{fig:phi_hm_grid}, $\chi_{\text{PS}}=0.5$, $d=8$.
    }
    \label{fig:DeltaF_map}
\end{figure}

Figure \ref{fig:DeltaF_map} illustrates how colloids may be either repelled or attracted by the polymer meshwork, 
depending on the balance of osmotic and interfacial contributions to $\Delta F(z,r)$. 
Since both $\Delta F_{\text{osm}}(z,r)$ and $\Delta F_{\text{int}}(z,r)$ depend on the on local polymer concentration $\phi(z,r)$, 
the net insertion free energy $\Delta F(z,r)$ is position-dependent, and may exhibit quite large spatial variations. 
For example, the brush shown in Figure \ref{fig:DeltaF_map} switches from attraction at the caps to repulsion inside the pore at $\chi_{\text{PC}}=-0.75$.


%%%%%%%%%%
\subsection{Diffusivity is determined by the ratio of polymer mesh size and colloid size}
%%%%%%%%%%

The presence of a semi-dilute polymer meshwork also slows the rate of colloid movement compared to plain solution, leading to a position-dependent diffusion coefficient $D\{\phi(z,r)\}$ \cite{Laktionov2023}. 
We capture this effect through the scaling relation

\begin{equation}
    D\{\phi(z,r)\} = \frac{D_0}{1+d^2/\xi^{2}\{\phi(z,r)\}}
    \label{eq:Rubinstein}
\end{equation}

\noindent where the correlation length (or 'mesh size') $\xi$ is controlled by the local polymer concentration $\phi(z,r)$. 
As follows from Eq. (\ref{eq:Rubinstein}), particles smaller than the mesh size diffuse virtually unhindered ($D\{\phi(z,r)\}\approx D_0$ for $d\leq \xi$). 
In contrast, large colloids are significantly slowed down by the polymer medium compared to plain solvent ($D\cong D_0 (\xi/d)^2\ll D_0$ for $d\gg \xi$). 
We approximate the dependence of the mesh size on polymer concentration by the scaling dependence valid close to ideal solvent conditions, $\xi\cong \phi^{-1}$. 

%Do we need a figure here that shows illustrative map of D/D0? 

%%%%%%%%%%
\subsection{Linking local resistivities to global transport}
%%%%%%%%%%

We have derived an approximate solution to the stationary diffusion equation, 
which effectively amounts to neglecting the radial component of colloid fluxes inside the pore (see SI for details). 
This approach is inspired by the fact that the net transport across the membrane is associated only with the axial component of the fluxes. 
The solution enables the explicit calculation of the resistance of the pore itself

\begin{equation}
    R_{pore}=\int_{0}^{L}\left(\int_{0}^{r_{\text{pore}}}2\pi rdrD(z',r)e^{-\Delta F(z',r)}\right)^{-1}dz'
    \label{eq:res_with_brush}
\end{equation}
    
\noindent The product $D(z',r)e^{-\Delta F(z',r)}$ here has the meaning of local conductivity, 
and collects the local effects of the polymer meshwork on diffusivity and insertion free energy. 
Integration of the conductivy over the pore cross-section gives the inverse resistance per unit length, as appropriate for resistors connected in parallel.
Further integration of this resistance over the axial coordinate simply adds contributions from all the slices, as appropriate for resistors connected in series. 

Provided that the brush is entirely contained in the interior of the pore, 
the overall resistance can be presented as $R=R_{\text{convergent}}+ R_{\text{pore}}$ (see Eq. \ref{eq:R_tot_tot}),
where the first term represents the contribution of convergent/divergent flows at the entrance/exit ofthe pore, 
and the second term accounts for the resistance of the pore itself.

Whilst the assumption that the brush is entirely contained in the interior of the pore is well justified under poor solvent conditions, 
a brush in good or $\theta$-solvent conditions produces caps outside the pore (Figure \ref{fig:phi_hm_grid}). 
In these cases, the flow lines at the entrance to and exit from the pore are modified, 
and the Rayleygh resistance specified by Eq. \ref{eq:flux_Ral} does not fairly represent the corresponding contribution. 
To calculate the contribution of the brush caps to the overall resistance, we implemented an approximate numerical integration scheme (see SI for details).

To calculate the exterior and interior resistance of the pore generally, we consider an orthogonal curvilinear coordinate system $x_{\theta}, x_{r}, x_{z}$ with iso-surfaces defined as follows:
    half-planes with a constant azimuthal angle $x_{\theta}$;
    stream surfaces of diffusing particles constant $x_{r}$ indexed with a radius $r$ of its intersection at $z=0$;
    the level set of the potential function $\psi$ with a constant $x_{z}$ indexed with its intersection with $z$-axis

For a thin empty pore this curvilinear coordinate system is a variant of oblate spheroidal coordinate system, where surfaces of constant $x_{\theta}$ are half-planes, surfaces of constant $x_{r}$ are confocal hyperboloids of revolution, and surfaces of constant $x_{z}$ are confocal oblate spheroids. The focal circle is a pore circumference.

For an empty pore with a finite thickness the iso-surfaces can be approximated such that the resulting curvilinear coordinate system is a joint of cylindrical and oblate spheroidal coordinate system.
Where cylindrical coordinates are used inside the pore $z \in [-L/2, L/2]$ and oblate spheroidal otherwise.

Oblate spheroid coordinate system slice the exterior space into curved layers.
Similar to the integration of the resistance of the interior, with the exception that the layers are curved, we first integrate conductivity over the spheroid layers, which gives us the inverse resistance per unit length.
Further we integrate the axial coordinate to sum up the contribution of each curved layer.

The inverse resistance of each layer scales with the square distance to the pore entrance $(z-L/2)^2$.
The integral over the axial coordinate for the exterior layers is a $R_{\textrm{convergent}} = 1/2 D_0 r_{\textrm{pore}}$, following Rayleigh's analytical solution.
Finally, given the nature of the effective pore shape (rounded corners when excluded volume is accounted) we assume that the resistance of each layer in the exterior and in the interior is a continuous function.

\begin{eqnarray}
    % R_{z} = \rho^{-1}(z) \oint_{\Phi_z} h_{z} dA =\rho^{-1}(z) \int_{0}^{r_{\textrm{pore}}} \int_0^{2\pi} h_{r} h_{\theta} h^{-1}_{z} dx_{r} dx_{\theta}
    R_{z, \textrm{empty}}^{-1} = 
    \begin{cases}
        \pi r_{\textrm{pore}}^2 \textrm{, if } |z| < L/2
        \\
        \pi (4(z-L/2)^2 + r_{\textrm{pore}}^2) \textrm{, otherwise} 
    \end{cases}
    \\
    R_{\textrm{empty}} = \int_{-\infty}^{+\infty }R_z dz = \frac{L}{D_0\pi r_{\text{pore}}^{2}}+\frac{1}{2D_0r_{\text{pore}}}
\end{eqnarray}

The presence of the polymer brush and the change in the effective pore shape modulate the transport properties of the pore and result in the different flux streamlines and iso-potential surfaces.
The coordinate system for a pore with the polymer brush for a finite size particle diverges from the coordinate system for an empty pore for a point-like particle in the interior of the pore and in the exterior proximal to the entrance.
One can account for these effects but it is a rather complicated and results only in minor correction to the pore resistance.

We disregard this effects and consider that there is no change in iso-potential surfaces and flux streamlines.
Furthermore, the local conductivities we calculated with SCF are discrete values on a regular lattice, in this discrete cylindrical lattice when integrating we treat iso-potential lines as a surface of half-cylinder rather than a surface of an oblate spheroid.
Finally, a half-cylinder have a larger surface area than an oblate spheroid for the same $x_{z}$, which is corrected with a prefactor in the integrand.

The resistance integration on the regular cylindrical lattice:
\begin{gather}
    R_z^{-1} =
    \begin{cases}
         \pi \sum_{r=0}^{r_{\textrm{pore}}} \rho^{-1}_{[r,z]} (2r+1) \textrm{, if } z\in[-L/2,L/2]
         \\
         \pi \left(\sum_{r=0}^{r_{\textrm{base}}} \rho^{-1}_{[r,z]} (2r+1) + 2 r_{\textrm{base}} \sum_{z^{\prime} = z_{a}}^{z_{b}}\rho^{-1}_{[r,z^{\prime}]}\right) f(z) \textrm{, otherwise }
    \end{cases}
    \\
    \begin{cases}
        \textrm{if } z < -L/2, z_{a} = z, z_{b} = -L/2-1
        \\
        \textrm{if } z > L/2, z_{a} = L/2, z_{b} = z
    \end{cases}
    \\
    r_{\textrm{base}} = r_{\textrm{pore}} + |z| - L/2
    \\
    f(z) = \frac{r_{\textrm{pore}}^2 + 4(z-L/2)^2}{(\textrm{pore} + 3|z-L/2|)(\textrm{pore} + |z-L/2|)}
\end{gather}
To account the resistance of semi-infinite reservoirs outside the integration region
\begin{eqnarray}
    R_{(z, \pm\infty)} = \frac{\pi - 2\arctan\left(\frac{2(|z|-L/2)}{r_{\textrm{pore}}}\right)}{4\pi r_{\textrm{pore}}}
\end{eqnarray}
Finally, the total resistance of the pore, integrated on the discrete cylindrical lattice
\begin{equation}
    R = \sum_{z=z_{a}}^{z_{b}} R_z + R_{(z_{a}, -\infty)} + R_{(z_{b}, +\infty)}
\end{equation}

(The details are in SI).


% \noindent where $h_{z}, h_{\theta}, h_{r}$ are scale factors of the curvilinear coordinates (a ratio of the distance between two iso-surfaces for a given coordinate relative to a unit step along the normal axis) and ${\Phi_z}$ a level set value for $\psi$.

% We further simplify it with few assumptions assumptions. 
% First, the total area of oblate spheroids $x_z$ iso-surfaces are proportional to the distance the pore entrance $(z-L/2)^2$ coordinate.
% Second, the distance between two $x_z$ iso-surfaces are constant, so the scale factor $h_z = 1$.
% Both assumptions are precise relatively far from the entrance, as a focal circle degenerates to a focal point, compared to the oblate spheroid size.



% 
{\bf Mikhail to add a brief summary}
%


%%%%%%%%%%
\section{An attractive polymer filling dramatically enhances colloid fluxes through a mesopore}
%%%%%%%%%%

\begin{figure}
    \centering
    %\includegraphics[scale = 0.7]{fig/R.png}
    \includegraphics[scale = 0.7]{fig/resistivity_on_z_and_hm.png}
    \caption{
    (A) Resistivity profiles along the pore axis for selected polymer-colloid interaction strengths $\chi_{\text{PC}}$, as indicated with coloured lines. 
    The resistivity of a plain pore without polymers (black line) and the location of the membrane (green background) are shown for comparison. 
    Conditions: pore and polymer brush as in Figure \ref{fig:phi_hm_grid}, $\chi_{\text{PS}}=0.5$, $d=8$. \todo{It is not resistivity along z axis stricly speaking}
    (C) Maps of normalized resistivity $\rho D_0$ exemplifying a transition between hindered and enhanced permeability upon a subtle increase in the polymer-colloid interaction strength.
    $\chi_{\text{PC}}$ was varied from $-1.0$ (left) to $-1.1$ (right); resistivities are colour coded as indicated. 
    Conditions: pore and polymer brush as in Figure \ref{fig:phi_hm_grid}, $\chi_{\text{PS}}=\dots$, $d=\dots$.
    }
    \label{fig:R_map}
\end{figure}

Figure \ref{fig:R_map}A provides example resistivity profiles along the pore axis for a few selected polymer-colloid interaction strengths $\chi_{\text{PC}}$, and for an empty pore. 
The area under the curves represents the total resistance of the pore.
Several features are notable. 

Firstly, an attractive pore interior can enhance the permeability such that diffusive fluxes through the polymer filled pore exceed the limit of the empty pore.
This can be clearly appreciated for $\chi_{\text{PC}}=-1.2$ and $-1.3$, 
where the reistivities within the membrane width  ($-28\le z\le 28$) are consistently inferior to the resistivities for the empty pore.
This result may at first appear surprising, given that the polymer medium is expected to slow down the diffusion of colloids.
However, this effect is counteracted by the attractive potential of the pore which facilitates a higher colloid flux and thus reduces the pore resistance.

Secondly, attractive caps can further enhance permeability.
This can again be clearly appreciated for $\chi_{\text{PC}}=-1.2$ and $-1.3$ (and to a lesser extent also for $\chi_{\text{PC}}=-1.1$), 
where the resistivities outside the membrane ($-50 \lesssim z<-28$ and $28<z \lesssim 50$) are consistently inferior to the resistivities for the empty pore. 
Clearly, this effect stems form the attractive potential of the caps, which in this case reach to $z \approx \pm 50$ and again facilitate higher colloid flux. 

As the colloid attraction increases further (not shown here), the resistivities across the $z$ range spanning the pore interior and the caps become negligible, 
such that the convergent flow outside the polymer-filled volume dominates the resistance against colloid transport.
Without caps, the colloid flux may thus increase by up to approximately 3-fold. 
Attractive caps can increase the flux substantially further.
When reaching $z=\pm 50$ an extra 5-fold enhancement is possible, for example, and even larger caps would facilitate further enhancement.

In conclusion, polymer brushes can readily enhance permeability of the pore by one (and more) orders of magnitude.


%%%%%%%%%%
\section{Polymer filled mesopores sharply select colloids by their attraction to the polymer}
%%%%%%%%%%

Figure \ref{fig:R_map}A also demonstrates that a rather modest change in $\chi_{\text{PC}}$ (from $-1.0$ to $-1.1$) entails a substantial (approximately twofold) change in resistance. 
In this specific case, the pore interior controls the modulation of brush permeability. 
This is apparent with more detail in \ref{fig:R_map}B where the central channel has a resistivity higher than an empty pore (red colour) for $\chi_{\text{PC}}=-1.0$, 
but lower (blue colour) for $\chi_{\text{PC}}=-1.1$. 

\textbf{Need to expand on this aspect.}


%%%%%%%%%%
\section{Polymer filled mesopores can effectively gate colloids by their size}
%%%%%%%%%%

\begin{figure}
    \centering
    %\includegraphics[scale = 0.5]{fig/R_vs_d.png}
    \includegraphics[scale = 0.5]{fig/permeability_on_d.png}
    \caption{
    Resistivity as a function of colloid size for selected polymer-colloid interaction strengths ($\chi_{\text{PC}}$, as indicated with coloured lines) 
    and solvent qualities ($\chi_{\text{PS}}$, as indicate above each panel). 
    The resistivity of a plain pore without polymers (black thick line) is shown for comparison. 
    Conditions: pore and polymer brush as in Figure \ref{fig:phi_hm_grid}. 
    }
    \label{fig:R_vs_d}
\end{figure}

Figure \ref{fig:R_vs_d} illustrates how the total resistivity varies with colloid size for a range of polymer-colloid interaction strengths and solvent qualities. 
Two distinct trends are generally observed.

At sufficiently low polymer-colloid interaction strengths, the polymer filled pore tends to be more resistant to colloid transport than the empty pore across  all colloid sizes. 
In this regime, the resistance increases gradually yet substantially with colloid size, owing to a combination of enhanced osmotic repulsion and reduced diffusivity.

As the polymer-colloid interaction strength increases ($\chi_{\text{PC}}$ decreases), permeability is enhanced compared to an empty pore for small colloids. 
Interestingly, the resistance remains approximately constant over a range of colloid size, until a critical colloid size emerges above which the resistance becomes very high, 
effectively impeding permeation. 
In this regime, the polymer filled pore thus acts as a gate that sharply selects colloids below from colloids above a certain threshold size.

It can be seen that with decreasing solvent strength, the level of attraction required for sharp size selectivity decreases. 
Moreover, the threshold for impeded permeation (at any given $\chi_{\text{PC}}$) is pushed towards larger sizes.


%%%%%%%%%%
\section{DISCUSSION}
%%%%%%%%%%

% RR: I have not revised this section in detail.
% RR: May here add a paragraph that highlights the main finding.

%%%%%%%%%%%%%%%%%%%%%%%%%%%%%%%%%%%%%%%
%Validity of the approximations
%%%%%%%%%%%%%%%%%%%%%%%%%%%%%%%%%%%%%%%
To validate our approximate analytical solution, we have also performed full numerical solution of the Smoluchowsky diffusion equation, which is illustrated by Figure X 
(details in SI) and corresponding point are presented, together with analytical results in Figure 7. A very good quantitative agreement proves accuracy of our analytical theory
within the limit of the particle size sufficiently smaller than the pore radius. This is applicable to NPCs where facilitated transport starts from 5 nm while the
NP diameter is about 40 nm.

%%%%%%%%%%%%%%%%%%%%%%%%%%%%%%%%%%%%%%%
%Many pores problem
%%%%%%%%%%%%%%%%%%%%%%%%%%%%%%%%%%%%%%%
The question of how several pores in the same membrane interfere affecting their permeability was first posed by Rayleigh himself \cite{Strutt1878}. 
Fabrikant  proposed a quantitative theory for a negligibly thin membrane with several circular apertures of different radii and arbitrary mutual positions \cite{Fabrikant1985}. 
The resultant effect of the pore interference is an increase in the pore permeability since the Rayleigh resistance is partially shared by the neighboring pores. 
However, the effect is quite small (a few percent) whenever the distance between the pore centers is larger than their diameters by an order of magnitude or more. 
It is intuitively clear that once the resistance due to a finite pore length and due to the brush is non-negligible, the mutual interference effect 
becomes even smaller. Hence, we are not concerned with this aspect of the problem.
%RR: May here elaborate that the NPC density in the nuclear envelope is relatively low compared to the limits presented here. 
%Hence, selective permeation across NPCs should be well represented by our theory. 
%%%%%%%%%%%%%%%%%%%%%%%%%%%%%%%%%%%%%%%

% RR: May here discuss relevance of our findings to the NPC.

% RR: Should we discuss the question of high colloid concentration? This definitely is relevant to the NPC.

%%%%%%%%%%
\section{Conclusion}
%%%%%%%%%%

% RR: I have not revised this section in detail.

We have explored physical mechanisms for facilitated and selective diffusive transport of colloids through mesopores filled by polymer brushes grafted onto the inner pore wall. 
To this end, we have applied approximate analytical and numerical solutions of the Smoluchowsky diffusion equation 
in the external potential field experienced by diffusing colloids interacting with a brush that fills the interior of the pore
and protrudes outside the pore forming caps. In addition, enhanced resistance of the semidilute polymer solution, to which the brush-filled pore
interior can be locally assimilated, to the diffusing species was taken into account. 

Depending on the strength of the polymer-colloid interaction and/or solvent quality, diffusion of particles of different sizes
through the pore can be either blocked by the resistance of the brush or enhanced compared to the diffusion through the plain (polymer-free) pore. 

The maximal size of the particles for which diffusion through the pore is enhanced (gating threshold) increases with polymer-colloid attractiveness and decreasing solvent strength.

Moreover, in the case of polymer fringes protruding outside the pore and attractive polymer-particle interaction, the total resistance composed of that of the pore itself 
and convergent/divergent flow regions can be substantially lower than that of the plain (polymer-free) pore due to reduced resistance of the entrance/exit regions proximal to the pore.

Altogether, our findings shed the light onto possible mechanisms of selective transport through NPC and, at the same time, suggest a molecular design strategy for controlling selective
permeability through artificial mesoporous membranes with the eye on applications in...


\printbibliography
\end{document}

%%%%%%%%%%%%%%%%%%%%%%%%%%%%%%%%%%%%%%%%%%%%%%%%%%%%%%%%%%%%%%%%%%%%%%%%%%%%%%%%%%%%%%%%%%%%%%%%%%%%%%%%%%
%RR: In the following, I have left some pieces of text that could be inserted into the main text where desired
%%%%%%%%%%%%%%%%%%%%%%%%%%%%%%%%%%%%%%%%%%%%%%%%%%%%%%%%%%%%%%%%%%%%%%%%%%%%%%%%%%%%%%%%%%%%%%%%%%%%%%%%%%

%Under good (or \theta-) solvent conditions we may consider separately the situations with positive and negative insertion free energies. 
%Negative insertion free energies are rather exceptional under good solvent conditions. As we see below, in this case $R_{caps}\leq R_{convergent}$ and the total resistance
%is lower than that of the empty pore.
%Positive insertion free energies under good solvent conditions are more common. 
%In this case, the resistance of the pore interior is always dominant, 
%$$
%R_{tot}\approx R_{pore}
%$$
%and the accuracy in estimating the resistance contributions from the entrance/exit regions is not of a major concern. 

%In Figure \ref{fig:fe_scf_grid} the insertion free energy profiles $\Delta F(z,r=0)$ calculated by analytical scheme and by SF-SCF method 
%are presented as a function of position of a spherical particle along the pore axis.
%While the SF-SCF method provides the net free energy, the analytical scheme allows decoupling of the free energy into osmotic and surface contributions, 
%which are shown separately in Figure \ref{fig:fe_scf_grid}.
%The numerical coefficients $b_0$ and $b_1$ in eq \ref{} are chosen by the best fit, but appear to be fairly universal and independent of the particle size 
%and interaction parameters $\chi_{PS,PC}$.
%Remarkably, the fit fails when the size $d$ became comparable with the pore diameter or in the case of extreme $\chi_{ads}$ values 
%when analytical scheme is not applicable because of strong perturbation 
%of the brush structure by inserted particle, while SF-SCF method can still be safely used
%for the evaluation of the insertion free energy.

%The 2D insertion free energy $\Delta F(z,r)$ patterns have rather complex shape. However, we can trace their evolution upon changing interaction parameters
%looking at the position-dependent free energy of the particle on the pore axis, $\Delta F(z,r=0)$.
%As one can see from Figure \ref{fig:fe_scf_grid}, the insertion free energy profiles evolve upon changing the interaction parameters $\chi_{PS,PC}$ as follows:
%At $\chi_{ads}\geq \chi_{crit}$ which is the case under good or theta-solvent conditions and weak or absent polymer-particle attraction, $|\chi_{PC}|\leq 1$, the positive osmotic
%term, $\Delta F_{osm}\geq 0$ dominates in the insertion free energy, which is positive and reach maximal value in the pore center, where polymer concentration is maximal.
%Hence, polymer-particle interaction has overall repulsive character and $\Delta F(z,r)$ has the shape of the free energy barrier preventing penetration and accumulation of particles in the pore.
%By using the insertion free energy $\Delta F(z,r)$ one can calculate the equilibrium partition coefficient 
%$$
%P=\int_{0}^{r_{pore}}2\pi rdr\int_{0}^{L}dz\exp (-\Delta F(z,r)/k_BT)/\pi r^{2}_{pore}L
%$$
%is larger than unity, $P\geq 1$. Noticably the repulsive free energy profiles extends beyond the edges of the pore, because of the fringes in the polymer density distribution in swollen brush.

%A decrease in $\chi_{ads}$ triggered by a decrease in  $\chi_{PC}$ or/and an increase in $\chi_{PS}$ leads to qualitative changes in the insertion free energy 
%$\Delta F(z,r)$ patterns: At $\chi_{ads}\leq \chi_{crit}$ the particle surface becomes
%adsorbing for the polymer, $\gamma \leq 0$, that gives rise to a negative contribution $\Delta F_{surf}(z,r)$ to the insertion free energy. 
%When $\chi_{PS}$ increases (the solvent is getting worse for the polymer)
%the osmotic pressure inside the brush decreases that leads to a decrease in the 
%magnitude of $\Delta F_{osm}(z,r)$ with the concomitant shrinkage of the  protruding outside the pore parts of the brush where  $\Delta F(z,r)\neq 0$.
%As a result, the $\Delta F_{surf}(z,r)$ aquires two minima with negative values near the endtance and the exit of the pore, separated by a maximum centered in the middle of the pore
%where polymer concentration is larger and the osmotic repulsive term  $\Delta F_{osm}(z,r)$ dominates.
%Finally, at strong polymer-particle attraction $\chi_{ads} < \chi_{crit}$, the negative surface contribution $\Delta F_{surf}(z,r)\leq 0$ overperform osmotic repulsion everywhere inside the pore
%and the $\Delta F(z,r)$ aquires the shape of the potential well centered in the middle of the pore, which gives rise to preferential accumulation of particles in the pore, $P\geq 1$.