\documentclass[12pt, a4paper]{article}
\usepackage{graphicx}
\usepackage{amsmath}
\usepackage[
backend=biber,
natbib=true,
style=numeric,
sorting=none
]{biblatex}
\usepackage{xcolor}

\newcommand\todo[1]{\textcolor{red}{#1}}

\addbibresource{flow.bib}
\title{Physical Principles of Nanocolloids Selective Transport through Polymer Brush Mesoporous Membranes. \\ DRAFT}

\author{M. Laktionov}

\date{August 2023}
\begin{document}
\maketitle

\section{Introduction}
\dots

\section{Theoretical model behind colloid transport}
\subsection{Definition of the system. Polymer brush structure}

\begin{figure}
    \centering
    \includegraphics[scale = 1.0]{fig/pore_cartoon.png}
    \caption{
        Schematic illustration of colloid particle interaction scenario and diffusive transport through a polymer brush pore. 
        The brush (red strands) with a degree of polymerization $N$, grafting density $\sigma$ and polymer segment length $a$ is grafted to the inner surface of a cylindrical pore in an impermeable membrane. The pore has the radius $r_{pore}$ and the thickness $s$.
        The membrane (grey) separates semi-infinite solvent (blue) reservoirs such that any transport occur through the pore. 
        Transport of colloid particles (green hue) is affected by the brush in a complex manner.
        The local mobility of the particles in the solvent $D_{0}$ is defined by their size ($d$), the presence of polymer chains impedes colloid transport by decreasing local mobility.
        When a colloid placed in the brush it changes system free energy, which interprets in local insertion free energy.
        The negative landscape of local insertion free facilitates colloid transport, whilst free energy barrier can halt the transport.
        The local insertion free energy arise from the contact of colloid particle with brush, and depends on Flory interaction parameters $\chi$ (lower inset). To calculate insertion free energy polymer brush, solvent and particle coarse-grained into regular elements (upper inset) having finite volume and contact area.
    }
    \label{fig:colloid_transport}
\end{figure}

\begin{figure}
    \centering
    \includegraphics[scale = 1.0]{fig/phi_hm_grid.png}
    \caption{
    Polymer density profiles in cylindrical coordinates depending on solvent quality. 
    Solvent quality is defined by interaction parameter $\chi_{PS}$ ranging from 0.1 to 1.1, the value of $\chi_{PS}$ is written on the top of each frame.
    Polymer density profiles $\phi(z,r)$ are presented as a colormaps, with a color code is universal for the all frames, where white color corresponds to pure solvent, yellow to magenta is low polymer concentration, blue to black corresponds to high concentration.
    To trace $\phi(z,r)$ values, the colorbar is shown on under the frames.
    The membrane body is drawn with the green color.
    Horizontal axes are longitudinal axis $z$, vertical axes are radial axis $r$.
    For illustrative reasons the colormaps are mirrored along $z$ axis, in a cylindrical coordinate system radial coordinate is always positive.
    To remind about the axial symmetry the axis is drawn as orange dash-dotted line in the last frame.
    }
    \label{fig:phi_hm_grid}
\end{figure}

\begin{figure}
    \centering
    \includegraphics[width = 4in]{fig/fe_scf_grid.png}
    \caption{
    Results of mapping from SF-SCF to analytical insertion free energy method for a cylindrical particle with diameter and height $d=8$ moving along the main axis of the pore with radius $r_{pore} = 26$ and membrane thickness $s=26$.
    Solvent quality is varied near $\theta$-point with $\chi_{PS} = [0.4, 0.5, 0.6]$, ordered from left to right column, respectively.
    Colloid particle affinity ranges from attractive to inert particle with $\chi_{PC} = [-1.0, -0.5, 0.0]$, ordered from first to last row, respectively.
    \\
    Horizontal axes corresponds to position of the particle's center $z_c$, vertical axes corresponds to free energy value.
    \\
    Insertion free energy calculated with SF-SCF scheme $\Delta F_{SF-SCF}$ is drawn with red squares, the results of analytical scheme are presented with solid red line for the total insertion free energy $\Delta F_{tot}$ and dashed green and blue line for the osmotic and surface term, respectively.
    The gray area marks values of $z$ that corresponds to the volume inside the pore $s\in [-13, 13]$.
    \label{fig:fe_scf_grid}
    }
\end{figure}


\begin{figure}
    \centering
    \includegraphics[scale = 1.0]{fig/streamlines.png}
    \caption{
    Flux streamlines
    }
    \label{fig:streamlines}
\end{figure}

\begin{figure}
    \centering
    \includegraphics[width = \textwidth]{fig/permeability_on_d.png}
    \caption{
        Permeability coefficient $P$ for a spherical colloid particle as a function of particle size for different solvent quality and particle affinity.
        \\
        Solvent quality is defined with polymer-solvent interaction parameter $\chi_{PS}$ ranging from good to moderately poor solvent, four frames from left to right corresponds to a set of $\chi_{PS} = {0.3, 0.4, 0.5, 0.6}$.
        \\
        Colored solid lines corresponds to different values of polymer-colloid interaction parameter $\chi_{PC} = {-1.5, -1.25, -1.00, 0.00}$ from attractive to inert particle. 
        The color code explained in the legend.
        \\
        Dashed black line traces permeability of an empty pore in a membrane of a finite thickness $s$.
        Dotted black line traces permeability of an empty pore in an infinitely thin membrane.
        Both results are calculated using analytical solution of diffusion through a pore problem.
        \\
        Selected cases were calculated numerically using CFD approach, the results shown as circle markers that shares the same color as solid lines.
        \\
        \todo{remove redundant lines}
        }
        \label{fig:permeability_on_d}
\end{figure}


\end{document}