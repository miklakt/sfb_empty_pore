\documentclass[12pt, a4paper]{article}
\usepackage{graphicx}
\usepackage{amsmath}
\usepackage[
backend=biber,
natbib=true,
style=numeric,
sorting=none
]{biblatex}
\usepackage{xcolor}

\newcommand\todo[1]{\textcolor{red}{#1}}

\addbibresource{biblio.bib}
\title{Physical principles of the colloid selective permeation through polymer-filled mesopores.}

\author{Mikhail Y. Laktionov$^1$, Leonid I.Klushin$^{2}$,\\Ralf P.Richter$^3$, France A.M. Leermakers$^4$, Oleg V.Borisov$^1$\\
$^{1}$CNRS, Universit\'e de Pau et des Pays de l'Adour UMR 5254,\\
Institut des Sciences Analytiques et de Physico-Chimie\\
pour l'Environnement et les Mat\'eriaux, 64053 Pau, France \\
$^{2}$Institute of Macromolecular Compounds \\
of the Russian Academy of Sciences, \\
199004 St.Petersburg, Russia,\\
$^{3}$University of Leeds, School of Biomedical Sciences, \\
Faculty of Biological Sciences, 
School of Physics and Astronomy, \\
Faculty of Engineering and Physical Sciences,\\  
Astbury Centre for Structural Molecular Biology,\\ 
and Bragg Center for Materials Research,\\ 
Leeds, LS2 9JT, United Kingdom\\
$^{4}$ University of Wageningen, the Netherlands
}

\begin{document}
\maketitle

\begin{abstract}
Physical mechanisms of selective facilitated permeation of nanocolloidal particles 
through polymer-grafted mesopores are unravelled on the basis of self-consistent field theoretical modelling.
We predict that diffusive transport of particle can be accelerated compared to that through a bare pore due to
cohesive polymer-particle interactions, while penetration of inert with respect to the polymer particles of even smaller size can be 
efficiently impeded. We formulate thermodynamic criteria for unrestricted gating threshold through the pore and anticipate, that underlying
physical mechanisms may apply for faciliated permeation of biologically active molecules in complex with NTR thorugh NPC.   
\end{abstract}

%%%%%%%%%%%%%%%%%%%%%%%%%
\section{Introduction}
%%%%%%%%%%%%%%%%%%%%%%%%%

Polymer-modified mesoporous materials and membranes belong to a new class of functional nanostructured materials with great potential in a number of key technologies. 
The interaction and absorption of (macro)molecules and nanocolloidal particles by porous media, as well as their transport through macro- and mesoporous membranes, 
are important elements of many technological processes (chromatography, heterogeneous catalysis, micro- and ultrafiltration, protein separation and purification etc.) 
and, therefore, have been the subject of intensive research for decades. 

Advances in macromolecular chemistry have made it possible to significantly improve functional properties of mesoporous (with a pore diameter within 100 nm) 
materials by modifying them with covalently (or strongly non-covalently) bound to the pore walls macromolecules of various chemical nature,  
forming a “soft” , solvated physical polymer meshwork that fills the pore volume or only the near-wall regions, 
depending on molecular mass and conformational state of the polymer chains. 
The interaction of this polymer meshwork with guest molecules/nanoparticles 
%and, in particular, the presence or absence of a hollow (polymer-free) channel in the center of the pore, 
essentially determine the absorption and separation properties of the polymer-modified mesoporous materials and membranes, respectively. 
These interactions can be attractive or repulsive, short- or long-range (in the presence of charges on the chains and on guest molecules/particles), 
and most importantly, they can be controlled by a complex of external stimuli, such as temperature, pH and/or ionic strength of the medium, valence of ions , 
solvent composition, etc. This opens up a unique opportunity for highly selective and controlled uptake and transport of guest molecules/nanoparticles 
through polymer-filled mesoscopic channels. 
%For example, mesopores modified with ionic polymer molecules can potentially be used to separate molecules/nanocolloids that are almost identical in size and shape, 
%but differ in a small number of charged groups. 

Nature uses the principle of controlling the selective transport of biological molecules 
between the nucleus and cytoplasm of eukaryotic cells through the so-called nucleopores (cylindrical channels of about 40 nm in diameter), 
which perforate the nuclear envelope and are filled with a swollen meshwork consisting of natively denatured proteins attached to the pore walls. 
A similar structural motif was recently found in the internal channels of microtubules (about 15 nm in diameter) 
decorated with so-called microtubule intrinsic proteins (MEPs), presumably modifying microtubule stability and rigidity.
A popular nowadays paradigm suggests that the accuracy and efficiency of many processes in nature are ensured 
not so much by specific (molecular recognition) interactions, but due to a fine balance of fundamental (electrostatic, hydrophobic...) 
interactions between biomacromolecules and (bio) nanocolloids.  

However, up to date, 
%the theoretical knowledge and systematic 
understanding of the relationship between molecular architecture of the brush 
decorating the pore walls and the spatial structure, 
cohesive and rheological properties of the resulting "soft" meshwork and its ability to selectively absorb in the 
volume of pores or modulate the diffusion transport of nanocolloidal particles through the pores is lacking. 

In this paper we present results of analytical theory and numerical modelling of 
%a self-consistent field theory describing equilibrium absorption and 
diffusive transport of nanocolloidal particles
thorugh mesoscopic channel in a planar membrane filled with a polymer brush grafted onto the innder channel walls. We investigate how an interplay of the particle size,
polymer-particle interaction and thermodynami solvent quality for the brush-forming chains affect the diffusion rate of the particles through the pore. 
The main focus of the research is to predict how competition of repulsive (osmotic) force with polymer cohesion to the particle surface in combination with
enhansed viscosity of the polymer medium do control the  pore permeability threshold as a function of the particle size.


The rest of the paper is organized as follows: 



%%%%%%%%%%%%%%%%%%%%%%%%%%%%%%%%%%%%%%%%%%%%%%%%%%%%%%%%%%%%%%%%%%%%%%%%%%%%%%%%%%%%%%%
%\section{Colloid interaction with a polymer brush in a cylindrical pore}
%%%%%%%%%%%%%%%%%%%%%%%%%%%%%%%%%%%%%%%%%%%%%%%%%%%%%%%%%%%%%%%%%%%%%%%%%%%%%%%%%%%%%%

%%%%%%%%%%%%%%%%%%%%%%%%%%%%%%%%%%%%%
\section{Diffusive transport of colloidal particles through polymer-filled pore}
%%%%%%%%%%%%%%%%%%%%%%%%%%%%%%%%%%%%% 


\begin{figure}
    \centering
    \includegraphics[scale = 1.0]{fig/pore_cartoon.png}
    \caption{
        Schematic illustration of the colloid particle diffusive transport through a pore filled with polymer brush. 
        The brush is formed by linear polymer chains (red strands) with a degree of polymerization $N$, uniformly grafted with grafting density $\sigma$  
        to the inner surface of a cylindrical pore in an impermeable membrane. The pore radius is $r_{pore}$ and the thickness of the membrane is $s$.
        Polymer chains are flexible with a statistical segment length $a$ and volume $\sim a^3$.
        %The membrane (grey) separates semi-infinite solvent (blue) reservoirs such that any transport occur through the pore. 
        %Transport of colloid particles (green hue) is affected by the brush in a complex manner.
        %The local mobility of the particles in the solvent $D_{0}$ is defined by their size ($d$), the presence of polymer chains impedes colloid transport by decreasing local mobility.
        %When a colloid placed in the brush it changes system free energy, which interprets in local insertion free energy.
        %The negative landscape of local insertion free facilitates colloid transport, whilst free energy barrier can halt the transport.
        %The local insertion free energy arise from the contact of colloid particle with brush, and depends on Flory interaction parameters $\chi$ (lower inset). 
        %To calculate insertion free energy polymer brush, solvent and particle coarse-grained into regular elements (upper inset) having finite volume and contact area.
}
    \label{fig:colloid_transport}
\end{figure}

The goal of this work is to understand mechanism of the transport of colloidal particles though a cylindrical pore in a membrane, 
the pore being decorated by a polymer brush grafted to its inner surface. 
We consider a nanoscale cylindrical pore perforating planar membrane and filled by a polymer brush which is formed  
by flexible polymer chains end-grafted to the inner surface of the pore, Figure \ref{fig:colloid_transport}. 
We aim to understand how the diffusion of solute particles between separated by the porous
membrane semi-infinite solution reservoirs is modulated by interaction of the particles with the 
polymer brush filling the pore and, in the highly swollen state, protruding beyond the pore edgies.

We anticipate, and prove this below, that transport of colloid particles through the pore is affected by the brush in a complex manner: First, local mobility of the particles in a polymer medium
(semidilute solution to which the brush can be assimilated) is slowing down  as compared to that in pure solvent. Second, insertion of the particle into the brush at certain position
(characterized by longitudinal $z$ and radial $r$ coordinates of the particle center) leads to change in the system free energy that can be interpreted as position-dependent differential insertion free energy $\Delta F(z,r)$. 
The latter can be either positive or negative depending on the 2D polymer density distribution $\phi(z,r)$, particle volume, 
thermodynamic solvent quality for polymers and short-range interactions (repulsive or attractive) between polymer segments and particle surface. 
As demonstrated below, the negative landscape of local insertion free facilitates colloid transport, whilst positive free energy barrier can impede the transport.

%%%%%%%%%%%%%%%%%%%%%%%%%%%%%%%%%%%%%%%%%%%%%%%%%%%%%%%%%%%%%%%%%
%****************************************************************
%%%%%%%%%%%%%%%%%%%%%%%%%%%%%%%%%%%%%%%%%%%%%%%%%%%%%%%%%%%%%%%%
For understanding the effect of the polymer brush on transport of colloidal particles though a cylindrical pore,
%in a membrane, the pore being decorated by a polymer brush grafted to its inner surface. For that purpose, 
we consider the stationary diffusive flux of colloidal particles through the pore and analyze how it is affected by the parameters of the pore, the brush, and the colloid. 

A natural starting point is the diffusive flux through an empty pore without any brush. The earliest approach to that problem goes back to Lord Rayleigh 
who analyzed a potential flow though a circular aperture (pore) in a planar membrane of negligible thickness while recognizing and exploiting its electrostatic 
and gravitational analogies\cite{Strutt1878}.  
Let the position of the membrane coincides with the XY plane at  $z=0$, and the pore is a circle of radius $r_{pore}$. 
The concentration of the diffusing species is fixed to be zero and $c$ far away from the membrane (at   $z\rightarrow\mp\infty$, respectively).  
The equiconcentration (equipotential) surfaces are oblate spheroids and the streamlines form confocal hyperboloids of revolution\cite{Cooke1966}.
The net flux through the pore is given by
\begin{equation}
J=2D_0r_{pore}c\label{eq:flux_Ral}
\end{equation}

\noindent where $D_0$ is the diffusion coefficient of the diffusing species in pure solvent. 
%The fact that the flux is proportional to the linear size of the pore rather than its area was a subject of some historical discussion \cite{Cooke1966}.
Diffusion through a cylindrical pore in a membrane of finite thickness $L$ also allows an analytical solution but in this case it involves an implicit infinite series \cite{Brunn1984}. 
The lowest order approximation turns out to be quite accurate (with an error of less than 6 percent in the full range of the $\frac{L}{r_{pore}}$ ratio) and reads:

\begin{equation}
    J=\frac{2D_0r_{pore}c}{1+\frac{2L}{\pi r_{pore}}}\label{eq:flux_finlength}
\end{equation}

At this point it is convenient to introduce the pore resistance $R$ to the flow as
\begin{equation}
\label{R}
J=\frac{c}{R}
\end{equation}

Then Eq (\ref{eq:flux_finlength}) admits a most natural interpretation in terms of the total resistance of the pore to the flow, as

\begin{equation}
R=\frac{L}{D_0\pi r_{pore}^{2}}+\frac{1}{2D_0r_{pore}}\label{eq:resistance}
\end{equation}

The first term in eq \ref{eq:resistance} is the resistance of the cylindrical pore itself (the resistivity of the medium being $D^{-1}$) while the second term is the 
Rayleigh resistance of the pore of infinitesimal thickness as deduced from Eq  (\ref{eq:flux_Ral}) . 
The latter represents the effects of the convergent flow at the entrance of the pore and its symmetric counterpart on the exit side of the membrane, 
while the flow lines inside the cylindrical pore turn out to be approximately axial. 
The relatively small error carried by the approximate solution  (\ref{eq:resistance})  is due to deviations from flow axiality inside the pore 
and to the corresponding minor modification of the convergent flow at the entrance/exit as compared to the case of a membrane of negligible thickness.
To account for the excluded volume of the diffusing particles, assuming them being spheres of diameter $d$, one should replace $r_{pore}\rightarrow r_{pore}-d/2$
in eqs \ref{eq:flux_finlength} and \ref{eq:resistance}

%Altogether the resistance of the setup with a membrane of finite thickness and the boundary conditions imposed at $z\rightarrow\mp\infty$ coincides with 
%that of an equivalent cylinder of the same radius $a$ and of total length $L_{eq}=L+l_{R}$   where the additional length,  $l_{R}=\frac{\pi}{2}a$ , 
%accounts for the Rayleigh resistance contribution. The boundary conditions of fixed concentration are now imposed at the caps of the equivalent cylinder, 
%see the cartoon illustration in Figure \ref{fig:flow_cartoon}. In what follows, we will refer to the additional cylindrical sections outside the membrane, 
%each of length $\frac{l_{R}}{2}=\frac{\pi}{4}a$ , as the Rayleigh cylinders.
 
 
%\begin{figure}
    %\centering
    %\includegraphics[width=0.9\linewidth]{flowcartoon.pdf}
    %\caption{ (a) Cartoon representing the flow lines for a pore in a thick membrane with the boundary conditions imposed far way from the membrane (at $\pm\infty$). (b) An equivalent cylinder with the boundary conditions imposed at the caps leading to strictly axial flow lines; additional cylindrical sections (transparent) represent the Rayleigh resistance and have the length of  $\frac{l_{R}}{2}=\frac{\pi}{4}a$  each, where $a$ is the radius of the pore. The equivalent cylinder accurately approximates the total flux in the situation depicted in panel (a).}
    %\label{fig:flow_cartoon}
%\end{figure}


Filling the pore with polymer brush gives rise to an  effective  potential experienced by the diffusing particles. 
Then diffusion of colloidal particles through the pore filled by the brush is described by the Smoluchowsky equation which represents 
a high friction limit of the Fokker-Planck equation \cite{Risken1996}
\begin{equation}
    \frac{\partial c(\textbf{r},t)}{\partial t}=\nabla\cdotp D(\textbf{r})\left(\nabla c(\textbf{r},t)+c(\textbf{r},t)\nabla\Delta F(\textbf{r})\right)
    \label{eq:smoluchowsky}
\end{equation}
where $c(\textbf{r},t)$ is the concentration of colloidal particles.
%The notion of the equivalent cylinder is very helpful for estimating the effects of the brush in the interior of the pore on the diffusive flux. 
Interaction of the brush with the diffusing particles is described via the insertion free energy profile, $\Delta F(\textbf{r})$ which is in turn linked to the profiles 
of the polymer concentration in the brush $\phi(\textbf{r})$  and of the osmotic pressure $\Pi\{\phi(\textbf{r})\}$ \cite{Laktionov2023}. 
On top of that, we introduce the position-dependent diffusion coefficient $D\{\phi(\textbf{r})\}$
which depends on the local polymer concentration and accounts for slower diffusion through a semidilute polymer meshwork \cite{Laktionov2023}.
In the following, we apply scaling relation
\begin{equation}
    D\{\phi(\textbf{r})\} = \frac{D_{0}}{1+d^2/\xi^{2}\{\phi(\textbf{r})\}}
\label{Rubinstein}
\end{equation}
where $\xi\{\phi(\textbf{r})\}$ is correlation length (the "mesh size") controlled by local polymer concentration $\phi(\textbf{r})$. 
As follows from eq \ref{Rubinstein}, for particles smaller than the mesh size $d\leq \xi$, the diffusion coefficient $D\{\phi(\textbf{r})\}\approx D_0$ is the same as in pure solvent,
that is, polymer solution does not hinder the particle dissusion. However, for larger particles of size, $d\gg \xi$, particle diffusion in polymer medium is significantly slowed down,
$D\cong D_0 (\xi/d)^2\ll D_0$ as compared to the diffusion in pure solvent. 
Below we adopt the scaling dependence of the correlation length on polymer concentration valid
close to theta-solvent conditions, $\xi\cong \phi^{-1}$, which quantitatively captures increasing resistance to the particle diffusion upon increasing local polymer concentration
leading to a decrease in the mesh size. 
%Diffusion of colloidal particles in the presence of an effective potential is described by the Smoluchowsky equation which represents 
%a high friction limit of the Fokker-Planck equation \cite{Risken1996}:
%Here $c$ is the concentration of the colloidal particles, $D$ is the local (position-dependent) diffusion coefficient, 
%and $\Delta F$ is the position-dependent free energy of insertion which plays the role of the potential of mean force.

We assume the axial (cylindrical) symmetry of the flow in the pore. Together with the stationary conditions, this implies that all the relevant functions,i.e. $c(z,r)$, $\Delta F(z,r)$, and $D(z,r)$ 
depend on the axial coordinate $z$ and the radial coordinate $r$ but not on the azimuthal angle.
The stationary flux density has two components linked to the corresponding components of the gradients of the particle concentration and the insertion free energy:

\begin{equation}
j_{z}(z,r)=-D(z,r)\left(\frac{\partial c(z,r)}{\partial z}+c(z,r)\frac{\partial\Delta F(z,r)}{\partial z}\right)\label{eq:flux_axial}
\end{equation}

\begin{equation}
j_{r}(z,r)=-D(z,r)\left(\frac{\partial c(z,r)}{\partial r}+c(z,r)\frac{\partial\Delta F(z,r)}{\partial r}\right),
\label{eq:flux_radial}
\end{equation}
\noindent where $c(z,r)$ is the stationary colloid concentration.

A general analytical solution of the stationary diffusion equation, eq \ref{eq:smoluchowsky} is not available. 
%to our best knowledge. 
Therefore, here we present an approximate solution which amounts to neglecting the radial component of the flux density within the pore. 
which is inspired by the fact that the net transport across the membrane is associated only with the axial component of the flux density. 
%and by the notion of the equivalent cylinder with the boundary conditions imposed at its caps as introduced above. 
%We seek the solution for the stationary colloid concentration in the form of a modified Boltzmann distribution, similar to the planar case explored earlier \cite{Laktionov2023}:                                                         

\begin{equation}
c(z,r)=\psi(z)e^{-\Delta F(z,r)}\label{eq:stationary_c_ansatz}
\end{equation}

For the axial flux density, we obtain:

\begin{equation}
j_{z}(z,r)=-D(z,r)\frac{d\psi(z)}{dz}e^{-\Delta F(z,r)}\label{eq:flux_ansatz}
\end{equation}

\noindent 
%Here the prime in  $\psi'(z)$ stands for the derivative with respect to $z$. 
Stationarity implies that the net flux over any cross-section of the pore is the same, independent of the position  $z$ :

\begin{equation}
J=\int_{0}^{r_{pore}}2\pi rdrj_{z}(z,r)=\frac{d\psi(z)}{dz}\int_{0}^{r_{pore}}2\pi rdrD(z,r)e^{-\Delta F(z,r)}=const,\label{eq:fi_const}
\end{equation}

\noindent where $r_{pore}$ is the radius of the pore as introduced above. Solving for $\psi(z)$  we obtain 

\begin{equation}
\psi(z)=C-J\int_{0}^{z}\left(\int_{0}^{r_{pore}}2\pi rdrD(z',r)e^{-\Delta F(z',r)}\right)^{-1}dz'\label{eq:psi}
\end{equation}

%\noindent Here the origin of the z-axis,  $z=0$, is placed at one of the caps of the equivalent cylinder  of total length $L_{eq}= L+\frac{\pi}{2}a$  . 

\noindent Here the origin of the z-axis,  $z=0$, is placed at one edge of the pore  of total length $L$ (the membrane thickness). 
Assuming the insertion free energy outside the pore is zero, $C$  can be recognized as the colloid concentration at the boundary with $z=0$ 
which plays the role of the source.   We impose the zero boundary condition at the opposite edge of the pore 
(the sink) and obtain for the resistance of the pore, $R_{pore}=C/J$ : 

%\begin{equation}
%R_{pore}=\int_{0}^{L_{eq}}\left(\int_{0}^{a}2\pi rdrD(z',r)e^{-\Delta F(z',r)}\right)^{-1}dz'\label{eq:res_with_brush}
%\end{equation}

\begin{equation}
R_{pore}=\int_{0}^{L}\left(\int_{0}^{r_{pore}}2\pi rdrD(z',r)e^{-\Delta F(z',r)}\right)^{-1}dz'\label{eq:res_with_brush}
\end{equation}


The product $D(z',r)e^{-\Delta F(z',r)}$ has the meaning of local conductivity. 
Then integration over the pore cross-section gives the inverse resistance per unit length 
(as appropriate for resistors connected in parallel) and the integration over the axial coordinate simply adds contributions 
from all the slices connected in series. 
This simple interpretation is of course consistent with neglecting the radial component of the flux density. 
Naturally, if the brush is absent and the insertion free energy vanishes everywhere, $\Delta F(z',r)\equiv 0$, and $D(z,r)\equiv D_0$
then Eq  (\ref{eq:res_with_brush}) reduces to the first term Eq (\ref{eq:resistance}) which describes resistance of the empty pore. 

Therefore, if the brush is entirely contained in the interior of the pore, the overall resistance can be presented as

\begin{equation}
R_{tot}=R_{convergent}+ R_{pore}=\frac{1}{2D_0r_{pore}}+ \int_{0}^{L}\left(\int_{0}^{r_{pore}}2\pi rdrD(z',r)e^{-\Delta F(z',r)}\right)^{-1}dz'
\label{R_tot_1}
\end{equation}
where the first term represents contribution of convergent/divergent flows at the entrance/exit from the pore and the second term
account for the resistance for the pore itself. At $\Delta F(z',r)\equiv 0$ (that is, in the absence of a brush in the pore) eq \ref{R_tot_1} is reduced to eq \ref{eq:resistance}.

As we demonstrate below, the assumption that the brush is entirely contained in the interior of the pore is well justified under poor solvent conditions. 
Contrary to that, under good or $\theta$-solvent conditions the brush would swell producing a fringe that protrudes outside the pore, 
see Figure %\ref{fig:}
. In this case, the flow lines at the entrance to and exit from the pore are modified and the Rayleygh resistance specified by eqs \ref{eq:flux_Ral},\ref{R} 
does not fairly represent the corresponding contribution. 

In order to calculate contribution to the overall resistance from the brush fringes protruding outside the pore, we implement an approximate numerical integration scheme.....

%%%%%%%%%%%%%%%%%%%%%%%%%%%%% 
{\bf HERE INSERT BY MIKHAIL}
%%%%%%%%%%%%%%%%%%%%%%%%%%%%%

Finally, the overall resistance is given by

\begin{equation}
R_{tot}=R_{caps}+R_{pore}
\label{R_tot_tot}
\end{equation}
and $R_{caps}\rightarrow R_{convergent}$ in the limit of $\Delta F(r,z)|_{|z|\geq 0}=0$. 



Under good  (or theta-) solvent conditions we may consider separately the situations with positive and negative insertion free energies. 
Negative insertion free energies are rather exceptional under good solvent conditions. As we see below, in this case $R_{caps}\leq R_{convergent}$ and the total resistance
is lower than that of the empty pore.
%We propose that in this case  the resistance of the entrance/exit regions 
%is bounded between the Rayleigh resistance (without any brush effects) and the resistance of the Rayleigh cylinder filled with the actual brush fringe, 
%and use both these estimates in our calculations.
Positive insertion free energies under good solvent conditions are more common. 
In this case, the resistance of the pore interior is always dominant, 
$$
R_{tot}\approx R_{pore}
$$
and the accuracy in estimating the resistance contributions from the entrance/exit regions is not of a major concern. 



%Hence neglect the tentative changes in the picture of the 
%flow lines and evaluate both the pore interior and the brush fringe contributions by applying  Eq (\ref{eq:res_with_brush}) with the 
%insertion free energy profile defined everywhere within the equivalent cylinder.
%Another computational aspect that must be addressed in the case when the brush fringe extends not just beyond the membrane 
%but beyond the caps of the additional Ryleigh cylinders as well. Then the insertion free energy is non-zero at the source and the 
%sink boundaries, and Eq (\ref{eq:res_with_brush}) must be modified such that the free energy of insertion 
%$\Delta F(z,r)$ is counted from the reference state that represents the colloid free energy averaged over the 
%different radial positions along the boundary cap of the Ryleigh cylinder. Another way out is to shift the position 
%of the boundary cap away from membrane so that the brush fringe does not touch it. Then the reference state of the 
%colloid in a pure solvent is restored. Our results are rather insensitive to the choice of treatment of the fringe problem for the reasons discussed above.


%The question of how several pores in the same membrane interfere affecting their permeability was first posed by Rayleigh himself \cite{Strutt1878}. 
%Fabrikant  proposed a quantitative theory for a negligibly thin membrane with several circular apertures of different radii and arbitrary mutual positions \cite{Fabrikant1985}. 
%The resultant effect of the pore interference is an increase in the pore permeability since the Rayleigh resistance is partially shared by the neighboring pores. 
%However, the effect is quite small (a few percent) whenever the distance between the pore centers is larger than their diameters but an order of magnitude or more. 
%It is intuitively clear that once the resistance due to a finite pore length and due to the brush is non-negligible, the mutual interference effect 
%becomes even smaller. Hence, we are not concerned with this aspect of the problem.




%%%%%%%%%%%%%%%%%%%%%%%%%%%%%%%%%%%%%%%%%%%%%%%%%%%%%%%%%%%%%%%%%%%%
%*******************************************************************
%%%%%%%%%%%%%%%%%%%%%%%%%%%%%%%%%%%%%%%%%%%%%%%%%%%%%%%%%%%%%%%%%%%%
%\begin{figure}
    %\centering
    %\includegraphics[width = 2.5in]{fig/coordinates.png}
    %\caption{
    %Cylindrical coordinate system with the origin in the pore center.
    %The longitudinal axis $z$ is the pore's main axis.
    %The radial axis $r$ is perpendicular to axis $z$.
    %As the system exhibits axial symmetry (orange dash-dotted line) with respect to axis $z$, angular coordinate is redundant and is not defined.}
    %\label{fig:coordinate_system}
%\end{figure}

%%%%%%%%%%%%%%%%%%%%%%%%%%%%%%%%%%%%%%%%%%%%%%%%%%%%%%%%%%%%%%%%%%
%Consider cylindrical coordinate system with the origin in the pore center (Figure \ref{fig:coordinate_system}), as the system exhibit axial symmetry with respect to pore main axis, 
%polymer brush properties uniform along angular coordinate. Thus, all system properties are three dimensional they are represented as two dimensional profiles $f(z,r)$.
%%%%%%%%%%%%%%%%%%%%%%%%%%%%%%%%%%%%%%%%%%%%%%%%%%%%%%%%%%%%%%%%%%


%%%%%%%%%%%%%%%%%%%%%%%%%%%%%%%%%%%%%%%%%%%%%%%%%%%%%%%%%%%%%%%%%%%%%%%%%%%%%%%%%%%%%%%%%%%%%%
\section{Conformational structure of the polymer brush in the pore and insertion free energy}
%%%%%%%%%%%%%%%%%%%%%%%%%%%%%%%%%%%%%%%%%%%%%%%%%%%%%%%%%%%%%%%%%%%%%%%%%%%%%%%%%%%%%%&&&&&&&&



\begin{figure}
    \centering
    \includegraphics[scale = 1.0]{fig/phi_hm_grid.png}
    \caption{
    Polymer volume fraction $\phi(z,r)$ profiles in cylindrical coordinates for a polymer brush grafted inside a cylindrical pore at varied solvent quality calculated using SF-SCF theory. 
    The solvent quality is quantified  by Flory-Huggins parameter $\chi_{PS}$ ranging from 0.1 to 1.1, that is, from good to poor solvent conditions.
    (the value of $\chi_{PS}$ is written on the top of each frame).
    The specific choice of the brush and pore parameters...
    Polymer volume fraction profiles $\phi(z,r)$ are presented as a colormaps, 
    %with a color code is universal for the all frames, 
    where white color corresponds to pure solvent, yellow to magenta is low polymer concentration, blue to black correspond to high concentration.
    %To trace $\phi(z,r)$ values, the colorbar is shown on under the frames.
    %The membrane body is drawn with the green color.
    Horizontal axis corresponds to longitudinal coordinate $z$, vertical axis corresponds to the radial coordinate $r$.
    For illustrative reasons the colormaps are mirrored along $z$ axis, in a cylindrical coordinate system radial coordinate is always positive.
    %To remind about the axial symmetry the axis is drawn as orange dash-dotted line in the last frame.
    }
    \label{fig:phi_hm_grid}
\end{figure}

% \begin{figure}
%     \centering
%     \includegraphics[scale = 1]{fig/phi_open.png}
% \end{figure}

%\begin{figure}
    %\centering
    %\includegraphics[scale=1]{fig/phi_center.png}
    %\caption{Polymer density profile $\phi(z, r=0)$ along a pore's main axis $z$ for a set of solvent quality parameters. 
    %Solvent quality is defined by interaction parameter $\chi_{PS}$ ranging from 0.1 to 1.1, the value of $\chi_{PS}$ is written on the top of each curve and explained in the legend.
    %The region where the $z$ coordinate corresponds to the inner space of a pore is highlighted with gray annotated rectangle. 
    %}
    %\label{fig:phi_center}
%\end{figure}

Below we apply an approximate analytical scheme for calculating $\Delta F(z,r)$ as $\Delta F\{\phi(z,r)\}$, where $\phi(z,r)$ is polymer density distribution in a particle-free brush.
The particle is thus considered within this analytical approach as a "probe" which does not perturbed the density distribution $\phi(z,r)$ in the brush. 

The 2D polymer density profile $\phi(z,r)$ in the particle-free brush can be calculated 
for given brush architecture at arbitrary solvent quality for polymers using two-gradient SF-SCF theory.
The advantage of this scheme is that it enables evaluating of the insertion free energy $\Delta F\{\phi(z,r)\}$ at arbitrary position of the particle in the brush.
A more accurate calculation of the insertion free energy $\Delta F\{\phi(z,r=0)\}$ with the account of actual perturbation of the brush structure by inserted particle can be performed by SF-SCF method
for the particle positioned at the axis of the filled by the brush pore, $r=0$, that is, when the whole system possesses cylindrical symmetry, 
that enables using two-gradient SF-SCF theory for the brush with a particle inside it.
Details of the SF-SCF calculations are presented in SI.
An excellent quantitative matching of the results obtained by these two approaches for the free energy $\Delta F\{\phi(z,r=0)\}$
of insertion of the particle along the axis of the pore justifies using of $\Delta F(z,r)=\Delta F\{\phi(z,r)\}$ calculated by approximate analytical scheme for analysis of diffusive transport of particles 
through polymer-filled pore. 


%For spherical colloidal particles of finite size the coordinates $(z,r)$ refer to the position of its center while the insertion 
%free energy is obtained by integrating the volume and the surface contributions 
%over the volume and the surface of the colloid, respectively.

%%%%%%%%%%%%%%%%%%%%%%%%%%%%%%%%%%%%%%%%
\subsection{Conformational transition in the brush grafted to the pore walls triggered by inferior solvent strength}
%%%%%%%%%%%%%%%%%%%%%%%%%%%%%%%%%%%%%%%%



The conformations adopted by polymer chains grafted to the pore walls are controlled by strong (under overlapping conditions) intermolecular interactions and depend on the solvent quality. 
The latter can be quantified by Flory-Huggins solubility parameter $\chi_{PS}$ (here subscript "PS" denotes polymer-solvent interaction). 
The values of $\chi_{PS}<0.5$ and $\chi_{PS}>0.5$ correspond to good or poor solvent, respectively, whereas $\chi_{PS}=0.5$ in $\theta-$solvent.


The swelling/deswelling conformational transition triggered by variation in the solvent quality in the brush grafted inside the pore is illustrated in 
Figures \ref{fig:phi_hm_grid}, \ref{fig:phi_center}.
As one can see from Figure \ref{fig:phi_hm_grid}, \ref{fig:phi_center},
under good or theta-solvent conditions ($\chi_{PS}\leq 0.5$) the brush is swollen and fills not only interior of the pore but also protrudes outside the pore. 
We remind that swelling of the brush-forming chains with respect to unperturbed Gaussian dimensions 
occurs  under both good ($\chi_{PS}\leq 0.5$) and theta-solvent ($\chi_{PS}= 0.5$) conditions due to binary or ternary monomer-monomer repulsions, rerspectively.
On the contrary, in poor solvent, polymer brush is predominantly condensed inside the pore though (for chosen set of the pore and brush parameters) at moderately poor solvent strength
a fraction of the collapsed polymer chains form condensed "caps" at both ends of the pore.
It is worth noting that in the case of sufficiently wide pore and small polymerization degree/grafting density, an open channel free of polymer 
may appear under poor solvent strength conditions in the pore center, as discussed in details in ref \cite{Laktionov2021}.



%In the Figure \ref{fig:phi_hm_grid} first two frames corresponds to a swollen polymer brush in a good solvent $\chi_{PS}<0.5$, and the third one to a $\theta$-solvent $\chi_{PS}=0.5$. 
%For those conditions a large portion of the brush is outside the pore.
%Polymer brush in a good and $\theta$-solvent has wide region with polymer concentration smoothly decays, note the wide yellow halo in the first three frames in Figure \ref{fig:phi_hm_grid}.

%In a moderately poor solvent $\chi_{PS}=0.7$ the brush still protrudes outside the pore, however, the transition between the brush and solvent is more sharp, as one can see in the Figure \ref{fig:phi_hm_grid} as a yellow halo.
%In the poor solvent the brush collapses inside the pore forming concave dish-like surface with a step-like change in polymer density (small correlation length).

%These effects are illustrated in the Figure \ref{fig:phi_center}, which traces the evolution of polymer density profiles along the pore's main axis $z$. 
%Polymer brush in a good and $\theta$-solvent exhibit slow smooth change in polymer density when moving along $z$-axis, the polymer density profiles are wide bell-like curves with wide tails.
%In a poor solvent solvent polymer brush exhibit sharp transition in polymer concentration when moving along $z$-axis, with a box-like curves.
%****************************

%Colloid particles has a bulk concentration $c_0$ in one of the reservoirs and allowed to diffuse into the other reservoir with pure solvent only through the pore.
%Colloid particle interaction with solvent and polymer are very complex, we consider approximate model, where the particle has homogenous surface with a given affinity to polymer and solvent.
%When a particle come into contact with a polymer, the contact colloid-solvent (CS) and polymer-solvent (PS) are substituted to the contact polymer-colloid (PC).
%The particles, approximated with a sphere with diameter $d$, the affinity to the polymer is controlled with Flory-Huggins interaction parameter $\chi_{PC}$, 
%where subscript PC denotes polymer-colloid interaction. Without loss in generality $\chi_{CS}=0.0$, where subscript PS denotes colloid-solvent interaction.



%%%%%%%%%%%%%%%%%%%%%%%%%%%%%%%%%%%%%%%%%%%%%
\subsection{Free energy of the colloid insertion into the brush}
%%%%%%%%%%%%%%%%%%%%%%%%%%%%%%%%%%%%%%%%%%%%%



%\subsubsection*{Analytical method}

When a colloidal particle is moved from the bulk solvent into the brush, there is a change in the system free energy $\Delta F(z,r)$, which defines insertion free energy penalty. 
Positive $\Delta F$ thus implies that the brush repels the particle, and vice versa.

As long as the particle size is larger than the correlation length in semidilute polymer solution, to which the brush can be locally assimilated, but significanltly smaller than the characteristic 
pore size (coinciding under pore filling conditions with characteristic dimensions of individual macromolecules), 
the insertion free energy $\Delta F(z,r)$ can be decoupled into two contributions: osmotic $\Delta F_{osm}(z,r)$ and surface $\Delta F_{sur}(z,r)$ free energy,

\begin{eqnarray}
    \Delta F (z,r)= \Delta F_{osm}(z,r) + \Delta F_{sur}(z,r)
    \\
    \Delta F_{osm}(z,r) = \int_{V} \Pi(z,r) dV
    \\
    \Delta F_{sur}(z,r) = \int_{S} \gamma (z,r) dS
\label{Delta_F}
\end{eqnarray}
For spherical (or cylindrical) colloidal particle of finite size the coordinates $(z,r)$ refer to the position of its center while the insertion 
free energy is obtained by integrating the particle volume and the particle surface, respectively.

The osmotic contribution, $\Delta F_{osm}(z,r)$, is proportional to the particle volume and accounts for the work performed against excess osmotic pressure upon insertion of the partivle into the brush
The osmotic pressure in the brush can be calculated from local polymer density using Flory mean field approach, that leads to 
$$
\Pi(z,r)=  \phi(z,r)\frac{\partial f\{\phi(z,r)\}}{\partial \phi(z,r)} - f\{\phi(z,r)\}= 
$$
\begin{equation}
	%\Pi(z,r)=  \phi(z,r)\frac{\partial f\{\phi(z,r)\}}{\partial \phi(z,r)} - f\{\phi(z,r)\}= 
	k_BT[-\ln(1-\phi(z,r)) - \phi(z,r) -\chi_{PS}\phi^2(z,r)]
\end{equation}
where
$$
f\{\phi(z,r)\}=(1-\phi(z,r))\ln(1-\phi(z,r)) +\chi_{PS}\phi(z,r)(1-\phi(z,r))
$$
is the mean-field Flory expression for the interaction free energy per unit volume of the polymer solution of concentration (volume fraction) $\phi(z,r)$.
As long as osmotic pressure inside the brush is positive, $\Pi(z,r)$, the $\Delta F_{osm}(z,r)$ term is positive as well and dominating for sufficiently large particle. 

The surface contribution is proportional to the particle surface, with surface tension coefficient $\gamma (z,r)$ approximated as
\begin{eqnarray}
    \gamma (z,r)= \frac{1}{6}(\chi_{ads} - \chi_{crit})\phi^{\ast}(z,r)
    \\
    \chi_{ads} = \chi_{PC} - \chi_{PS}(1-\phi^{\ast})
    \\
    \phi^{\ast}(z,r)= (b_{0} + b_{1}\chi_{PC})\phi (z,r)
\end{eqnarray}
Here $\gamma$ is a free energy change upon replacement of a contact of the unit surface area of the particle with the pure solvent by a contact with polymer solution of concentration $\phi (z,r)$.
%When the particle placed into the brush, it creates a region depleted from polymer, if the particle is substantially attractive the region is enriched in polymer.
Coefficients $b_0$ and $b_1$ are introduced to account for  depletion/accumulation of polymer in the proximity of the colloid particle surface, thus correcting local polymer concentration in an empty brush 
to the apparent concentration $\phi^{\ast} (z,r)$. Coefficients $b_0$ and $b_1$ are subject to fitting.
Depending on the strength of polymer particle interaction parameter $\chi_{PC}$ and solvent strength quantified by  $\chi_{PS}$, the sign of $\gamma \sim (\chi_{ads} - \chi_{crit})\phi^{\ast}$ may be either politive or negative.
In the former case there is a depleated of polymer layer near the particle surface, that gives rise to positive $\gamma$ and, in combination with $\Delta F_{osm}(z,r)\geq 0$, overall positive insertion free energy $\Delta F(z,r)$: 
The particle is repeled from the brush. The latter case of negative $\gamma$ corresponds to polymer adsorption on the particle surface and to competition between $\Delta F_{osm}(z,r)\geq 0$ and $\Delta F_{surf}(z,r)\leq 0$, 
that may result in $\Delta F(z,r) = \Delta F_{osm} (z,r)+ \Delta F_{sur}(z,r)\leq 0$ and preferential accumulation of particles inside the brush. Since both $\Delta F_{osm}\{\phi(z,r)\}$ and $\Delta F_{surf}\{\phi(z,r)\}$ depend
on local polymer density $\phi(z,r)$, the net insertion free energy $\Delta F(z,r)$ is position-dependent as well, and it may exhibit rough 2D patterns.

The 2D distributions of the polymer volume fraction $\phi(z,r)$ in an unperturbed brush are calculated using lattice SF-SCF method, as explained in SI.
Consequently, a special discretization scheme was emploied for integration 
of the the volume and the surface of either a spherical particle of diameter $d$ or a cylindrical particle with the base diameter and height both equal to $d$ (See SI for details).
%with $z_c,r_c$ being coordinates ot its center .


%%%%%%%%%%%%%%%%%%%%%%%%%%%%%%%%%%%%%%%%%%%%%%%%%%%%%%%%%%%%%%%%%%%%%%%%%%%
%%%%%%%%%%%%%%%%%%%%%%%%%%%%%%%%%%%%%%%%%%%%%%%%%%%%%%%%%%%%%%%%%%%%%%%%%%%
%Consider a cylindrical particle with a center on $z$-axis embedded in a polymer brush. If one can neglect gradient in polymer density inside the brush on the length scale of the order of the particle size, $d\cdot \vert \nabla\phi(z,r) \vert \ll \phi(z)$, then the insertion free energy (with the reference state of the particle outside the brush) can be approximated as

%\begin{equation}
%	\begin{aligned}
%		\Delta F(z_c)= &\Delta F_{osm}(z_c) + \Delta F_{surf}(z_c)=
%		\\
%		&\Pi(z_c)\cdot V + \gamma\{\phi(z_c)\}\cdot A
%	\end{aligned}
%	\label{F_ins_1}
%\end{equation}
%where $z_c$ is the coordinate of the center.

%A more rigorous expression for the insertion free energy we use in numerical calculation for the cylindrical and spherical particles.
%Insertion free energy for cylindrical particle particle co-axial to the pore's main axis are formulated as

%\begin{align}
 %   \label{eq:cylinder_osm_fe}
  %  \Delta F_{osm}(z_c) =& 
   % 2 \pi \int_{z_c-d/2}^{z_c+d/2} \int_{0}^{d/2} \Pi(z,r) r dr dz
    %\\
    %\label{eq:cylinder_sur_fe}
    %\Delta F_{sur}(z_c) = & 
    %\\
    %\nonumber
    %\pi &\int_{0}^{d/2} \left[\gamma(z_c-d/2, r) + \gamma(z_c+d/2,r)\right] dr +
    %\\
    %\nonumber
    %2 &\pi d \int_{z_c-d/2}^{z_c+d/2} \gamma(z,d/2) dz 
%\end{align}

%In the Eq.\ref{eq:cylinder_sur_fe} the first term integrates surface tension coefficient over cylindrical particle base, in the second it is integrated over the element of the cylinder.

%The expressions for integration over spherical particle volume and surface are less trivial.
%Consider spherical particle placed with an offset to the pore's main axis with coordinates in cylindrical coordinates $(z_c, r_c)$.
%To formulate the simplest expression one has to integrate using local spherical coordinates with an origin placed in the spherical particle center with the zenith axis coaxial to $z$-axis.

%Consider a point with a local spherical coordinates $(\rho, \theta_{pol}, \phi_{az})$, where $\rho, \theta_{pol}, \phi_{az}$ are radial, polar and azimuthal coordinate, respectively.
%To translate it to global cylindrical coordinates one apply the next transformation 
%\begin{align}
 %   &z=z_c + \rho \cos \theta_{pol}
  %  \\
   % &r=\sqrt{r_c^2 + \rho^2 \sin^2 \theta_{pol} - 2 r_c \rho \sin \theta_{pol} \cos \phi_{az}}
%\end{align} 

%To calculate the osmotic term in the insertion free energy, one has to integrate osmotic pressure over the particle volume.

%\begin{equation}
%    \label{eq:sphere_osm_fe}
   % \begin{aligned}
    %    \Delta F_{osm}(z_c, r_c) =&
     %   \\ 
      %  2 \int_{0}^{d/2} \int_{0}^{\pi} \int_{0}^{\pi} \Pi \left\{ z_c + \rho \cos \theta_{pol}, \sqrt{r_c^2 + \rho^2 \sin^2 \theta_{pol} - 2 r_c \rho \sin \theta_{pol} \cos \phi_{az}} \right\}
       % \\
        %\rho^2 \sin\theta_{pol} d\theta_{pol} d\phi_{az} d\rho&
    %\end{aligned}
%\end{equation}

%To calculate the osmotic term in the insertion free energy, one has to integrate osmotic pressure over the particle volume.

%\begin{equation}
    %\label{eq:sphere_sur_fe}
    %\begin{aligned}
    %\Delta F_{sur}(z_c, r_c) =&
    %\\
    %\frac{d^2}{2} \int_{0}^{\pi} \int_{0}^{\pi} \gamma \left\{ z_c + \rho \cos \theta_{pol}, \sqrt{r_c^2 + \frac{d^2}{4} \sin^2 \theta_{pol} - 2 r_c \frac{d}{2} \sin \theta_{pol}} \right\} &
    %\\
    %\sin \theta_{pol} d\theta_{pol} d\phi_{az}&
    %\end{aligned}
%\end{equation}

%Note that in Eq. \ref{eq:sphere_osm_fe}, \ref{eq:sphere_osm_fe}  $\theta_{pol}, \phi_{az}$ are dummy variables and will not present in the final result, so the insertion free energy for a given particle depends only on its position $(z_c, r_c)$.

%To calculate this integrals we employed numerical integration, for the technical details please visit \emph{SUPPLEMENTARY INFORMATION}

%%%%%%%%%%%%%%%%%%%%%%%%%%%%%%%%%%%%%%%%%%%%%%%%%%%%%%%%%%%%%%
%%%%%%%%%%%%%%%%%%%%%%%%%%%%%%%%%%%%%%%%%%%%%%%%%%%%%%%%%%%%%%
%%%%%%%%%%%%%%%%%%%%%%%%%%%%%%%%%%%%%%%%%%%%%%%%%%%%%%
%\subsubsection{Free energy calculation by numerical SF-SCF method}
%%%%%%%%%%%%%%%%%%%%%%%%%%%%%%%%%%%%%%%%%%%%%%%%%%%%%%

The SF-SCF method in its 2 gradient version can be used for direct calculating of the free energy $\Delta F(z,r=0)$ of the spherical or cylindrical particle insertion into the
polymer-filled pore along its axis ($r=0$)
%As follows from \todo{eq.} to construct analytical scheme one has to specify coefficient $b_0$ and $b_1$.
%This was done by mapping insertion free energy obtained by analytical method 
%to the free energy of the probe particle inserted at arbitrary $z$ position at the pore axis ($r=0)$ 
%calculated with Scheutjens-Fleer self-consistent field (SF-SCF) numerical method.
We remind that calculation of the free energy of the pore with a particle in it by SF-SCF method  can be performed only 
for cylindrically symmetrical configuration of the system, that is, when the particle of cylindrical or spherical shape is placed at arbitrary position at the axis of the pore, $r=0$ and
the (cylindrical) particle axis coincides with the axis of the pore (See Figure in SI).
  

%Similar to the procedure the authors employed in \todo{ref}.
In the SF-SCF scheme the colloid particles is defined as an array of the lattice cells impermeable for the polymer embedded at the pore axis.
%with some surface affinity to the polymer. 
%The particles were moved from bulk solution to the center of the pore, by defining impermeable for the polymer lattice cells.
%Not to break axial symmetry of the cylindrical pore, we consider cylindrical particle embedded in the polymer brush coaxial with the system's main axis.
The particle position is specified by the $z$-coordinate of its center. 
%of the cylinder $z_c$, where $z=0$ corresponds to a cylindrical particle placed in the pore center.
For each particle position the total free energy $F_{SF}(z_c)$ was calculated with SF-SCF scheme. 
The free energy of the system at the particle position far away from the pore (ground state) was subtrated to 
assure reference value $\Delta F_{SF}(z_c \rightarrow \pm \infty) = 0$
%These values were ground state corrected such that the system total free energy of a particle in bulk solution equal to zero $F_{SF}(z_c \gg 0) = 0$.

%For fixed pore radius $r$, membrane thickness $s$, polymer grafting density $\sigma$ and polymerization degree $N$ insertion free energy along $z$ 
%were explored for different values of interaction parameters $\chi_{PC}$, $\chi_{PS}$ and particle size $d$.
%Polymer-colloid interaction parameter $\chi_{PC}$ ranged from -1.5 (attractive colloid particle) to 0.0 (inert colloid particle).
%Polymer-solvent interaction parameter $\chi_{PS}$ ranged from 0.0 (good solvent) to 1.0 (poor solvent).
%Cylindrical particles with equal diameter $d$ and height were explored for size ranging from 4 to 24 unit lengths.


\begin{figure}
    \centering
    \includegraphics[width = 4in]{fig/free_energy_hm.png}
    \caption{ 
The insertion free energy calculated by analytical and SF-SCF methods for a cylindrical particle with diameter and height $d=8$ moving along the main axis of the pore with radius $r_{pore} = 26$ and membrane thickness $s=26$.
    Solvent quality is varied near $\theta$-point with $\chi_{PS} = [0.4, 0.5, 0.6]$, ordered from left to right column, respectively.
    Colloid particle affinity to polymer chains ranges from attractive to inert particle with $\chi_{PC} = [-1.0, -0.5, 0.0]$, ordered from first to last row, respectively.
    \\
    Horizontal axes corresponds to position of the particle's center $z_c$, vertical axes corresponds to free energy value.
    \\
    Insertion free energy calculated with SF-SCF scheme $\Delta F_{SF-SCF}$ is shown with red squares, the results of analytical scheme are presented with solid red line for the total insertion free energy $\Delta F_{tot}$ and dashed green and blue line for the osmotic and surface term, respectively.
    The gray area marks values of $z$ that corresponds to the volume inside the pore $s\in [-13, 13]$.
    \label{fig:fe_scf_grid}
    }
\end{figure}

%\begin{figure}
 %   \centering
  %  \includegraphics[width = 3in]{fig/phi_correction.png}
   % \caption{
    %    Apparent local polymer density correction factor $\phi^{\ast}/\phi$ as a function of polymer-colloid interaction parameter $\chi_{PC}$ (particle affinity to polymer).
     %   The value $\chi_{PC}^{\ast} \approx -1.0$ is value when no correction needed. 
      %  For the less attractive particles $\chi_{PC}<\chi_{PC}^{\ast}$, apparent concentration is lower then calculated with SF-SCF scheme for the empty brush $\phi^{\ast}/\phi<1$ and vice versa}
    %\label{fig:phi_correction}
%\end{figure}

%To exclude any effects caused by perturbation in polymer brush when embedding larger particles the fit were performed for the smallest particles with $d=4$ moving along the main axis of the pore with radius $r_{pore} = 26$ 
%and membrane thickness $s=26$ for a ranging values of interaction parameters $\chi_{PC}$ and $\chi_{PS}$, the other results verified and exposed limitation of the fit.

In Figure \ref{fig:fe_scf_grid} the insertion free energy profiles $\Delta F(z,r=0)$ calculated by analytical scheme and by SF-SCF method 
are presented as a function of position of a spherical particle along the pore axis.
While the SF-SCF method provides the net free energy, the analytical scheme allows decoupling of the free energy into osmotic and surface contributions, 
which are shown separately in Figure \ref{fig:fe_scf_grid}.
The numerical coefficients $b_0$ and $b_1$ in eq \ref{} are chosen by the best fit, but appear to be fairly universal and independent of the particle size 
and interaction parameters $\chi_{PS,PC}$.
Remarkably, the fit fails when the size $d$ became comparable with the pore diameter or in the case of extreme $\chi_{ads}$ values 
when analytical scheme is not applicable because of strong perturbation 
of the brush structure by inserted particle, while SF-SCF method can still be safely used
for the evaluation of the insertion free energy.

The 2D insertion free energy $\Delta F(z,r)$ patterns have rather complex shape. However, we can trace their evolution upon changing interaction parameters
looking at the position-dependent free energy of the particle on the pore axis, $\Delta F(z,r=0)$.
As one can see from Figure \ref{fig:fe_scf_grid}, the insertion free energy profiles evolve upon changing the interaction parameters $\chi_{PS,PC}$ as follows:
At $\chi_{ads}\geq \chi_{crit}$ which is the case under good or theta-solvent conditions and weak or absent polymer-particle attraction, $|\chi_{PC}|\leq 1$, the positive osmotic
term, $\Delta F_{osm}\geq 0$ dominates in the insertion free energy, which is positive and reach maximal value in the pore center, where polymer concentration is maximal.
Hence, polymer-particle interaction has overall repulsive character and $\Delta F(z,r)$ has the shape of the free energy barrier preventing penetration and accumulation of particles in the pore.
By using the insertion free energy $\Delta F(z,r)$ one can calculate the equilibrium partition coefficient 
$$
P=\int_{0}^{r_{pore}}2\pi rdr\int_{0}^{L}dz\exp (-\Delta F(z,r)/k_BT)/\pi r^{2}_{pore}L
$$
is larger than unity, $P\geq 1$. Noticably the repulsive free energy profiles extends beyond the edges of the pore, because of the fringes in the polymer density distribution in swollen brush.

A decrease in $\chi_{ads}$ triggered by a decrease in  $\chi_{PC}$ or/and an increase in $\chi_{PS}$ leads to qualitative changes in the insertion free energy 
$\Delta F(z,r)$ patterns: At $\chi_{ads}\leq \chi_{crit}$ the particle surface becomes
adsorpbing for the polymer, $\gamma \leq 0$, that gives rise to a negative contribution $\Delta F_{surf}(z,r)$ to the insertion free energy. 
When $\chi_{PS}$ increases (the solvent is getting worse for the polymer)
the osmotic pressure inside the brush decreases that leads to a decrease in the 
magnitude of $\Delta F_{osm}(z,r)$ with the concomitant shrinkage of the  protruding outside the pore parts of the brush where  $\Delta F(z,r)\neq 0$.
As a result, the $\Delta F_{surf}(z,r)$ aquires two minima with negative values near the endtance and the exit of the pore, separated by a maximum centered in the middle of the pore
where polymer concentration is larger and the osmotic repulsive term  $\Delta F_{osm}(z,r)$ dominates.
Finally, at strong polymer-particle attraction $\chi_{ads} < \chi_{crit}$, the negative surface contribution $\Delta F_{surf}(z,r)\leq 0$ overperform osmotic repulsion everywhere inside the pore
and the $\Delta F(z,r)$ aquires the shape of the potential well centered in the middle of the pore, which gives rise to preferential accumulation of particles in the pore, $P\geq 1$.


%The result of the mapping is presented on the Figure \ref{fig:fe_scf_grid} for a larger particle $d=8$ 
%then the fit had been performed for, this verifies that the analytical method is invariant of the particle size $d$.
%The fit will fail, eventually, when the size $d$ became comparable with the pore diameter or in the case of extreme $\chi_{ads}$ values.

%The main result of the fit is coefficients $b_0$ and $b_1$ introduced in \todo{eq}.
%The coefficients were employed to calculate surface term of the insertion free energy with a correction in proximal to the particle polymer density (apparent density) $\phi^{\ast}$.
%The correction is illustrated with the Figure \ref{fig:phi_correction}.









%%%%%%%%%%%%%%%%%%%%%%%%%%%%%%%%%%%%RESULTS%%%%%%%%%%%%%%%%%%%%%%%%%%%%%%%%%%%%%%%%%%%%%%%%%%%%%%%%%%%%%%%%%%%%%%%%%%%%%


%\subsection{Permeability as function of size}
%\begin{figure}
 %   \centering
 %   \includegraphics[width = \textwidth]{fig/permeability.pdf} \caption{
        %Permeability coefficient $P$ for a spherical colloid particle as a function of particle size for different solvent quality and particle affinity.
        %\\
        %Solvent quality is defined with polymer-solvent interaction parameter $\chi_{PS}$ ranging from good to moderately poor solvent, four frames from left to right corresponds to a set of $\chi_{PS} = {0.3, 0.4, 0.5, 0.6}$.
        %\\
        %Colored solid lines corresponds to different values of polymer-colloid interaction parameter $\chi_{PC} = {-1.5, -1.25, -1.00, 0.00}$ from attractive to inert particle. 
        %The color code explained in the legend.
        %\\
        %Dashed black line traces permeability of an empty pore in a membrane of a finite thickness $s$.
        %Dotted black line traces permeability of an empty pore in an infinitely thin membrane.
        %Both results are calculated using analytical solution of diffusion through a pore problem.
        %\\
        %Selected cases were calculated numerically using CFD approach, the results shown as circle markers that shares the same color as solid lines.
        %\\}
       % \label{fig:partition_on_d}
%\end{figure}

%\subsection{Permeability versus partitioning}
%\subsection{Critical values}
%\begin{figure}
%    \centering
%    \includegraphics[width = \textwidth]{fig/chi_PC_crit_on_chi_PS.pdf}
%    \caption{
%        Critical value of polymer-colloid interaction parameter $\chi{PC}^{crit}$ as a function of solvent quality defined by polymer-solvent interaction parameter $\chi_{PS}$ and particle size $d$.
%    }
%    \label{fig:chi_pc_crit_on_chi_ps}
%\end{figure}

%\begin{figure}
%    \centering
%    \includegraphics[width = \textwidth]{fig/chi_PC_crit_on_d.pdf}
%    \caption{
%        Critical value of polymer-colloid interaction parameter $\chi{PC}^{crit}$ as a function of solvent quality defined by polymer-solvent interaction parameter $\chi_{PS}$ and particle size $d$.
%        \\
%        \todo{check non-monotonicity}
%    }
%    \label{fig:chi_pc_crit_on_d}
%\end{figure}

%%%%%%%%%%%%%%%%%%%%%%%%%%%%%%%%%%%%%%%%SI%%%%%%%%%%%%%%%%%%%%%%%%%%%%%%%%%%%%%%%%%%%%%%%%%%%%%%%%%%%%%%%%%%%%%%%%%%%%%%
%\begin{equation}
%    D = \frac{D_{0}}{1+\phi^2 d^2}
%\end{equation}

%\begin{equation}
%    P_{theory} = \frac{2 D r_{pore}}{\pi + 2 s / r_{pore}}
%\end{equation}

%\begin{equation}
%    P_{convergent} = \frac{2 D r_{pore}}{\pi}
%\end{equation}

%\begin{eqnarray}
%    j_0 = c_0 P
%    \\
%    P_{channel} = \left[\int_{-s/2}^{s/2} \left( \int_{0}^{r_{pore}} D e^{-\Delta F / kT} r dr \right)^{-1} dz \right]^{-1}
%    \\
%    P = P_{channel}  + P_{convergent}^
%\end{eqnarray}

%\begin{equation}
%    P_{channel} = \left[\sum_{k=-s/2}^{s/2} \left( \sum_{i=0}^{r_{pore}} \Delta F_{[i,k]} \cdot (2i+1) \right)^{-1} \right]^{-1}
%\end{equation}


%Diffusion of colloid particle in the presence of potential field governed by Smoluchowski equation. The equation is closely connected to advection-diffusion and drift-diffusion equation.

%\begin{equation}
%    \partial_{t} c(r,z,t) = \nabla \cdot D(r,z)(\nabla c(r, z, t) +  c(r, z, t) \nabla U)
%\end{equation}

%Here $c$ is the concentration of the colloidal particles, $D$ is the local diffusion coefficient, and $U \equiv \Delta F$ is the position-dependent free energy of insertion which plays the role of the potential of mean force.



%%%%%%%%%%%%%%%%%%%%%%%%%%%%%%%%%%%%%%%%%%%%%%%%%%%%%%%%%%%%%%%%%%%%%%%%
%%%%%%%%%%%%%%%%%%%%%%%%%%%%%%%%%%%%%%%%%%%%%%%%%%%%%%%%%%%%%%%%%%%%%%%%
\section{Discussion}
%%%%%%%%%%%%%%%%%%%%%%%%%%%%%%%%%%%%%%%%%%%%%%%%%%%%%%%%%%%%%%%%%%%%%%%%
%%%%%%%%%%%%%%%%%%%%%%%%%%%%%%%%%%%%%%%%%%%%%%%%%%%%%%%%%%%%%%%%%%%%%%%%
In Figure 7 the total resistance of the pore to the diffusive flow calculated analytically as a function of the particle size is presented for 
different combinations of the interactions parameters, $\chi_{PC}, \chi_{PS}$.
While a decrease in $\chi_{PC}$ (increasing strength of the polymer-particle attraction) leads only to a decreasing resistance (enhancing permeability) 
due to a decrease in $\Delta F(z,r)$ both inside the pore and within fringes, a decrease in the solvent strength (increase in $\chi_{PS}$) has more complex effect: First,
$\chi_{ads}$ and $\Delta F(z,r)$ decrease, similarly to how it is triggered by a decrease in $\chi_{PC}$, that facilitate diffusion. The opposite, though weaker effect is cause by
decreasing local diffusivity (see eq \ref{}) due to an increase in local polymer concentration in the brush $\phi(z,r)$. Finally, the span of the fringes providing attractive for the particles
regions outside the pore decreases that also diminishes the effect of enhancing of the diffusion in the pre-membrane region.
The correlations between conformational re-arrangements in the brush and effect of $\chi_{PS}$ on the diffusion are illustrated by Figure...

To validate our approximate analytical solution, we have also performed full numerical solution of the Smoluchowsky diffusion equation, which is illustrated by Figure X 
(details in SI) and corresponding point are presented, together with analytical results in Figure 7. A very good quantitative agreement proves accuracy of our analytical theory
within the limits of the particle size sufficiently smaller than the pore radius (Note that this is applicable to NPC where facilitated transport starts from 5 nm while the
NP diameter is about 40 nm).

In Figure XX we present the dependence of the resistance $R_{tot}$ on the polymer-particle affinity parameter $\chi_{PC}$.


%%%%%%%%%%%%%%%%%%%%%%%%%%%%%%%%%%%%%%%
%Many pores problem
%%%%%%%%%%%%%%%%%%%%%%%%%%%%%%%%%%%%%%%
The question of how several pores in the same membrane interfere affecting their permeability was first posed by Rayleigh himself \cite{Strutt1878}. 
Fabrikant  proposed a quantitative theory for a negligibly thin membrane with several circular apertures of different radii and arbitrary mutual positions \cite{Fabrikant1985}. 
The resultant effect of the pore interference is an increase in the pore permeability since the Rayleigh resistance is partially shared by the neighboring pores. 
However, the effect is quite small (a few percent) whenever the distance between the pore centers is larger than their diameters but an order of magnitude or more. 
It is intuitively clear that once the resistance due to a finite pore length and due to the brush is non-negligible, the mutual interference effect 
becomes even smaller. Hence, we are not concerned with this aspect of the problem.
%%%%%%%%%%%%%%%%%%%%%%%%%%%%%%%%%%%%%%%


%%%%%%%%%%%%%%%%%%%%%%%%%%%%%%%
\section{Conclusion}
%%%%%%%%%%%%%%%%%%%%%%%%%%%%%%

In the present paper we have explored physical mechanisms for facilitated selective diffusive transport of nanocolloidal particles through nanoscale mesopores
filled by polymer brushes grafted onto the inner pore wall. 
For that we have implied approximate analytical and numerical solutions of the 
Smoluchowsky diffusion equation in the external potential field experienced by diffusing particle interacting with the brush that fills the interior of the pore
and protrudes outside the pore into the surrounding solution. In addition, enhanced resistance of the semidilute polymer solution, to which the brush-filled pore
interior can be locally assimilated, to the diffusing species was taken into account. 

The inspiration for this theoretical work was provided by selective permeability barrier of NPC... 

As we demonstrated, depending on the strength of the polymer-colloid interaction or/and solvent quality of the polymer chains, diffusion of particles of different sizes
through the pore can be either blocked by the resistance of the brush or enhanced compared to the diffusion through the empty (polymer-free) pore. 

Mximal size of the particles for diffusion through the pore is unhindered (gating threshold)  depends on the set of control interaction parameters and increases as a function of polymer-particle
affinity and decreasing solvent strength.
Moreover, we demonstrated that in the case of polymer fringes protruding outside the pore and attractive polymer-particle interaction, the total resistance composed of that of the pore itself 
and convergent/divergent flow regions can be even lower than that of the
empty (polymer-free) pore due to reduced resistance of the proximal to the pore entrance/exit regions.



Alltogether, our findings shed the light onto possible mechanisms of selective transport through NPC and, at the same time, suggests an molecular design strategy for controlling selective
permeability through artificial mesoporous membranes with the eye on applications in...



\printbibliography
\end{document}