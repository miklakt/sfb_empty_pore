\documentclass[12pt, a4paper]{article}
\usepackage{graphicx}
\usepackage{amsmath, amssymb, amsfonts, mathtools}
\usepackage[
backend=biber,
natbib=true,
style=numeric,
sorting=none
]{biblatex}
\usepackage{xcolor}
\usepackage{bm}

\newcommand\todo[1]{\textcolor{red}{#1}}

\addbibresource{biblio.bib}
\title{Physical principles of selective colloid permeation through polymer-filled mesopores}

\author{Mikhail Y. Laktionov$^1$, Leonid I.Klushin$^{2}$,\\Ralf P.Richter$^3$, France A.M. Leermakers$^4$, Oleg V.Borisov$^1$\\
$^{1}$CNRS, Universit\'e de Pau et des Pays de l'Adour UMR 5254,\\
Institut des Sciences Analytiques et de Physico-Chimie\\
pour l'Environnement et les Mat\'eriaux, 64053 Pau, France \\
$^{2}$Institute of Macromolecular Compounds \\
of the Russian Academy of Sciences, \\
199004 St.Petersburg, Russia,\\
$^{3}$University of Leeds, School of Biomedical Sciences, \\
Faculty of Biological Sciences, 
School of Physics and Astronomy, \\
Faculty of Engineering and Physical Sciences,\\  
Astbury Centre for Structural Molecular Biology,\\ 
and Bragg Center for Materials Research,\\ 
Leeds, LS2 9JT, United Kingdom\\
$^{4}$ University of Wageningen, the Netherlands
}

\begin{document}
\maketitle

\begin{abstract}

% RR: I have note worked on the abstract yet.

Physical mechanisms of selective facilitated permeation of nanocolloidal particles 
through polymer-grafted mesopores are unravelled on the basis of self-consistent field theoretical modelling.
We predict that diffusive transport of particle can be accelerated compared to that through a bare pore due to
cohesive polymer-particle interactions, while penetration of inert with respect to the polymer particles of even smaller size can be 
efficiently impeded. We formulate thermodynamic criteria for unrestricted gating threshold through the pore and anticipate, that underlying
physical mechanisms may apply for facilitated permeation of biologically active molecules in complex with NTR through NPC.   
\end{abstract}

%%%%%%%%%%
\section{INTRODUCTION}
%%%%%%%%%%

% RR: I have not worked on this part of the introduction yet.

Polymer-modified mesoporous materials and membranes belong to a new class of functional nanostructured materials with great potential in a number of key technologies. 
The interaction and absorption of (macro)molecules and nanocolloidal particles by porous media, as well as their transport through macro- and mesoporous membranes, 
are important elements of many technological processes (chromatography, heterogeneous catalysis, micro- and ultrafiltration, protein separation and purification etc.) 
and, therefore, have been the subject of intensive research for decades. 

Advances in macromolecular chemistry have made it possible to significantly improve functional properties of mesoporous (with a pore diameter within 100 nm) 
materials by modifying them with covalently (or strongly non-covalently) bound to the pore walls macromolecules of various chemical nature,  
forming a “soft” , solvated physical polymer meshwork that fills the pore volume or only the near-wall regions, 
depending on molecular mass and conformational state of the polymer chains. 
The interaction of this polymer meshwork with guest molecules/nanoparticles 
%and, in particular, the presence or absence of a hollow (polymer-free) channel in the center of the pore, 
essentially determine the absorption and separation properties of the polymer-modified mesoporous materials and membranes, respectively. 
These interactions can be attractive or repulsive, short- or long-range (in the presence of charges on the chains and on guest molecules/particles), 
and most importantly, they can be controlled by a complex of external stimuli, such as temperature, pH and/or ionic strength of the medium, valence of ions , 
solvent composition, etc. This opens up a unique opportunity for highly selective and controlled uptake and transport of guest molecules/nanoparticles 
through polymer-filled mesoscopic channels. 
%For example, mesopores modified with ionic polymer molecules can potentially be used to separate molecules/nanocolloids that are almost identical in size and shape, 
%but differ in a small number of charged groups. 

Nature uses the principle of controlling the selective transport of biological molecules 
between the nucleus and cytoplasm of eukaryotic cells through the so-called nucleopores (cylindrical channels of about 40 nm in diameter), 
which perforate the nuclear envelope and are filled with a swollen meshwork consisting of natively denatured proteins attached to the pore walls. 
A similar structural motif was recently found in the internal channels of microtubules (about 15 nm in diameter) 
decorated with so-called microtubule intrinsic proteins (MEPs), presumably modifying microtubule stability and rigidity.
A popular nowadays paradigm suggests that the accuracy and efficiency of many processes in nature are ensured 
not so much by specific (molecular recognition) interactions, but due to a fine balance of fundamental (electrostatic, hydrophobic...) 
interactions between biomacromolecules and (bio) nanocolloids.  

However, up to date, 
%the theoretical knowledge and systematic 
understanding of the relationship between molecular architecture of the brush 
decorating the pore walls and the spatial structure, 
cohesive and rheological properties of the resulting "soft" meshwork and its ability to selectively absorb in the 
volume of pores or modulate the diffusion transport of nanocolloidal particles through the pores is lacking. 

% RR: I have prepared the following final paragraph.

Our analysis focuses on pores that are pervaded by a dynamic meshwork of flexible polymers. 
This meshwork effectively increases the local viscosity and thereby slows down transport of colloids compared to an open pore. 
On the other hand, an attractive polymer phase recruits colloids into the pore, thus increasing colloid transport, 
and such recruitment is further enhanced when attractive polymers extend outside the pore. 
Intriguingly, the solvent strength through its influence on the density and compactness of the polymer plug impacts all these effects. 
Here, using a self-consistent field approach, we define how solvent quality and colloid attraction to the polymer may be tuned 
to maximise the transport rate (including beyond the rate for an open pore) or for the highly selective transport of colloids of a desired size or attraction to the polymers.


%%%%%%%%%%
\section{RESULTS}
%%%%%%%%%%


%%%%%%%%%%
\subsection{Defining the transport scenario}
%%%%%%%%%%

\begin{figure}
    \centering
    \includegraphics[width = 0.7\linewidth]{fig/pore_cartoon.png}
    \caption{
        Schematic illustration of colloid diffusive transport through a pore filled with polymer brush. 
        The brush is formed by linear polymer chains (red strands) with a degree of polymerization $N$, uniformly grafted with grafting density $\sigma$ 
        to the inner surface of a cylindrical pore in an impermeable membrane. The pore radius is $r_{\text{pore}}$ and the thickness of the membrane is $L$.
        Polymer chains are flexible with a statistical segment length $a$ and volume $\sim a^3$. 
        Spherical colloids with diameter $d$ are free to diffuse in the surrounding solvent.
        All length scales are normalized by the statistical segment length $a$.
        \\
        The pore is permselective to larger particles; larger magenta-colored particles are drawn only on the left side of the membrane to symbolize this effect.
        }
    \label{fig:colloid_transport}
\end{figure}

The salient features of our simulated mesopore (Figure \ref{fig:colloid_transport}) are inspired by the nuclear pore complex.
Pores with such features can though also be realized synthetically.
The pore has a cylindrical shape with diameter $r_{\text{pore}}$.
The pore perforates an otherwise impermeable, planar membrane of thickness $L$, and thus is the sole conduit for colloids between two semi-infinite solution reservoirs.
Flexible polymer chains are end-grafted to the inner pore walls, at a density sufficient to form a polymer brush that penetrates the entire pore cross-section.
We aim to define how the diffusive transport of colloids across the pore is modulated by the polymer brush. 
The interaction strength between a polymer segment and a unit surface area of the colloid is represented by the Flory-Huggins parameter $\chi_{\text{PC}}$. 


%%%%%%%%%%
\subsection{Colloid transport is defined by the sum of resistances outside and inside the pore}
%%%%%%%%%%

To understand how the polymer brush affects the transport of colloids, we consider the stationary diffusive flux of colloids through the pore 
and analyze how it is affected by the parameters of the pore, the brush, and the colloid.
To this end, the colloid concentration is fixed to be zero and $c$ far away from the membrane (at $z\rightarrow\mp\infty$, respectively). 
We assume axial (cylindrical) symmetry of the flow in the pore. Together with the stationary conditions, 
this implies that parameters relevant to colloid transport depend on the axial coordinate $z$ and the radial coordinate $r$, but not on the azimuthal angle.

%%%%%%%%%%
\subsubsection{Empty pore as a reference case}
%%%%%%%%%%

A natural reference is the diffusive flux through an empty pore without any polymer. The earliest approach to that problem goes back to Lord Rayleigh 
who analyzed the flux through a circular pore in a planar membrane of negligible thickness \cite{Strutt1878}. 
The equiconcentration surfaces in this case are oblate spheroids and the streamlines form confocal hyperboloids of revolution \cite{Cooke1966}.
The net flux through the pore is given by

\begin{equation}
    J=2D_0r_{\text{pore}}\Delta c
    \label{eq:flux_Ral}
\end{equation}

\noindent where $D_0$ is the diffusion coefficient of the colloid in plain solvent. 

Diffusion through a pore in a membrane of finite thickness $L$ allows an approximate analytical solution (with an error of less than 6\% in the full range of the $\frac{L}{r_{\text{pore}}}$ ratio) \cite{Brunn1984}: 

\begin{equation}
    J=\frac{2D_0r_{\text{pore,eff}}}{1+\frac{2L}{\pi r_{\text{pore,eff}}}}\Delta c
    \label{eq:flux_finlength}
\end{equation}

\noindent To account for the excluded volume of the diffusing colloids, assuming them being spheres of diameter $d$, we here use the effective pore radius $r_{\text{pore,eff}}=r_{\text{pore}}-d/2$.

Introducing the resistance $R$ to colloid flow as $J=\frac{\Delta c}{R}$ admits a most natural interpretation of Eq. (\ref{eq:flux_finlength}) in terms of the total resistance of the pore

\begin{equation}
    R_{\text{empty}}=\frac{L_{\text{eff}}}{D_0\pi r_{\text{pore,eff}}^{2}}+\frac{1}{2D_0r_{\text{pore,eff}}}=R_{\text{pore, empty}}+R_{\text{conv, empty}}
    \label{eq:resistance}
\end{equation}

\noindent The first term in Eq. (\ref{eq:resistance}) is the resistance of the interior of the empty pore, 
while the second term is the Rayleigh resistance of a pore of infinitesimal thickness (Eq. (\ref{eq:flux_Ral})). 
The latter represents the effects of the convergent flow towards the pore entrance and its symmetric counterpart at the pore exit, 
while the flow lines inside the cylindrical pore turn out to be approximately axial. 
In the empty pore scenario, the inverse of the diffusion constant ($\rho_0=D_0^{-1}$) represents the resistivity of the medium both inside and outside the pore. 

%%%%%%%%%%
\subsubsection{A polymer filling affects the resistance of the pore itself, and also of regions outside the pore}
%%%%%%%%%%

\begin{figure}
    \centering
    \includegraphics[width = 0.7\linewidth]{fig/phi_hm_grid.png}
    \caption{
    Maps of the polymer volume fraction $\phi(r,z)$ for a polymer brush in a cylindrical pore at a range of solvent qualities, as predicted by self-consistent field theory. 
    The solvent quality is quantified by the Flory-Huggins parameter $\chi_{\text{PS}}$ ranging from 0.1 (good solvent) to 1.1 (poor solvent).
    Polymer volume fractions are mapped in cylindrical coordinates (as shown in the inset), colour coded as indicated and with selected iso-concentration lines shown. 
    For illustrative purposes the colormaps are mirrored along the $z$ axis, and the membrane is indicated in green.
    Conditions: $L=2r_{\text{pore}}=56$, $\sigma=0.02$, $N=300$.
    }
    \label{fig:phi_hm_grid}
\end{figure}

The conformations adopted by polymer chains grafted to the pore walls are controlled by strong (under overlapping conditions) intermolecular interactions, and depend on the solvent quality. 
The solvent quality is quantified by the Flory-Huggins solubility parameter $\chi_{\text{PS}}$. 
Values of $\chi_{PS}<0.5$ and $\chi_{PS}>0.5$ correspond to good and poor solvent, respectively, whereas $\chi_{PS}=0.5$ represents the ideal (or $\theta-$)solvent.

The polymer density profile $\phi(r,z)$ in the pore was calculated by two-gradient self-consistent field theory according to Scheutens and Fleer 
%UPDATE MANUALLY%%%%%%%%%%%%%%%%%%%%%%%%%%%%%%%%%%%%%%%%%%%%%%%%%%%%%%%%%%%%%%%%%%%%%%%%%%%%%%%%%%%%%%%%%%%%%%%%%%%%%%%%
(SF-SCF; see Section 2 of SI for details).
%%%%%%%%%%%%%%%%%%%%%%%%%%%%%%%%%%%%%%%%%%%%%%%%%%%%%%%%%%%%%%%%%%%%%%%%%%%%%%%%%%%%%%%%%%%%%%%%%%%%%%%%%%%%%%%%%%%%%%%%
In Figure \ref{fig:phi_hm_grid}, one can appreciate the expected increase in polymer concentration with decreasing solvent quality (increasing $\chi_{\text{PS}}$). 
It is notable that the variations in polymer concentration as a function of solvent strength are substantial, 
and that the concentration is more or less homogeneous across the polymer-filled space. 
The polymer brush penetrates the entire pore cross-section across the full range of solvent qualities explored ($0.1\le\chi_{\text{PS}}\le1.1$), 
implying that colloids need to navigate the polymer medium to reach across the pore.

We remind that swelling of the brush-forming chains with respect to unperturbed Gaussian dimensions occurs under both good ($\chi_{PS} \leq 0.5$) and $\theta$-solvent ($\chi_{PS}= 0.5$) conditions due to binary or ternary monomer-monomer repulsions, respectively.
In contrast, in poor solvent, the polymer brush is predominantly condensed.
It is worth noting that in the case of sufficiently wide pore and small polymerization degree/grafting density, 
an open channel free of polymer may appear under poor solvent conditions in the pore center, as discussed in details in~\cite{Laktionov2021}. 
This scenario would lead to a distinct permeation behaviour as the colloids can move through the pore without traversing the brush, and is not further considered here. 

Figure \ref{fig:phi_hm_grid} further illustrates that whilst the brush remains confined within the pore lumen in poor solvent ($\chi_{\text{PS}}=0.9$ and $\chi_{\text{PS}}=1.1$) 
it protrudes substantially into the surrounding space in ideal and good solvent ($\chi_{\text{PS}}\le0.5$), thus forming 'caps' on either side of the pore.  
The polymer brush therefore will impact on colloid flow within as well as outside the pore, such that

\begin{equation}
    R=R_{\text{pore}}+R_{\text{conv}}
    \label{eq:R_tot_tot}
\end{equation}
    
\noindent with $R_{\text{pore}}\rightarrow R_{\text{pore, empty}}$ and $R_{\text{conv}}\rightarrow R_{\text{conv, empty}}$ in the limit of the empty pore. 


%%%%%%%%%%
\subsection{Insertion free energy and diffusivity define transport locally}
%%%%%%%%%%

Zooming in on the local scale, we can define how colloids are accumulated or depleted due to attractive or repulsive interactions, respectively, by the presence of the polymer meshwork, 
and how the polymer meshwork affects the rate of diffusion.

%%%%%%%%%%
\subsubsection{Insertion free energy is a balance of volume and surface effects}
%%%%%%%%%%

The insertion free energy $\Delta F(r,z)$ defines the energy penalty upon moving a colloid from plain solvent into the polymer meshwork.
A positive $\Delta F$ thus implies that the brush repels the particle, and vice versa.

For colloids that are significantly smaller than the size of the pore, the insertion free energy is determined entirely by the local polymer concentration (i.e., wall effects can be neglected), 
and made up of two distinct contributions, one osmotic and the other interfacial.

\begin{eqnarray}
    \Delta F (r,z)= \Delta F_{\text{osm}}(r,z) + \Delta F_{\text{int}}(r,z)
    \\
    \Delta F_{\text{osm}}(r,z) = \int_{V} \Pi(r,z) dV
    \\
    \Delta F_{\text{int}}(r,z) = \oint_{S} \gamma (r,z) dS
    \label{eq:Delta_F}
\end{eqnarray}

\noindent The coordinates $(r,z)$ here refer to the center of the colloid, whilst the insertion free energy is obtained by integrating over the colloid volume and surface, respectively.

The osmotic contribution, $\Delta F_{\text{osm}}(r,z)$, is proportional to the colloid volume 
and accounts for the work performed against excess osmotic pressure upon insertion of the particle into the brush. 
The local osmotic pressure is calculated from the local polymer concentration using a Flory mean field approach 

$$
\Pi(r,z)=  \phi(r,z)\frac{\partial f\{\phi(r,z)\}}{\partial \phi(r,z)} - f\{\phi(r,z)\}= 
$$
\begin{equation}
	k_{\text{B}}T[-\ln(1-\phi(r,z)) - \phi(r,z) -\chi_{\text{PS}}\phi^2(r,z)]
\end{equation}

\noindent where
$$
f\{\phi(r,z)\}=(1-\phi(r,z))\ln(1-\phi(r,z)) +\chi_{\text{PS}}\phi(r,z)(1-\phi(r,z))
$$

\noindent is the mean-field Flory expression for the interaction free energy per unit volume of the polymer solution of concentration (volume fraction) $\phi(r,z)$.
As long as the osmotic pressure inside the brush is positive, $\Delta F_{\text{osm}}$ is positive as well and dominates for sufficiently large colloids. 

The interfacial contribution, $\Delta F_{\text{int}}(r,z)$, is proportional to the colloid surface, 
with the surface tension $\gamma (r,z)$ approximated as

\begin{eqnarray}
    \gamma (r,z)= \frac{1}{6}(\chi_{\text{ads}} - \chi_{\text{crit}})\phi^{\ast}(r,z)
    \\
    \chi_{\text{ads}} = \chi_{\text{PC}} - \chi_{\text{PS}}(1-\phi^{\ast})
    \\
    \phi^{\ast}(r,z)= (b_{0} + b_{1}\chi_{\text{PC}})\phi(r,z)
\end{eqnarray}

\noindent Here $\gamma$ is a free energy change upon replacement of a contact of the unit surface area of the colloid with plain solvent by a contact with a polymer solution of concentration $\phi(r,z)$.
Coefficients $b_0$ and $b_1$ are adjustable parameters to account for depletion/accumulation of polymer in the proximity of the colloid surface, 
thus adjusting the local polymer concentration in a plain brush to the effective concentration $\phi^{\ast}(r,z)$.

Depending on the relative strengths of polymer-colloid ($\chi_{\text{PC}}$) and polymer-solvent ($\chi_{\text{PS}}$) interactions, 
the sign of $\gamma\sim(\chi_{\text{ads}}-\chi_{\text{crit}})\phi^{\ast}$ may be either positive or negative.

SF-SCF results in a more accurate calculation of the insertion free energy with account of the actual perturbation of the brush structure by the inserted colloid is possible with SF-SCF simulations, but only for cylindrical colloids positioned along the pore axis ($r=0$).

As the maps of the polymer volume fraction $\phi(r,z)$ in an unperturbed brush (Figure \ref{fig:phi_hm_grid}) were calculated using a lattice based method, a special discretization scheme was employed for the integration of volumes and surfaces for cylindrical particles in
%UPDATE MANUALLY%%%%%%%%%%%%%%%%%%%%%%%%%%%%%%%%%%%%%%%%%%%%%%%%%%%%%%%%%%%%%%%%%%%%%%%%%%%%%%%%%%%%%%%%%%%%%%%%%%%%%%%%
Section 3 of SI, Eqs.~(18,19).
%%%%%%%%%%%%%%%%%%%%%%%%%%%%%%%%%%%%%%%%%%%%%%%%%%%%%%%%%%%%%%%%%%%%%%%%%%%%%%%%%%%%%%%%%%%%%%%%%%%%%%%%%%%%%%%%%%%%%%%%
Coefficients $b_0$ and $b_1$ were tuned to fit approximate analytical scheme to cylindrical particles with SF-SCF results.
%UPDATEMANUALLY%%%%%%%%%%%%%%%%%%%%%%%%%%%%%%%%%%%%%%%%%%%%%%%%%%%%%%%%%%%%%%%%%%%%%%%%%%%%%%%%%%%%%%%%%%%%%%%%%%%%%%%%
The details on the fitting of $b_0$ and $b_1$ coefficients are disclosed in Section 4 of SI.
%%%%%%%%%%%%%%%%%%%%%%%%%%%%%%%%%%%%%%%%%%%%%%%%%%%%%%%%%%%%%%%%%%%%%%%%%%%%%%%%%%%%%%%%%%%%%%%%%%%%%%%%%%%%%%%%%%%%%%%%

In the next step using the $b_0$ and $b_1$ coefficients we generalize an approximate analytical scheme to arbitrary placed spherical particles, with yet another discretization scheme for the integration of volumes and surfaces for for spherical particles.
%UPDATE MANUALLY%%%%%%%%%%%%%%%%%%%%%%%%%%%%%%%%%%%%%%%%%%%%%%%%%%%%%%%%%%%%%%%%%%%%%%%%%%%%%%%%%%%%%%%%%%%%%%%%%%%%%%%%
Further details is in Section 5 of SI.
%%%%%%%%%%%%%%%%%%%%%%%%%%%%%%%%%%%%%%%%%%%%%%%%%%%%%%%%%%%%%%%%%%%%%%%%%%%%%%%%%%%%%%%%%%%%%%%%%%%%%%%%%%%%%%%%%%%%%%%%

In what follows, we apply an approximate analytical scheme for calculating the insertion free energy $\Delta F(r,z)$ as $\Delta F\{\phi(r,z)\}$, 
where $\phi(r,z)$ is the polymer density distribution in a colloid-free brush. 
The colloid is thus considered within this analytical approach as a 'probe' which does not perturb the global concentration distribution $\phi(r,z)$ in the brush. 
The advantage of this scheme is that it enables evaluating the insertion free energy at any position of the colloid in the brush.

Comparison of the approximate analytical approach for cylindrical colloid with the SF-SCF simulations for colloids on the pore axis demonstrated good quantitative agreement (see Section 4 of SI, Figure~6), 
thus justifying the use of the more versatile analytical approach.

\begin{figure}
    \centering
    \includegraphics[width = 0.7\linewidth]{fig/free_energy_hm.png}
    %\includegraphics[scale = 1.0]{fig/DeltaF_map.png}
    \caption{
    Maps of the insertion free energy $\Delta F(r,z)$ for a polymer brush in a cylindrical pore for a range of polymer-colloid interaction strengths. 
    The polymer-colloid interaction strength is quantified by the Flory-Huggins parameter $\chi_{\text{PC}}$ ranging from -0.50 (least attractive) to -1.25 (most attractive), as indicated. 
    Insertion free energies are mapped in cylindrical coordinates (as in Figure \ref{fig:phi_hm_grid}), and colour coded as indicated. 
    For illustrative purposes the colormaps are mirrored along the $z$ axis, and the membrane is indicated in green. 
    Conditions: pore and polymer brush as in Figure \ref{fig:phi_hm_grid}, $\chi_{\text{PS}}=0.5$, $d=8$.
    }
    \label{fig:DeltaF_map}
\end{figure}

Figure \ref{fig:DeltaF_map} illustrates how colloids may be either repelled or attracted by the polymer meshwork, 
depending on the balance of osmotic and interfacial contributions to $\Delta F(r,z)$. 
Since both $\Delta F_{\text{osm}}(r,z)$ and $\Delta F_{\text{int}}(r,z)$ depend on the on local polymer concentration $\phi(r,z)$, 
the net insertion free energy $\Delta F(r,z)$ is position-dependent, and may exhibit quite large spatial variations. 
For example, the brush shown in Figure \ref{fig:DeltaF_map} switches from attraction at the caps to repulsion inside the pore at $\chi_{\text{PC}}=-0.75$.


%%%%%%%%%%
\subsubsection{Diffusivity is determined by the ratio of polymer mesh size and colloid size}
%%%%%%%%%%

The presence of a semi-dilute polymer meshwork also slows the rate of colloid movement compared to plain solution, leading to a position-dependent diffusion coefficient $D\{\phi(r,z)\}$ \cite{Laktionov2023}. 
We capture this effect through the scaling relation

\begin{equation}
    D\{\phi(r,z)\} = \frac{D_0}{1+d^2/\xi^{2}\{\phi(r,z)\}}
    \label{eq:Rubinstein}
\end{equation}

\noindent where the correlation length (or 'mesh size') $\xi$ is controlled by the local polymer concentration $\phi(r,z)$. 
As follows from Eq. (\ref{eq:Rubinstein}), particles smaller than the mesh size diffuse virtually unhindered ($D\{\phi(r,z)\}\approx D_0$ for $d\leq \xi$). 
In contrast, large colloids are significantly slowed down by the polymer medium compared to plain solvent ($D\cong D_0 (\xi/d)^2\ll D_0$ for $d\gg \xi$). 
We approximate the dependence of the mesh size on polymer concentration by the scaling dependence valid close to ideal solvent conditions, $\xi\cong \phi^{-1}$. 

%Do we need a figure here that shows illustrative map of D/D0? 

\subsubsection{Linking Local Resistivity to Global Transport}

We develop an analytical method to estimate the resistance of a cylindrical pore with a grafted polymer brush to the diffusion of colloidal particles. 
This approach builds upon the classic solution for diffusion through an empty pore and incorporates the effects of the polymer brush by modifying the local diffusion coefficient and introducing a free energy landscape.

In an empty pore, the steady-state concentration profile of diffusing particles features iso-concentration lines that form oblate spheroids, with the pore acting as a focal circle.
The flux density is given by $\mathbf{j} = -D_0 \nabla c$, where $D_0$ is the constant diffusion coefficient.
When a polymer brush is grafted inside the pore, the local diffusion coefficient $D$ becomes position-dependent, depending on the local polymer concentration and accounting for slower diffusion through a semidilute polymer mesh~\cite{Cai2011}.
Particles also experience a mean force, with an insertion free energy $\Delta F$ playing the role of a potential. To account for these effects, we introduce a scalar potential function $\psi = c\exp(\Delta F / k_B T)$ in the form of a modified Boltzmann distribution, which satisfies $\mathbf{j} = -D \nabla \psi$ and maintains a structure similar to the empty pore solution.
As we noted in our previous paper, for the solution in the form of a modified Boltzmann distribution, the product $D \exp\left( \frac{-\Delta F}{k_B T} \right) \equiv \rho^{-1}$ has the meaning of local conductivity and encapsulates the local effects of the polymer meshwork on diffusivity and insertion free energy~\cite{Laktionov2023}.

To simplify the analysis, we utilize an orthogonal curvilinear coordinate system aligned with the iso-surfaces of $\psi$ and stream surfaces of $\bm{j}$.
Inside the pore, we employ cylindrical coordinates, while outside the pore, we approximate the iso-surfaces using a variant of oblate spheroidal coordinates, as seen in Figure~\ref{fig:R_map}B. The chosen coordinate system has axial, radial, and azimuthal axes.

By integrating the local conductivity $\rho^{-1}$ over the radial and azimuthal coordinates at each axial position, we calculate the inverse resistance per unit length $R_z^{-1}$.
The total resistance $R$ is then obtained by integrating $R_z$ along the axial coordinate $z$:
\begin{eqnarray}
    R_z^{-1} = \int_{0}^{r_{\text{pore}}}\int_{0}^{2\pi}\rho^{-1} \, \tilde{h} \, d\theta \, dr\\
    R_{\text{pore}} = \int_{-L/2}^{+L/2} R_z \, dz\\
    R_{\text{conv}} = \int_{-\infty}^{-L/2} R_z \, dz + \int_{+L/2}^{+\infty} R_z \, dz\\
    R = R_{\text{pore}} + R_{\text{conv}}
\end{eqnarray}
where $\tilde{h}$ depends on the coordinate system metric. Here, $R_{\textrm{pore}}$ is the resistance of the pore channel, and $R_{\textrm{conv}}$ is the resistance due to convergent flow in the exterior regions.

% The resistances of the channel $R_{\text{pore}}$ and the pore's exterior $R_{\text{conv}}$ can be translated to respective dwell-times for permeating species.
% When permeation is enhanced the dwell time inside the pore lumen is shorter that time spend to reach the pore entrance.
% It was reported in \cite{Gruenwald2010} that 'docking' dwell time for mRNA transport is about an order of magnitude longer than the time spend in a nucleopore itself.
% Conversely, when the permeation is hindered ($R_{\text{pore}} > R_{\text{pore, empty}}$) the permeating species spend the most time traversing through the pore channel.

For the interior region, the integration is straightforward:
\begin{eqnarray}
    R_{\text{pore}} = 2\pi\int_{-L/2}^{+L/2}\left(\int_{0}^{r_{\text{pore}}} r \, dr \, \rho^{-1}(r, z')\right)^{-1} dz'
    \label{eq:R_pore}
\end{eqnarray}
The resistance due to the convergent flow, $R_{\text{conv}}$, in the exterior region requires integration over the surfaces of a series of oblate spheroids. 
See the details in Section~7 of the Supplementary Information.

For an empty pore without a polymer brush, this method recovers the classic expression for the total resistance (Eq.~\ref{eq:resistance}).

For the brush entirely contained within the interior of the pore -- which is well justified under poor solvent conditions -- the total resistance is found as $R = R_{\textrm{pore}} + R_{\textrm{conv,empty}}$, as the exterior is not modified by the brush.
Conversely, a brush in good or $\Theta$-solvent conditions produces caps outside the pore (Figure~\ref{fig:phi_hm_grid}), requiring a more elaborate calculation of the resistance due to modulated convergent flow, $R_{\textrm{conv}}$.

Because of the discrete nature of the numerical data, we perform the integration on a cylindrical lattice, approximating the iso-potential surfaces as disks in the interior of the pore (as in Eq.~\ref{eq:R_pore}) and half-cylinders in the exterior regions.
We introduce a correction factor to account for differences between this approximation and the actual oblate spheroidal geometry.
%UPDATE MANUALLY%
See the details in Section~7 of the Supplementary Information.

This analytical method provides a tool to analyze and compare the resistance experienced by particles during axial transport with that of an empty pore.
The $R_z$ profiles can provide insight into the pore resistance structure in a compact form, as demonstrated in Figure~\ref{fig:R_map}.


\subsection{An Attractive Polymer Filling Dramatically Enhances Colloid Fluxes Through a Mesopore}

\begin{figure}
    \centering
    %\includegraphics[scale = 0.7]{fig/R.png}
    \includegraphics[width = 0.7\linewidth]{fig/resistivity_on_z_and_hm.png}
    \caption{
    \textbf{(A)} Resistance per unit length $R_{z}$ along the axial coordinate of the curvilinear coordinate system for selected polymer-colloid interaction strengths $\chi_{\textrm{PC}} \in \{ -0.9, -1.0, -1.1, -1.2, -1.3\}$, as indicated with colored lines.
    The resistance per unit length of a plain pore without polymers $R_{z, \textrm{empty}}$ (black thick line), a pore with only the diffusion coefficient modulated ($\Delta F = 0$, dashed black line), and the location of the membrane (green background) are shown for comparison. 
    Conditions: pore and polymer brush as in Figure~\ref{fig:phi_hm_grid}, $\chi_{\textrm{PS}}=0.5$, $d=8$.
    \\
    \textbf{(B)} Special orthogonal curvilinear coordinate system aligned with the flux density $\bm{j}$ stream surfaces (radial coordinate) and level sets of the potential function $\psi$ (axial coordinate).
    Red lines correspond to constant values of the axial coordinate; gray lines are tangential to the flux density field and correspond to constant values of the radial coordinate.
    The lines define bodies of revolution along the $z$-axis; the angular coordinate is not shown.
    In the exterior of the pore, constant radial coordinates are confocal hyperboloids of revolution, constant axial coordinates are confocal oblate spheroids, and constant angular coordinates are half-planes.
    In the interior of the pore, the coordinate system is equivalent to the cylindrical coordinate system.
    \\
    \textbf{(C)} Maps of normalized resistivity $\rho D_0$ exemplifying a transition between hindered and enhanced permeability upon a subtle increase in the polymer-colloid interaction strength.
    $\chi_{\textrm{PC}}$ was varied from $-1.0$ (left) to $-1.1$ (right); resistivities are color-coded as indicated. 
    Conditions: pore and polymer brush as in Figure~\ref{fig:phi_hm_grid}, $\chi_{\textrm{PS}}=0.5$, $d=8$.
    }
    \label{fig:R_map}
\end{figure}

Figure~\ref{fig:R_map}A provides example resistance per unit length profiles along the pore axis for a few selected polymer-colloid interaction strengths $\chi_{\textrm{PC}}$ and for an empty pore. 
For comparison, we also show the case where the polymer brush has an effect only on the diffusion coefficient of the medium, as if the colloid particles have no interactions with the polymer brush but only experience increased viscosity of the medium.
The area under the curves represents the total resistance of the pore.
Several features are notable. 

Firstly, an attractive pore interior can enhance the permeability such that the diffusive fluxes through the polymer-filled pore exceed the limit of the empty pore.
This can be clearly appreciated for $\chi_{\textrm{PC}} = -1.2$ and $-1.3$, where the resistance per unit length within the membrane width ($-28 \leq z \leq 28$) is consistently lower than the resistance per unit length of the empty pore.
This result may at first appear surprising, given that the polymer medium is expected to slow down the diffusion of colloids.
However, this effect is counteracted by the attractive potential of the pore, which facilitates a higher colloid flux and thus reduces the pore resistance.

Secondly, attractive caps can further enhance permeability.
This can again be clearly appreciated for $\chi_{\textrm{PC}} = -1.2$ and $-1.3$ (and to a lesser extent for $\chi_{\textrm{PC}} = -1.1$), where the resistance per unit length outside the membrane ($-50 \lesssim z < -28$ and $28 < z \lesssim 50$) is consistently lower than the resistance per unit length for the empty pore. 
Clearly, this effect stems from the attractive potential of the caps, which in this case extend to $z \approx \pm 50$ and again facilitate higher colloid flux. 

As the colloid attraction increases further (not shown here), the resistance per unit length across the $z$ range spanning the pore interior and the caps becomes negligible, such that the convergent flow outside the polymer-filled volume dominates the resistance against colloid transport.
Without caps, the colloid flux may thus increase by up to approximately 3-fold. 
Attractive caps can increase the flux substantially further.
When reaching $z = \pm 50$, an extra 5-fold enhancement is possible, for example, and even larger caps would facilitate further enhancement.

Thirdly, enhanced permeability is not solely determined by the existence of a path with negative insertion free energy but the path with high conductivity $D\exp(\frac{-\Delta F}{k_B T})>D_0$. 
The criterion for the enhance permeability is existence of an effective channel with
$
\frac{\Delta F}{k_B T} < \ln\left( \frac{D}{D_0} \right) + \ln\left( \frac{r_{\textrm{channel}}}{r_{\textrm{pore}}} \right),
$
where $r_{\textrm{channel}}$ is the radius of the channel with negative insertion free energy.

For instance, although $\chi_{\textrm{PC}} = -1.0$ at $\chi_{\textrm{PS}} = 0.5$ ensures the presence of a negative insertion free energy channel (see Figure~\ref{fig:phi_hm_grid}), it is insufficient to overcome the reduction in diffusion due to the polymer mesh. 
When $\chi_{\textrm{PC}} = -0.9$, the insertion free energy becomes only slightly negative, and the negative insertion free energy channel narrows, with a positive insertion free energy penalty near the interior walls. 
This leads to a resistance per unit length that exceeds even the case when only the increased viscosity of the medium is considered, as shown in Figure~\ref{fig:R_map}.

Furthermore, a further decrease in polymer-colloid interaction strength (not shown here) may completely inhibit permeation, as every possible path a colloid particle could take would involve encountering a free energy barrier.

An estimate of the required surface tension coefficient $\gamma$ to overcome osmotic pressure and the reduced diffusion due to the polymer mesh is given by
\begin{equation}
\gamma < \frac{4}{\pi d^2} \left( \ln\frac{D_0}{D} + \ln\left( \frac{r_{\textrm{channel}}}{r_{\textrm{pore}}} \right) \right) - \frac{\Pi d}{6}.
\end{equation}

In conclusion, polymer brushes can readily enhance the permeability of the pore by one or more orders of magnitude.

\subsection{Polymer-Filled Mesopores Sharply Select Colloids Based on Their Attraction to the Polymer}
\begin{figure}
    \centering
    \includegraphics[width=0.9\linewidth]{fig/resistivity_on_chi_PC.png}
    \caption{
        Total pore resistance $R$, normalized by the viscosity of the medium $\eta_0$, as a function of polymer-colloid interaction strength $\chi_{\textrm{PC}}$ (thin black lines) for selected particle sizes $d \in \{2, 4, 6, \dots, 18\}$, as indicated with inline labels on the black lines, and solvent strength $\chi_{\textrm{PS}}$ as indicated above each panel.
        The blue curve separates the parameter space of enhanced permeability ($R < R_{\textrm{empty}}$, left and below) and hindered permeability ($R > R_{\textrm{empty}}$, right and above).
        Each intersection point of the blue curve with the black lines corresponds to the critical polymer-colloid interaction strength $\chi_{\textrm{PC}}^{\textrm{crit}}$ for a given particle size $d$, where the pore resistance is equal to the resistance of an empty pore ($R = R_{\textrm{empty}}$).
        \\
        Conditions: pore and polymer brush as in Figure~\ref{fig:phi_hm_grid}.
    }
    \label{fig:R_vs_chi_PC}
\end{figure}

Figure~\ref{fig:R_vs_chi_PC} illustrates how the total resistance of the pore varies with the particle's affinity to the polymer brush, characterized by $\chi_{\textrm{PC}}$.
As expected, increasing the polymer-colloid interaction strength (i.e., decreasing $\chi_{\textrm{PC}}$) results in a decrease in the pore's total resistance, since the interfacial term in the insertion free energy becomes more negative, thereby increasing the local conductivity $\rho^{-1}$.
Naturally, this effect is minor for small particles at every solvent strength, as can be seen from Figure~\ref{fig:R_vs_chi_PC} for particles up to about $d = 4$.
In contrast, for larger particles, a pore with a polymer brush can exhibit high selectivity based on the polymer-colloid interaction strength, observed as curves with steep slopes in Figure~\ref{fig:R_vs_chi_PC}.

As the insertion free energy becomes more negative with decreasing $\chi_{\textrm{PC}}$, the resistance of the pore channel $R_{\textrm{pore}}$ decreases, and the resistance due to the convergent flow in the exterior, $R_{\textrm{conv}}$, becomes the major contribution to the total pore resistance ($R \approx R_{\textrm{conv}}$).
As we have already discussed, protruding attractive polymer brushes can reduce the resistance even below that of an infinitely thin empty pore ($R < R_{\textrm{empty, conv}}$).
In such cases, the total resistance is defined by the resistance of the semi-infinite reservoir, which is not modulated by the pore.
Consequently, the black lines plateau at sufficiently negative $\chi_{\textrm{PC}}$.

For larger particles, these plateaus in Figure~\ref{fig:R_vs_chi_PC} are followed by lines with steep slopes, indicating that even a tiny change in the polymer-colloid interaction $\chi_{\textrm{PC}}$ results in a large change in permeability.
This transition from plateau to steep slope signifies a gating behavior of the pore with respect to the polymer-colloid interaction $\chi_{\textrm{PC}}$.
Moreover, the larger the particle size, the higher the value of $\chi_{\textrm{PC}}$ at which this transition from plateau to high permeation selectivity occurs.

To analyze the pore resistance in Figure~\ref{fig:R_vs_chi_PC}, we trace $\chi_{\textrm{PC}}^{\textrm{crit}}(d)$ (blue curve), which bisects the lines of total resistance into regions where $R < R_{\textrm{empty}}$ when $\chi_{\textrm{PC}} < \chi_{\textrm{PC}}^{\textrm{crit}}$, and regions with impeded permeation ($R > R_{\textrm{empty}}$).
Although the region to the right of the blue curve exhibits high selectivity with respect to the polymer-colloid interaction $\chi_{\textrm{PC}}$, it is also characterized by high pore resistance and low colloid flux.
Thus, the traced $\chi_{\textrm{PC}}^{\textrm{crit}}(d)$ effectively defines the pore's operative range.

Notably, as seen from Figure~\ref{fig:R_vs_chi_PC}, the gating with respect to $\chi_{\textrm{PC}}$ occurs at values of $\chi_{\textrm{PC}}$ that correspond to enhanced colloid transport ($R < R_{\textrm{empty}}$), which manifests as the plateaus found on the left side to the blue curves.

%%%%%%%%%%
\subsection{Polymer filled mesopores can effectively gate colloids by their size}
%%%%%%%%%%

\begin{figure}
    \centering
    %\includegraphics[scale = 0.5]{fig/R_vs_d.png}
    \includegraphics[width = 0.7\linewidth]{fig/permeability_on_d.png}
    \caption{
    Total pore resistance as a function of colloid size for selected polymer-colloid interaction strengths ($\chi_{\text{PC}}$, as indicated with coloured lines) 
    and solvent qualities ($\chi_{\text{PS}}$, as indicate above each panel). 
    The resistance of a plain pore without polymers (black thick line) is shown for comparison. 
    Conditions: pore and polymer brush as in Figure \ref{fig:phi_hm_grid}. 
    }
    \label{fig:R_vs_d}
\end{figure}

Figure \ref{fig:R_vs_d} illustrates how the total resistance varies with colloid size for a range of polymer-colloid interaction strengths and solvent qualities. 
Two distinct trends are generally observed.

At sufficiently low polymer-colloid interaction strengths, the polymer filled pore tends to be more resistant to colloid transport than the empty pore across  all colloid sizes. 
In this regime, the resistance increases gradually yet substantially with colloid size, owing to a combination of enhanced osmotic repulsion and reduced diffusivity.

As the polymer-colloid interaction strength increases ($\chi_{\text{PC}}$ decreases), permeability is enhanced compared to an empty pore for small colloids. 
Interestingly, the resistance remains approximately constant over a range of colloid size, until a critical colloid size emerges above which the resistance becomes very high, 
effectively impeding permeation. 
In this regime, the polymer filled pore thus acts as a gate that sharply selects colloids below from colloids above a certain threshold size.

It can be seen that with decreasing solvent strength, the level of attraction required for sharp size selectivity decreases. 
Moreover, the threshold for impeded permeation (at any given $\chi_{\text{PC}}$) is pushed towards larger sizes.


\subsection{High Colloid Flux Implies Crowding}

\begin{figure}
    \centering
    \includegraphics[width=0.9\linewidth]{fig/streamlines.png}
    \caption{
    Stationary solution of the Smoluchowski diffusion equation for colloidal particles diffusing in a potential field defined by the position-dependent insertion free energy $\Delta F$, with a modulated diffusion coefficient, through a cylindrical pore in an impermeable membrane.
    The concentration of colloidal particles is defined in the bulk infinitely far from the membrane as $c(z = -\infty) = c_{b}$ on one side and $c(z = +\infty) = 0$ on the other.
    The normalized colloidal particle concentration $c / c_{b}$ is presented as a colormap, where white corresponds to the absence of colloidal particles, yellow to red indicates concentrations below $c_{b}$, and violet to black indicates concentrations above $c_{b}$.
    \\
    In the presence of polymer chains, the diffusion coefficient $D$ in and near the pore decreases compared to the diffusion coefficient in the bulk $D_0$.
    Additionally, short-range polymer-colloid interactions create a positive or negative insertion free energy landscape.
    \\
    Isoconcentration surfaces are shown with contours for values from 0.99 to 0.90 and from 0.10 to 0.00 in steps of 0.01.
    Most contours are labeled.
    The flux is represented by streamlines marked with small arrows, indicating the average trajectory of a massless particle.
    \\
    Far from the pore, the solution to the equation is symmetrical; streamlines are perpendicular to isoconcentration contours, which are oblate spheroids, similar to the analytical solution for an empty pore~\cite{Brunn1984}.
    \\
    Conditions: spherical colloidal particle with diameter $d = 12$, polymer-colloid interaction parameter $\chi_{\textrm{PC}} = -1.5$ (high affinity), in a good solvent with $\chi_{\textrm{PS}} = 0.3$.
    The pore has a radius $r_{\textrm{pore}} = 26$ and membrane thickness $L = 52$.
    Brush-forming chains have a length $N = 300$ with a grafting density $\sigma = 0.02$ chains per unit area.
    }
    \label{fig:colloid_concentration}
\end{figure}

It is well-known that high concentrations of biomacromolecules that have affinity for FG motifs—such as importins, exportins, and cargo complexes—are found in the vicinity of nuclear pore complexes (NPCs)~\cite{Beck2007, Gruenwald2010, Tu2011} \todo{refs to be checked}.
Moreover, it is speculated that such high concentrations are required for effective transport through the pore [\dots].

In Figure~\ref{fig:colloid_concentration}, we present a map of the steady-state concentration of diffusing colloidal particles for a representative pore, as in Figure~\ref{fig:phi_hm_grid}, in a good solvent ($\chi_{\textrm{PS}} = 0.3$).
The particle size is $d = 12$ with a polymer-colloid interaction parameter $\chi_{\textrm{PC}} = -1.5$.
This condition corresponds to facilitated transport with a total resistance to transport about 10-fold lower compared to an empty pore.

The colloid concentration decays rapidly to the respective bulk concentrations of the semi-infinite reservoirs, $c(z = -\infty) = c_{b}$ and $c(z = +\infty) = 0$, as seen from the contour lines.
In the absence of a polymer brush, the diffusion coefficient is not modulated, and there is no insertion free energy penalty; thus, the potential function $\psi = c$, and the isoconcentration surfaces become oblate spheroids, as can be seen from Figure~\ref{fig:colloid_concentration}.
The solution is symmetric in the sense that the isoconcentration surfaces are mirrored with respect to the pore's center, and the mirrored isoconcentration values sum up to $c_{b} = 1$.
This is demonstrated in Figure~\ref{fig:colloid_concentration}, where the contour line labels are paired accordingly.

The average colloid trajectories are shown with streamlines (contours with arrows) in Figure~\ref{fig:colloid_concentration}.

The colloid concentration is up to 30 times larger near the pore entrance at $z \approx -L/2 = -26$ and about 8 times larger in the pore channel and proximal space ($|z| \lesssim 50$), which is caused by the negative insertion free energy in the space occupied by the polymer brush.
We remind that we disregard any colloid-colloid interactions, so the presented partitioning is valid only for low concentrations in the bulk.

Notably, an increase in polymer-colloid interaction (lower $\chi_{\textrm{PC}}$) increases the colloid concentration in the polymer brush but also shifts the maximum concentration toward the pore center, as the relative osmotic contribution to the insertion free energy drops.
At the same time, an increase in particle size with no change in polymer-colloid interaction $\chi_{\textrm{PC}}$ also increases the colloid concentration near the pore entrance but inhibits colloid transport because of the higher osmotic barrier in the pore interior.

A higher magnitude of the insertion free energy also brings the steady-state solution closer to equilibrium partitioning with no flux, $c = c_{b} \exp\left( \frac{-\Delta F}{k_B T} \right)$, as the drift fluxes caused by the free energy gradient dominate diffusion fluxes driven by the colloid concentration gradient.

One can imagine a case where a transported species has a sufficiently negative surface tension coefficient, such that in regions with lower polymer concentration in the protruding part of the brush, the insertion free energy is negative and dominated by the interfacial term, while there is an osmotic barrier in the middle (as in the upper right panel in Figure~\ref{fig:DeltaF_map}).
Such species would still dock but would not permeate the pore.

\todo{This could explain}
It has been reported that the cargo-imp$\beta$ complex strongly stains the nuclear envelope but does not efficiently enter the nuclear interior; however, when RanGDP was added, the cargos then efficiently exited the NPC and accumulated in the nucleus~\cite{Lowe2015}.
It was also reported that imp$\beta$ is seldom found in the pore channel until CAS is present~[\dots].


%%%%%%UPDATE_MANUALLY%%%%%%%%%%%%%%%%%%%%%%%%%%%%%%%%%%%%%%%%%%%%%%%%%%%%%%%%%%%
Presented colloid concentration map was calculated with computational fluid dynamics, see the details in Section 8 of Supplementary Information.
%%%%%%%%%%%%%%%%%%%%%%%%%%%%%%%%%%%%%%%%%%%%%%%%%%%%%%%%%%%%%%%%%%%%%%%%%%%%%%%%
%%%%%%%%%%
\section{DISCUSSION}
%%%%%%%%%%

% RR: I have not revised this section in detail.
% RR: May here add a paragraph that highlights the main finding.

%%%%%%%%%%%%%%%%%%%%%%%%%%%%%%%%%%%%%%%
%Validity of the approximations
%%%%%%%%%%%%%%%%%%%%%%%%%%%%%%%%%%%%%%%
To validate our approximate analytical solution, we have also performed full numerical solution of the Smoluchowsky diffusion equation, which is illustrated by Figure X 
(details in SI) and corresponding point are presented, together with analytical results in Figure 7. A very good quantitative agreement proves accuracy of our analytical theory
within the limit of the particle size sufficiently smaller than the pore radius. This is applicable to NPCs where facilitated transport starts from 5 nm while the
NP diameter is about 40 nm.

%%%%%%%%%%%%%%%%%%%%%%%%%%%%%%%%%%%%%%%
%Many pores problem
%%%%%%%%%%%%%%%%%%%%%%%%%%%%%%%%%%%%%%%
The question of how several pores in the same membrane interfere affecting their permeability was first posed by Rayleigh himself \cite{Strutt1878}. 
Fabrikant  proposed a quantitative theory for a negligibly thin membrane with several circular apertures of different radii and arbitrary mutual positions \cite{Fabrikant1985}. 
The resultant effect of the pore interference is an increase in the pore permeability since the Rayleigh resistance is partially shared by the neighboring pores. 
However, the effect is quite small (a few percent) whenever the distance between the pore centers is larger than their diameters by an order of magnitude or more. 
It is intuitively clear that once the resistance due to a finite pore length and due to the brush is non-negligible, the mutual interference effect 
becomes even smaller. Hence, we are not concerned with this aspect of the problem.
%RR: May here elaborate that the NPC density in the nuclear envelope is relatively low compared to the limits presented here. 
%Hence, selective permeation across NPCs should be well represented by our theory. 
%%%%%%%%%%%%%%%%%%%%%%%%%%%%%%%%%%%%%%%

% RR: May here discuss relevance of our findings to the NPC.

% RR: Should we discuss the question of high colloid concentration? This definitely is relevant to the NPC.

%%%%%%%%%%
\section{Conclusion}
%%%%%%%%%%

% RR: I have not revised this section in detail.

We have explored physical mechanisms for facilitated and selective diffusive transport of colloids through mesopores filled by polymer brushes grafted onto the inner pore wall. 
To this end, we have applied approximate analytical and numerical solutions of the Smoluchowsky diffusion equation 
in the external potential field experienced by diffusing colloids interacting with a brush that fills the interior of the pore
and protrudes outside the pore forming caps. In addition, enhanced resistance of the semidilute polymer solution, to which the brush-filled pore
interior can be locally assimilated, to the diffusing species was taken into account. 

Depending on the strength of the polymer-colloid interaction and/or solvent quality, diffusion of particles of different sizes
through the pore can be either blocked by the resistance of the brush or enhanced compared to the diffusion through the plain (polymer-free) pore. 

The maximal size of the particles for which diffusion through the pore is enhanced (gating threshold) increases with polymer-colloid attractiveness and decreasing solvent strength.

Moreover, in the case of polymer fringes protruding outside the pore and attractive polymer-particle interaction, the total resistance composed of that of the pore itself 
and convergent/divergent flow regions can be substantially lower than that of the plain (polymer-free) pore due to reduced resistance of the entrance/exit regions proximal to the pore.

Altogether, our findings shed the light onto possible mechanisms of selective transport through NPC and, at the same time, suggest a molecular design strategy for controlling selective
permeability through artificial mesoporous membranes with the eye on applications in...


\printbibliography
\end{document}

%%%%%%%%%%%%%%%%%%%%%%%%%%%%%%%%%%%%%%%%%%%%%%%%%%%%%%%%%%%%%%%%%%%%%%%%%%%%%%%%%%%%%%%%%%%%%%%%%%%%%%%%%%
%RR: In the following, I have left some pieces of text that could be inserted into the main text where desired
%%%%%%%%%%%%%%%%%%%%%%%%%%%%%%%%%%%%%%%%%%%%%%%%%%%%%%%%%%%%%%%%%%%%%%%%%%%%%%%%%%%%%%%%%%%%%%%%%%%%%%%%%%

%Under good (or \theta-) solvent conditions we may consider separately the situations with positive and negative insertion free energies. 
%Negative insertion free energies are rather exceptional under good solvent conditions. As we see below, in this case $R_{caps}\leq R_{convergent}$ and the total resistance
%is lower than that of the empty pore.
%Positive insertion free energies under good solvent conditions are more common. 
%In this case, the resistance of the pore interior is always dominant, 
%$$
%R_{tot}\approx R_{pore}
%$$
%and the accuracy in estimating the resistance contributions from the entrance/exit regions is not of a major concern. 

%In Figure \ref{fig:fe_scf_grid} the insertion free energy profiles $\Delta F(z,r=0)$ calculated by analytical scheme and by SF-SCF method 
%are presented as a function of position of a spherical particle along the pore axis.
%While the SF-SCF method provides the net free energy, the analytical scheme allows decoupling of the free energy into osmotic and surface contributions, 
%which are shown separately in Figure \ref{fig:fe_scf_grid}.
%The numerical coefficients $b_0$ and $b_1$ in eq \ref{} are chosen by the best fit, but appear to be fairly universal and independent of the particle size 
%and interaction parameters $\chi_{PS,PC}$.
%Remarkably, the fit fails when the size $d$ became comparable with the pore diameter or in the case of extreme $\chi_{ads}$ values 
%when analytical scheme is not applicable because of strong perturbation 
%of the brush structure by inserted particle, while SF-SCF method can still be safely used
%for the evaluation of the insertion free energy.

%The 2D insertion free energy $\Delta F(r,z)$ patterns have rather complex shape. However, we can trace their evolution upon changing interaction parameters
%looking at the position-dependent free energy of the particle on the pore axis, $\Delta F(z,r=0)$.
%As one can see from Figure \ref{fig:fe_scf_grid}, the insertion free energy profiles evolve upon changing the interaction parameters $\chi_{PS,PC}$ as follows:
%At $\chi_{ads}\geq \chi_{crit}$ which is the case under good or theta-solvent conditions and weak or absent polymer-particle attraction, $|\chi_{PC}|\leq 1$, the positive osmotic
%term, $\Delta F_{osm}\geq 0$ dominates in the insertion free energy, which is positive and reach maximal value in the pore center, where polymer concentration is maximal.
%Hence, polymer-particle interaction has overall repulsive character and $\Delta F(r,z)$ has the shape of the free energy barrier preventing penetration and accumulation of particles in the pore.
%By using the insertion free energy $\Delta F(r,z)$ one can calculate the equilibrium partition coefficient 
%$$
%P=\int_{0}^{r_{pore}}2\pi rdr\int_{0}^{L}dz\exp (-\Delta F(r,z)/k_BT)/\pi r^{2}_{pore}L
%$$
%is larger than unity, $P\geq 1$. Noticably the repulsive free energy profiles extends beyond the edges of the pore, because of the fringes in the polymer density distribution in swollen brush.

%A decrease in $\chi_{ads}$ triggered by a decrease in  $\chi_{PC}$ or/and an increase in $\chi_{PS}$ leads to qualitative changes in the insertion free energy 
%$\Delta F(r,z)$ patterns: At $\chi_{ads}\leq \chi_{crit}$ the particle surface becomes
%adsorbing for the polymer, $\gamma \leq 0$, that gives rise to a negative contribution $\Delta F_{surf}(r,z)$ to the insertion free energy. 
%When $\chi_{PS}$ increases (the solvent is getting worse for the polymer)
%the osmotic pressure inside the brush decreases that leads to a decrease in the 
%magnitude of $\Delta F_{osm}(r,z)$ with the concomitant shrinkage of the  protruding outside the pore parts of the brush where  $\Delta F(r,z)\neq 0$.
%As a result, the $\Delta F_{surf}(r,z)$ aquires two minima with negative values near the endtance and the exit of the pore, separated by a maximum centered in the middle of the pore
%where polymer concentration is larger and the osmotic repulsive term  $\Delta F_{osm}(r,z)$ dominates.
%Finally, at strong polymer-particle attraction $\chi_{ads} < \chi_{crit}$, the negative surface contribution $\Delta F_{surf}(r,z)\leq 0$ overperform osmotic repulsion everywhere inside the pore
%and the $\Delta F(r,z)$ aquires the shape of the potential well centered in the middle of the pore, which gives rise to preferential accumulation of particles in the pore, $P\geq 1$.