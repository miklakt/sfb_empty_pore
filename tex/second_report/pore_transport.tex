\documentclass[12pt, a4paper]{article}
\usepackage{graphicx}
\usepackage{amsmath}
\usepackage[
backend=biber,
natbib=true,
style=numeric,
sorting=none
]{biblatex}
\usepackage{xcolor}

\newcommand\todo[1]{\textcolor{red}{#1}}

\addbibresource{flow.bib}
\title{Colloid transport through polymer brush nanopores. \\ DRAFT}

\author{M. Laktionov}

\date{August 2023}
\begin{document}
\maketitle

\section{Introduction}
\dots

\section{Theoretical model behind colloid transport}
\subsection{Definition of the system. Polymer brush structure}

\begin{figure}
    \centering
    \includegraphics[scale = 1.0]{fig/pore_cartoon.png}
    \caption{
        Schematic illustration of colloid particle interaction scenario and diffusive transport through a polymer brush pore. 
        The brush (red strands) with a degree of polymerization $N$, grafting density $\sigma$ and polymer segment length $a$ is grafted to the inner surface of a cylindrical pore in an impermeable membrane. The pore has the radius $r_{pore}$ and the thickness $s$.
        The membrane (grey) separates semi-infinite solvent (blue) reservoirs such that any transport occur through the pore. 
        Transport of colloid particles (green hue) is affected by the brush in a complex manner.
        The local mobility of the particles in the solvent $D_{0}$ is defined by their size ($d$), the presence of polymer chains impedes colloid transport by decreasing local mobility.
        When a colloid placed in the brush it changes system free energy, which interprets in local insertion free energy.
        The negative landscape of local insertion free facilitates colloid transport, whilst free energy barrier can halt the transport.
        The local insertion free energy arise from the contact of colloid particle with brush, and depends on Flory interaction parameters $\chi$ (lower inset). To calculate insertion free energy polymer brush, solvent and particle coarse-grained into regular elements (upper inset) having finite volume and contact area.
    }
    \label{fig:colloid_transport}
\end{figure}

In this work we consider a system schematically shown in Figure \ref{fig:colloid_transport}. We consider a membrane with thickness $s$ perforated by a cylindrical pore with radius $r$. %with a polymer brush grafted to the inner surface.
The membrane separates two semi-infinite reservoirs and impermeable for polymer and colloid particles. 

Polymer chains with degree of polymerization $N$, segment length $\sigma$, segment volume $\sigma$ are densely grafted to the inner surface of the pore forming polymer brush.

\begin{figure}
    \centering
    \includegraphics[width = 2.5in]{fig/coordinates.png}
    \caption{
    Cylindrical coordinate system with the origin in the pore center.
    The longitudinal axis $z$ is the pore's main axis.
    The radial axis $r$ is perpendicular to axis $z$.
    As the system exhibits axial symmetry (orange dash-dotted line) with respect to axis $z$, angular coordinate is redundant and is not defined.}
    \label{fig:coordinate_system}
\end{figure}

Consider cylindrical coordinate system with the origin in the pore center (Figure \ref{fig:coordinate_system}), as the system exhibit axial symmetry with respect to pore main axis, polymer brush properties uniform along angular coordinate.
Thus, all system properties are three dimensional they are represented as two dimensional profiles $f(z,r)$.

\begin{figure}
    \centering
    \includegraphics[scale = 1.0]{fig/phi_hm_grid.png}
    \caption{
    Polymer density profiles in cylindrical coordinates depending on solvent quality. 
    Solvent quality is defined by interaction parameter $\chi_{PS}$ ranging from 0.1 to 1.1, the value of $\chi_{PS}$ is written on the top of each frame.
    Polymer density profiles $\phi(z,r)$ are presented as a colormaps, with a color code is universal for the all frames, where white color corresponds to pure solvent, yellow to magenta is low polymer concentration, blue to black corresponds to high concentration.
    To trace $\phi(z,r)$ values, the colorbar is shown on under the frames.
    The membrane body is drawn with the green color.
    Horizontal axes are longitudinal axis $z$, vertical axes are radial axis $r$.
    For illustrative reasons the colormaps are mirrored along $z$ axis, in a cylindrical coordinate system radial coordinate is always positive.
    To remind about the axial symmetry the axis is drawn as orange dash-dotted line in the last frame.
    }
    \label{fig:phi_hm_grid}
\end{figure}

\begin{figure}
    \centering
    \includegraphics[scale = 1]{fig/phi_open.png}
\end{figure}

\begin{figure}
    \centering
    \includegraphics[scale=1]{fig/phi_center.png}
    \caption{Polymer density profile $\phi(z, r=0)$ along a pore's main axis $z$ for a set of solvent quality parameters. 
    Solvent quality is defined by interaction parameter $\chi_{PS}$ ranging from 0.1 to 1.1, the value of $\chi_{PS}$ is written on the top of each curve and explained in the legend.
    The region where the $z$ coordinate corresponds to the inner space of a pore is highlighted with gray annotated rectangle. 
    }
    \label{fig:phi_center}
\end{figure}

Depending on solvent quality polymer brush swell/deswell. 
In good solvent swollen brush fills not only the pore but also protrudes outside the pore. In poor solvent, polymer brush deswell and in the case of sufficiently large pore may form open channel free of polymer [CITE].

Solvent quality is controlled by Flory-Huggins parameter $\chi_{PS}$, subscript PS denotes polymer-solvent interaction. The value $\chi_{PS}<0.5$ corresponds to good solvent, $\chi_{PS}>0.5$ corresponds to poor solvent, and $\chi_{PS}=0.5$ corresponds to $\theta-$solvent.
The swell/deswell behavior of a brush depending on solvent quality illustrated on Figure \ref{fig:phi_hm_grid}, \ref{fig:phi_center}.

In the Figure \ref{fig:phi_hm_grid} first two frames corresponds to a swollen polymer brush in a good solvent $\chi_{PS}<0.5$, and the third one to a $\theta$-solvent $\chi_{PS}=0.5$. 
For those conditions a large portion of the brush is outside the pore.
Polymer brush in a good and $\theta$-solvent has wide region with polymer concentration smoothly decays, note the wide yellow halo in the first three frames in Figure \ref{fig:phi_hm_grid}.

In a moderately poor solvent $\chi_{PS}=0.7$ the brush still protrudes outside the pore, however, the transition between the brush and solvent is more sharp, as one can see in the Figure \ref{fig:phi_hm_grid} as a yellow halo.
In the poor solvent the brush collapses inside the pore forming concave dish-like surface with a step-like change in polymer density (small correlation length).

These effects are illustrated in the Figure \ref{fig:phi_center}, which traces the evolution of polymer density profiles along the pore's main axis $z$. 
Polymer brush in a good and $\theta$-solvent exhibit slow smooth change in polymer density when moving along $z$-axis, the polymer density profiles are wide bell-like curves with wide tails.
In a poor solvent solvent polymer brush exhibit sharp transition in polymer concentration when moving along $z$-axis, with a box-like curves.


Colloid particles has a bulk concentration $c_0$ in one of the reservoirs and allowed to diffuse into the other reservoir with pure solvent only through the pore.
Colloid particle interaction with solvent and polymer are very complex, we consider approximate model, where the particle has homogenous surface with a given affinity to polymer and solvent.
When a particle come into contact with a polymer, the contact colloid-solvent (CS) and polymer-solvent (PS) are substituted to the contact polymer-colloid (PC).
The particles, approximated with a sphere with diameter $d$, the affinity to the polymer is controlled with Flory-Huggins interaction parameter $\chi_{PC}$, where subscript PC denotes polymer-colloid interaction. Without loss in generality $\chi_{CS}=0.0$, where subscript PS denotes colloid-solvent interaction.

\subsection{Colloid insertion free energy}
\subsubsection*{Analytical method}
When a colloid particle is moved from the bulk solvent into the brush, there is a change in free energy $\Delta F$, which defines insertion free energy penalty. A positive $\Delta F$ thus implies that the brush repels the colloid, and vice versa.

For a relatively small particles insertion free energy $\Delta F(z,r)$ can be defined as a sum of two contributions: osmotic $\Delta F_{osm}(z,r)$ and surface $\Delta F_{sur}(z,r)$ free energy.

To calculate osmotic and surface terms, one has to integrate polymer brush osmotic pressure profile over colloid particle volume and surface tension coefficient over colloid particle surface.

\begin{eqnarray}
    \Delta F(z,r) = \Delta F_{osm} + \Delta F_{sur}
    \\
    \Delta F_{osm} = \int_{V} \Pi dV
    \\
    \Delta F_{sur} = \int_{S} \Pi dS
\end{eqnarray}

Local osmotic pressure is calculated from local polymer density
\begin{equation}
	\Pi(z,r)=  \phi(z,r)\frac{\partial f\{\phi(z,r)\}}{\partial \phi(z,r)} - f\{\phi(z,r)\}= 
	k_BT[-\ln(1-\phi(z,r)) - \phi(z,r) -\chi_{PS}\phi^2(z,r)]
\end{equation}

The surface tension coefficient is approximated as
\begin{eqnarray}
    \gamma = \frac{1}{6}(\chi_{ads} - \chi_{crit})\phi^{\ast}
    \\
    \chi_{ads} = \chi_{PC} - \chi_{PS}(1-\phi^{\ast})
    \\
    \phi^{\ast}= (b_{0} + b_{1}\chi_{PC})\phi
\end{eqnarray}
the coefficient $\gamma$ is interpreted as an free energy excess/deficiency when a contact of colloid with the pure solvent replaced by a contact with polymer solution.

\todo{Add some physical insight on $1/6$ and $\chi_{crit}$.}

When the particle placed into the brush, it creates a region depleted from polymer, if the particle is substantially attractive the region is enriched in polymer.
Coefficients $b_0$ and $b_1$ are introduced to account proximal to colloid particle depletion/accumulation of polymer, thus corrects local polymer concentration of an empty brush to the apparent concentration $\phi^{\ast}$. 
Coefficients $b_0$ and $b_1$ are subject to fitting.

Consider a cylindrical particle with a center on $z$-axis embedded in a polymer brush. If one can neglect gradient in polymer density inside the brush on the length scale of the order of the particle size, $d\cdot \vert \nabla\phi(z,r) \vert \ll \phi(z)$, then the insertion free energy (with the reference state of the particle outside the brush) can be approximated as

\begin{equation}
	\begin{aligned}
		\Delta F(z_c)= &\Delta F_{osm}(z_c) + \Delta F_{surf}(z_c)=
		\\
		&\Pi(z_c)\cdot V + \gamma\{\phi(z_c)\}\cdot A
	\end{aligned}
	\label{F_ins_1}
\end{equation}
where $z_c$ is the coordinate of the center.

A more rigorous expression for the insertion free energy we use in numerical calculation for the cylindrical and spherical particles.
Insertion free energy for cylindrical particle particle co-axial to the pore's main axis are formulated as

\begin{align}
    \label{eq:cylinder_osm_fe}
    \Delta F_{osm}(z_c) =& 
    2 \pi \int_{z_c-d/2}^{z_c+d/2} \int_{0}^{d/2} \Pi(z,r) r dr dz
    \\
    \label{eq:cylinder_sur_fe}
    \Delta F_{sur}(z_c) = & 
    \\
    \nonumber
    \pi &\int_{0}^{d/2} \left[\gamma(z_c-d/2, r) + \gamma(z_c+d/2,r)\right] dr +
    \\
    \nonumber
    2 &\pi d \int_{z_c-d/2}^{z_c+d/2} \gamma(z,d/2) dz 
\end{align}

In the Eq.\ref{eq:cylinder_sur_fe} the first term integrates surface tension coefficient over cylindrical particle base, in the second it is integrated over the element of the cylinder.

The expressions for integration over spherical particle volume and surface are less trivial.
Consider spherical particle placed with an offset to the pore's main axis with coordinates in cylindrical coordinates $(z_c, r_c)$.
To formulate the simplest expression one has to integrate using local spherical coordinates with an origin placed in the spherical particle center with the zenith axis coaxial to $z$-axis.

Consider a point with a local spherical coordinates $(\rho, \theta_{pol}, \phi_{az})$, where $\rho, \theta_{pol}, \phi_{az}$ are radial, polar and azimuthal coordinate, respectively.
To translate it to global cylindrical coordinates one apply the next transformation 
\begin{align}
    &z=z_c + \rho \cos \theta_{pol}
    \\
    &r=\sqrt{r_c^2 + \rho^2 \sin^2 \theta_{pol} - 2 r_c \rho \sin \theta_{pol} \cos \phi_{az}}
\end{align} 

To calculate the osmotic term in the insertion free energy, one has to integrate osmotic pressure over the particle volume.

\begin{equation}
    \label{eq:sphere_osm_fe}
    \begin{aligned}
        \Delta F_{osm}(z_c, r_c) =&
        \\ 
        2 \int_{0}^{d/2} \int_{0}^{\pi} \int_{0}^{\pi} \Pi \left\{ z_c + \rho \cos \theta_{pol}, \sqrt{r_c^2 + \rho^2 \sin^2 \theta_{pol} - 2 r_c \rho \sin \theta_{pol} \cos \phi_{az}} \right\}
        \\
        \rho^2 \sin\theta_{pol} d\theta_{pol} d\phi_{az} d\rho&
    \end{aligned}
\end{equation}

To calculate the osmotic term in the insertion free energy, one has to integrate osmotic pressure over the particle volume.

\begin{equation}
    \label{eq:sphere_sur_fe}
    \begin{aligned}
    \Delta F_{sur}(z_c, r_c) =&
    \\
    \frac{d^2}{2} \int_{0}^{\pi} \int_{0}^{\pi} \gamma \left\{ z_c + \rho \cos \theta_{pol}, \sqrt{r_c^2 + \frac{d^2}{4} \sin^2 \theta_{pol} - 2 r_c \frac{d}{2} \sin \theta_{pol}} \right\} &
    \\
    \sin \theta_{pol} d\theta_{pol} d\phi_{az}&
    \end{aligned}
\end{equation}

Note that in Eq. \ref{eq:sphere_osm_fe}, \ref{eq:sphere_osm_fe}  $\theta_{pol}, \phi_{az}$ are dummy variables and will not present in the final result, so the insertion free energy for a given particle depends only on its position $(z_c, r_c)$.

To calculate this integrals we employed numerical integration, for the technical details please visit \emph{SUPPLEMENTARY INFORMATION}



\subsubsection*{Numerical method. Fitting procedure}
As follows from \todo{eq.} to construct analytical scheme one has to define coefficient $b_0$ and $b_1$.
This was done by a procedure where the result of insertion free energy from analytical method were mapped to the results of a probe particle insertion calculated with Scheutjens-Fleer self-consistent field (SF-SCF) numerical method.

Similar to the procedure the authors employed in \todo{ref}.
In SF-SCF scheme colloid particles defined as lattice cells impermeable for a polymer with some surface affinity to the polymer. 
The particles were moved from bulk solution to the center of the pore, by defining impermeable for the polymer lattice cells.
Not to break axial symmetry of the cylindrical pore, we consider cylindrical particle embedded in the polymer brush coaxial with the system's main axis.

Particle position is controlled by the center of the cylinder $z_c$, where $z=0$ corresponds to a cylindrical particle placed in the pore center.
For each particle position there certain value of total free energy $F_{SF}(z_c)$ calculated with SF-SCF scheme. These values were ground state corrected such that the system total free energy of a particle in bulk solution equal to zero $F_{SF}(z_c \gg 0) = 0$.

For fixed pore radius $r$, membrane thickness $s$, polymer grafting density $\sigma$ and polymerization degree $N$ insertion free energy along $z$ were explored for different values of interaction parameters $\chi_{PC}$, $\chi_{PS}$ and particle size $d$.
Polymer-colloid interaction parameter $\chi_{PC}$ ranged from -1.5 (attractive colloid particle) to 0.0 (inert colloid particle).
Polymer-solvent interaction parameter $\chi_{PS}$ ranged from 0.0 (good solvent) to 1.0 (poor solvent).
Cylindrical particles with equal diameter $d$ and height were explored for size ranging from 4 to 24 unit lengths.


\begin{figure}
    \centering
    \includegraphics[width = 4in]{fig/fe_scf_grid.png}
    \caption{
    Results of mapping from SF-SCF to analytical insertion free energy method for a cylindrical particle with diameter and height $d=8$ moving along the main axis of the pore with radius $r_{pore} = 26$ and membrane thickness $s=26$.
    Solvent quality is varied near $\theta$-point with $\chi_{PS} = [0.4, 0.5, 0.6]$, ordered from left to right column, respectively.
    Colloid particle affinity ranges from attractive to inert particle with $\chi_{PC} = [-1.0, -0.5, 0.0]$, ordered from first to last row, respectively.
    \\
    Horizontal axes corresponds to position of the particle's center $z_c$, vertical axes corresponds to free energy value.
    \\
    Insertion free energy calculated with SF-SCF scheme $\Delta F_{SF-SCF}$ is drawn with red squares, the results of analytical scheme are presented with solid red line for the total insertion free energy $\Delta F_{tot}$ and dashed green and blue line for the osmotic and surface term, respectively.
    The gray area marks values of $z$ that corresponds to the volume inside the pore $s\in [-13, 13]$.
    \label{fig:fe_scf_grid}
    }
\end{figure}

\begin{figure}
    \centering
    \includegraphics[width = 3in]{fig/phi_correction.png}
    \caption{
        Apparent local polymer density correction factor $\phi^{\ast}/\phi$ as a function of polymer-colloid interaction parameter $\chi_{PC}$ (particle affinity to polymer).
        The value $\chi_{PC}^{\ast} \approx -1.0$ is value when no correction needed. 
        For the less attractive particles $\chi_{PC}<\chi_{PC}^{\ast}$, apparent concentration is lower then calculated with SF-SCF scheme for the empty brush $\phi^{\ast}/\phi<1$ and vice versa
        }
    \label{fig:phi_correction}
\end{figure}

To exclude any effects caused by perturbation in polymer brush when embedding larger particles the fit were performed for the smallest particles with $d=4$ moving along the main axis of the pore with radius $r_{pore} = 26$ and membrane thickness $s=26$ for a ranging values of interaction parameters $\chi_{PC}$ and $\chi_{PS}$, the other results verified and exposed limitation of the fit.

The result of the mapping is presented on the Figure \ref{fig:fe_scf_grid} for a larger particle $d=8$ then the fit had been performed for, this verifies that the analytical method is invariant of the particle size $d$.
The fit will fail, eventually, when the size $d$ became comparable with the pore diameter or in the case of extreme $\chi_{ads}$ values.

The main result of the fit is coefficients $b_0$ and $b_1$ introduced in \todo{eq}.
The coefficients were employed to calculate surface term of the insertion free energy with a correction in proximal to the particle polymer density (apparent density) $\phi^{\ast}$.
The correction is illustrated with the Figure \ref{fig:phi_correction}.


\subsection{Theory behind permeability}
\subsubsection{Analytical solution for empty pore problem}
\todo{Here text from Leonid}
The goal of this work is to understand the transport of colloidal particles though a cylindrical pore in a membrane, the pore being decorated by a polymer brush grafted to its inner surface. For that purpose, we find the stationary diffusive flux of colloidal particles through the pore and analyze how it is affected by the parameters of the pore, the brush, and the colloid. A natural starting point is the diffusive flux through an empty pore without any brush. The earliest approach to that problem goes back to Lord Rayleigh who analyzed a potential flow though a circular aperture (pore) in a planar membrane of negligible thickness while recognizing and exploiting its electrostatic and gravitational analogies\cite{Strutt1878}.  In a standard setup, the position of the membrane coincides with the XY plane at  $z=0$, and the pore is a circle of radius a. The concentration of the diffusing species is fixed to be 0 and c far away from the membrane (at   $z\rightarrow\mp\infty$, respectively).  The equipotential surfaces are oblate spheroids and the streamlines form confocal hyperboloids of revolution\cite{Cooke1966}.
The net flux through the pore is given by


\begin{equation}
\Phi=2Dac\label{eq:flux_Ral}
\end{equation}

\noindent where $D$ is the diffusion coefficient. The fact that the flux is proportional to the linear size of the pore rather than its area was a subject of some historical discussion \cite{Cooke1966}.
Diffusion through a cylindrical pore in a membrane of finite thickness $L$ also allows an analytical solution but in this case it involves an implicit infinite series \cite{Brunn1984}. The lowest order approximation turns out to be quite accurate (with an error of less than 6 percent in the full range of the $\frac{L}{a}$ ratio) and reads:

\begin{equation}
    \Phi=\frac{2Dac}{1+\frac{2L}{\pi a}}\label{eq:flux_finlength}
\end{equation}

Eq (\ref{eq:flux_finlength}) admits a most natural interpretation in terms of the total resistance, $R=\frac{c}{\Phi}$:

\begin{equation}
R=\frac{L}{D\pi a^{2}}+\frac{1}{2Da}\label{eq:resistance}
\end{equation}

The first term can be recognized as the resistance of the cylindrical pore itself (the resistivity of the medium being $D^{-1}$ while the second term is the Rayleigh resistance of the pore of infinitesimal thickness as deduced from Eq  (\ref{eq:flux_Ral}) . The latter represents the effects of the convergent flow at the entrance of the pore and its symmetric counterpart on the exit side of the membrane, while the flow lines inside the cylindrical pore turn out to be approximately axial. The relatively small error carried by the approximate solution  (\ref{eq:resistance})  is due to deviations from flow axiality inside the pore and to the corresponding minor modification of the convergent flow at the entrance/exit as compared to the case of a membrane of negligible thickness.. 
Altogether the resistance of the setup with a membrane of finite thickness and the boundary conditions imposed at $z\rightarrow\mp\infty$ coincides with that of an equivalent cylinder of the same radius $a$ and of total length $L_{eq}=L+l_{R}$   where the additional length,  $l_{R}=\frac{\pi}{2}a$ , accounts for the Rayleigh resistance contribution. The boundary conditions of fixed concentration are now imposed at the caps of the equivalent cylinder, see the cartoon illustration in Figure \ref{fig:flow_cartoon}. In what follows, we will refer to the additional cylindrical sections outside the membrane, each of length $\frac{l_{R}}{2}=\frac{\pi}{4}a$ , as the Rayleigh cylinders.
 
 
\begin{figure}
    \centering
    \includegraphics[width=0.9\linewidth]{fig/flowcartoon.pdf}
    \caption{ (a) Cartoon representing the flow lines for a pore in a thick membrane with the boundary conditions imposed far way from the membrane (at $\pm\infty$). (b) An equivalent cylinder with the boundary conditions imposed at the caps leading to strictly axial flow lines; additional cylindrical sections (transparent) represent the Rayleigh resistance and have the length of  $\frac{l_{R}}{2}=\frac{\pi}{4}a$  each, where $a$ is the radius of the pore. The equivalent cylinder accurately approximates the total flux in the situation depicted in panel (a).}
    \label{fig:flow_cartoon}
\end{figure}


The notion of the equivalent cylinder is very helpful for estimating the effects of the brush in the interior of the pore on the diffusive flux. Interaction of the brush with the diffusing particles is described via the insertion free energy profile, which is in turn linked to the profiles of the brush concentration and of the osmotic pressure \cite{Laktionov2023}. On top of that, we introduce the position-dependent diffusion coefficient which depends on the local polymer concentration and accounts for slower diffusion through a semidilute polymer mesh \cite{Laktionov2023}.
Diffusion of colloidal particles in the presence of an effective potential is described by the Smoluchowsky equation which represents a high friction limit of the Fokker-Planck equation \cite{Risken1996}:

\begin{equation}
    \frac{\partial c(\textbf{r},t)}{\partial t}=\nabla\cdotp D(\textbf{r})\left(\nabla c(\textbf{r},t)+c(\textbf{r},t)\nabla\Delta F(\textbf{r})\right)
    \label{eq:smoluchowsky}
\end{equation}
Here $c$ is the concentration of the colloidal particles, $D$ is the local (position-dependent) diffusion coefficient, and $\Delta F$ is the position-dependent free energy of insertion which plays the role of the potential of mean force.
We assume the axial (cylindrical) symmetry of the pore. Together with the stationary conditions, this implies that all the relevant functions,i.e. $c$, $\Delta F$, and $D$ depend on the axial coordinate $z$ and the radial coordinate $r$ but not on the azimuthal angle.
The stationary flux density has two components linked to the corresponding components of the gradients of the particle concentration and the insertion free energy:


\begin{equation}
j_{z}(z,r)=-D(z,r)\left(\frac{\partial c(z,r)}{\partial z}+c(z,r)\frac{\partial\Delta F(z,r)}{\partial z}\right)\label{eq:flux_axial}
\end{equation}

\begin{equation}
j_{r}(z,r)=-D(z,r)\left(\frac{\partial c(z,r)}{\partial r}+c(z,r)\frac{\partial\Delta F(z,r)}{\partial r}\right),
\label{eq:flux_radial}
\end{equation}
\noindent where $c(z,r)$ is the stationary colloid concentration.

A general analytical solution of the stationary equation is not available to our best knowledge. Here we discuss an approximate solution which amounts to neglecting the radial component of the flux density within the pore. This is inspired by the fact that the net transport across the membrane is associated only with the axial component of the flux density, and by the notion of the equivalent cylinder with the boundary conditions imposed at its caps as introduced above. 
We seek the solution for the stationary colloid concentration in the form of a modified Boltzmann distribution, similar to the planar case explored earlier \cite{Laktionov2023}:                                                         

\begin{equation}
c(z,r)=\psi(z)e^{-\Delta F(z,r)}\label{eq:stationary_c_ansatz}
\end{equation}

For the axial flux density, we obtain:

\begin{equation}
j_{z}(z,r)=-D(z,r)\psi'(z)e^{-\Delta F(z,r)}\label{eq:flux_ansatz}
\end{equation}

\noindent Here the prime in  $\psi'(z)$ stands for the derivative with respect to $z$. Stationarity implies that the net flux over any cross-section of the pore is the same, independent of the position  $z$ :

\begin{equation}
\Phi=\int_{0}^{a}2\pi rdrj_{z}(z,r)=\psi'(z)\int_{0}^{a}2\pi rdrD(z,r)e^{-\Delta F(z,r)}=const,\label{eq:fi_const}
\end{equation}

\noindent where $a$ is the radius of the pore as introduced above. Solving for $\psi(z)$  we obtain 

\begin{equation}
\psi(z)=C-\Phi\int_{0}^{z}\left(\int_{0}^{a}2\pi rdrD(z',r)e^{-\Delta F(z',r)}\right)^{-1}dz'\label{eq:psi}
\end{equation}

\noindent Here the origin of the Z-axis,  $z=0$, is placed at one of the caps of the equivalent cylinder  of total length $L_{eq}= L+\frac{\pi}{2}a$  . Assuming the insertion free energy outside the pore is zero, $ C$  can be recognized as the colloid concentration at the boundary with $z=0$ which plays the role of the source.   We impose the zero boundary condition at the opposite cap of the equivalent cylinder (the sink) and obtain for the total resistance, $R=C/\Phi$ : 

\begin{equation}
R=\int_{0}^{L_{eq}}\left(\int_{0}^{a}2\pi rdrD(z',r)e^{-\Delta F(z',r)}\right)^{-1}dz'\label{eq:res_with_brush}
\end{equation}

In our previous paper we noted that the product $D(z',r)e^{-\Delta F(z',r)}$ has the meaning of local conductivity. Then integration over the pore cross-section gives the inverse resistance per unit length (as appropriate for resistors connected in parallel) and the integration over the axial coordinate simply adds contributions from all the slices connected in series. This simple interpretation is of course consistent with neglecting the radial component of the flux density. Naturally, if the brush is absent and the insertion free energy vanishes everywhere, Eq  (\ref{eq:res_with_brush}) reduces to Eq (\ref{eq:resistance}). 
The assumption that the brush is entirely contained in the interior of the pore is well justified under poor solvent conditions. Contrary to that, in a $\Theta$- or good solvent the brush would swell producing a fringe that resides outside the pore, see Figure (NEEDS AN ILLUSTRATION!!). In this case, the flow lines at the entrance to the pore are modified and the Rayleygh resistance may not fairly represent the corresponding contribution. 
In order to produce a reliable approximate scheme for good solvent conditions we consider separately the situations with positive and negative insertion free energies. Negative insertion free energies are rather exceptional under good solvent conditions. We propose that in this case  the resistance of the entrance/exit regions is bounded between the Rayleigh resistance (without any brush effects) and the resistance of the Rayleigh cylinder filled with the actual brush fringe, and use both these estimates in our calculations.
Positive insertion free energies are much more common. In this case, the resistance of the pore interior is always dominant, and the accuracy in estimating the resistance contributions from the entrance/exit regions is not of a major concern. Hence neglect the tentative changes in the picture of the flow lines and evaluate both the pore interior and the brush fringe contributions by applying  Eq (\ref{eq:res_with_brush}) with the insertion free energy profile defined everywhere within the equivalent cylinder.
Another computational aspect that must be addressed in the case when the brush fringe extends not just beyond the membrane but beyond the caps of the additional Ryleigh cylinders as well. Then the insertion free energy is non-zero at the source and the sink boundaries, and Eq (\ref{eq:res_with_brush}) must be modified such that the free energy of insertion $\Delta F(z,r)$ is counted from the reference state that represents the colloid free energy averaged over the different radial positions along the boundary cap of the Ryleigh cylinder. Another way out is to shift the position of the boundary cap away from membrane so that the brush fringe does not touch it. Then the reference state of the colloid in a pure solvent is restored. Our results are rather insensitive to the choice of treatment of the fringe problem for the reasons discussed above.

For spherical colloidal particles of finite size the coordinates $(z,r)$ refer to the position of its center while the insertion free energy is obtained by integrating the volume and the surface contributions (see Eqs. (?)-(SEE TEXT ON FREE ENERGY EVALUATION))  over the volume and the surface of the colloid, respectively.

The question of how several pores in the same membrane interfere affecting their permeability was first posed by Rayleigh himself \cite{Strutt1878}. Fabrikant  proposed a quantitative theory for a negligibly thin membrane with several circular apertures of different radii and arbitrary mutual positions \cite{Fabrikant1985}. The resultant effect of the pore interference is an increase in the pore permeability since the Rayleigh resistance is partially shared by the neighboring pores. However, the effect is quite small (a few percent) whenever the distance between the pore centers is larger than their diameters but an order of magnitude or more. It is intuitively clear that once the resistance due to a finite pore length and due to the brush is non-negligible, the mutual interference effect becomes even smaller. Hence, we are not concerned with this aspect of the problem.

%%%%%%%%%%%%%%%%%%%%%%%%%%%%%%%%%%%%RESULTS%%%%%%%%%%%%%%%%%%%%%%%%%%%%%%%%%%%%%%%%%%%%%%%%%%%%%%%%%%%%%%%%%%%%%%%%%%%%%
\section{Results}
\subsection{Permeability as function of size}

\begin{figure}
    \centering
    \includegraphics[width = \textwidth]{fig/permeability.pdf}
    \caption{
        Permeability coefficient $P$ for a spherical colloid particle as a function of particle size for different solvent quality and particle affinity.
        \\
        Solvent quality is defined with polymer-solvent interaction parameter $\chi_{PS}$ ranging from good to moderately poor solvent, four frames from left to right corresponds to a set of $\chi_{PS} = {0.3, 0.4, 0.5, 0.6}$.
        \\
        Colored solid lines corresponds to different values of polymer-colloid interaction parameter $\chi_{PC} = {-1.5, -1.25, -1.00, 0.00}$ from attractive to inert particle. 
        The color code explained in the legend.
        \\
        Dashed black line traces permeability of an empty pore in a membrane of a finite thickness $s$.
        Dotted black line traces permeability of an empty pore in an infinitely thin membrane.
        Both results are calculated using analytical solution of diffusion through a pore problem.
        \\
        Selected cases were calculated numerically using CFD approach, the results shown as circle markers that shares the same color as solid lines.
        \\
        \todo{remove redundant lines}
        }
        \label{fig:partition_on_d}
\end{figure}

\subsection{Permeability versus partitioning}
\todo{Permeability-partitioning plot}

\subsection{Critical values}
\begin{figure}
    \centering
    \includegraphics[width = \textwidth]{fig/chi_PC_crit_on_chi_PS.pdf}
    \caption{
        Critical value of polymer-colloid interaction parameter $\chi{PC}^{crit}$ as a function of solvent quality defined by polymer-solvent interaction parameter $\chi_{PS}$ and particle size $d$.
    }
    \label{fig:chi_pc_crit_on_chi_ps}
\end{figure}

\begin{figure}
    \centering
    \includegraphics[width = \textwidth]{fig/chi_PC_crit_on_d.pdf}
    \caption{
        Critical value of polymer-colloid interaction parameter $\chi{PC}^{crit}$ as a function of solvent quality defined by polymer-solvent interaction parameter $\chi_{PS}$ and particle size $d$.
        \\
        \todo{check non-monotonicity}
    }
    \label{fig:chi_pc_crit_on_d}
\end{figure}

%%%%%%%%%%%%%%%%%%%%%%%%%%%%%%%%%%%%%%%%SI%%%%%%%%%%%%%%%%%%%%%%%%%%%%%%%%%%%%%%%%%%%%%%%%%%%%%%%%%%%%%%%%%%%%%%%%%%%%%%

\section*{SUPPLEMENTARY INFORMATION}
\subsubsection{Numerical simulation of Smoluchowski equation}
\todo{Almost no text here yet. Mostly modified typical CFD equations}
\begin{equation}
    D = \frac{D_{0}}{1+\phi^2 d^2}
\end{equation}

\begin{equation}
    P_{theory} = \frac{2 D r_{pore}}{\pi + 2 s / r_{pore}}
\end{equation}

\begin{equation}
    P_{convergent} = \frac{2 D r_{pore}}{\pi}
\end{equation}

\begin{eqnarray}
    j_0 = c_0 P
    \\
    P_{channel} = \left[\int_{-s/2}^{s/2} \left( \int_{0}^{r_{pore}} D e^{-\Delta F / kT} r dr \right)^{-1} dz \right]^{-1}
    \\
    P = P_{channel}  + P_{convergent}
\end{eqnarray}


\begin{equation}
    P_{channel} = \left[\sum_{k=-s/2}^{s/2} \left( \sum_{i=0}^{r_{pore}} \Delta F_{[i,k]} \cdot (2i+1) \right)^{-1} \right]^{-1}
\end{equation}


Diffusion of colloid particle in the presence of potential field governed by Smoluchowski equation. The equation is closely connected to advection-diffusion and drift-diffusion equation.

\begin{equation}
    \partial_{t} c(r,z,t) = \nabla \cdot D(r,z)(\nabla c(r, z, t) +  c(r, z, t) \nabla U)
\end{equation}

Here $c$ is the concentration of the colloidal particles, $D$ is the local diffusion coefficient, and $U \equiv \Delta F$ is the position-dependent free energy of insertion which plays the role of the potential of mean force.


Consider an element of the regular grid with a constant volume $V(i,k)$ and surface area $S(i,k)$.
The change in the concentration in the element is defined by the net flux of colloid particle through the element surface which can be expressed using divergence theorem.

\begin{figure}
    \centering
    \includegraphics[width = \textwidth]{fig/regular_grid.png}
    \caption{Spatial discretization, indexing and boundary conditions}
\end{figure}

\begin{figure}
    \centering
    \includegraphics[width = \textwidth]{fig/neigboring.png}
    \caption{Stencil, grid values}
\end{figure}

\begin{equation}
    \partial_{t}c(i,k) = \nabla_{V} j = \int_{S(i,k)} j dS = \int_{V(i,k)} \nabla \cdot j dV
\end{equation}


\begin{figure}
    \centering
    \includegraphics[]{fig/element_divergence.png}
    \caption{Grid element and net flux (flux divergence) trough the borders}
\end{figure}

\begin{figure}
    \centering
    \includegraphics[]{fig/element_divergence_2.png}
\end{figure}


\begin{eqnarray}
    \lambda_{n}[i,k] = A[i,k]/V[i,k] = 1+r_[i,k]^{-2}
    \\
    \lambda_{s}[i,k] = A[i,k-1]/V[i,k] = 1-r_[i,k]^{-2}
    \\
    \lambda_{e} = \lambda_{w} = 1
    \\
    \nabla_V[i, k] j = \lambda_{n}[i,k] j_r[i,k] - \lambda_{s}[i,k] j_r[i,k-1] + \lambda_{e}[i,k] j_z[i,k] - \lambda_{w}[i,k] j_z[i,k-1]
    \\
\end{eqnarray}
\begin{eqnarray}
    Pe_{z}[i,k] = \Delta_{z} U[i,k] = \Delta F[i+1, k] - \Delta F[i, k]
    \\
    Pe_{r}[i,k] = \Delta_{r} U[i,k] = \Delta F[i, k+1] - \Delta F[i, k]
    \\
    D_{z}[i,k] = \frac{D[i+1,k] + D[i,k]}{2}
    \\
    D_{r}[i,k] = \frac{D[i,k+1] + D[i,k]}{2}
    \\
    \alpha_{z,r} = \frac{e^{Pe_{z,r}/2} - 1}{e^{Pe_{z,r}} - 1}
\end{eqnarray}
\begin{eqnarray}
    c_{e}[i,k] = c[i+1,k] \alpha_{z}[i,k] + c[i, k] (1 - \alpha_{z}[i,k])
    \\
    c_{w}[i,k] = c[i-1,k] (1-\alpha_{z}[i-1,k]) + c[i, k] \alpha_{z}[i-1,k]
    \\
    c_{n}[i,k] = c[i,k+1] \alpha_{r}[i,k] + c[i, k] (1 - \alpha_{r}[i,k])
    \\
    c_{s}[i,k] = c[i,k+1] (1-\alpha_{z}[i,k-1]) + c[i, k] \alpha_{z}[i,k-1]
\end{eqnarray}
\begin{eqnarray}
    j_{e, drift}[i,k] = D_z[i,k] \Delta_{z} U[i,k] c_e[i,k]
    \\
    j_{w, drift}[i,k] = D_z[i-1,k] \Delta_{z} U[i-1,k] c_w[i,k]
    \\
    j_{n, drift}[i,k] = D_r[i,k] \Delta_{r} U[i,k] c_n[i,k]
    \\
    j_{s, drift}[i,k] = D_r[i,k-1] \Delta_{r} U[i,k-1] c_s[i,k]
\end{eqnarray}

\begin{eqnarray}
    j_{e, diff}[i,k] = D_z[i,k]  (c[i+1,k] - c[i,k])
    \\
    j_{w, diff}[i,k] = D_z[i,k]  (c[i,k] - c[i-1,k])
    \\
    j_{n, diff}[i,k] = D_r[i,k]  (c[i,k+1] - c[i,k])
    \\
    j_{s, diff}[i,k] = D_z[i,k]  (c[i,k] - c[i,k-1])
\end{eqnarray}

\begin{eqnarray}
    j = j_{drift}+j_{diff}
\end{eqnarray}

\begin{equation}
    \nabla_{V} j = -\lambda_w j_w + \lambda_e j_e - \lambda_s j_s + \lambda_n j_n
\end{equation}





\subsection{Numerical methods. Discretization scheme.}
Polymer brush density profiles and insertion free energy profiles is a main ingredient to study the transport through the pore.  
Scheutjens-Fleer SCF numerical method was employed to study polymer brush density profiles and a probe particle insertion free energy profiles.
The method based on free energy functional minimization, the result is discrete density profile with minimum system free energy.

This method make use of discrete space coordinates, as the system exhibit axial symmetry, the space is discretized into homogeneously curved two gradient lattices \emph{i. e.} cylindrical lattice.

There are two coordinate axes: longitudinal $z$ and radial $r$ (Figure \ref{fig:main_discretization}(1,2)). 
Gradient in longitudinal direction produces planar lattice layers, gradient in radial direction produces curved coaxial cylindrical walls.

While $rz$ coordinate system looks identical to the two dimensional Cartesian coordinates on the figures, each element of the lattice is square toroid (Figure \ref{fig:main_discretization}(5)), the mean-field approximation is applied in angular direction, means properties in angular direction are uniform.
For the two dimensional Cartesian coordinates each element is square box with an infinite height.

\begin{figure}
    \centering
    \includegraphics[width=\textwidth]{fig/main_figure.pdf}
    \caption{Schematic cutaway diagram of system geometry, space discretization, boundary conditions and polymer brush morphology.
    The radial axis $r$ (1) and the axis $z$ (2) defines cylindrical coordinates with degenerate angular coordinates.
    If $z$ is held constant a flat circular plane is traced, $z=0$ traces a plane at the origin (3). 
    If $r$ is held constant a cylindrical surface is traced.
    The space is discretized using regular 2D grid (4) with element size $\delta z = \delta r = a$, where $a$ is Kuhn segment length.
    Each element of the grid has a finite volume equal to a square toroid traced by the cylindrical and circular surfaces (5).
    \\
    The studied system is a pore in an infinite solvent reservoir (6).
    The pore walls (7) are rigid body impermeable for particles and polymer brush. The inner surface of a pore is grafted with homopolymer chains (8) with grafting density $\sigma$ and degree of polymerization $N$, 
    the grafting surface (9) is pictured as red straight line on the horizontal cross-section plane.
    \\
    Spherical particles (10) with diameter $d$ diffuse from the left semi-infinite reservoir. 
    The ingress of particles shown with green arrows(11) and simulated as a source with constant concentration of particles $c_0$ placed far from the pore.
    \\
    The egress of particles shown with red arrows(12) and simulated as a perfect sink with constant concentration of particles $c=0$ placed far from the pore.
    \\
    Particles are discretized by grid elements occupancy, e.g. how much of a particle volume occupy grid element volume. The discretization of particle volume is shown as a heatmap (13) under the particle.  
    \\
    The polymer volume concentration profile ($\phi$) has axial symmetry the profile is presented as a heatmap on the vertical cross-section plane. 
    The particle stationary concentration profile normalized by the bulk concentration ($c/c_0$) is a heatmap on the horizontal cross-section plane.}
    \label{fig:main_discretization}
\end{figure}

Ideally, to extract ground state one has place a particle infinitely far from the membrane, in practice the particle has to placed far enough from the membrane.
Obviously, semi-infinite reservoirs can not be calculated with lattice method explicitly, one has to find some adequate system lattice size when any increase in lattice size does not translates to the better precision.

The geometrical features of the simulation volume is illustrated in Figure \ref{fig:sim_box_layout}. 
There are additional number of layers $l_1$ and $l_2$ in longitudinal direction to model two semi-infinite reservoirs, the number of layers selected such that in the good solvent when the brush is the most swollen one could place colloid particle without any contact with the polymer.
Number of layers in radial direction beyond the pore $h$ were chosen such that the swollen brush can not extend to the simulation volume edge.

\begin{figure}
    \centering
    \includegraphics[width = 4in]{fig/sim_box_size.png}
    \caption{
        Schematic layout of lattice size and geometrical features in $rz$-coordinates of cylindrical pore model for the SF-SCF method.
        The axes vectors are drawn in the lower left corner of the figure with vertical blue arrow for $r$ axis and horizontal black axis for $z$ which is also the axis of system symmetry (orange dash-dot line).
        The origin of cylindrical coordinate system is in the center of the pore, see the black cross with coordinates $(0,0)$.
        \\
        The pore radius is $r_{pore}$ and the membrane thickness is $s$. 
        Two semi-infinite reservoirs modelled to have $l_1$ and $l_2$ longitudinal layers, respectively.  
        Total number of lattice layers in longitudinal direction $z_l = l_1 + s + l_2$,
        number of radial layers $r_l = r_{pore} + h$.
        \\    
        In the utilized SF-SCF scheme only coaxial cylindrical colloid particle (yellow rectangle) moving along axis $z$ can be modelled to calculate insertion free energy. Particle position is controlled by the particle center coordinate $z_c$, the coordinates $(z_c, 0)$ are marked with black cross.
        \\
        Analytical scheme allows to calculate insertion free energy for a colloid particle of spherical shape (green circle) with arbitrary coordinates of the center, the coordinates $(z_c, r_c)$ marked with black cross.
        \\
        The volume in the pore marked with red tinted rectangle that corresponds to cylindrical body with cross-section area $\pi r_{pore}^2$ and total volume $\pi r_{pore}^2 s$.
        The membrane is shown as gray hatched rectangle.
        The model of colloid transport accounts for the volume that colloid particle cannot physically occupy by introducing excluded volume.  
        As a colloid particle can not come closer than $d/2$ to the membrane the excluded volume is defined by the equidistant surface and is shown as greenish area that envelops the membrane.
        Naturally, the cross-section area of the pore that supports transport shrinks to $r_{pore}^{\ast} = r_{pore} - d/2$ 
    }
    \label{fig:sim_box_layout}
\end{figure}

\subsubsection{Insertion free energy for cylindrical particle}

To build and verified analytical insertion free energy scheme cylindrical particle insertion were analyzed with SF-SCF numeric scheme.

Insertion free energy were calculated analytically for a cylindrical particle and compared with the results of SF-SCF.
Consider cylindrical particle with diameter and height $d$, to calculate osmotic term (\todo{eq}) one has to integrate osmotic pressure profile over particle volume.
To calculate the integral numerically, a particle discretized spatially into the volume elements, using cylindrical lattice constructed in SF-SCF scheme.
%The discretization were done such that the particle shares the the same volume elements as the ones used in SF-SCF scheme.
Thus, lattice element either fully occupied by the particle or not.
The volume of discretization elements schematically shown on Figure \ref{fig:cylindrical_kernel} on vertical cross-section plane.
Suppose the particle has coordinate of the center ${(0, z_c)}$, the particle is divided into $d^2/2$ elements, $d$ - horizontally and $d/2$ - vertically, the volume of each element forms a matrix $v[d \times d/2]$.
\begin{equation}
    v[k,i] = \pi(2i+1)
\end{equation}
where $v[k,i]$ is $i$-th element volume.
The sum of $v$ is equal to cylindrical particle volume.
\begin{equation}
    \sum_{i=0}^{d/2-1} \sum_{k=0}^{d-1} v[k,i] = \frac{\pi d^3}{4}
\end{equation}

To calculate osmotic term in insertion free energy one has to find the next dot product for the each position of $z$.
\begin{equation}
    \Delta F_{osm}[z_c] = \sum_{i=0}^{d/2-1} \sum_{k=0}^{d-1} v[k,i] \cdot \Pi[z_c-d/2+k, i]
\end{equation}


The surface term is calculated similar to the osmotic term, but rather than volume one found surface of the discretization element. 
Obviously, inner part of the particle has no surface, which is illustrated in the horizontal cutaway surface in Figure \ref{fig:cylindrical_kernel}.
The surface of each discretization element forms a matrix $s[d \times d/2]$, such that only outer layer has non-zero values.

\begin{eqnarray}
    s_{element}[i,k] = 
    \begin{cases}
        2 \pi i,   & \text{if}\ i=d/2-1 \\
        0,         & \text{otherwise}
    \end{cases}
    \\
    s_{base}[i,k] = 
    \begin{cases}
        2\pi(i+1), & \text{if}\ k=0 \text{ or } k=d-1 \\
        0,         & \text{otherwise}
    \end{cases}
    \\
    s[i,k] = s_{element}[i,k] + s_{base}[i,k]
\end{eqnarray}
where $s_{element}$ - element surface if the element contains \emph{element} of the cylinder,
$s_{base}$ - element surface if the element contains \emph{base} of the cylinder.

The sum of $s$ is equal to cylindrical particle surface
\begin{equation}
    \sum_{i=0}^{d/2-1} \sum_{k=0}^{d-1} s[k,i] = \frac{3 \pi d^2}{2}
\end{equation}

To calculate surface term in insertion free energy one has to find the next dot product for the each position of $z$.
\begin{equation}
    \Delta F_{sur}[z_c] = \sum_{i=0}^{d/2-1} \sum_{k=0}^{d-1} s[k,i] \cdot \gamma[z_c-d/2+k, i]
\end{equation}


% \begin{figure}
%     \centering
%     \includegraphics[]{fig/cylindrical_kernel.png}
%     \caption{
%         Cylindrical particle volume and surface discretization with cylindrical lattice.
%         \\
%         The particle is a cylindrical body (1), with a center(2) in $rz$-coordinates $(z_c,0)$, marked with a red cross. \todo{FIX COORDINATES}.
%         \\
%         The heatmap on the vertical cross-section is volume occupancy of the particle within lattice elements(3) $v[k,i]$
%         \\ 
%         The heatmap on the horizontal cross-section is surface occupancy of the particle within lattice elements(4) $s[k,i]$
%         \\
%         Vertical axis $i$ is parallel to system coordinates axis $r$. Axis $k$ is coaxial to axis $z$.
%         \\
%         The particle has height and diameter $d=16$, so it is divided in $16 \times 8$ elements in longitudinal ($k \in [0,15]$) and radial direction ($i \in [0,7]$), respectively.
%     }
%     \label{fig:cylindrical_kernel}
% \end{figure}

\subsubsection{Insertion free energy for spherical particle}
To calculate osmotic term of the particle insertion free energy osmotic pressure is summed over the all elements occupied by the particle weighted by the occupancy. 

\todo{It is a convolution operation ($\ast$) only in some sence, since v[l, m] and s[l,m] will change with the particle offset, they are rather 3d matrices $m[i,i,k]$.
This would be a true convolution for Cartesian lattice.}
\begin{align}
    &F_{osm}[r_c, z_c] = \sum_{i=z_c-d/2}^{z_c+d/2-1} \sum_{k=\max\{r_c-d/2, 0\}}^{r_c+d/2-1} \Pi[k,i] \cdot v[k,i] = \Pi \ast v
    \\
    &F_{sur}[r_c, z_c] = \sum_{i=z_c-d/2}^{z_c+d/2-1} \sum_{k=\max\{r_c-d/2, 0\}}^{r_c+d/2-1} \gamma[k,i] \cdot s[k,i] = \gamma \ast s
\end{align}
where $r_c, z_c$ is the particle's center position, square brackets denotes discrete nature of the function.
The summation limit ${k=\max\{r_c-d/2, 0\}}$ depends on the offset from $z$-axis.


The sum of $v$ and $s$ is equal to spherical particle volume and surface, respectively.
\begin{eqnarray}
    \sum_{i,k} v[k,i] = \frac{\pi d^{3}}{6}
    \\
    \sum_{i,k} s[k,i] = \pi d^{2}
\end{eqnarray}

\begin{figure}
    \centering
    \includegraphics[scale = 0.7]{fig/spherical_kernel.png}
    \caption{
        Spherical particle volume and surface discretization with cylindrical lattice.
        The particle has the center offset from the cylindrical lattice $z$ axis used in discretization.
        \\
        To construct particle volume element matrix $v[k,i]$, for each lattice element (yellow opaque square toroid) one determines volume of the lattice element occupied by the particle (red body, marked with the word 'volume').
        \\
        To construct particle surface element matrix $s[k,i]$, for each lattice element (yellow opaque square toroid) one determines surface of the lattice element occupied by the particle (green surfaces, marked with the word 'surface').
        \\
        As an example, volume element matrix $v[k,i]$ is shown as a blue-green-yellow colormap, 
        where blue color means the lattice element is not occupied by the particle.
        \\
        For this figure a spherical particle with $d=16$ were selected, the center of the particle has a radial offset $r_c=8$ and arbitrary $z_c$ coordinate,
        $i \in [0,15]$, $k \in [0,15]$.
        \\
        \todo{RENAME integration element to discretization element?}
    }
    \label{fig:spherical_kernel}
\end{figure}


\printbibliography


\end{document}