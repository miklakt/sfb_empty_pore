\documentclass[10pt, a4paper]{article}
\usepackage{geometry}
\geometry{
    a4paper,
    left= 15mm,
    right = 15mm,
    top=20mm,
    bottom = 25mm,
    }
\usepackage{caption}
\captionsetup[figure]{
    font=small, 
    labelfont=bf
    }

\usepackage{graphicx}
\usepackage{amsmath, amssymb, amsfonts, mathtools}
\usepackage[
    backend=biber,
    natbib=true,
    style=numeric,
    sorting=none
    ]{biblatex}
\usepackage{xcolor}
\usepackage{bm}
\usepackage{multicol}
\usepackage{ltablex}
\usepackage{wrapfig}
\usepackage{float}
\usepackage[utf8]{inputenc}

\usepackage{titlesec}
\titleformat{\section}{\normalfont\fontsize{12pt}{12pt}\selectfont\bfseries}{\thesection}{0.5em}{}
\titlespacing*{\section}{0pt}{0ex}{0.0ex}

\newcommand\todo[1]{\textcolor{red}{#1}}
\newcommand\scalemath[2]{\scalebox{#1}{\mbox{\ensuremath{\displaystyle #2}}}}
\newcommand{\sign}{\text{sign}}

% Add S prefix to figures, equations and tables
\renewcommand{\thefigure}{S\arabic{figure}}
\renewcommand{\theequation}{S\arabic{equation}}
\renewcommand{\thetable}{S\arabic{table}}
\makeatletter
\renewcommand{\fnum@table}{\textbf{Table~\thetable}}
\makeatother

\addbibresource{biblio.bib}
\title{SUPPLEMENTARY INFORMATION}

\author{}
\date{}

\begin{document}
\maketitle

\textbf{This file includes:} Supplementary Methods 1 to 8, and Supplementary References.

\pagebreak
%%%%%%%%%%%%%%%%%%%%%%%%%%%%%%%%%%%%%%%%%%%%%%%%%%%%%%%%%%%%%%%%%%%%%%%%%%%%%
\section{Supplementary Method. Estimates of nucleopore geometry and polymer filling}
\begin{multicols}{2}

Ralf to add a short description of what is known about the nucleopore geometry, and the length and number of FG domains in the nucleopore, and a comparison with the assumptions in our model. 

\end{multicols}

%%%%%%%%%%%%%%%%%%%%%%%%%%%%%%%%%%%%%%%%%%%%%%%%%%%%%%%%%%%%%%%%%%%%%%%%%%%%%
\section{Supplementary Method. Overview of the computational process to simulate colloid transport through polymer-filled mesopores}

Figure~\ref{fig:paper_roadmap} outlines all major steps in the computational process. Technical details for individual steps are described in subsequent sections.

\begin{figure}[H]
    \centering
    \includegraphics[width = 0.4\linewidth]{fig/roadmap.png}
    \caption{Outline of the processing steps involved in the simulation of colloid transport through polymer-filled mesopores.}
    \label{fig:paper_roadmap}
\end{figure}

%%%%%%%%%%%%%%%%%%%%%%%%%%%%%%%%%%%%%%%%%%%%%%%%%%%%%%%%%%%%%%%%%%%%%%%%%%%%%%%%
\section{Supplementary Method. Computing polymer density maps with the Scheutjens-Fleer self-consistent field (SF-SCF) numerical method}
\begin{multicols}{2}

%COMMENT RR: This section looks very detailed, and some aspects, e.g. with regard to Eq. (S2) seem to be standard. Can these be omitted, stasting instead an authoritative reference?

Predicting the spatial distribution of polymers necessitated numerical calculations owing to the complex geometry of the mesopore. 
The SF-SCF numerical method was deployed to this end, with pore, polymer and colloid parameters as defined in Figure~\ref{fig:sf-scf_scheme}.
The method is based on the minimization of the excess Helmholtz energy with the constraint that all volume concentrations sum up to unity (incompressibility condition).
Space is discretized into a regular lattice, and the temporal average of the polymer volume fraction is calculated at each lattice site, such that the corresponding equilibrium distribution of polymers minimizes the overall system free energy.

As our problem exhibits axial symmetry, space was discretized into a cylindrical lattice with a degenerate angular direction,
implemented as a homogeneously curved two-gradient lattice defined by longitudinal $z$ and radial $r$ coordinates (Figure~\ref{fig:sf-scf_scheme}).
This $rz$ coordinate system is visualized as a two-dimensional Cartesian coordinate system (Figure~\ref{fig:sf-scf_scheme}, right); however, each element of the lattice represents a square toroid (of volume $2 \pi r a^3$) instead of a square (of area $a^2$).
The mean-field approximation is applied in the angular direction, meaning properties in the angular direction are uniform.

The membrane and any colloids were coarse-grained.
The membrane was modeled as a toroid with a rectangular cross-section of physical height $L$, a physical inner radius $r_{\text{p}}$, and an outer radius large enough to be effectively infinite with respect to the polymer distribution (Figure~\ref{fig:sf-scf_scheme}, green).
The colloid was modeled as a cylinder with height and diameter $d$ (Figure~\ref{fig:sf-scf_scheme}, yellow).
Membrane and colloid lattice elements were modeled as impermeable to the solvent and the polymers, illustrated as crossed-out cells in matching colors in Figure~\ref{fig:sf-scf_scheme} (right).

\begin{figure}[H]
    \centering
    \includegraphics[width = 0.95\linewidth]{fig/sf-scf_scheme.png}
    \caption{
    Left - Three-dimensional schematic of the lattice and geometrical features of the cylindrical pore model in the SF-SCF method.
    Right - Key with color code of the modeled objects (top) and their representation on a discrete two-gradient lattice (bottom).
    }
    \label{fig:sf-scf_scheme}
\end{figure}

Each polymer chain was represented as a freely jointed chain (FJC) with $N$ segments of length $a$, modelled as a step-weighted random walk on the lattice (Figure~\ref{fig:sf-scf_scheme}, right, red circles with black connecting lines).
The weights of each step direction were set according to the boundary conditions and the lattice curvature.
Steps into impermeable lattice elements had zero weight.
Other steps were weighted according to the change in volume per lattice element, with steps towards increasing $r$ consequently being favoured, and steps towards decreasing $r$ being disfavoured, compared to steps along $z$.
The resulting local polymer concentration $\phi$ is a weighted sum of all possible paths the chain can take, Eq.~(\ref{eq:sum_to_phi}).

The excess Helmholtz energy was minimized through a Lagrangian with multipliers $\alpha(r,z)$:
\begin{equation}
    \label{eq:fe_lagrangian}
    \begin{aligned}
        &F[\bm{u}, \boldsymbol{\phi}, \boldsymbol{\alpha}] =\\
        &= F_{\text{ent}}[\bm{u}] - \sum\limits_{r,z} \sum\limits_X u_X(r, z) \phi_X(r, z) + \\
        &+ F_{\text{int}} [\boldsymbol{\phi}] 
        + \sum\limits_{r,z} \alpha(r, z) \left( \sum\limits_X \phi_X(r, z) - 1 \right),
    \end{aligned} 
\end{equation}
where $\phi_X(r,z)$ is the local volume concentration function of segment type $X$ (polymer, colloid, or solvent), $u_X(r, z)$ is the potential field of segment type $X$, the functional $F_{\text{ent}}[\bm{u}]$ is the mixing entropy term, and the functional $F_{\text{int}} [\boldsymbol{\phi}]$ is the interaction term of the free energy.

The condition for the minimum of the functional is a system of three variational equations:
\begin{equation}
    \label{eq:energy_min_system}
    \begin{cases}
        \frac{\partial F}{\partial \boldsymbol{\alpha}} = 0 \\
        \frac{\partial F}{\partial \boldsymbol{\phi}} = 0 \\
        \frac{\partial F}{\partial \bm{u}} = 0 \\
    \end{cases}
\end{equation}
The first condition in Eq.~(\ref{eq:energy_min_system}) ensures system incompressibility.

The second condition in Eq.~(\ref{eq:energy_min_system}) results in the segment potential field equation for a regular solution:
\begin{equation}
    \label{eq:u-phi}
    u_A(r, z) =\sum\limits_{B} \chi_{A,B} \left(\phi_B(r,z) - \phi_B^b \right) + \alpha(r, z),
\end{equation}
where $\chi_{A,B}$ is the Flory interaction parameter between segments $A$ and $B$, and $\phi_B^b$ is the volume fraction of $B$ in the bulk (equal to 1 for the solvent and zero otherwise).

The third condition in Eq.~(\ref{eq:energy_min_system}) (minimization with respect to potentials) links the chain partition function with the local polymer concentration $\phi$ in a diffusion-like equation (Eq.~(\ref{eq:propagation})).
Any subchain of the FJC can be considered a Markov process starting at some segment $s_i$ at coordinates $r_i, z_i$ that goes through intermediate steps to segment $s_k$ at coordinates $r_k, z_k$ (Figure~\ref{fig:sf-scf_scheme}, bottom right).
Such a process has a statistical weight $G(\{r_k, z_k\}, s_k | \{r_i, z_i\}, s_i)$.
All the Markov processes that start with segment $s_i$ and end with segment $s_k$ at fixed coordinates $\{r, z\}$ are found as the sum over all possible starting coordinates:
\begin{equation}
    \label{eq:sum_to_phi}
    G(\{r, z\}, s_k | s_i) = \sum_{r^{\prime}, z^{\prime}} G(\{r, z\}, s_i | \{r^{\prime}, z^{\prime}\}, s_i)
\end{equation}


The statistical weight of all possible processes that start from segment $s_i$ and end with segment $s_k$ is the sum over all possible coordinates:
\begin{equation}
    G(s_k | s_i) = \sum_{r, z} G(\{r, z\}, s_k | s_i)
\end{equation}

When $s_i=1$ and $s_k=N$, the result contains the statistical weight of all possible and allowed conformations of the chain and is the single-chain partition function $G(N|1)$.

Let $G(r, z) = G(\{r, z\}, 1|1)$ be the initial condition of the Markov process, which contains just one segment (starts and ends at segment $1$).
The segment potential $\bm{u}$ acts on this segment; thus, Boltzmann statistical weights are applied:
\begin{equation}
    G(r, z) = \exp(-u(r,z))
\end{equation}

The volume density distribution of segment $s_i$ at coordinates $r, z$ is found from the composition law:
\begin{equation}
    \label{eq:propagation}
    \begin{aligned}
        &\phi(\{r, z\}, s_i) = \\
        &\frac{2 \pi r_{\text{p}}^{0} \sigma N}{G(N|1)}
        \frac{G(\{r, z\}, s_n | 1) G(\{r, z\}, (N - s_n + 1) | 1)}{G(r, z)}
    \end{aligned}
\end{equation}
where $G(\{r, z\}, s_n | 1)$ and $G(\{r, z\}, (N - s_n + 1) | 1)$ are forward and backward propagators, respectively; $\sigma$ is the grafting density.

Finally, the volume concentration at coordinates $r, z$ is found as the sum over all chain segments:
\begin{equation}
    \phi(r, z) = \sum_{i}^{N} \phi(\{r, z\}, s_i)
\end{equation}

The numerical algorithm solves the Scheutjens-Fleer system of nonlinear equations such that the segment potential $\bm{u}$ is consistent with the volume concentration $\boldsymbol{\phi}$.
The relationship between the segment potential and the volume concentration is defined in Eq.~(\ref{eq:u-phi}).

The SF-SCF scheme can be summarized as:
\begin{equation}
    \boldsymbol{u}[\boldsymbol{\phi}] \xleftrightarrow[]{\sum_{X} \phi_X = 1} \boldsymbol{\phi}[\boldsymbol{u}]
\end{equation}

For the calculations, we used the package \emph{SFbox} developed in the University of Wageningen.
%COMMENT RR: A reference to the literature for details on this software is needed here.
The package contains several Newton/Quasi-Newton optimization routines to perform the minimization of the functional Eq.~(\ref{eq:fe_lagrangian}).
Each iteration returns a new improved approximation to the segment potential $\bm{u}$ and updates the volume concentrations $\bm{\phi}$.
The routine was looped until the desired accuracy was reached.

\end{multicols}


%%%%%%%%%%%%%%%%%%%%%%%%%%%%%%%%%%%%%%%%%%%%%%%%%%%%%%%%%%%%%%%%%%%%%%%%%%%%%
\section{Supplementary Method. Extracting surface and volume contributions to the colloid insertion free energy from SF-SCF data}

\begin{multicols}{2}

The two-gradient SF-SCF numerical model limits the cylindrical colloid to be positioned coaxially along the $z$-axis.
The insertion free energy pofile $\Delta F_{\text{cyl}}(z)$ is composed of the volume contribution $\Delta F_{\text{cyl}}^{\text{osm}}$ and the surface contribution $\Delta F_{\text{cyl}}^{\text{sur}}$.
We here define how these two contributions are extracted from the SC-SCF data.
In their continuous form, the contributions are:
\begin{equation}
    \Delta F_{\text{cyl}}^{\text{osm}}(z_{\text{c}}) = 2 \pi \int_{z_{\text{c}} - d/2}^{z_{\text{c}} + d/2} \int_{0}^{d/2} \Pi(r,z) \, r \, dr \, dz
\end{equation}
and
\begin{equation}\label{eq:continuous_surf_int}
    \begin{aligned}
        \Delta F_{\text{cyl}}^{\text{sur}}(z_{\text{c}}) = 2 \pi d \int_{z_{\text{c}} - d/2}^{z_{\text{c}} + d/2} \gamma(d/2,z) \, dz +\\
        + \pi \int_{0}^{d/2} \left[ \gamma(z_{\text{c}} - d/2, r) + \gamma(z_{\text{c}} + d/2,r) \right] dr,
    \end{aligned}
\end{equation}
where the first term in Eq.~(\ref{eq:continuous_surf_int}) integrates over the lateral surface, and the second term integrates over the top and bottom faces, of the cylinder.

To represent the lattice discretization in SF-SCF, we follow the convention for indices: $i \le 0$ iterates in the direction of the $r$-axis, with indices starting from $i=0$ at the $z$-axis, and $k$ iterates in the direction of the $z$-axis.
As illustrated in Figure~\ref{fig:sf-scf_scheme}, the cylindrical colloid occupies $d$ lattice elements along the $z$-axis (from $z_{\text{c}} - d/2$ to $z_{\text{c}} + d/2$) and $d/2$ elements along the $r$-axis (from $r = 0$ to $r = d/2$), with $d$ being an even integer to match the lattice.

We define the colloid volume projection matrix $\bm{V}_{\text{cyl}}[d/2 \times d]$ for a cylindrical colloid of size $d$, such that each element of the matrix equals the volume of the colloid contained within the corresponding lattice element:
\begin{equation}
    V_{\text{cyl}}[i, k] = \pi(2i + 1)
\end{equation}
Obviously, the sum of all matrix elements equals the volume of the cylinder.
\begin{equation}
    \sum_{i=0}^{d/2 - 1} \sum_{k=0}^{d - 1} V_{\text{cyl}}[i, k] = \frac{\pi d^3}{4}
\end{equation}

Analogously, we define the colloid surface projection matrix $\bm{S}_{\text{cyl}}[d/2 \times d]$, such that each element of the matrix equals the surface area of the colloid within the corresponding lattice element.
\begin{align}
    \begin{split}
        S_{\text{cyl}}[i, k] = 
        &\begin{cases}
            2 \pi i,   & \text{if } i = d/2 - 1 \\
            0,         & \text{otherwise}
        \end{cases}
        \\
        &+
        \begin{cases}
            2 \pi (i + 1), & \text{if } k = 0 \text{ or } k = d - 1 \\
            0,             & \text{otherwise}
        \end{cases}
    \end{split}
\end{align}
Here, the first term accounts for the top and bottom faces, and the second term accounts for the lateral surface elements, of the cylinder.
Again, the sum of all matrix elements equals the surface area of the cylinder:
\begin{equation}
    \sum_{i=0}^{d/2 - 1} \sum_{k=0}^{d - 1} S_{\text{cyl}}[i, k] = \frac{3 \pi d^2}{2}
\end{equation}

Figure~\ref{fig:cylindrical_kernel_SI} provides a color-coded map of the volume $\bm{V}_{\text{cyl}}$ and surface $\bm{S}_{\text{cyl}}$ projection matrices for a selected colloid sice ($d = 16$).

\begin{figure}[H]
    \centering
    \includegraphics[width = \linewidth]{fig/cylindrical_kernel_SI.png}
    \caption{
    Volume (left) and surface (right) projection matrices for a cylindrical particle with diameter and height $d = 16$. The element values of the matrices are rainbow color-coded, with violet representing zero and yellow the highest values.
    }
    \label{fig:cylindrical_kernel_SI}
\end{figure}

To calculate the colloid insertion free energy, we integrated the osmotic pressure over the colloid volume and the surface tension over the colloid surface (Eq.~(5)).
The equivalent operation on a discrete lattice is the matrix dot product.
The two contributions to the insertion free energy are thus calculated as:
\begin{equation}\label{eq:cyl_fe_osm}
    \begin{split}
        \Delta F_{\text{cyl}}^{\text{osm}}(z_{\text{c}}) = \bm{V}_{\text{cyl}} \cdot \boldsymbol{\Pi}\{z_{\text{c}},\} \text{ and}
        \\
        \Delta F_{\text{cyl}}^{\text{sur}}(z_{\text{c}}) = \bm{S}_{\text{cyl}} \cdot \boldsymbol{\gamma}\{z_{\text{c}}\},
    \end{split}
\end{equation}
where the matrix elements for $\boldsymbol{\Pi}\{z_{\text{c}}\}$ and $\boldsymbol{\gamma}\{z_{\text{c}}\}$ run across $0 \leq i < d/2$ and $z_{\text{c}} - d/2 \leq k < z_{\text{c}} + d/2$.

The discretized insertion free energy profile $\Delta F_{\text{cyl}}$ is obtained from a series of insertion free energy calculations across all possible colloid center positions $z_{\text{c}}$.
Such a series of sequential integrations (matrix dot products) is equivalent to the convolution with the colloid volume/surface projection matrix acting as a kernel:
\begin{equation}
    \Delta F_{\text{cyl}}^{\text{osm}} = \boldsymbol{\Pi} \ast \bm{V}_{\text{cyl}} \text{ and }
    \Delta F_{\text{cyl}}^{\text{sur}} = \boldsymbol{\gamma} \ast \bm{S}_{\text{cyl}}.
\end{equation}
Convolution was performed via fast Fourier transform for computational efficiency.

\end{multicols}


%%%%%%%%%%%%%%%%%%%%%%%%%%%%%%%%%%%%%%%%%%%%%%%%%%%%%%%%%%%%%%%%%%%%%%%%%%%%%
\section{Supplementary Method. Estimating the effective polymer concentration near a colloid from SF-SCF data}
\begin{multicols}{2}

Depending on the magnitude of the polymer-colloid interaction strength $\chi_{\text{PC}}$, a colloid may attract or repel the polymer around it, causing local perturbations within a distance of a few segments.
In Eq. (7), the effective polymer volume fraction (concentration) near the colloid was approximated as $\phi^{\ast} = (b_0 + b_1 \chi_{\text{PC}}) \phi$.
Here, we justify this approximation based on SF-SCF data and quantify the coefficients $b_0$ and $b_1$, which capture the local perturbation to the polymer volume fraction profile.

As illustrated in Figure~\ref{fig:sf-scf_scheme}, a cylindrical particle is positioned coaxially along the $z$‑axis explicitly as impermeable lattice elements.
Consequently, the polymer‑brush chains adjust to the available space and the polymer-colloid interaction strength, producing a change in the system's total free energy, $F_{\text{cyl}}(z_{\text{c}})$.

We performed a series of SF-SCF calculations to determine the insertion free energy profiles $\Delta F_{\text{SF-SCF}}(z_{\text{c}})$ for a range of $d$, $\chi_{\text{PS}}$ and $\chi_{\text{PC}}$ values with a pore geometry and polymer brush parameters as set in Figure~1.
For each condition, we performed a ground state free energy correction, subtracting the total free energy for a colloid far away from the pore $F_{\text{cyl}}(z_{\text{c}} \to \pm \infty)$, to ensure the reference value $\Delta F_{\text{SF-SCF}}(z_{\text{c}} \to \pm \infty) = 0$.

The insertion free energies thus computed were compared to $\Delta F_{\text{cyl}}(z_{\text{c}})$, obtained as described in the previous section with $\gamma_{i,k}$ parametrized according to Eq. (7).
The optimal coefficients $b_0$ and $b_1$ were found using the least-squares method, i.e, minimizing $\sum [\Delta F_{\text{SF-SCF}} - \Delta F_{\text{cyl}}(b_0,b_1)]^2$ across a range of $\chi_{\text{PS}} \in [0,1]$  and $\chi_{\text{PC}} \in [-2,0]$ for a relevant particle positions $-60 \leq z_{\text{c}} \leq 0$.
Fits were performed with small colloids ($d=4$), to focus on local effects and avoid added effects that may arise due to global perturbations of the polymer distribution in the pore.

% The sum covered the relevant range of the profiles ($-60 \leq z_{\text{c}} \leq 0$) and $\chi_{\text{PC}}$ values of 0.00, -0.75 and -1.50 in $\theta$-solvent ($\chi_{\text{PS}} = 0.5$).
%COMMENT RR: Mikhail, please can you check if the above is correct and revise as necessary. We here need to provide a clear 'recipe' how the b parameters were quantified. Your text was a bit confusing in this regard as the natrue of the some was not clearly spelled out. Also, can we justify wht the selected chi_PC and chi_PS values were selected? This was also not entirely clear.

In a $\theta$‑solvent (chosen here as a representative case) and for fixed pore geometry and brush parameters, the three interaction parameters
$\chi_{\text{PC}} = 0.00,; -0.75,; \text{and}; -1.50$ span the regimes of net repulsion, near‑critical adsorption, and net attraction, respectively.
Figure~\ref{fig:fit_SI} demonstrates that a satisfactory fit (thick solid line) to the $\Delta F_{\text{SF-SCF}}$ profiles (circles) could be obtained across all three representative datasets with a single parameter set $b_0 = 0.7$ and $b_1 = -0.3$.
The insertion free energies are consistently close to zero for $\chi_{\text{PC}} = -0.75$ illustrates that this value is close to $\chi_{\text{PC}}^{\text{crit}} = \chi_{\text{crit}} + \chi_{\text{PS}} (1 - \phi)$, where polymer attraction and osmotic repulsion just cancel each other.
Consequently, the least-squares fits considered the regimes of net repulsion and net attraction approximately equally through $\chi_{\text{PC}} = 0.00$ and $\chi_{\text{PC}} = -1.50$, respectively.
In contrast, neglect of local perturbations to the polymer concentration ($b_0 = 1.0$ and $b_1 = 0.0$) reproduces $\Delta F_{\text{SF-SCF}}$ rather poorly for repulsive and attractive colloids (Figure~\ref{fig:fit_SI}, thin dashed line), thus demonstrating the importance of the correction.

\begin{figure}[H]
    \centering
    \includegraphics[width = 0.9\linewidth]{fig/fit_SI.png}
    \caption{
    Comparison of $\Delta F_{\text{SF-SCF}}$ profiles (circles) with $\Delta F_{\text{cyl}}(b_0,b_1)$, for the best-fit values of $b_0 = 0.7$ and $b_1 = -0.3$ (optimally accounting for local perturbations in the polymer concentration; thick solid lines) and for $b_0 = 1.0$ and $b_1 = 0.0$ (neglecting any local perturbations in polymer concentration; thin dashed lines).
    Pore and brush parameters are as given in Figure 1; $d = 4$, $\chi_{\text{PS}} = 0.5$, and $\chi_{\text{PC}}$ values are color-coded and indicated on the right side of the figure.
    The light green hatched area marks values of $z_{\text{c}}$ that are located inside the pore lumen ($|z| \leq 26$).
    }
    \label{fig:fit_SI}
\end{figure}

Although the fit was performed only for small colloids, the resulting parameters $b_{0}$ and $b_{1}$ still successfully account for the local perturbations to the polymer concentration when calculating the insertion free energy of larger colloids.
This is illustrated in Figure~\ref{fig:fe_scf_grid} for two selected $\chi_{\text{PC}}$ values (-0.5 and -1.0) and three selected $\chi_{\text{PS}}$ values (0.4, 0.5 and 0.6).
% We vary the solvent strength $\chi_{\text{PS}}$ around the $\theta$‑condition, thereby shifting an attractive particle from repulsion in a moderately good solvent to attraction in a moderately poor solvent.
% Changing $\chi_{\text{PS}}$ in either direction affects only the magnitude of the profile, not its overall trend.
% For colloids with $d \leq 12$, $\Delta F_{\text{cyl}}$ (solid lines) reproduces the full $\Delta F_{\text{SF-SCF}}$ profiles well.
Deviations become notable, however, for $d = 16$ under conditions of strong repulsion or attraction.
This size regime thus marks the limit of validity of the local perturbation approximation.
Instead, attractive or repulsive interactions entail non-local changes that impact the polymer concentration across the entire pore cross-section.
As a consequence, the pore walls also influence colloid insertion. 

A similar profiles to Figure~\ref{fig:fe_scf_grid} for position-dependent insertion were obtained by Molecular Dynamics in ref \cite{Ananth2018, Tagliazucchi2018}.
Another interesting effect can also observed for strongly attractive particles, even when small.
The SF-SCF results predict that, to reach the minimum of the insertion free energy $\Delta F_{\text{SF-SCF}}$, the polymer brush changes conformation to reach the particle at a greater distance $|z_{\text{c}}|$ from the pore.
In Figure~\ref{fig:fit_SI}, this effect is apparent for $\chi_{\text{PC}} = -1.5$, where the SF-SCF results (blue circles) predict systematically lower insertion free energies than the analytical approach (blue solid line) outside the pore ($|z_{\text{c}}| > 26$).
This effect is mild and only slightly increases the region with negative insertion free energy.
It may entail a slight reduction in the total resistance to diffusive transport: for pores with regions of negative insertion free energy the total resistance is defined by the resistance of the bulk solution (Figure 4), and the capture of colloids by the polymers at a larger distance from the pore reduces the bulk resistivity.

\begin{figure}[H]
    \centering
    \includegraphics[width = 0.95\linewidth]{fig/fe_scf_grid2.png}
    \caption{ 
    Comparison of $\Delta F_{\text{SF-SCF}}$ profiles (squares) with $\Delta F_{\text{cyl}}(b_0,b_1)$ for the best-fit values $b_0 = 0.7$ and $b_1 = -0.3$ (solid lines), for $d = [8, 12, 16]$ (color coded).
    Pore and brush parameters are as given in Figure 1. $\chi_{\text{PC}} = -1.0$ (top row) and -0.5 (bottom row); the solvent quality was varied near the $\theta$-point with $\chi_{\text{PS}} = [0.4, 0.5, 0.6]$, as indicated.
    The light green hatched area marks values of $z_{\text{c}}$ that are located inside the pore lumen ($|z| \leq 26$).
    \label{fig:fe_scf_grid}
    }
\end{figure}

\begin{figure}[H]
    \centering
    \includegraphics[width = 0.95\linewidth]{fig/perturbation.png}
    \caption{ 
    Perturbation to polymer concentration $\Delta\phi = \phi^{\text{ins}} - \phi$, where $\phi^{\text{ins}}$ is perturbed by the colloid particle polymer concentration.
    \label{fig:perturbation}
    }
\end{figure}

\end{multicols}

%%%%%%%%%%%%%%%%%%%%%%%%%%%%%%%%%%%%%%%%%%%%%%%%%%%%%%%%%%%%%%%%%%%%%%%%%%%%%
\section{Supplementary Method. Computing insertion free energies for arbitrarily placed spherical colloids}

\begin{multicols}{2}

The SF-SCF method considered in the previous sections is limited to colloids moving along the main axis of the pore.
In reality, colloids may be located off the pore axis.
Here, we generalize the analytical approach to calculate insertion free energies based on volume and surface contributions (assuming localized perturbations to the polymer concentration) to arbitrarily placed colloids.
In doing so, we also change the shape of the colloid, from a cylinder to a simpler sphere.

As the polymer and pore geometrical features are uniform in the angular direction, we use $r, z$ cylindrical coordinates with a degenerate angular axis.
Any property such as the polymer volume fraction can hence be expressed as a function $f(r,z,\theta) = f(r,z)$.
The transformation from cylindrical coordinates $r$, $z$ and $\theta$ to Cartesian coordinates $x$, $y$ and $z$ is:
\begin{equation}
    \bm{T}(r,z,\theta) = 
    \begin{pmatrix}
        x = r \cos \theta\\
        y = r \sin \theta\\
        z = z
    \end{pmatrix}
\end{equation}
$\Delta_{\text{center}} = \sqrt{r^2 + r_{\text{c}}^2 - 2 r r_{\text{c}} \cos(\theta) + (z - z_{\text{c}})^2}$ is the distance to the colloid center $(r_{\text{c}}, z_{\text{c}}, \theta_{\text{c}})$ in the cylindrical coordinate system; without loss of generality, we set $\theta_{\text{c}} = 0$.

We note identities that will be used in later expressions:
\begin{gather}
    \bm{J} = 
    \begin{pmatrix}
        \cos{\theta} & -r\sin\theta & 0 \\
        \sin{\theta} &  r\cos\theta & 0 \\
        0           & 0            & 1
    \end{pmatrix}\\
    \det(\bm{J}) = r \\
    \left| \hat{n} \cdot \frac{\partial \bm{T}}{\partial \theta} \right| = \frac{-r^2 \sin \theta + r r_{\text{c}} \sin \theta \cos \theta}{\Delta_{\text{center}}}\\
    \left| \frac{\partial \bm{T}}{\partial r} \times \frac{\partial \bm{T}}{\partial z} \right| = 1\\
    dA = \frac{\left|\frac{\partial \bm{T}}{\partial r} \times \frac{\partial \bm{T}}{\partial z} \right|}{\left|\hat{n}(r, z, \theta) \cdot \frac{\partial\bm{T}}{\partial \theta}\right|} dr dz
\end{gather}
where $\bm{J}$ is the Jacobian matrix, $\hat{n}$ is a normal to the surface, and $dA$ is a differential area element.

We define the volume $V$ with an indicator function:
\begin{eqnarray}
    \bm{1}_V =
    \begin{cases}
        1 & \text{if } (r, z, \theta) \in V\\
        0 & \text{if } (r, z, \theta) \notin V
    \end{cases}
\end{eqnarray}

We further integrate $f(r,z)$, over a volume defined by the indicator function:
\begin{eqnarray}
    \label{eq:int_indicator_V}
    \begin{split}
        \int\limits_{V} f dV
        =\int\displaylimits_{0}^{+\infty} \int\displaylimits_{-\infty}^{+\infty} \int\displaylimits_{0}^{2\pi} f(r, z) \bm{1}_V  dr dz d\theta =\\
        =\int\displaylimits_{0}^{+\infty} \int\displaylimits_{-\infty}^{+\infty} f(r, z) \int\displaylimits_{0}^{2\pi}  \bm{1}_V dr dz d\theta =\\
        =\int\displaylimits_{0}^{+\infty} \int\displaylimits_{-\infty}^{+\infty} f(r, z)  V_{\theta \downarrow} dr dz
    \end{split}
\end{eqnarray}

\begin{figure}[H]
    \centering
    \includegraphics[width=0.95\linewidth]{fig/sphere_volume_and_surface_projection.png}
    \caption{
        Surface and volume projections for a spherical colloid with diameter $d = 8$ in a cylindrical $r$ vs. $z$ lattice, shown for a set of distances $r_{\text{c}} = \{0, 2, 4, 6\}$ from the sphere center to the $z$-axis (shown side by side).
        The sphere centers are indicated with a red cross.
        The heatmaps are drawn with equal aspect ratio; the $z$-axis is omitted in the figure, as the center coordinates $z_{\text{c}}$ are arbitrary.
        The color code is shown with the colorbar on top, where blue corresponds to zero and yellow corresponds to values above 25.
        Intensities for surface and volume are normalized by $a^2$ and $a^3$, respectively, where $a$ is the lattice unit length.
        The sphere surface projection $S_{\theta \downarrow}\{r_{\text{c}}\}$ on the $rz$-plane is shown in the upper set of heatmaps, and the volume projection $V_{\theta \downarrow}\{r_{\text{c}}\}$ is shown in the lower set of heatmaps, as indicated.
        The discretization results in the surface and volume projection matrices $\bm{S}\{r_{\text{c}}\}$ and $\bm{V}\{r_{\text{c}}\}$, respectively, which are shown on the right.
    }
    \label{fig:sphere_volume_and_surface_projection}
\end{figure}

Similarly, when a surface is defined via an indicator function $\bm{1}_S$, we can integrate a function $f(r, z)$, over a surface $S$:
\begin{eqnarray}
    \label{eq:int_indicator_S}
    \begin{split}
        \int\limits_{S} f dS = \int\displaylimits_{0}^{+\infty} \int\displaylimits_{-\infty}^{+\infty} \int\displaylimits_{0}^{2\pi} f(r, z) \bm{1}_S  dA =\\
        =\int\displaylimits_{0}^{+\infty} \int\displaylimits_{-\infty}^{+\infty} f(r, z)  S_{\theta \downarrow} dr dz
    \end{split}
\end{eqnarray}
We here dropped the angular dimension by projecting the integration volume $V_{\theta \downarrow}$ and surface $S_{\theta \downarrow}$ onto the $rz$-plane.
This reduction simplifies the integration, allowing to perform it in two dimensions rather than three, as the angular dependence is factored out.

The volume $V_{\theta \downarrow}$ and surface $S_{\theta \downarrow}$ projections of a spherical colloid onto the $rz$-plane in cylindrical coordinates are:
\begin{gather}
    V_{\theta \downarrow}(r, z, r_{\text{c}}, z_{\text{c}}) = 2\int_{0}^{\pi} H\left( \Delta_{\text{center}} - \frac{d}{2} \right) r \, d\theta
    \\
    S_{\theta \downarrow}(r, z, r_{\text{c}}, z_{\text{c}}) = 2\int_{0}^{\pi} \frac{\delta \left( \Delta_{\text{center}} - \frac{d}{2} \right) \Delta_{\text{center}}}{-r \sin \theta + r_{\text{c}} \sin \theta \cos \theta} \, d\theta
\end{gather}
where $H$ is the Heaviside function and $\delta$ is the Dirac delta function.
The function domain, values and discretization for $V_{\theta \downarrow}$ and $S_{\theta \downarrow}$ are exemplified in Figure~\ref{fig:sphere_volume_and_surface_projection} for a set of colloid with radial center positions $r_{\text{c}}$.

To find the elements of the projection matrices, we discretize the projections by integrating over a set of domains $r, z \in [i, i + \delta r] \times [k, k + \delta z]$ for possible indices $i,k$:
\begin{eqnarray}
    V(r_{\text{c}})_{[i, k]} = \iint \limits_{i, k}^{\substack{i+\delta r\\ k+\delta z}} V_{\theta \downarrow} (r, z, r_{\text{c}}, z_{\text{c}}) dr dz
    \\
    S(r_{\text{c}})_{[i, k]} = \iint \limits_{i, k}^{\substack{i+\delta r\\ k+\delta z}} S_{\theta \downarrow} (r, z, r_{\text{c}}, z_{\text{c}}) dr dz
\end{eqnarray}
where $z_{\text{c}}$ has an arbitrary value.
The size of the projection matrices is $\min(d, r_{\text{c}} + d/2) \times d$.
In contrast to Cartesian projections, cylindrical projections depend on the radial coordinate of the center $r_{\text{c}}$.

The discretized form of Eq. (6) to calculate the osmotic term in the insertion free energy for a spherical particle is:
\begin{eqnarray}
    \begin{split}
        \Delta F_{\text{osm}}(r_{\text{c}}, z_{\text{c}}) =\\
        = \mathop{\sum\sum}_{\mathclap{\substack{i \in [0, \min(r_{\text{c}}+d/2,d)-1] \\ k \in [0, d-1]}}} V{r_{\text{c}}}_{[i, k]} \cdot \Pi_{[\max(r_{\text{c}}-d/2,0)+i, z_{\text{c}}-d/2+k]} =\\[-15pt]
        = \bm{V}\{r_{\text{c}}\} \cdot \bm{\Pi}\{r_{\text{c}}, z_{\text{c}}\} \\[5pt]
        \text{where } \bm{\Pi}\{r_{\text{c}},z_{\text{c}}\} =\left(\bm{\Pi}_{i,k}\right) {\substack{\max(r_{\text{c}}d/2,0) \le i < r_{\text{c}}+d/2 \\ z_{\text{c}}-d/2 \le k < z_{\text{c}}+d/2}}
    \end{split}
\end{eqnarray}
Similarly, the surface term in the insertion free energy is the following matrix multiplication:
\begin{eqnarray}
    \begin{split}
        \Delta F_{\text{sur}}(r_{\text{c}}, z_{\text{c}}) = \bm{S}\{r_{\text{c}}\} \cdot \bm{\gamma}\{r_{\text{c}}, z_{\text{c}}\} \\[5pt]
        \text{where } \bm{\gamma}\{r_{\text{c}},z_{\text{c}}\} =\left(\bm{\gamma}_{i,k}\right) {\substack{\max(r_{\text{c}}-d/2,0) \le i < r_{\text{c}}+d/2 \\ z_{\text{c}}-d/2 \le k < z_{\text{c}}+d/2}}
    \end{split}
\end{eqnarray}

\begin{figure}[H]
    \centering
    \includegraphics[width=0.9\linewidth]{fig/spherical_kernel.png}
    \caption{
        Illustration of a spherical colloid's volume and surface projection matrices in a cylindrical lattice, exemplifying the volume projection matrix $\bm{V}\{r_c\}$ for a spherical colloid with diameter $d = 12$ and $r_{\text{c}}= 8$.
        The colored tiles encode the matrix elements' values, where violet means zero and yellow represents the largest values.
        The geometrical meaning of the matrix element is the colloid volume (red body) or surface (green small patch) found in the domain $r,z \in [i, i + \delta r] \times [k, k + \delta z]$ (opaque blue toroid).
        The smaller drawing on the right complements the main drawing with a $z$-view for clarity.
        %Comment RR: Mikhail, I was left wondering what themeaning is of the bright green circle in the illustration on the right?
    }
    \label{fig:spherical_kernel}
\end{figure}

Volume and surface projections are explained geometrically in Figure~\ref{fig:spherical_kernel}.
The colloid center has an offset $r_{\text{c}}$ from the cylindrical lattice's $z$-axis.
To construct the colloid volume projection elements $V\{r_{\text{c}}\}[i, k]$, for each lattice element (blue opaque square toroid), we determine the volume of the lattice element occupied by the colloid (red body, marked with the word 'volume')
The colloid surface projection element $S\{r_{\text{c}}\}[i, k]$ is the piece of the colloid surface (green patch, marked with the word 'surface') encompassed within a lattice element.
As an example, a $12 \times 12$ volume projection matrix $V\{r_{\text{c}} = 8\}[i, k]$ for a spherical colloid with diameter $d = 12$ is shown as a blue-green-yellow colormap, where blue color indicates that the lattice element is not occupied by the colloid.

\end{multicols}

%%%%%%%%%%%%%%%%%%%%%%%%%%%%%%%%%%%%%%%%%%%%%%%%%%%%%%%%%%%%%%%%%%%%%%%%%%%%%
\section{Effects of Volume Exclusion Due to Pore Walls}

%COMMENT RR: This section is not well placed here. It may either be placed in the front, or omitted alltogether? I have not reviewed this section in detail.

\begin{multicols}{2}

\begin{figure}[H]
    \centering
    \includegraphics[width=0.9\linewidth]{fig/excluded_volume_SI.png}
    \caption{
        In the left frame, the effective pore shape is traced with a dashed red line
        The excluded volume is created by a spherical particle with diameter $d$ and is shown with a red shade.
        In the right frame, the excluded volume is shown on the regular lattice on the $rz$-plane.
        % The excluded volume on the regular lattice is a result of a morphological binary dilation $\bm{W}^{\ast} = \bm{W} \bigoplus \bm{V}$.
        }
    \label{fig:excluded_volume}
\end{figure}

Naturally, a colloid particle cannot occupy the space near the membrane walls.
In general, for a particle of arbitrary shape, there are translational and orientational constraints; a region close to the walls does not allow for certain orientations of the particle or positions of its center.

Spherical particles have only translational non-interference constraints; the excluded volume can be seen as a morphological dilation of the membrane body.
The morphological dilation in this case is the locus of points covered by a sphere when the center of the sphere moves inside the membrane body.

The space impermeable to a spherical particle is an effective pore with a radius smaller than the actual pore radius, $r_{\text{p}} = r_{\text{p}}^{0} - \frac{d}{2}$, and thicker walls, $L = L_{0} + d$, with rounded corners, as shown in the upper frame of Figure~\ref{fig:excluded_volume}.

Since the space is discretized into a regular grid, the membrane impermeable to the brush is defined as a two-dimensional Boolean array $\bm{W}$, where $W[i, k] = \text{True}$ indicates the wall.

Let us introduce another coarse-grained representation for a spherical particle:
\begin{equation}
    V[i, k] = 
    \begin{cases}
            1, & \text{if } \left( \dfrac{d}{2} - i - \dfrac{1}{2} \right)^2 + \left( \dfrac{d}{2} - k - \dfrac{1}{2} \right)^2 \le \dfrac{d^2}{4} \\
            0, & \text{otherwise}
        \end{cases}
\end{equation}
The two-dimensional Boolean array $\bm{V}$ is shown in the right frame of Figure~\ref{fig:excluded_volume} as lattice elements filled with orange color; the spherical particle is shown in the left frame.

The array $\bm{V}$ indicates whether any piece of the particle present in a domain $r,z \in [i, i + \delta r] \times [k, k + \delta z]$ for a given indices $i,k$.

The space impermeable to a spherical particle includes the membrane and the excluded volume, and is defined as a two-dimensional Boolean array $\bm{W}^{\ast} = \bm{W} \bigoplus \bm{V}$, which is the result of a binary morphological dilation, as shown in the right frame of Figure~\ref{fig:excluded_volume}.

\end{multicols}



\begin{figure*}[h]
    \centering
    \includegraphics[width=0.85\linewidth]{fig/coordinate_system.png}
    \caption{
        (Left)
        The steady-state solution of the diffusion equation for a point-like particle diffusing through a finite-thickness empty cylindrical pore.
        The iso-concentration surfaces $c = \text{const}$ are shown with a contour plot, with concentration values labeled.
        For an empty pore, $\psi = c$; for a brush-filled pore, $\psi = c \exp(\Delta F / k_B T)$.
        The blue and red axes are radial and radial and axial coordinates, respectively.
    }
    \label{fig:empty_pore_solution}
    \caption{
        (Right)
        Special orthogonal curvilinear coordinate system for a pore with radius $r_{\text{p}}^{0} = 20$ and thickness $L_{0} = 20$.
        \\
        Solid lines represent surfaces of rotation along the pore axis.
        Red lines show the constant $x_{z}$ surfaces; blue lines are the constant $x_{r}$ surfaces.
        The constant $x_{\theta}$ semi-planes are not shown.
        \\
        A local basis of the coordinate system $\hat{e}_r, \hat{e}_z$ is shown with arrows.
        The local basis defines Lam\'e coefficients $h_r = |\hat{e}_r|$, $h_z = |\hat{e}_z|$, $h_{\theta} = |\hat{e}_{\theta}|$.
        }
    \label{fig:coordinate_system}
\end{figure*}

\pagebreak
%%%%%%%%%%%%%%%%%%%%%%%%%%%%%%%%%%%%%%%%%%%%%%%%%%%%%%%%%%%%%%%%%%%%%%%%%%%%%%%%
\section{Supplementary Methods 7. Analytical Pore Resistance Estimation}
\begin{multicols}{2}

% \begin{figure}[H]
%     \centering
%     \includegraphics[width=0.8\linewidth]{fig/empty_pore_contour.png}
%     \caption{

%         }
%     \label{fig:empty_pore_solution}
% \end{figure}

Consider a medium with a constant diffusion coefficient $D_0$, separated by an impermeable membrane with a cylindrical pore.
The pore is placed at the origin of the cylindrical coordinate system, with the main axis of the pore coaxial with the $z$-axis.
Let the concentration of the solute in the semi-infinite bulk on one side be $c(z = -\infty) = 1$ and $c(z = +\infty) = 0$ in the bulk on the other side.
The steady-state solution $\frac{\partial c}{\partial t} = 0$ of the diffusion equation is a concentration profile shown in Figure~\ref{fig:empty_pore_solution}.
It is known that for a pore in an infinitely thin membrane (an orifice), the iso-concentration lines form oblate spheroids with the pore being a focal circle, and even for a membrane with finite thickness, this remains a good approximation.
Trivially, for a long cylindrical channel the iso-concentration surfaces are equally-spaced disks, which approximate the solution in the pore lumen.
The flux intensity field is the gradient of concentration, $\bm{j} = -D_0 \nabla c$.

A polymer brush grafted in the pore modulates the local diffusion coefficient $D$ and creates a free energy landscape.
The iso-concentration lines in the steady state are no longer oblate spheroids, although the steady-state solution still shares the structure with an empty pore in a medium with a constant diffusion coefficient.
As the flux intensity is a conservative vector field, we introduce a scalar potential function $\psi$, such that $\bm{j} = -D \nabla \psi$.
This scalar field $\psi = c \exp(\Delta F / k_B T)$ is a modified Boltzmann distribution and has a structure similar to the steady-state solution of an empty pore.

Consider an orthogonal curvilinear coordinate system $x_{\theta}, x_{r}, x_{z}$ with iso-surfaces defined as follows:
the level set of the potential function $\psi$ with a constant $x_{z}$ indexed by its intersection with the $z$-axis;
stream surfaces of diffusing particles with constant $x_{r}$ indexed by the radius $r$ of its intersection at $z = 0$;
half-planes with a constant azimuthal angle $x_{\theta}$:
\begin{gather}
    x_z = \left\{ (r, z) \mid \psi = \psi(r = 0, z) \right\}
    \\
    x_r = \left\{ (r, z) \mid \nabla f \cdot \nabla \psi = 0, f = f(r, z = 0) \right\}
    \\
    x_{\theta} = \theta
\end{gather}

For an orifice, this curvilinear coordinate system is a variant of the oblate spheroidal coordinate system, where surfaces of constant $x_{\theta}$ are half-planes, surfaces of constant $x_{r}$ are confocal hyperboloids of revolution, and surfaces of constant $x_{z}$ are confocal oblate spheroids.
The focal circle is the pore circumference.

For an empty pore with finite thickness, the iso-surfaces can be approximated such that the resulting curvilinear coordinate system is a joint of cylindrical and oblate spheroidal coordinate systems,
where cylindrical coordinates are used inside the pore $z \in [-L/2, L/2]$ and oblate spheroidal coordinates otherwise.

We exemplify the curvilinear coordinate system for a pore with finite thickness $L_{0} = 20$ and radius $r_{\text{p}}^{0} = 20$ in Figure~\ref{fig:coordinate_system}.

The interior cylindrical coordinate transformations and Lam\'e coefficients are defined as follows:
\begin{gather}
    \label{eq:cyl_transformation_1}
    x = x_r \cos(x_{\theta}), \quad
    y = x_r \sin(x_{\theta}), \quad
    z = x_z
    \\
    \label{eq:cyl_transformation_2}
    h_r = 1, \quad
    h_z = 1, \quad
    h_{\theta} = x_r
\end{gather}

The exterior oblate spheroidal coordinate system transformations and Lam\'e coefficients are:
\begin{gather}
    \begin{aligned}\label{eq:oblate_spheroid_transformation_1}
        x &= x_r
        \sqrt{1 + \frac{(|x_z| - L/2)^2}{r_{\text{p}}^2}}
        \cos(x_{\theta})
        \\
        y &= x_r
        \sqrt{1 + \frac{(|x_z| - L/2)^2}{r_{\text{p}}^2}}
        \sin(x_{\theta})
        \\
        z &= (|x_z| - L/2) \frac{\sqrt{r_{\text{p}}^2 - x_r^2}}{r_{\text{p}}} + \operatorname{sgn}(x_z) \frac{L}{2}
    \end{aligned}
    \\[4pt]
    \begin{aligned}\label{eq:oblate_spheroid_transformation_2}
        h_r &= \frac{\sqrt{r_{\text{p}}^2 + (|x_z| - L/2)^2 - x_r^2}}{\sqrt{r_{\text{p}}^2 - x_r^2}}
        \\
        h_z &= \frac{\sqrt{r_{\text{p}}^2 + (|x_z| - L/2)^2 - x_r^2}}{\sqrt{r_{\text{p}}^2 + (|x_z| - L/2)^2}}
        \\
        h_{\theta} &= \frac{x_r \sqrt{r_{\text{p}}^2 + (|x_z| - L/2)^2}}{r_{\text{p}}}
    \end{aligned}
\end{gather}

We assume that in the selected coordinate system, the flux is always normal to constant $x_z$ surfaces, which is exact for point-like particles passing through an orifice.

When the steady-state solution to the Smoluchowski equation is postulated to be a modified Boltzmann distribution, the product $\rho = D e^{-\Delta F / k_B T}$ can be recognized as local conductivity, as we also noted in our previous paper~\cite{Laktionov2023}.

Integration over the constant $x_z$ surfaces gives the inverse resistance per unit length (as appropriate for resistors connected in parallel), and the integration over each $x_z$ coordinate simply adds contributions from all the slices connected in series.

\begin{gather}
    \label{eq:R_z_analyt}
    \varrho_{z}^{-1} = \int\displaylimits_{0}^{r_{\text{p}}} \int\displaylimits_{0}^{2\pi} \rho^{-1} h_r h_{\theta} h_z^{-1} dx_{\theta} dx_r
    \\
    \label{eq:h_integrand}
    \begin{aligned} 
        &h_r h_{\theta} h_z^{-1} = \\ &=
        \begin{cases}
            x_r, \text{ if } x_z \in [-L/2,L/2]
            \\[2pt]
            \dfrac{x_r}{r_{\text{p}}}\dfrac{r_{\text{p}}^2 + (|x_z|-L/2)^2}{\sqrt{r_{\text{p}}^2 - x_r^2}}, \text{ otherwise}
        \end{cases}
    \end{aligned}
    \\
    \label{eq:R_analyt}
    R = \int_{-\infty}^{+\infty} \varrho_{z} \, dx_z
\end{gather}

Here, $\varrho_{z} \, dx_z$ is the resistance of an oblate spheroid shell in the exterior region $x_z \notin [-L/2, L/2]$ and a circular disk in the interior of the pore $x_z \in [-L/2, L/2]$.

The physical meaning of the integrand in eq.~\ref{eq:R_z_analyt} is the conductivity of a conductor with cross-sectional area $h_r h_{\theta}$, length $h_z$, and specific conductivity $\rho^{-1}$.
In eq.~\ref{eq:h_integrand}, we write the part of the integrand corresponding to the interior and exterior regions of the pore, using Lam\'e coefficients from eqs.~\ref{eq:cyl_transformation_2} and \ref{eq:oblate_spheroid_transformation_2}, respectively.
The total resistance is the sum of the resistances of all the $x_z$ layers, as in eq.~\ref{eq:R_analyt}.

Consider an empty pore with finite thickness; let us apply eq.~\ref{eq:R_z_analyt} to find its resistance to a point-like particle.
For a pore without a polymer brush, the local conductivity is not modulated, so $\rho^{-1} = D_0^{-1}$.
\begin{gather}
    \label{eq:r_z_empty}
    (\varrho_{z}^{0})^{-1} = 
    \begin{cases}
        D_0 \pi r_{\text{p}}^2 \text{, if } |x_z| < L/2
        \\
        D_0 2 \pi \left((|x_z|-L/2)^2 + r_{\text{p}}^2\right) \text{, otherwise} 
    \end{cases}
\end{gather}

Finally, when we integrate over the $x_z$ domain, we get the classic equation~\cite{Brunn1984}:
\begin{equation}
    \label{eq:r_empty}
    R_{0} = \int\displaylimits_{-\infty}^{+\infty} \varrho_{z}^{0} dx_z
    =\frac{L}{D_0 \pi r_{\text{p}}^2} + \frac{1}{D_0 2 r_{\text{p}}}
\end{equation}

We inherently use discrete cylindrical lattice from the SF-SCF calculations. 
When integrating, we treat iso-potential lines as surfaces of half-cylinders rather than surfaces of oblate spheroids, as shown in Figure~\ref{fig:integration_scheme}.
Finally, a half-cylinder shell has a larger surface area than an oblate spheroid shell for the same $x_z$, which is corrected by introducing a prefactor $p$ in the integrand.

The resistance integration on the regular cylindrical lattice:
\begin{gather}
    \label{eq:r_z_num}
    \begin{aligned} 
        &\varrho_{z}^{-1} =
        \\
        &=\begin{cases}
             \pi \sum_{r=0}^{r_{\text{p}}^{0}} \rho^{-1}_{[r,z]} (2r+1) \text{, if } z\in[-L/2,L/2]
             \\[4pt]
             \begin{aligned}
                &\text{otherwise}
                \\
                &\pi f(z) \left(\sum_{r=0}^{r_{\text{base}}} \rho^{-1}_{[r,z]} (2r+1) + 2 r_{\text{base}} \sum_{z^{\prime} = z_{a}}^{z_{b}}\rho^{-1}_{[r,z^{\prime}]}\right)
             \end{aligned}
        \end{cases}
    \end{aligned}
    \\
    \begin{cases}
        z_{a} = z, z_{b} = -L/2-1, \text{if } z < -L/2
        \\
        z_{a} = L/2, z_{b} = z, \text{if } z > L/2
    \end{cases}
    \\
    \label{eq:r_base}
    r_{\text{base}} = r_{\text{p}}^{0} + |z| - L/2
    \\
    \label{eq:prefactor}
    \begin{aligned}
        &p(x_z) =\\
        & = \frac{2(|x_z|-L/2)^2 + 2(r_{\text{p}}^{0})^2}{(3(|x_z|-L/2)+r_{\text{p}}^{0})((|x_z|-L/2)+r_{\text{p}}^{0})}
    \end{aligned}
\end{gather}

where $p$ represents the ratio of the conductivities of an oblate spheroid shell and a half-cylinder shell for shells that share the same cross-section with the $z$-axis.

The way we iterate over cylindrical lattice elements in eq.~\ref{eq:r_z_num} is shown in Figure~\ref{fig:integration_scheme}.
In the interior of the pore $x_z \in [-L/2, L/2]$, the conductors are thin disks with inhomogeneous conductivity.
We assume no radial flux in the interior, so the conductivity of a disk centered at $z$ is approximated by $\pi \sum_{r=0}^{r_{\text{p}}^{0}} \rho^{-1}_{[r,z]} (2r + 1)$.
In the exterior of the pore, $x_z \notin [-L/2, L/2]$, to mimic the oblate spheroid shape, the conductors are thin half-cylinder shells with radius $r_{\text{base}}$ and element length $|z| - L/2$, with inhomogeneous conductivity.
\begin{figure}[H]
    \centering
    \includegraphics[width=\linewidth]{fig/resistance_integration.png}
    \caption{
        Numerical integration of local conductivity/resistance on the cylindrical lattice.
        For the exterior region, conductivities are summed over half-cylinder shells, shown in red.
        In the interior region, the conductivities are summed over cylindrical disks, shown in blue.
        Half-cylinder shells mimic the oblate spheroid iso-potential profiles of the analytical solution to the empty pore problem.
        }
    \label{fig:integration_scheme}
\end{figure}

To account for the resistance of semi-infinite reservoirs outside the integration region on the cylindrical lattice, we integrate the analytical eq.~\ref{eq:R_z_analyt} from the integration boundary $z$ to infinity over $x_z$:

\begin{eqnarray}
    \label{eq:r_reservoir}
    R_{(z, \pm\infty)} = \pm \int\displaylimits_{z}^{\pm\infty} \varrho_{z} \, dx_z = \dfrac{\pi - 2 \arctan\left( \dfrac{|z| - L/2}{r_{\text{p}}^{0}} \right)}{4 \pi r_{\text{p}}^{0}}
\end{eqnarray}

Finally, the total resistance of the pore, integrated on the discrete cylindrical lattice, is:

\begin{eqnarray}
    R_{\text{int}} = \sum_{z=-L/2}^{L/2} R_z
    \\
    R_{\text{ext}} = R_{(z_{a}, -\infty)} + \sum\limits_{\mathclap{\substack{z \in [z_{a},-L/2)\\z \in (L/2, z_b]}}} \varrho_{z} + R_{(z_{b}, +\infty)}
    \\
    \label{eq:R_total}
    R = R_{\text{ext}} + R_{\text{int}}
\end{eqnarray}
where $R_{\text{int}}$ is the resistance of the pore channel, $R_{\text{ext}}$ is a resistance of the convergent flux in the pore exterior.
The eq.\ref{eq:R_total} concludes the analytical estimation with the total resistance of the pore.


\end{multicols}

\section{Validating Analytical Approaches with Numerical Simulations}

\begin{multicols}{2}
The diffusion of nanocolloids in the presence of an effective potential is governed by the Smoluchowski diffusion equation:
\begin{equation}
    \label{eq:Smoluchowski}
    \frac{\partial c}{\partial t} = \nabla \cdot D(\nabla c + c \nabla \Delta F)
\end{equation}
where $c$ is the concentration of the colloid particles, $D$ is the local diffusion coefficient, and $\Delta F$ is the position-dependent insertion free energy, which plays the role of the potential of mean force
Our interest is in a stationary solution to the Smoluchowski diffusion equation, i.e., when $\frac{\partial c}{\partial t} = 0$.

To the best of our knowledge, a general analytical solution of the stationary equation is not available.
In the previous section, we discussed an analytical approximation we constructed with a set of assumptions
Here, we present results of numerical simulation on a discrete regular grid.
The method requires no assumptions about concentration profiles and results directly in a flux density field $\bm{j}$.

The mass conservation equation connects the rate of change in the colloid concentration and the flux density field:
\begin{equation}
    \frac{\partial c}{\partial t} = -\nabla \cdot \bm{j}
\end{equation}

From the position-dependent flux density, the total flux through the pore is found by integration over a control cross-section.
The control cross-section is an arbitrary surface that divides the system with the pore into two separated parts.
For convenience, we select a cross-section at the pore center ($z = 0$), as there is only an axial component $j_z$ to the flux.
Then the total flux through the pore is
\begin{equation}
    \label{eq:total_flux_1}
    J = \oint\limits_{S} \bm{j} \cdot d\bm{S} = 2 \pi \int_0^{r_{\text{p}}} j_z(r, z = 0) \, r \, dr
\end{equation}

We employ the finite volume method to simulate transport of the nanocolloid particles through the pore.

The key step in the finite volume method is the control volume integration using the divergence theorem:
\begin{equation}
    \label{eq:CFD_integration_1}
    \left( \frac{\partial c}{\partial t} \right)_{\text{CV}} = -\int\limits_{\text{CV}} \nabla \cdot \bm{j} \, dV = -\oint\limits_{\text{CV}} \bm{j} \cdot d\bm{S}
\end{equation}

To update the average colloid concentration $c$ in a control volume, we use the forward Euler time integration scheme:
\begin{equation}
    c_{t + \Delta t} = c_t + \left( \frac{\partial c}{\partial t} \right)_{\text{CV}} \Delta t
\end{equation}
where $c_{t + \Delta t}$ and $c_t$ are the average colloid concentrations in a control volume at times $t + \Delta t$ and $t$, respectively
We stop updating $c$ when $\left( \frac{\partial c}{\partial t} \right)_{\text{CV}}$ becomes negligible, indicating that the stationary solution has been reached.

For the finite volume method, the domain has to be divided into discrete control volumes.
Identical to the previous section, we inherit a discrete cylindrical lattice; the lattice has two dimensions with each control volume has four neighboring control volumes.

The usual convention is to assign letters for each neighboring control volume and shared faces.
The properties averaged over some selected control volume are subscripted with $P$ or with $E$, $W$, $N$, $S$ for the neighboring control volumes to the east, west, north, and south, respectively.
The radial coordinate $r$ points to the north and axial coordinate points to the east.
The properties defined on the faces of a control volume are subscripted with $e$, $w$, $n$, $s$, depending on the neighbor the face is shared with (see Figure~\ref{fig:CFD_element}).


\begin{figure}[H]
    \centering
    \includegraphics[width=\linewidth]{fig/CFD_element.png}
    \caption{
        Schematic diagram of control volume, domain discretization, and labeling convention.
        The control volume (red square on the left frame) is surrounded by four neighbors to the east, west, north, and south (E, W, N, S), with faces labeled with lowercase letters.
        The discretized domain is a regular cylindrical lattice with $\delta r = \delta z = a$, identical to the lattice in SF-SCF calculations.
        The fluxes across the faces are shown in the right frame with arrows.
        The dilution effect of the larger control volumes is accounted for with $\lambda_{e,w,n,s}$.
        }
    \label{fig:CFD_element}
\end{figure}

In the cylindrical lattice, the control volumes increase with radial coordinate $r$; hence, when a quantity of colloid particles is transported in the radial direction, the colloid particles get diluted.
The dilution effect is accounted for with $\lambda_{e,w,n,s}$, which is the ratio between control volumes that share a face, with the selected control volume in the denominator:
\begin{eqnarray}
    \lambda_n =\begin{cases}
        1 + \dfrac{1}{r}, & \text{if } r \ne 0\\
        2, & \text{if } r = 0
    \end{cases}
    \\
    \lambda_s =\begin{cases}
        1 - \dfrac{1}{r}, & \text{if } r \ne 0\\
        0, & \text{if } r = 0
    \end{cases}
    \\
    \lambda_{e} = \lambda_{w} = 1
\end{eqnarray}

The discrete form of the divergence theorem in eq.~\ref{eq:CFD_integration_1} is
\begin{equation}
    \label{eq:CFD_integration_2}
    \left( \frac{\partial c}{\partial t} \right)_{\text{CV}} = \lambda_w j_w + \lambda_s j_s - \lambda_e j_e - \lambda_n j_n
\end{equation}
the signs before each term is chosen to reflect the fluxes direction according to the divergence rules (see Figure~\ref{fig:CFD_element}).

The flux of colloid particles, as one can see from the Smoluchowski diffusion equation, is caused by the gradient in the colloid concentration and the gradient in the insertion free energy
Let us call these contributions the diffusion and drift fluxes, $j_{\text{diffusion}}$ and $j_{\text{drift}}$, respectively.

The following equations are written only for the northern direction and are identical for the other directions.
The diffusion flux and drift terms contributing to the flux intensity are
\begin{eqnarray}
    j_n = j_{\text{diffusion}, n} + j_{\text{drift}, n} \\
    j_{\text{diffusion}, n} = - D_n (c_N - c_P) \\
    j_{\text{drift}, n} = - D_n c_n (\Delta F_N - \Delta F_P)
\end{eqnarray}
where the capital letter subscripts denote the property average values over the control volumes, and the lowercase letters denote the property values at the faces, as shown in Figure~\ref{fig:CFD_element}.
The diffusion coefficient at the face is found as a simple arithmetic mean:
\begin{equation}
    D_n = \frac{D_N + D_P}{2}
\end{equation}

When the flux is dominated by the insertion free energy gradient, the colloid concentration at face $c_n$ cannot be found as a simple arithmetic mean.
The choice of $c_n$ must reflect the dominant direction of the flux.

So called \emph{exponential scheme} uses the exact analytical solution of a one-dimensional equation for diffusion in a potential field to estimate the colloid concentration at face $c_n$.
This estimate is suitable in a wide range of conditions, from diffusion-dominated to drift-dominated transport \cite{Patankar1980,Versteeg2007}.

It does so by modifying the weight between the colloid concentrations $c_N$ and $c_P$ in the neighboring control volumes based on the local insertion free energy gradient.

The colloid concentration at the face is given by the formula:
\begin{equation}
    \label{eq:upwind}
    c_n = c_N + (c_P - c_N) \frac{\exp\left( \dfrac{\Delta F_N - \Delta F_P}{2} \right) - 1}{\exp\left( \Delta F_N - \Delta F_P \right) - 1}
\end{equation}

At $\Delta F_N = \Delta F_P$, eq.~\ref{eq:upwind} reduces to a simple arithmetic mean, while for $\Delta F_N - \Delta F_P > 10$, the neighboring control volume does not contribute to the colloid concentration at face $c_n \approx c_P$, as the dominant drift flux is directed outwards.
Conversely, when $\Delta F_N - \Delta F_P < -10$, the dominant drift flux is directed inwards, so the colloid concentration at face $c_n \approx c_N$.




Finally, when the stationary state is reached, we calculate the total flux through the pore with the discrete form of eq.~\ref{eq:total_flux_1}:
\begin{equation}
    \label{eq:total_flux_2}
    J = \frac{\pi}{4} j_z[ r = 0, z = 0 ] + \pi \sum_{r = 1}^{r_{\text{p}}} 2 r \, j_z[ r, z = 0 ]
\end{equation}

The resistance of the pore is found from the simple equation $R = \dfrac{\Delta c}{J}$, similar to Ohm's law, with a correction from the eq. (\ref{eq:r_reservoir}) to account for the finite size of the simulation box.

While, running numerical simulations to calculate the resistance is more computationally demanding than the approximate analytical scheme explained in the previous section and hides away some of the details on the pore resistance structure.
Nevertheless, there are major benefits to using it, since it requires no assumptions about the shape of the position-dependent colloid concentration $c$.
We employ numerical simulation as a guide and sanity check for the developed approximate analytical scheme.
The comparison for a set of parameters is given in Figure~\ref{fig:CFD_comparison}.

\begin{figure}[H]
    \centering
    \includegraphics[width=0.9\linewidth]{fig/R_vs_d_SI.png}
    \caption{%
        Total resistance compared between numerical simulation and approximate analytical approaches as a function of particle size.
        Total resistance is calculated for a good solvent $\chi_{\text{PS}} = 0.5$ and a set of 
        $\chi_{\text{PC}} = \{0.0, -1.00, -1.4, -1.8\}$ 
        interaction parameters ranging from inert to adsorbing particles.
        The pore radius and length are kept at $r_{\text{p}}^{0} = 26$, $L_{0} = 52$.
        \\
        The thick black line is the resistance of a pore with effective shape (volume exclusion).
        \\
        The resistance is normalized by the viscosity $\eta_{S}$ of the pure solvent and particle size $d$, Stokesian behaviour from consideration.
        \\
        \todo{Will be updated as more simulation be ready}
        }
    \label{fig:CFD_comparison}
\end{figure}

\todo{Also I soft clipped free energy from below to be more than -10kT, because any more negative free energy does not change conductivity much, but make simulation stale or break due to high partitioning}
\begin{equation}
    \Delta F_{\text{clipped}} = \Delta F_{\text{min}}+ \ln[\exp(\Delta F - \Delta F_{\text{min}})]
\end{equation}

\end{multicols}
%#########This chapter is to be rewritten to include sticky particles and fix prefactor###############
% \section{Diffusion Coefficient Impact on the Trends in Pore Resistance}
% \begin{multicols}{2}
    
% Locally, the polymer brush resembles a semi-dilute polymer solution with segment concentration $\phi$, characterized by a correlation length $\xi$ that represents the typical mesh size formed by overlapping polymer coils.

% According to the scaling theory developed by Cai et al.\cite{Cai2011}, for colloids smaller than the correlation length $\xi$, diffusion is unhindered and follows the Stokes--Einstein equation:
% \begin{equation}
% D_0 \approx \frac{k_B T}{3 \pi \eta_0 d}
% \label{eq:D_0}
% \end{equation}
% where $\eta_0$ is the solvent viscosity.

% For colloids larger than $\xi$, diffusion is hindered and occurs via hopping between mesh cages, requiring polymer chain relaxation. The diffusion coefficient in this regime is given by \cite{Cai2011}:
% \begin{equation}
% D \approx D_0 \frac{\xi^2}{d^2}
% \label{eq:D_m}
% \end{equation}

% To interpolate between Eqs.~\ref{eq:D_0} and \ref{eq:D_m}, we assume that the retarding effects due to solvent friction and mesh relaxation are additive:
% \begin{equation}
% D^{-1} = D_0^{-1} \left( 1 + \frac{d^2}{\xi^2} \right)
% \label{eq:D_interp}
% \end{equation}

% To account for possible deviations, we introduce a prefactor $f$:
% \begin{equation}
% D^{-1} = D_0^{-1} \left( 1 + \frac{f^2 d^2}{\xi^2} \right)
% \label{eq:D_interp_2}
% \end{equation}
% where larger values of $f$ decrease the local diffusion coefficient, effectively representing a larger particle size.

% The prefactor $f$ is introduced to demonstrate the limited impact of variations in the local diffusion coefficient on the overall trends in pore resistance.

% Figure~\ref{fig:permeability_ond_low_D} compares the pore resistance calculated using the position-dependent diffusion coefficient from Eq.~\ref{eq:D_interp} (thin lines) and Eq.~\ref{eq:D_interp_2} (thick lines with markers) with $f = 10$.

% This comparison shows that even if the position-dependent diffusion coefficient is underestimated, the qualitative trends remain the same.

% Lower diffusion coefficients correspond to higher pore resistance, shifting the resistance-particle size curves upward for smaller particles.

% When the pore channel resistance $R_{\text{int}}$ is negligible due to large negative insertion free energy, and the total resistance is mainly determined by the convergent flow resistance $R \approx R_{\text{ext}}$, changes in the position-dependent diffusion coefficient $D$ have little impact on the total resistance. This effect is observed in Figure~\ref{fig:permeability_ond_low_D} for highly attractive ($\chi_{\text{PC}} = -2.0$) particles with larger sizes in a $\theta$-solvent and moderately good ($\chi_{\text{PS}} = 0.3$) solvent.

% Furthermore, a lower position-dependent diffusion coefficient exaggerates the portion of the trend with negative slope in the resistance-particle size dependency.

% Since the local diffusion coefficient depends not only on the local polymer concentration and particle size but also on the molecular-level interactions between the polymer brush and the particle, which are not captured by our coarse-grained approach, incorporating more complex models or empirical data into the calculation of $D$ will not drastically change the trends and structure of the pore resistance.


% \end{multicols}



%####################################################
%There should be one more chapter about experimental data
%####################################################
\pagebreak
\section{Kinetics of Pore-Mediated Equilibration: Comparison to Experimental Data}

\begin{multicols}{2}

\begin{figure}[H]
    \centering
    \includegraphics[width=0.9\linewidth]{fig/experiment_description.png}
    \caption{Reductionist view of relevant pore-mediated equilibration experiments from Refs.~\cite{Ribbeck2001, Mohr2009, Popken2015, Timney2016, Frey2018}.\\
    \textbf{Left:} A compartment with finite volume $V_{\text{in}}$ is separated from a finite ($V_\text{out}$) or semi-infinite reservoir ($V_\text{out}\to \infty$)  by an impermeable membrane (green contour) permeated by $N_{\text{pores}}$ polymer-filled pores. 
    Nanocolloids (yellow circles) are mobile particles, present at a concentration $c_{\text{out}}$ in the semi-infinite reservoir, and influx into the finite volume where the time-dependent concentration is $c_{\text{in}}(t)$.\\
    \textbf{Top right:} Isolated polymers phase-separate in a poor solvent with solvent strength $\chi_{\text{PS}}^{\text{FG}}$, forming polymer gel droplets (red irregular shapes) with polymer volume fraction $\phi_{\text{gel}}$.\\
    \textbf{Bottom right:} Nanocolloids equilibrate between the solvent and the polymer gel, characterized by the partition coefficient $\text{PC}_{\text{gel}} \equiv \left(c_{\text{in}}/c_{\text{out}}\right)_{\text{gel}}$.
    }
    \label{fig:experiments_overview}
\end{figure}
    
We adopt a reductionist view of NPC‐mediated transport (Figure \ref{fig:experiments_overview}, left). 
The nucleus is represented as a well-mixed compartment of volume $V_{\text{in}}$ bounded by an impermeable membrane that contains $N_{\text{pores}}$ identical, widely spaced cylindrical channels, each filled with a homogeneous polymer brush mimicking the FG-domain meshwork of nucleoporins. 

Colloid particles at concentration $c_{\text{out}} = c_{0}$ diffuse from an external reservoir, either finite ($V_{\text{out}}$) or effectively infinite ($V_{\text{out}}\!\to\!\infty$), into the 'nucleus' compartment, whose initial concentration is $c_{\text{in}}(0)=0$.
Because the combined pore volume is negligible ($N_{\text{pores}}V_{\text{pore}}\ll V_{\text{in}}$) the equilibration follows a first-order rate law.

\begin{eqnarray}
    \frac{\partial c}{\partial t} &=& k (c_{\text{out}} - c_{\text{in}}(t)) \\
    \frac{c_{\text{in}}(t)}{c_{\text{out}}} &=& 1 - e^{-kt} \label{eq:kinetics} \\
    k &=& \frac{N_{\text{pores}}}{R} V_{\text{in}}^{-1}
    \label{eq:rate_constant}
\end{eqnarray}
$k$ is the rate constant (with characteristic time $\tau = 1/k$).
The pore resistance $R$ is defined by Eq.~\ref{eq:R_total}.

If both compartments change concentration during equilibration
(a two-compartment model, \textit{i.e.} exchange between the nucleus and cytoplasm), the rate constant becomes
\begin{equation}
  k = \frac{N_{\text{pores}}}{R}
      \bigl(V_{\text{in}}^{-1}+V_{\text{out}}^{-1}\bigr),
  \label{eq:rate_constant_2}
\end{equation}

FG-domains of nucleoporins are slightly hydrophobic. 
In solution they can therefore phase-separate into hydrogel particles (Figure~\ref{fig:experiments_overview}, top right), whereas when end-grafted to the pore walls they form a dense polymer brush that fills the NPC channel.
The equilibrium partition coefficient of a probe colloid in these isolated FG hydrogels (Figure~\ref{fig:experiments_overview}, lower right) reflects its interaction with the same domains in the grafted brush, and can be translated into an insertion free energy, and hence into a predicted pore resistance.

To capture this behaviour we treat the chemically diverse FG-domains as identical homopolymer chains with averaged properties.  
The same model reproduces both the hydrogel phase separation of ungrafted chains and the observed correlation between hydrogel
partitioning and NPC resistance.


\textbf{Extracting pore resistance from the experimental data.}
    
We compiled transport data for NPCs from HeLa and \textit{Saccharomyces cerevisiae} cells.  
The human cells studies used digitonin-permeabilised HeLa cells and monitored nuclear fluorescence while fluorescently tagged probes equilibrated with a large external reservoir 
\cite{Ribbeck2001,Mohr2009,Frey2018}.  
The yeast studies followed nuclear-cytoplasmic equilibration after
photobleaching a reporter protein \cite{Popken2015,Timney2016}.

Although yeast and human NPCs differ in their protein composition, the dimensions of the pore channel are similar and only slightly smaller for yeast NPC \cite{Yang1998}, allowing a direct comparison.  
Table~\ref{tbl:experimental} lists the experimental observables we extracted. 
To compare them with one another, and with our theoretical predictions, we convert each observable to the single-pore resistance $R_{\text{exp}}$.  
For data that report a kinetic constant $k$ (or its reciprocal time constant $\tau$) we use Eqs.~(\ref{eq:rate_constant}, \ref{eq:rate_constant_2}).

Popken \textit{et al.}\,\cite{Popken2015} measured the
nucleus-to-cytoplasm fluorescence ratio after $t =  1\text{h}$, which directly gives the concentration ratio
$c_{\text{in}}/c_{\text{out}}$.  
With finite nuclear and cytoplasmic volumes, the corresponding rate
constant from Eq.~\ref{eq:rate_constant_2} is

\begin{equation}
    k = t^{-1}\ln\left(
        \cfrac{\cfrac{c_{\text{in}}(t)}{c_{\text{out}}(t)} \cfrac{V_{\text{in}}}{V_{\text{out}}} +1}
             {1 - \cfrac{c_{\text{in}}(t)}{c_{\text{out}}(t)}}
        \right)
\end{equation}

The single-pore resistance extracted from Eq.~\ref{eq:rate_constant_2}
for nucleus-cytoplasm equilibration is
\begin{equation*}
  R_{\text{exp}}
  = \frac{N_{\text{pores}}}{k}
    \bigl(V_{\text{in}}^{-1}+V_{\text{out}}^{-1}\bigr),
\end{equation*}
whereas for digitonin-permeabilised nuclei equilibrating with a large
external reservoir (Eq.~\ref{eq:rate_constant}) it reduces to
\begin{equation*}
  R_{\text{exp}}
  = \frac{N_{\text{pores}}}{k}\,V_{\text{in}}^{-1}.
\end{equation*}

Because NPCs operate on the nanometer scale, the resulting resistances are extremely small, typically
$R\sim10^{20}\,\text{s}/\text{m}^{-3}$.  
To express these values more intuitively, we convert them to a
translocation rate per one NPC, defined as the number of molecules that cross one pore per second under a unit concentration gradient of $\Delta c = 1\;\mu\text{M}$ (cf.~\cite{Ribbeck2001}):
\begin{equation}
    \begin{array}{c}
        \text{Translocation rate} \\
        \text{per one NPC}\\
        \text{at} \, \Delta c = 1\;\mu\text{M}
    \end{array}
    \hspace{-1em}= \,\frac{N_{\mathrm{A}}}{R}\,10^{-3}\;\text{s}^{-1},
\end{equation}

where \(N_{\mathrm{A}}\) is Avogadro's number.

%\textbf{}

\textbf{Translocation rate versus inert probe colloid molar weight.}

\begin{figure}[H]
    \centering
    \includegraphics[width=3.2in]{fig/flux_vs_MW_SI.png}
    \caption{%
    Single-NPC translocation rate plotted against probe molar mass:
    comparison of three models.  
    \textit{Diffusive barrier} (gray)-
    the FG mesh lowers the diffusivity but imposes no net insertion free
    energy;  
    \textit{Rigid barrier} (grey)-
    the pore is treated as an impermeable channel $\approx$\,4-5\,nm in
    diameter (cut-off $\approx$\,40-50\,kDa), so larger inert colloids are almost completely
    excluded;  
    \textit{Soft barrier} (orange, this work)-
    the FG domains are modelled explicitly, combining size-dependent
    diffusivity with the calculated insertion free energy.  
    Shaded symbols represent experimental rates; shaded bands indicate parameter
    uncertainty $1.0 \le \rho_{\text{probe}} \le 1.4\;\text{g/cm}^{-3}$.
    }
    \label{fig:flux_vs_MW}
\end{figure}

We compare our theoretical predictions with experimental translocation data—both the original measurements of Ribbeck \textit{et al.}\,\cite{Ribbeck2001} and values we deduced from other studies \cite{Mohr2009,Popken2015,Timney2016,Frey2018} (see Table~\ref{tbl:experimental}).  
For the calculations we adopt the number of pores per nucleus, nuclear and cytoplasmic volume reported for digitonin-permeabilised HeLa cells \cite{Ribbeck2001} and for \textit{S.\,cerevisiae} nuclei \cite{Timney2016} (Table~\ref{tbl:experimental}), the inner diameter and the channel length of the pores is set to be $r_{\text{pore}}\approx 40 \text{nm}$ (Figure~1).
The buffer viscosity at 20$^\circ$C is taken as $\eta_{\text{S}} = 1.45\times10^{-3}\,\text{Pa\,s}$, matching the assay composition (20 mM HEPES-KOH, pH 7.4; 120 mM KOAc; 5 mM Mg(OAc)$_2$; 0.5 mM EGTA; 250 mM sucrose) \cite{Ribbeck2001}.

Throughout we assume the nuclear-pore brush is in a moderately poor solvent, $\chi_{\text{PS}} = 0.6$. \todo{[add reference or brief rationale]}

All probe particles are treated as spheres with densities in the range $1.0 \le \rho_{\text{probe}} \le 1.4\;\text{g/cm}^{-3}$.  Their
diameter (in nm) is estimated from the molar mass $M_{w}$ (in kDa) via
\begin{equation}
  d_{\text{probe}}
  = 10^{8}
    \left(
      \frac{6}{\pi}\,
      \frac{M_{w}}{N_{\text{A}}\rho_{\text{probe}}}
    \right)^{\!1/3},
  \label{eq:d_probe}
\end{equation}
where $N_{\text{A}}$ is Avogadro's number; the corresponding molecular masses are listed in Tables~\ref{tbl:inert_probes} and~\ref{tbl:attr_probes}.

In Figure 8a the scatter represent the experimental translocation rates extracted for inert probes (Table~\ref{tbl:inert_probes}).
The matching shaded band shows our theoretical prediction over the size range $1 \le d \le 10$ nm, bounded by the lowest and highest probe densities.
For reference we also plot the rates expected for a bare pore, using Eq.~(3), again with a band bracketing the density range.

\todo{Here to write that rigid barrier does not fit the data well, and about correction of wall drag, as it is important here.
They have similar comparison in Timney et al \cite{Timney2016}.}

\begin{equation}
    R = \frac{LK\{\frac{d}{2r_{\text{pore}}}\}}{D_0 \pi (r_{\text{p}}^0-r_{\text{p}})^{2}} + \frac{1}{2 D_0 (r_{\text{p}}^0-r_{\text{p}})}
    \label{eq:resistance}
\end{equation}
where $K\{\frac{d}{2r_{\text{pore}}}\}$ account for wall drag and found from \cite{Haberman1958}.

\todo{Also reduced by the meshwork diffusivity does not explain the trend, even if $\beta = 8$ instead of $\beta = 5.5$. We change $\beta$ to fit trend to smaller particles}

\textbf{Translocation rate versus partition coefficient in FG-domain hydrogel correlation.}

\begin{figure}[H]
    \centering
    \includegraphics[width=3.5in]{fig/flux_vs_PC_SI.png}
    \caption{%
    Gating of colloids by their affinity as in Figure~8b ($\rho_{\text{probe}} =1.3 \text{g/cm}^{-3}$).
    Shaded bands indicate parameter uncertainty $1.0 \le \rho_{\text{probe}} \le 1.4\;\text{g/cm}^{-3}$.
    }
    \label{fig:flux_vs_PC}
\end{figure}

For particles with high affinity to the polymer (low $\chi_{\text{PC}}$), we expect enhanced permeability compared to inert particles, possibly even surpassing that of a bare pore.

Frey \textit{et al}~\cite{Frey2018} examined permeability of similarly sized colloidal particles with varying surface properties, ranging from inert to FG-domain-affine.
To characterize these surfaces, FG-domain hydrogels were prepared to study colloid partitioning.

The authors prepared droplets of Mac98A FG-domains via rapid dilution, inducing phase separation. 
%following the protocol of Ref.~\cite{Schmidt2015}.
The estimated protein concentration in the gel phase is $\approx 275 \, \text{mg}/\text{ml}$, which corresponds to a volume fraction $\phi_{\text{gel}} \approx 0.2$, assuming a typical dry protein density of $1.35 \, \text{g}/\text{ml}$.

Similarly, droplets of Nup116 FG-domains were prepared with an estimated intra-particle protein concentration of $\phi_{\text{gel}} \approx 400 \, \text{mg}/\text{ml}$, corresponding to $\phi_{\text{gel}} \approx 0.4$. A compromise value $\phi_{\text{gel}} = 0.3$ was used in Figure~\todo{Main text figure}.
Within the range $0.2 < \phi_{\text{gel}} < 0.4$, the results are not sensitive to the exact value of $\phi_{\text{gel}}$.

Upon dilution, FG-domains undergo phase separation due to chain cohesiveness, demixing into a dilute FG-poor phase and a condensed FG-rich gel.
The polymer concentration in the bulk phase, $\phi_{\text{gel}}^{\text{out}}$, approximately equals the critical concentration for phase separation.

For a two-component system (FG-domain + buffer), phase separation occurs at:
$
\chi_{\text{PS}}^{\text{FG}} > \frac{1}{2} + \frac{1}{\sqrt{N}} \gtrsim 0.5
$
according to mean-field theory (assuming $N \gg 1$). 
Both chemical potentials and osmotic pressures are equal in the rich and poor phases \cite{Vovk2016, Zilman2018}.

The estimated critical concentration for Nup116 and Nup98A FG-domains is $1 \, \mu\text{g}/\text{ml}$, corresponding to \mbox{$\phi_{\text{gel}}^{\text{out}} \approx 10^{-6}$} \cite{Schmidt2015}.
This extremely low value implies the osmotic pressure is effectively zero in both phases:

\begin{eqnarray}
    \Pi(\phi_{\text{gel}}) = \Pi(\phi_{\text{gel}}^{\text{out}}) \approx 0
\end{eqnarray}

This allows us to estimate the solvent quality $\chi_{\text{PS}}^{\text{FG}}$ using the condition for vanishing osmotic pressure Eq.~(\todo{EQ}):

\begin{eqnarray}
    \chi_{\text{PS}}^{\text{FG}} = \frac{-\ln(1-\phi_{\text{gel}}) - \phi_{\text{gel}}}{\phi_{\text{gel}}^2}
\end{eqnarray}

Assuming that FG-domains in the gel are chemically similar to those in nuclear pores, we take the polymer-colloid interaction to be unchanged: $\chi_{\text{PC}}^{\text{gel}} \approx \chi_{\text{PC}}$.

Combining this information, we estimate the insertion free energy $\Delta F_{\text{gel}}$ for particles with different surface properties (i.e., varying $\chi_{\text{PC}}$) into an FG-domain gel of volume fraction $\phi_{\text{gel}}$.
Since the osmotic pressure is negligible, only the surface interaction contributes:

\begin{eqnarray}
    \Delta F_{\text{gel}} =
    \frac{\pi d^3}{6} \cdot \gamma\left(
    \phi_{\text{gel}}, \chi_{\text{PC}},
    \chi_{\text{PS}}^{\text{FG}}\{ \phi_{\text{gel}} \}
    \right) \\
    \text{PC}_{\text{gel}} = \left(\frac{c_{\text{in}}}{c_{\text{out}}}\right)_{\text{gel}} = e^{-\Delta F_{\text{gel}}}
\end{eqnarray}

Owing to the uncertainty in the volume fraction of the FG-particles, $\phi_{\text{gel}}$, we adopt a compromise value of $\phi_{\text{gel}} = 0.30$ in Figure 8b,.  
With an average probe molar mass of \(\sim\!28\;\text{kDa}\) the particle diameter is estimated to lie in the range  $4.0\;\text{nm} \lesssim d\cdot a \lesssim 4.5\;\text{nm}$.
Because our theoretical model is implemented for even integer diameters expressed in segment lengths ($a = 0.76\;\text{nm}$), we use the
nearest value, $d = 6$.


\end{multicols}
\pagebreak

\begin{table}[htp]
    %Better assemble the table somewhere else and export as latex for the better layout
    \centering
    \resizebox{0.8\linewidth}{!}{
        \begin{tabular}{p{2.6cm}|p{1.9cm}|p{8cm}|p{0.7cm}|p{0.7cm}|p{0.4cm}}
            %\hline
            Study & Cell culture & Reported quantity & $\!\!N_{\text{pores}}$ & $V_{\text{in}}$ $[f\text{l}]$ & $\!V_{\text{out}}$ $[f\text{l}]$ \\
            \hline
            Ribbeck \textit{et al.}\cite{Ribbeck2001} & HeLa & Translocation rate per one NPC at $\Delta c = 1 \mu\text{M}$ & 2770 & 1130 & $\infty$ \\
            Mohr \textit{et al.}\quad\,\cite{Mohr2009} & HeLa & Rate constant $k$ & 2770 & 1130 & $\infty$ \\
            Popken \textit{et al.}\,\,\cite{Popken2015} & \textit{S. cerevisiae} & Nuclear/cytoplasm concentration $c_{\text{in}}/c_{\text{out}}$ at  $t\! =\!1\text{h}$  & 161 & 4.8 & 60 \\
            Timney \textit{et al.}\cite{Timney2016} & \textit{S. cerevisiae} & Characteristic time $\tau$ & 161 & 4.8 & 60 \\
            Frey \textit{et al.}\quad\;\;\cite{Frey2018} & HeLa & Rate constant $k$, partitioning in FG hydrogel $\text{PC}_{\text{gel}}$ & 2770 & 1130 & $\infty$ \\
            %\hline
        \end{tabular}
        }
        \caption{Transport-related quantities extracted from the experimental studies.}
        \label{tbl:experimental}
\end{table}

\begin{multicols}{2}

\begin{table}[H]
\resizebox{\linewidth}{!}{
\begin{tabular}{p{3cm}|p{1cm}|p{2cm}|p{2.6cm}}
Probe & Molar weight [kDa] & Translocation rate $[\text{s}^{-1}]$ \mbox{per one NPC} \mbox{at $\Delta c = 1\mu\text{M}$}& Study \\
\hline
Fluorescein-Cys & 0.5 & 231 & Mohr \textit{et al} \\
11 aa peptide & 1.4 & 130 & Mohr \textit{et al} \\
Insulin & 5.8 & 59 & Mohr \textit{et al} \\
Aprotinin & 6.5 & 21.1 & Mohr \textit{et al} \\
\todo{Profilin} & ? & 13.4 & Mohr \textit{et al} \\
Ubiquitin & 8.5 & 8.7 & Mohr \textit{et al} \\
z-domain & 8.2 & 9.9 & Mohr \textit{et al} \\
Thioredoxin & 13.9 & 5.0 & Mohr \textit{et al} \\
Lactalbumin & 14.2 & 3.54 & Mohr \textit{et al} \\
GFP & 27.0 & 0.50 & Mohr \textit{et al} \\
PBP & 37.0 & 0.064 & Mohr \textit{et al} \\
MBP & 43.0 & 0.054 & Mohr \textit{et al} \\
GFP-HIS & 26.8 & 1.11 & Timney \textit{et al} \\
GFP-1PrA & 34.2 & 0.268 & Timney \textit{et al} \\
GFP-2PrA & 40.7 & 0.146 & Timney \textit{et al} \\
GFP-3PrA & 46.8 & 0.092 & Timney \textit{et al} \\
GFP-4PrA & 53.6 & 0.067 & Timney \textit{et al} \\
GFP-6PrA & 66.8 & 0.040 & Timney \textit{et al} \\
GFP-1PrG & 34.7 & 0.83 & Timney \textit{et al} \\
GFP-2PrG & 42.3 & 0.25 & Timney \textit{et al} \\
MGM & 109 & 0.0091 & Popken \textit{et al} \\
MGM2 & 149 & 0.00250 & Popken \textit{et al} \\
MGM4 & 230 & 0.00116 & Popken \textit{et al} \\
MG3 & 122 & 0.0052 & Popken \textit{et al} \\
MG4 & 150 & 0.00368 & Popken \textit{et al} \\
MG5 & 177 & 0.00197 & Popken \textit{et al} \\
BSA & 68 & $<0.1^{*}$ & Ribbeck \textit{et al} \\
GFP & 29.0 & $2^{*}$ & Ribbeck \textit{et al} \\
mCherry & 28.0 & 0.140 & Frey \textit{et al} \\
EGFP & 28.0 & 0.49 & Frey \textit{et al} \\
efGFP\_8R & 30.0 & 2.66 & Frey \textit{et al} \\
sffrGFP4 18xR→K & 28.0 & 0.53 & Frey \textit{et al} \\
sffrGFP4 25xR→K & 27.0 & 0.224 & Frey \textit{et al} \\
MBP & 43.0 & 0.0112 & Frey \textit{et al} \\
MBP K→R & 43.0 & 0.45 & Frey \textit{et al} \\
\end{tabular}
}
\caption{Translocation rates of inert probe colloids calculated from the experimental data.
Data marked with $^{*}$ are taken directly from the source without modification.}
\label{tbl:inert_probes}
\end{table}

\begin{table}[H]
\resizebox{\linewidth}{!}{
\begin{tabular}{p{3cm}|p{1cm}|p{2cm}|p{1.2cm}|p{1.2cm}}
Probe & Molar weight [kDa] & Translocation rate $[\text{s}^{-1}]$ \mbox{per one NPC} \mbox{at $\Delta c = 1\mu\text{M}$}& PC in Mac98A & PC in Nup116 \\
\hline
yNTF2 (dimer) & 28 & 84 & 290 & 1400 \\
rNTF2 (dimer) & 28 & 95 & 3200 & 13000 \\
mCherry & 28 & 0.140 & - & 0.09 \\
EGFP & 28 & 0.49 & 0.11 & 0.33 \\
efGFP\_0W & 28 & 0.84 & 0.09 & 0.42 \\
efGFP\_3W* & 27.5 & 11.5 & 1.50 & 14 \\
efGFP\_5W* & 28 & 12.2 & 2.20 & 15 \\
efGFP\_8F* & 28.3 & 6.0 & 8.30 & 51 \\
efGFP\_8L* & 26.8 & 4.6 & 1.80 & 10 \\
efGFP\_8I* & 28.2 & 9.5 & 2.90 & 23 \\
efGFP\_8M* & 28.2 & 15.4 & 3 & 17 \\
efGFP\_8R & 30 & 2.66 & 0.50 & 1.90 \\
sffrGFP4 & 29 & 22.4 & 14 & 50 \\
sffrGFP4 & 29 & 22.4 & 14 & 50 \\
sffrGFP4 18xR→K & 28 & 0.53 & 0.12 & 0.31 \\
sffrGFP4 25xR→K & 27 & 0.22 & 0.06 & 0.10 \\
sffrGFP4 & 29 & 22.4 & 14 & 50 \\
sffrGFP4 & 29 & 22.4 & 14 & 50 \\
sffrGFP5 & 28 & 9.0 & 0.67 & 5.50 \\
sffrGFP6 & 29 & 39.2 & 100 & 160 \\
sffrGFP7 & 29 & 43 & 200 & 200 \\
GFP\_MaxR\_5W* & 28 & 116 & 2100 & 4000 \\
GFP\_MaxR\_8i* & 27.6 & 182 & 2000 & 4100 \\
GFPNTR\_2B7* & 27.1 & 224 & 1200 & 1600 \\
GFPNTR\_7B3* & 26.3 & 238 & 1700 & 1400 \\
GFPNTR\_3B1* & 28.5 & 60 & 290 & - \\
GFPNTR\_3B7* & 27.5 & 106 & 3000 & - \\
GFPNTR\_3B8* & 27.5 & 94 & 1800 & - \\
GFPNTR\_3B9 & 28.3 & 122 & 4800 & - \\
\end{tabular}
}
\caption{Translocation rates probe colloids with different surface features calculated from the experimental data and their partition coefficient in the FG-gel taken directly from the source without modification from Frey \textit{et al}.\cite{Frey2018}.
Probe colloids marked with $^{*}$ reported to form oligomers, molar weights presented for monomer state.
}
\label{tbl:attr_probes}
\end{table}


\end{multicols}


\pagebreak
\subsection*{List of variables and abbreviations}

%\begin{multicols}{2}
\begin{tabularx}{\linewidth}{l X}
    % \toprule
    % \textbf{Variable} & \textbf{Definition} \\
    % \midrule
    $a$ & Kuhn segment length \\
    $b_0$, $b_1$ & Polymer depletion/accumulation correction coefficients \\
    $c$ & Local concentration of diffusing colloid particles in the steady state \\
    $c_{\text{N}}$ & Concentration in the neighboring control volume \\
    $c_{\text{n}}$ & Concentration at the face between the current and neighboring control volumes \\
    $c_{t}$ & Concentration in a control volume at time $t$ \\
    $d$ & Diameter of the spherical or cylindrical colloid particle \\
    $D$ & Local diffusion coefficient of colloid particles \\
    $D_0$ & Diffusion coefficient of colloid particles in pure solvent \\
    $D_{\text{N}}$ & Local diffusion coefficient in the neighboring control volume \\
    $D_{\text{n}}$ & Local diffusion coefficient at the face between the current and neighboring control volumes \\
    $\delta r$, $\delta z$ & Size of the discretization step in radial and axial directions \\
    $\Delta F$ & Analytical insertion free energy penalty to place a spherical particle \\
    $\Delta F_{\text{cyl}}$ & Analytical insertion free energy penalty to place a cylindrical particle \\
    $\Delta F_{\text{cyl}}^{\text{osm}}$ & Osmotic contribution to $\Delta F_{\text{cyl}}$ \\
    $\Delta F_{\text{cyl}}^{\text{sur}}$ & Surface contribution to $\Delta F_{\text{cyl}}$ \\
    $\Delta F_{\text{osm}}$ & Osmotic contribution to $\Delta F$ for a spherical particle\\
    $\Delta F_{\text{SF-SCF}}$ & Insertion free energy penalty to place a cylindrical particle calculated using the Scheutjens-Fleer approach \\
    $\Delta F_{\text{sur}}$ & Surface contribution to $\Delta F$ for a spherical particle\\
    $\Delta \phi$ & Change in polymer segment volume concentration due to the presence of a colloid particle \\
    $p(x_z)$ & Prefactor introduced to correct the difference between the resistance calculated using half-cylinder shells and oblate spheroid shells \\
    $G$ & Statistical weight of a subchain \\
    $\gamma$ & Surface tension coefficient \\
    $h_{r}$, $h_{z}$, $h_{\theta}$ & Lam\'e coefficients (scale factors) in curvilinear coordinate transformations \\
    $i$, $k$ & Indices of the discretized grid in radial $r$ and axial $z$ directions, respectively \\
    $J$ & Net flux of colloid particles through the pore in the steady state \\
    $j$ & Colloid particle flux density in the steady state \\
    $j_{\text{diffusion}, n}$ & Diffusive flux density between control volumes in direction $n$ \\
    $j_{\text{drift}, n}$ & Drift flux density between control volumes in direction $n$ due to the insertion free energy gradient \\
    $j_{n}$ & Total flux density between control volumes in direction $n$ \\
    $k_B$ & Boltzmann constant \\
    $L_{0}$ & Membrane thickness \\
    $L$ & Effective length of the pore considering volume exclusion \\
    $\lambda_{n}$ & Ratio of neighboring control volume sizes in direction $n$ \\
    $N$ & Number of Kuhn segments in the brush-forming chains \\
    $\Pi$ & Flory mean-field local osmotic pressure \\
    $\phi$ & Local volume fraction of polymer segments in a polymer brush \\
    $\phi^{\ast}$ & Apparent local volume fraction of polymer segments \\
    $\phi^{\text{ins}}$ & Local volume fraction of polymer segments disturbed by an inserted particle \\
    $\psi$ & Scalar potential function introduced to express the steady state solution \\
    $\rho$ & Local resistivity (inverse conductivity) to colloid particle diffusion \\
    $r$ & Radial coordinate in cylindrical coordinates \\
    $r_{\text{base}}$ & Base radius of half-cylinder shells used in resistance calculations\\
    $r_{\text{p}}^{0}$ & Radius of the pore \\
    $r_{\text{p}}$ & Effective radius of the pore considering volume exclusion \\
    $r_{c}$ & Radial coordinate of the colloid particle center in cylindrical coordinates \\
    $R$ & Total resistance of the pore to colloid particle diffusion in a semi-infinite solution \\
    $R^{\text{empty}}$ & Total resistance of an empty pore (without polymer brush) \\
    $R_{(z, \pm\infty)}$ & Resistance from position $z$ to infinity along positive or negative $z$-direction \\
    $R_{\text{ext}}$ & Convergent flow contribution to the total resistance \\
    $R_{\text{int}}$ & Resistance contribution from the pore channel \\
    $\varrho_{z}$ & Resistance per unit length at position $z$ \\
    $\varrho_{z}^{0}$ & Resistance per unit length at position $z$ for an empty pore \\
    $\sigma$ & Grafting density (number of polymer chains per unit area) \\
    $T$ & Temperature \\
    $\theta$ & Angular coordinate in cylindrical coordinates \\
    $u$ & Segment potential in Scheutjens-Fleer method\\
    $\xi$ & Correlation length in a semi-dilute polymer solution \\
    $z$ & Axial coordinate in cylindrical coordinates \\
    $z_a$, $z_b$ & Integration limits along the $z$-axis \\
    $z_{c}$ & Axial coordinate of the colloid particle center in cylindrical coordinates \\
    $\bm{1}_{V}$, $\bm{1}_{S}$ & Volume and surface indicator functions \\
    $\bm{e}_r$, $\bm{e}_z$ & Local covariant basis vectors in curvilinear coordinates \\
    $\bm{J}$ & Jacobian matrix \\
    $\bm{S}\{r_{c}\}$ & Particle surface projection matrix for a spherical particle with center at $r_{c}$ \\
    $\bm{S}_{\theta\downarrow}$ & Particle surface projection onto the $rz$-plane for a spherical particle with center at $r_{c}$ \\
    $\bm{T}$ & Coordinate system transformation matrix \\
    $\bm{V}$ & Boolean array representing the presence of the spherical particle in the discretized grid \\
    $\bm{V}\{r_{c}\}$ & Particle volume projection matrix for a spherical particle with center at $r_{c}$ \\
    $\bm{V}_{\theta\downarrow}$ & Particle volume projection onto the $rz$-plane for a spherical particle with center at $r_{c}$ \\
    $\bm{W}$ & Boolean array representing the membrane walls in the discretized grid \\
    $\bm{W}^{\ast}$ & Boolean array representing the space impermeable to the spherical particle \\
    $\chi_{\text{CS}}$ & Flory colloid-solvent interaction parameter \\
    $\chi_{\text{PC}}$ & Flory polymer-colloid interaction parameter \\
    $\chi_{\textrm{PS}}$ & Flory polymer-solvent interaction parameter \\
    %\bottomrule
\end{tabularx}
%\end{multicols}

\printbibliography
\end{document}