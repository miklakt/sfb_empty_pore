\documentclass[10pt, a4paper, twocolumn]{article}
\raggedbottom
\usepackage{geometry}
\geometry{
    a4paper,
    left= 15mm,
    right = 15mm,
    top=20mm,
    bottom = 25mm,
    }
\usepackage{caption}
\captionsetup[figure]{
    font=small, 
    labelfont=bf
    }

\usepackage{graphicx}
\usepackage{amsmath, amssymb, amsfonts, mathtools}
\usepackage[
    backend=biber,
    natbib=true,
    style=numeric,
    sorting=none
    ]{biblatex}
\usepackage{xcolor}
\usepackage{bm}
\usepackage{multicol}
\usepackage{ltablex}
\usepackage{wrapfig}
\usepackage{float}
\usepackage[utf8]{inputenc}

%\REMOVE BEFORE SUBMISSION
\usepackage[right]{lineno}
\linenumbers


\usepackage{titlesec}
\titleformat{\section}[block]
  {\normalfont\bfseries\fontsize{12pt}{12pt}}  % Style
  {Supplementary Method \thesection.} % Label
  {0.5em}                 % Spacing between label and title
  {}                      % Before-code (empty)
  %[\vspace{0.5em}]        % After-code (optional vertical space)

\newcommand\todo[1]{\textcolor{red}{#1}}

\newcommand\scalemath[2]{\scalebox{#1}{\mbox{\ensuremath{\displaystyle #2}}}}
\newcommand{\sign}{\text{sign}}

% Add S prefix to figures, equations and tables
\renewcommand{\thefigure}{S\arabic{figure}}
\renewcommand{\theequation}{S\arabic{equation}}
\renewcommand{\thetable}{S\arabic{table}}
\makeatletter
\renewcommand{\fnum@table}{\textbf{Table~\thetable}}
\makeatother


\addbibresource{biblio.bib}
\onecolumn

\title{SUPPLEMENTARY INFORMATION}

\author{}
\date{}

\begin{document}
\maketitle


\begin{center}
    \textbf{This file includes:} Supplementary Methods 1 to 8, and Supplementary References.
\end{center}
\twocolumn

\pagebreak
%%%%%%%%%%%%%%%%%%%%%%%%%%%%%%%%%%%%%%%%%%%%%%%%%%%%%%%%%%%%%%%%%%%%%%%%%%%%%
\section{Estimates of nucleopore geometry and polymer filling}

\todo{Ralf to add a short description of what is known about the nucleopore geometry, and the length and number of FG domains in the nucleopore, and a comparison with the assumptions in our model.}

\pagebreak
\section{Computing polymer density maps with the Scheutjens-Fleer self-consistent field (SF-SCF) numerical method}


\begin{figure}[h]
    \centering
    \includegraphics[width = 0.95\linewidth]{fig/sf-scf_scheme.png}
    \caption{
    Left - Three-dimensional schematic of the lattice and geometrical features of the cylindrical pore model in the SF-SCF method.
    Right - Key with color code of the modeled objects (top) and their representation on a discrete two-gradient lattice (bottom).
    }
    \label{fig:sf-scf_scheme}
\end{figure}

%COMMENT RR: This section looks very detailed, and some aspects, e.g. with regard to Eq. (S2) seem to be standard. Can these be omitted, stasting instead an authoritative reference?

The SF-SCF numerical method was deployed to this end, with pore, polymer and colloid parameters as defined in Figure~\ref{fig:sf-scf_scheme} to calculate polymer spatial distribution $\phi(r,z)$ and as reference insertion free energy for cylindrical colloids to build an approximate analytical model.
The method is based on the minimization of the excess Helmholtz energy with the constraint that all volume concentrations sum up to unity (incompressibility condition).
Space is discretized into a regular lattice, and the temporal average of the polymer volume fraction is calculated at each lattice site, such that the corresponding equilibrium distribution of polymers minimizes the overall system free energy.

As our problem exhibits axial symmetry, space was discretized into a cylindrical lattice with a degenerate angular direction,
implemented as a homogeneously curved two-gradient lattice defined by longitudinal $z$ and radial $r$ coordinates (Figure~\ref{fig:sf-scf_scheme}).
This $rz$ coordinate system is visualized as a two-dimensional Cartesian coordinate system (Figure~\ref{fig:sf-scf_scheme}, right); however, each element of the lattice represents a square toroid (of volume $2 \pi r a^3$) instead of a square (of area $a^2$).
The mean-field approximation is applied in the angular direction, meaning properties in the angular direction are uniform.

The membrane and any colloids were coarse-grained.
The membrane was modeled as a toroid with a rectangular cross-section of physical height $L_0$, a physical inner radius $r_{\text{p}}$, and an outer radius large enough to be effectively infinite with respect to the polymer distribution (Figure~\ref{fig:sf-scf_scheme}, green).
The colloid was modeled as a cylinder with height and diameter $d$ (Figure~\ref{fig:sf-scf_scheme}, yellow).
Membrane and colloid lattice elements were modeled as impermeable to the solvent and the polymers, illustrated as crossed-out cells in matching colors in Figure~\ref{fig:sf-scf_scheme} (right).

Each polymer chain was represented as a freely jointed chain (FJC) with $N$ segments of length $a$, modelled as a step-weighted random walk on the lattice (Figure~\ref{fig:sf-scf_scheme}, right, red circles with black connecting lines).
The weights of each step direction were set according to the boundary conditions and the lattice curvature.
Steps into impermeable lattice elements had zero weight.
Other steps were weighted according to the change in volume per lattice element, with steps towards increasing $r$ consequently being favoured, and steps towards decreasing $r$ being disfavoured, compared to steps along $z$.
The resulting local polymer concentration $\phi$ is a weighted sum of all possible paths the chain can take, Eq.~(\ref{eq:sum_to_phi}).

The excess Helmholtz energy was minimized through a Lagrangian with multipliers $\alpha(r,z)$:
\begin{equation}
    \label{eq:fe_lagrangian}
    \begin{aligned}
        &F[\bm{u}, \boldsymbol{\phi}, \boldsymbol{\alpha}] =\\
        &= F_{\text{ent}}[\bm{u}] - \sum\limits_{r,z} \sum\limits_X u_X(r, z) \phi_X(r, z) + \\
        &+ F_{\text{int}} [\boldsymbol{\phi}] 
        + \sum\limits_{r,z} \alpha(r, z) \left( \sum\limits_X \phi_X(r, z) - 1 \right),
    \end{aligned} 
\end{equation}
where $\phi_X(r,z)$ is the local volume concentration function of segment type $X$ (polymer, colloid, or solvent), $u_X(r, z)$ is the potential field of segment type $X$, the functional $F_{\text{ent}}[\bm{u}]$ is the mixing entropy term, and the functional $F_{\text{int}} [\boldsymbol{\phi}]$ is the interaction term of the free energy.

The condition for the minimum of the functional is a system of three variational equations:
\begin{equation}
    \label{eq:energy_min_system}
    \begin{cases}
        \frac{\partial F}{\partial \boldsymbol{\alpha}} = 0 \\
        \frac{\partial F}{\partial \boldsymbol{\phi}} = 0 \\
        \frac{\partial F}{\partial \bm{u}} = 0 \\
    \end{cases}
\end{equation}
The first condition in Eq.~(\ref{eq:energy_min_system}) ensures system incompressibility.

The second condition in Eq.~(\ref{eq:energy_min_system}) results in the segment potential field equation for a regular solution:
\begin{equation}
    \label{eq:u-phi}
    u_A(r, z) =\sum\limits_{B} \chi_{A,B} \left(\phi_B(r,z) - \phi_B^b \right) + \alpha(r, z),
\end{equation}
where $\chi_{A,B}$ is the Flory interaction parameter between segments $A$ and $B$, and $\phi_B^b$ is the volume fraction of $B$ in the bulk (equal to 1 for the solvent and zero otherwise).

The third condition in Eq.~(\ref{eq:energy_min_system}) (minimization with respect to potentials) links the chain partition function with the local polymer concentration $\phi$ in a diffusion-like equation (Eq.~(\ref{eq:propagation})).
Any subchain of the FJC can be considered a Markov process starting at some segment $s_i$ at coordinates $r_i, z_i$ that goes through intermediate steps to segment $s_k$ at coordinates $r_k, z_k$ (Figure~\ref{fig:sf-scf_scheme}, bottom right).
Such a process has a statistical weight $G(\{r_k, z_k\}, s_k | \{r_i, z_i\}, s_i)$.

All the Markov processes that start with segment $s_i$ and end with segment $s_k$ at fixed coordinates $\{r, z\}$ are found as the sum over all possible starting coordinates:
\begin{equation}
    \label{eq:sum_to_phi}
    G(\{r, z\}, s_k | s_i) = \sum_{r^{\prime}, z^{\prime}} G(\{r, z\}, s_i | \{r^{\prime}, z^{\prime}\}, s_i)
\end{equation}


The statistical weight of all possible processes that start from segment $s_i$ and end with segment $s_k$ is the sum over all possible coordinates:
\begin{equation}
    G(s_k | s_i) = \sum_{r, z} G(\{r, z\}, s_k | s_i)
\end{equation}

When $s_i=1$ and $s_k=N$, the result contains the statistical weight of all possible and allowed conformations of the chain and is the single-chain partition function $G(N|1)$.

Let $G(r, z) = G(\{r, z\}, 1|1)$ be the initial condition of the Markov process, which contains just one segment (starts and ends at segment $1$).
The segment potential $\bm{u}$ acts on this segment; thus, Boltzmann statistical weights are applied:
\begin{equation}
    G(r, z) = \exp(-u(r,z))
\end{equation}

The volume density distribution of segment $s_i$ at coordinates $r, z$ is found from the composition law:
\begin{equation}
    \label{eq:propagation}
    \begin{aligned}
        &\phi(\{r, z\}, s_i) = \\
        &\frac{2 \pi r_{\text{p}}^{0} \sigma N}{G(N|1)}
        \frac{G(\{r, z\}, s_n | 1) G(\{r, z\}, (N - s_n + 1) | 1)}{G(r, z)}
    \end{aligned}
\end{equation}
where $G(\{r, z\}, s_n | 1)$ and $G(\{r, z\}, (N - s_n + 1) | 1)$ are forward and backward propagators, respectively; $\sigma$ is the grafting density.

Finally, the volume concentration at coordinates $r, z$ is found as the sum over all chain segments:
\begin{equation}
    \phi(r, z) = \sum_{i}^{N} \phi(\{r, z\}, s_i)
\end{equation}

The numerical algorithm solves the Scheutjens-Fleer system of nonlinear equations such that the segment potential $\bm{u}$ is consistent with the volume concentration $\boldsymbol{\phi}$.
The relationship between the segment potential and the volume concentration is defined in Eq.~(\ref{eq:u-phi}).

The SF-SCF scheme can be summarized as:
\begin{equation}
    \boldsymbol{u}[\boldsymbol{\phi}] \xleftrightarrow[]{\sum_{X} \phi_X = 1} \boldsymbol{\phi}[\boldsymbol{u}]
\end{equation}

For the calculations, we used the package \emph{SFbox} developed in the University of Wageningen.
%COMMENT RR: A reference to the literature for details on this software is needed here.

The package contains several Newton/Quasi-Newton optimization routines to perform the minimization of the functional Eq.~(\ref{eq:fe_lagrangian}).
Each iteration returns a new improved approximation to the segment potential $\bm{u}$ and updates the volume concentrations $\bm{\phi}$.
The routine was looped until the desired accuracy was reached.




%%%%%%%%%%%%%%%%%%%%%%%%%%%%%%%%%%%%%%%%%%%%%%%%%%%%%%%%%%%%%%%%%%%%%%%%%%%%%
% \pagebreak
\section{Extracting surface and volume contributions to the colloid insertion free energy from SF-SCF data}



The two-gradient SF-SCF numerical model limits the cylindrical colloid to be positioned coaxially along the $z$-axis with the coordinate of the colloid center $z_{\text{c}}$.
The arising insertion free energy $\Delta F_{\text{cyl}}(z_{\text{c}})$ is composed of the volume contribution $\Delta F_{\text{cyl}}^{\text{osm}}(z_{\text{c}})$ and the surface contribution $\Delta F_{\text{cyl}}^{\text{sur}}(z_{\text{c}})$.
We here define how these two contributions are extracted from the SF-SCF data.
In their continuous form, the contributions are:
\begin{equation}
    \Delta F_{\text{cyl}}^{\text{osm}}(z_{\text{c}}) = 2 \pi \int_{z_{\text{c}} - d/2}^{z_{\text{c}} + d/2} \int_{0}^{d/2} \Pi(r,z) \, r \, dr \, dz
\end{equation}
and
\begin{equation}\label{eq:continuous_surf_int}
    \begin{aligned}
        \Delta F_{\text{cyl}}^{\text{sur}}(z_{\text{c}}) = 2 \pi d \int_{z_{\text{c}} - d/2}^{z_{\text{c}} + d/2} \gamma(d/2,z) \, dz +\\
        + \pi \int_{0}^{d/2} \left[ \gamma(z_{\text{c}} - d/2, r) + \gamma(z_{\text{c}} + d/2,r) \right] dr,
    \end{aligned}
\end{equation}
where the first term in Eq.~(\ref{eq:continuous_surf_int}) integrates over the lateral surface, and the second term integrates over the top and bottom faces, of the cylinder.

Following the lattice discretization of SF-SCF outputs with discretization steps $\delta r = \delta z = 1$, we use the indexing: $0 \le i \le d/2-1$ iterates in the direction of the $r$-axis and $0 \le k \le d-1$ iterates in the direction of the $z$-axis, which corresponds to the physical coordinates $r,z \in [0, d/2]\times[z_{\text{c}} - d/2, z_{\text{c}} + d/2]$ (as illustrated in Figure~\ref{fig:sf-scf_scheme}) with $d$ being an even integer to match the lattice.

We define the volume projection matrix $\bm{V}_{\text{cyl}}[d/2 \times d]$ for a cylindrical colloid of size $d$, such that each element of the matrix equals the volume of the colloid contained within the corresponding lattice element:
\begin{equation}
    V_{\text{cyl}}[i, k] = \pi(2i + 1)
\end{equation}
Obviously, the sum of all matrix elements equals the volume of the cylinder.
\begin{equation*}
    \sum_{i=0}^{d/2 - 1} \sum_{k=0}^{d - 1} V_{\text{cyl}}[i, k] = \frac{\pi d^3}{4}
\end{equation*}

Analogously, we define the colloid surface projection matrix $\bm{S}_{\text{cyl}}[d/2 \times d]$, such that each element of the matrix equals the surface area of the colloid within the corresponding lattice element.
\begin{align}
    \begin{split}
        S_{\text{cyl}}[i, k] = 
        &\begin{cases}
            2 \pi i,   & \text{if } i = d/2 - 1 \\
            0,         & \text{otherwise}
        \end{cases} +
        \\
        &+
        \begin{cases}
            2 \pi (i + 1), & \text{if } k = 0 \text{ or } k = d - 1 \\
            0,             & \text{otherwise}
        \end{cases}
    \end{split}
\end{align}
Here, the first term accounts for the top and bottom faces, and the second term accounts for the lateral surface elements, of the cylinder.
Again, the sum of all matrix elements equals the surface area of the cylinder:
\begin{equation*}
    \sum_{i=0}^{d/2 - 1} \sum_{k=0}^{d - 1} S_{\text{cyl}}[i, k] = \frac{3 \pi d^2}{2}
\end{equation*}

Figure~\ref{fig:cylindrical_kernel_SI} provides a color-coded map of the volume $\bm{V}_{\text{cyl}}$ and surface $\bm{S}_{\text{cyl}}$ projection matrices for a selected colloid size ($d = 16$).

\begin{figure}[h]
    \centering
    \includegraphics[width = \linewidth]{fig/cylindrical_kernel_SI.png}
    \caption{
    Volume (left) and surface (right) projection matrices for a cylindrical particle with diameter and height $d = 16$. The element values of the matrices are color-coded, with violet representing zero and yellow the highest values.
    }
    \label{fig:cylindrical_kernel_SI}
\end{figure}

To calculate the colloid insertion free energy, we integrated the osmotic pressure over the colloid volume and the surface tension over the colloid surface (Eq.~(15)).
The equivalent operation on a discrete lattice is the matrix dot product.
The two contributions to the insertion free energy are thus calculated as:
\begin{equation}\label{eq:cyl_fe_osm}
    \begin{split}
        \Delta F_{\text{cyl}}^{\text{osm}}(z_{\text{c}}) = \bm{V}_{\text{cyl}} \cdot \boldsymbol{\Pi}\{z_{\text{c}},\} \text{ and}
        \\
        \Delta F_{\text{cyl}}^{\text{sur}}(z_{\text{c}}) = \bm{S}_{\text{cyl}} \cdot \boldsymbol{\gamma}\{z_{\text{c}}\},
    \end{split}
\end{equation}
where the matrix elements for $\boldsymbol{\Pi}\{z_{\text{c}}\}$ and $\boldsymbol{\gamma}\{z_{\text{c}}\}$ run across $0 \leq i < d/2$ and $z_{\text{c}} - d/2 \leq k < z_{\text{c}} + d/2$.

The discretized insertion free energy profile $\Delta F_{\text{cyl}}$ is obtained from a series of insertion free energy calculations across all possible colloid center positions $z_{\text{c}}$.
Such a series of sequential integrations (matrix dot products) is equivalent to the convolution with the colloid volume/surface projection matrix acting as a kernel:
\begin{equation*}
    \Delta F_{\text{cyl}}^{\text{osm}} = \boldsymbol{\Pi} \ast \bm{V}_{\text{cyl}} \text{ and }
    \Delta F_{\text{cyl}}^{\text{sur}} = \boldsymbol{\gamma} \ast \bm{S}_{\text{cyl}}.
\end{equation*}
Convolution was performed via fast Fourier transform for computational efficiency.




%%%%%%%%%%%%%%%%%%%%%%%%%%%%%%%%%%%%%%%%%%%%%%%%%%%%%%%%%%%%%%%%%%%%%%%%%%%%%
% \pagebreak
\section{Estimating the effective polymer concentration near a colloid from SF-SCF data}

\begin{figure}[]
    \centering
    \includegraphics[width = 0.9\linewidth]{fig/fit_SI.png}
    \caption{
    Comparison of $\Delta F_{\text{SF-SCF}}$ profiles (circles) with $\Delta F_{\text{cyl}}(b_0,b_1)$, for the best-fit values of $b_0 = 0.7$ and $b_1 = -0.3$ (optimally accounting for local perturbations in the polymer concentration; thick solid lines) and for $b_0 = 1.0$ and $b_1 = 0.0$ (neglecting any local perturbations in polymer concentration; thin dashed lines).
    Pore and brush parameters are as given in Figure 1; $d = 4$, $\chi_{\text{PS}} = 0.5$, and $\chi_{\text{PC}}$ values are color-coded and indicated on the right side of the figure.
    The light green hatched area marks values of $z_{\text{c}}$ that are located inside the pore lumen ($|z| \leq 26$).
    }
    \label{fig:fit_SI}
\end{figure}

Depending on the magnitude of the polymer-colloid interaction strength $\chi_{\text{PC}}$, a colloid may attract or repel the polymer around it, causing local perturbations within a distance of a few segments.
In Eq. (17), the effective polymer volume fraction (concentration) near the colloid was approximated as $\phi^{\ast} = (b_0 + b_1 \chi_{\text{PC}}) \phi$.
Here, we justify this approximation based on SF-SCF data and quantify the coefficients $b_0$ and $b_1$, which capture the local perturbation to the polymer volume fraction profile.

As illustrated in Figure~\ref{fig:sf-scf_scheme}, a cylindrical particle is positioned coaxially along the $z$-axis explicitly as impermeable lattice elements.
Consequently, the polymer-brush chains adjust to the available space and the polymer-colloid interaction strength, producing a change in the system's total free energy, $F_{\text{cyl}}(z_{\text{c}})$.

We performed a series of SF-SCF calculations to determine the insertion free energy profiles $\Delta F_{\text{SF-SCF}}(z_{\text{c}})$ for a range of $d$, $\chi_{\text{PS}}$ and $\chi_{\text{PC}}$ values with a pore geometry and polymer brush parameters as set in Figure~1.
% For each condition, we performed a ground state free energy correction, subtracting the total free energy for a colloid far away from the pore to ensure the reference value $\Delta F_{\text{SF-SCF}}(z_{\text{c}} \to \pm \infty) = 0$.

The insertion free energies thus computed were compared to $\Delta F_{\text{cyl}}(z_{\text{c}})$, obtained as described in the previous section with $\gamma_{i,k}$ parametrized according to Eq. (17).
The optimal coefficients $b_0$ and $b_1$ were found using the least-squares method, i.e, minimizing $\sum [\Delta F_{\text{SF-SCF}} - \Delta F_{\text{cyl}}(b_0,b_1)]^2$ across a range of $\chi_{\text{PS}} \in [0,1]$  and $\chi_{\text{PC}} \in [-2,0]$ for a relevant particle positions $-60 \leq z_{\text{c}} \leq 0$.
Fits were performed with small colloids ($d=4$), to focus on local effects and avoid added effects that may arise due to global perturbations of the polymer distribution in the pore.

In a $\theta$-solvent (chosen here as a representative case) and for fixed pore geometry and brush parameters, the three interaction parameters
$\chi_{\text{PC}} = 0.00,; -0.75,; \text{and}; -1.50$ span the regimes of net repulsion, near-critical adsorption, and net attraction, respectively.
Figure~\ref{fig:fit_SI} demonstrates that a satisfactory fit (thick solid line) to the $\Delta F_{\text{SF-SCF}}$ profiles (circles) could be obtained across all three representative datasets with a single parameter set $b_0 = 0.7$ and $b_1 = -0.3$.
Given the overall small volume contribution to the insertion free energy for small particles, the fact that $\Delta F$ remains consistently close to zero for $\chi_{\text{PC}} = -0.75$ illustrates that this interaction parameter is close to the critical value $\chi_{\text{PC}}^{\text{crit}} = \chi_{\text{crit}} + \chi_{\text{PS}} (1 - \phi)$, where the loss in polymer conformational entropy due to surface exclusion is exactly balanced by the free energy gain from favorable monomer–surface interactions. At this point, the surface tension $\gamma$ vanishes \cite{Fleer1993, Birshtein1979, Birshtein1983, Eisenriegler1982}.

In contrast, neglecting local perturbations in the polymer concentration (i.e., using $b_0 = 1.0$ and $b_1 = 0.0$) results in a poor reproduction of the SF-SCF-calculated insertion free energy $\Delta F_{\text{SF-SCF}}$ for both repulsive and attractive colloids ($\gamma \neq 0$), as illustrated by the thin dashed line in Figure~\ref{fig:fit_SI}. This highlights the importance of accounting for local corrections in the theoretical model.
% Consequently, the least-squares fits considered the regimes of net repulsion and net attraction approximately equally through $\chi_{\text{PC}} = 0.00$ and $\chi_{\text{PC}} = -1.50$, respectively.
% In contrast, neglect of local perturbations to the polymer concentration ($b_0 = 1.0$ and $b_1 = 0.0$) reproduces $\Delta F_{\text{SF-SCF}}$ rather poorly for repulsive and attractive colloids (Figure~\ref{fig:fit_SI}, thin dashed line), thus demonstrating the importance of the correction.

\begin{figure}[]
    \centering
    \includegraphics[width = 0.95\linewidth]{fig/fe_scf_grid2.png}
    \caption{ 
    Comparison of $\Delta F_{\text{SF-SCF}}$ profiles (squares) with $\Delta F_{\text{cyl}}(b_0,b_1)$ for the best-fit values $b_0 = 0.7$ and $b_1 = -0.3$ (solid lines), for $d = [8, 12, 16]$ (color coded).
    Pore and brush parameters are as given in Figure 1. $\chi_{\text{PC}} = -1.0$ (top row) and -0.5 (bottom row); the solvent quality was varied near the $\theta$-point with $\chi_{\text{PS}} = [0.4, 0.5, 0.6]$, as indicated.
    The light green hatched area marks values of $z_{\text{c}}$ that are located inside the pore lumen ($|z| \leq 26$).
    \label{fig:fe_scf_grid}
    }
\end{figure}

Although the fit was performed only for small colloids, the resulting parameters $b_{0}$ and $b_{1}$ still successfully account for the local perturbations to the polymer concentration when calculating the insertion free energy of larger colloids.
This is illustrated in Figure~\ref{fig:fe_scf_grid} for two selected $\chi_{\text{PC}}$ values (-0.5 and -1.0) and three selected $\chi_{\text{PS}}$ values (0.4, 0.5 and 0.6).
% We vary the solvent strength $\chi_{\text{PS}}$ around the $\theta$-condition, thereby shifting an attractive particle from repulsion in a moderately good solvent to attraction in a moderately poor solvent.
% Changing $\chi_{\text{PS}}$ in either direction affects only the magnitude of the profile, not its overall trend.

Deviations become notable, however, for $d = 16$ under conditions of strong repulsion or attraction.
This size regime thus marks the limit of validity of the local perturbation approximation.
Instead, attractive or repulsive interactions entail non-local changes that impact the polymer concentration across the entire pore cross-section.
As a consequence, the pore walls also influence colloid insertion. 

% A similar profiles to Figure~\ref{fig:fe_scf_grid} for position-dependent insertion were obtained by Molecular Dynamics in ref \cite{Ananth2018, Tagliazucchi2018}.
Another interesting effect can also observed for strongly attractive particles, even when small.
The SF-SCF results predict that, to reach the minimum of the insertion free energy $\Delta F_{\text{SF-SCF}}$, the polymer brush changes conformation to reach the particle at a greater distance $|z_{\text{c}}|$ from the pore.
In Figure~\ref{fig:fit_SI}, this effect is apparent for $\chi_{\text{PC}} = -1.5$, where the SF-SCF results (blue circles) predict systematically lower insertion free energies than the analytical approach (blue solid line) outside the pore ($|z_{\text{c}}| > 26$).
This effect is mild and only slightly increases the region with negative insertion free energy.
It may entail a slight reduction in the total resistance to diffusive transport: for pores with regions of negative insertion free energy the total resistance is defined by the resistance of the bulk solution (Figure 4), and the capture of colloids by the polymers at a larger distance from the pore reduces the bulk resistivity.



% \begin{figure}[]
%     \centering
%     \includegraphics[width = 0.95\linewidth]{fig/perturbation.png}
%     \caption{ 
%     Perturbation to polymer concentration $\Delta\phi = \phi^{\text{ins}} - \phi$, where $\phi^{\text{ins}}$ is perturbed by the colloid particle polymer concentration.
%     \label{fig:perturbation}
%     }
% \end{figure}



%%%%%%%%%%%%%%%%%%%%%%%%%%%%%%%%%%%%%%%%%%%%%%%%%%%%%%%%%%%%%%%%%%%%%%%%%%%%%
% \pagebreak
\section{Computing insertion free energies for arbitrarily placed spherical colloids}


The SF-SCF method considered in the previous sections is limited to colloids moving along the main axis of the pore.
In reality, colloids may be located off the pore axis.
Here, we generalize the analytical approach to calculate insertion free energies based on volume and surface contributions (assuming localized perturbations to the polymer concentration) to arbitrarily placed colloids.
In doing so, we also change the shape of the colloid, from a cylinder to a simpler sphere.

As the polymer and pore geometrical features are uniform in the angular direction, we use $r, z$ cylindrical coordinates with a degenerate angular axis.
Any property such as the polymer volume fraction can hence be expressed as a function $f(r,z,\theta) = f(r,z)$.

Let the distance to the center of spherical body of a colloid particle be 
\begin{eqnarray}
\Delta_{\text{c}} = \sqrt{r^2 + r_{\text{c}}^2 - 2 r r_{\text{c}} \cos(\theta) + (z - z_{\text{c}})^2},
\end{eqnarray} 
where $r_{\text{c}}, z_{\text{c}}, \theta_{\text{c}}$ are the position of the center in cylindrical coordinates, without the loss of generality, we set $\theta_{\text{c}} = 0$.

Then we can integrate a function $f(r,z)$ over the spherical volume as:

\begin{eqnarray}
    \label{eq:int_indicator_V}
    \begin{aligned}
        \int\limits_{V} f dV
        % =\int\displaylimits_{0}^{+\infty} \int\displaylimits_{-\infty}^{+\infty} \int\displaylimits_{0}^{2\pi} f(r, z) H(\Delta_{\text{c}} - d/2) r \text{d}r \text{d}z \text{d}\theta =\\
        =\int\displaylimits_{0}^{+\infty} \int\displaylimits_{-\infty}^{+\infty} f(r, z) \int\displaylimits_{0}^{2\pi}  H(\Delta_{\text{c}} - d/2) r \text{d}r \text{d}z \text{d}\theta =\\
        =\int\displaylimits_{0}^{+\infty} \int\displaylimits_{-\infty}^{+\infty} f(r, z)  V_{\theta \downarrow}(r,z) \text{d}r \text{d}z
    \end{aligned}
\end{eqnarray}

where $r \text{d}r \text{d}z \text{d}\theta$ is the differential volume element in cylindrical coordinates, $H(\Delta_{\text{c}} - d/2)$ is the Heaviside function, that evaluates to 1 inside the sphere.
The $V_{\theta \downarrow}(r,z)$ is a projection of sphere volume on the $rz$-plane, with $\theta$ being the projecting direction.

Similarly, to compute surface integrals over the spherical colloid, we apply the Dirac delta function $\delta(\Delta_{\text{c}} - d/2)$ to restrict the integration domain to the spherical surface.
This allows us to express the surface integral of a scalar function $f(r,z)$ as:
\begin{eqnarray}
    \label{eq:int_indicator_S}
    \begin{aligned}
        \int\limits_{S} f dS = \\
        = \int\displaylimits_{0}^{+\infty} \int\displaylimits_{-\infty}^{+\infty} \int\displaylimits_{0}^{2\pi} f(r, z) \delta(\Delta_{\text{c}} - d/2)  \frac{\Delta_{\text{c}}}{r_{\text{c}} |\sin\theta|} r \, \text{d}r \text{d}z \text{d}\theta=\\
        =\int\displaylimits_{0}^{+\infty} \int\displaylimits_{-\infty}^{+\infty} f(r, z)  S_{\theta \downarrow} \, \text{d}r \text{d}z
    \end{aligned}
\end{eqnarray}
where the factor $\frac{\Delta_{\text{c}}}{r_{\text{c}} |\sin\theta|}r \, dr dz$ corresponds to the surface area element expressed in cylindrical coordinates.


From the Eqs.~(\ref{eq:int_indicator_V}, \ref{eq:int_indicator_S}) the volume $V_{\theta \downarrow}$ and surface $S_{\theta \downarrow}$ projections of a spherical body onto the $rz$-plane in cylindrical coordinates are:
\begin{gather}
    V_{\theta \downarrow}(r, z, r_{\text{c}}, z_{\text{c}}) = 2\int_{0}^{\pi} H\!\left( \Delta_{\text{center}} - {d}/{2} \right) r \, \text{d}\theta
    \\
    S_{\theta \downarrow}(r, z, r_{\text{c}}, z_{\text{c}}) = 2\int_{0}^{\pi}\delta(\Delta_{\text{c}} - d/2)  \frac{\Delta_{\text{c}}}{r_{\text{c}} |\sin\theta|} r \, \text{d}\theta
\end{gather}

These expressions describe the angular integration of the volume and surface projected onto the $rz$-plane, effectively reducing the 3D geometry to a two-gradient description suitable for cylindrical symmetry.

\begin{figure}[]
    \centering
    \includegraphics[width=0.95\linewidth]{fig/sphere_volume_and_surface_projection.png}
    \caption{
        \textbf{Left:}
        Surface $S_{\theta \downarrow}\{r_{\text{c}}\}$ (upper) and volume projections $V_{\theta \downarrow}\{r_{\text{c}}\}$ (lower) for a spherical colloid with diameter $d = 8$ on $rz$-plane, shown for a set of sphere center offsets $r_{\text{c}} = \{0, 2, 4, 6\}$ to the $z$-axis (shown side by side).
        The sphere centers are indicated with a red cross.
        The $r$-axis is shown with blue arrow, $z$-axis is omitted, as the center coordinates $z_{\text{c}}$ are arbitrary.
        The color code is shown with the colorbar below, where blue corresponds to zero and yellow corresponds to values above $25$.
        \\
        \textbf{Right:}
        The discretization results of the surface and volume projection matrices $\bm{S}\{r_{\text{c}}\}$ and $\bm{V}\{r_{\text{c}}\}$ as heatmaps, respectively, shown for $r_{\text{c}} = 6$.
        The color code with the colorbar below, where blue corresponds to zero and yellow rescaled to $8$ for clarity.
        \\
        Intensities for surface and volume are normalized by $a^2$ and $a^3$, respectively, where $a$ is the lattice unit length.
    }
    \label{fig:sphere_volume_and_surface_projection}
\end{figure}

To find the elements of the projection matrices (each representing the volume or surface area of the spherical colloid intersecting a given lattice cell), we discretize the angularly projected volume and surface area over finite lattice elements.
For each grid element indexed by $i, k$, corresponding to the domain $r \in [i, i + \delta r]$ and $z \in [k, k + \delta z]$, the matrix entries  are defined as
\begin{eqnarray}
    V\{r_{\text{c}}\}{[i, k]} = \! \iint \limits_{i, k}^{\quad \substack{i+\delta r\\ k+\delta z}} \! V_{\theta \downarrow} (r, z, r_{\text{c}}, z_{\text{c}})\, \text{d}r \text{d}z
    \\
    S\{r_{\text{c}}\}{[i, k]} = \! \iint \limits_{i, k}^{\quad \substack{i+\delta r\\ k+\delta z}} \! S_{\theta \downarrow} (r, z, r_{\text{c}}, z_{\text{c}})\, \text{d}r \text{d}z
\end{eqnarray}
where $z_{\text{c}}$ has an arbitrary value.
The size of the projection matrices is $\min(d, r_{\text{c}} + d/2) \times d$.
In contrast to Cartesian projections, cylindrical projections depend on the radial coordinate of the center $r_{\text{c}}$.

Naturally, the sum of all matrix elements equals the surface and the volume of the sphere
\begin{eqnarray*}
    \mathop{\sum\sum}_{\mathclap{\substack{i \in [0, \min(r_{\text{c}}+d/2,d)-1] \\ k \in [0, d-1]}}}  V\{r_{\text{c}}\}{[i, k]} = \pi d^2,
    \\
    \mathop{\sum\sum}_{\mathclap{\substack{i \in [0, \min(r_{\text{c}}+d/2,d)-1] \\ k \in [0, d-1]}}}  S\{r_{\text{c}}\}{[i, k]} = \frac{\pi d^3}{6} .
\end{eqnarray*}

The discretized form of Eq.~(15) to calculate the osmotic term in the insertion free energy for a spherical particle is:
\begin{eqnarray}
    \begin{aligned}
        \Delta F_{\text{osm}}(r_{\text{c}}, z_{\text{c}}) =\\
        = \mathop{\sum\sum}_{\mathclap{\substack{i \in [0, \min(r_{\text{c}}+d/2,d)-1] \\ k \in [0, d-1]}}} V{r_{\text{c}}}_{[i, k]} \cdot \Pi_{[\max(r_{\text{c}}-d/2,0)+i, z_{\text{c}}-d/2+k]} =\\[-15pt]
        = \bm{V}\{r_{\text{c}}\} \cdot \bm{\Pi}\{r_{\text{c}}, z_{\text{c}}\} \\[5pt]
        \text{where } \bm{\Pi}\{r_{\text{c}},z_{\text{c}}\} =\left(\bm{\Pi}_{i,k}\right) {\substack{\max(r_{\text{c}}d/2,0) \le i < r_{\text{c}}+d/2 \\ z_{\text{c}}-d/2 \le k < z_{\text{c}}+d/2}}
    \end{aligned}
\end{eqnarray}
Similarly, the surface term in the insertion free energy is the following matrix multiplication:
\begin{eqnarray}
    \begin{aligned}
        \Delta F_{\text{sur}}(r_{\text{c}}, z_{\text{c}}) = \bm{S}\{r_{\text{c}}\} \cdot \bm{\gamma}\{r_{\text{c}}, z_{\text{c}}\} \\[5pt]
        \text{where } \bm{\gamma}\{r_{\text{c}},z_{\text{c}}\} =\left(\bm{\gamma}_{i,k}\right) {\substack{\max(r_{\text{c}}-d/2,0) \le i < r_{\text{c}}+d/2 \\ z_{\text{c}}-d/2 \le k < z_{\text{c}}+d/2}}
    \end{aligned}
\end{eqnarray}

The function domain, values and discretization for $V_{\theta \downarrow}$ and $S_{\theta \downarrow}$ are exemplified in Figure~\ref{fig:sphere_volume_and_surface_projection} for a set of colloid with varying radial center positions $r_{\text{c}}$.

The method allow us to find insertion free energy $\Delta F(r,z)$ from inherently discrete SF-SCF outputs $\Pi, \gamma$, preserving the cylindrical lattice.

Volume and surface projection matrices are explained geometrically in Figure~\ref{fig:spherical_kernel}.

\begin{figure}[H]
    \centering
    \includegraphics[width=0.9\linewidth]{fig/spherical_kernel.png}
    \caption{
        Illustration of a spherical colloid's volume and surface projection matrices in a cylindrical lattice, exemplifying the volume projection matrix $\bm{V}\{r_c\}$ for a spherical colloid with diameter $d = 12$ and $r_{\text{c}}= 8$.
        The colored tiles encode the matrix elements' values, where violet means zero and yellow represents the largest values.
        The geometrical meaning of the matrix element is the colloid volume (red body) or surface (green small patch) found in the domain $r,z \in [i, i + \delta r] \times [k, k + \delta z]$ (opaque blue toroid).
        The smaller drawing on the right complements the main drawing with a $z$-view for clarity, using consistent color-coding for volume and surface elements, with pale blue annulus being the domain of lattice element.
        The yellow circle indicates the colloid cross-section within the current $z$-slice, while the larger pale yellow circle shows the rest of the colloid body lying behind the plane of the cross-section.
    }
    \label{fig:spherical_kernel}
\end{figure}




%%%%%%%%%%%%%%%%%%%%%%%%%%%%%%%%%%%%%%%%%%%%%%%%%%%%%%%%%%%%%%%%%%%%%%%%%%%%%%%%%%%%%%%%%%%%%%%%%
% \pagebreak
\section{Analytical Estimation of Pore Resistance}
\begin{figure*}[h]
    \centering
    \includegraphics[width=0.85\linewidth]{fig/coordinate_system.png}
    \caption{%
        \textbf{Left:}
        Steady-state solution of the diffusion equation for a point-like particle diffusing through an empty cylindrical pore of finite thickness.
        Iso-concentration surfaces, $c = \text{const}$, are represented by contour lines with labeled concentration values.
        % For an empty pore, $\psi = c$; for a polymer brush-filled pore, $\psi = c\exp(\Delta F/k_B T)$.
        Blue and red axes indicate radial and axial coordinates, respectively.
        Pore radius $r_{\text{p}}^{0} = 20$ and thickness $L_{0} = 20$.
    }
    \label{fig:empty_pore_solution}
    \caption{
        \textbf{Right:}
        Intrinsic orthogonal curvilinear coordinate system for the pore.
        Radial and axial coordinates are parameterized as $r'(r,z)$ and $z'(r,z)$, respectively.
        Solid lines indicate surfaces of rotation about the pore axis.
        Red lines correspond to surfaces of constant $z'$; blue lines correspond to surfaces of constant $r'$.
        Semi-planes with constant angular coordinate $\theta$ are not shown.
        Local basis vectors of the intrinsic coordinate system ($\hat{e}_r$, $\hat{e}_z$) are illustrated by arrows.
        Lamé coefficients are defined by the magnitudes of the local basis vectors as $h_r = |\hat{e}_r|$, $h_z = |\hat{e}_z|$, and $h_{\theta} = |\hat{e}_{\theta}|$.
        Pore radius $r_{\text{p}}^{0} = 20$ and thickness $L_{0} = 20$.
        }
    \label{fig:coordinate_system}
\end{figure*}



To construct an approximate analytical solution, we first consider a bare cylindrical pore of radius $r_{\text{p}}^{0}$ in an impermeable membrane separating two semi-infinite bulk solutions.
The pore axis coincides with the $z$-axis, and the membrane has thickness $L_{0}$.
We assume constant diffusivity $D_0$ and impose boundary conditions with concentrations $c(z=-\infty)=1$ and $c(z=+\infty)=0$ on opposite sides of the membrane.

The steady-state solution of the diffusion equation, defined by $\partial c/\partial t = 0$, produces the concentration profile shown in Figure~\ref{fig:empty_pore_solution}.

For a pore in an infinitely thin membrane (an orifice), the iso-concentration surfaces are known to form oblate spheroids with the pore rim as their focal circle.
This approximation remains valid for the iso-concentration lines in the exterior region even for membranes of finite thickness.

For a cylindrical channel, the solution in the pore lumen is approximated by equally spaced, disk-shaped iso-concentration surfaces. 

The flux density field, is directly related to the concentration gradient via Fick's law \mbox{$\bm{j} = -D_0 \nabla c$}.

Polymer filling within the pore modifies the local diffusion coefficient $D$ and generates a free-energy landscape, resulting in an effective position-dependent diffusion coefficient (conductivity) $\tilde{D}(r,z) = D\,e^{-\Delta F/k_B T}$ or local resistivity $\rho = \tilde{D}^{-1}$.
Since the flux density $\bm{j}$ is a conservative vector field, we define a scalar potential function $\psi = c\,e^{\Delta F/k_B T}$, such that $\bm{j} = -\tilde{D}\nabla\psi$.
Consequently, the steady-state iso-concentration surfaces differ from the oblate spheroids of the bare pore; however, they retain a similar structure for iso-values of $\psi$.

For the exterior region, we introduce an intrinsic curvilinear coordinate system $(r', z', \theta)$ aligned with approximate iso-surfaces of \(\psi\) (as depicted in Figure~\ref{fig:coordinate_system}), defined as follows:
level sets of the potential function $\psi$ form a family of oblate hemispheroids $z'$, indexed by their intersection points with the $z$-axis;
flux-density stream surfaces, perpendicular to $\psi$, form a family of hyperboloids of revolution $r'$, indexed by their intersection radii with the plane $z=0$:
\begin{gather}
    r' = \left\{(r,z) \mid \nabla f \cdot \nabla \psi = 0,\; f=f(r,0)\right\},
    \\
    z' = \left\{(r,z) \mid \psi(r,z)=\psi(0,z)\right\}.
\end{gather}
Half-planes of constant azimuthal angle \(\theta\) remain unchanged from the original cylindrical coordinate system.

In the exterior region ($|z| > L/2$), the intrinsic coordinate system $(r', z', \theta)$ is parameterized using the original cylindrical coordinates $(r,z,\theta)$ and the distance from the pore opening $z_{\text{ext}} = |z| - L/2$:
\begin{align}
    r'(r,z) &= r\sqrt{1 + \frac{z_{\text{ext}}^2}{r_{\text{p}}^2}},\\[4pt]
    z'(r,z) &= z_{\text{ext}}\frac{\sqrt{r_{\text{p}}^2 - r^2}}{r_{\text{p}}} + \text{sign}(z)\frac{L}{2}.
\end{align}
The corresponding Lam\'e  coefficients are:
\begin{align}
    h_r &= \frac{\sqrt{r_{\text{p}}^2 + z_{\text{ext}}^2 - r^2}}{\sqrt{r_{\text{p}}^2 - r^2}},\\[4pt]
    h_z &= \frac{\sqrt{r_{\text{p}}^2 + z_{\text{ext}}^2 - r^2}}{\sqrt{r_{\text{p}}^2 + z_{\text{ext}}^2}},\\[4pt]
    h_{\theta} &= \frac{r\sqrt{r_{\text{p}}^2 + z_{\text{ext}}^2}}{r_{\text{p}}},\\[4pt]
    \tilde{h}(r,z) &= h_r h_{\theta} h_z^{-1} = \frac{r}{r_{\text{p}}}\frac{r_{\text{p}}^2 + z_{\text{ext}}^2}{\sqrt{r_{\text{p}}^2 - r^2}}.
\end{align}

The conductivity integrated over the oblate hemispheroids in the exterior region is given by:
\begin{equation}
  \label{eq:rho_ext}
  \varrho_{\text{ext}}^{-1}(z)= 2\pi\int_{0}^{r_{\text{p}}^{}} 
  \tilde{D}\left( r'(r,z), z'(r,z) \right)\tilde{h}(r,z)\,dr.
\end{equation}

Inside the pore ($|z|\leq L/2$), assuming no significant radial flux, the conductivity of a disk cross-section at position $z$ is approximated as:
\begin{equation}
  \varrho_{\text{int}}^{-1}(z)= 2\pi\int_{0}^{r_{\text{p}}^{}} \tilde{D}(r,z)\,r\,dr.
\end{equation}

Therefore, the total resistances of the exterior and interior regions are respectively obtained by:
\begin{align}
   \label{eq:R_ext}
   R_{\text{ext}} &=2\int_{+L/2}^{+\infty}\varrho_{\text{ext}}(z)\,dz,\\[5pt]
   \label{eq:R_int}
   R_{\text{int}} &=\int_{-L/2}^{+L/2}\varrho_{\text{int}}(z)\,dz.
\end{align}

We now revisit the resistance of a bare pore of finite thickness to the diffusion of a point-like colloid using Eqs.~(\ref{eq:R_ext},~\ref{eq:R_int}).
For a bare pore, the effective diffusion coefficient is simply $\tilde{D} = D_0$:
\begin{eqnarray}
    \varrho^{0}_{\text{int}}(z) &=& \frac{1}{D_0 \pi r_{\text{p}}^2},\\[4pt]
    \varrho^{0}_{\text{ext}}(z) &=& \frac{1}{2\pi D_0\left(z_{\text{ext}}^2 + r_{\text{p}}^2\right)}.
\end{eqnarray}

Integration over the full domains of $z$ yields the classical result~\cite{Brunn1984}:
\begin{eqnarray}
    \label{eq:r_empty}
    R_{\text{ext}}^{0} &=& 2 \int_{-\infty}^{-L/2} \varrho_{\text{ext}}^{0}(z)\,dz = \frac{1}{2 D_0 r_{\text{p}}},\\[4pt]
    R_{\text{int}}^{0} &=& \int_{-L/2}^{+L/2} \varrho_{\text{int}}^{0}(z)\,dz = \frac{L}{D_0 \pi r_{\text{p}}^2}.
\end{eqnarray}

Our numerical approach inherently employs the discrete cylindrical lattice from the SF-SCF calculations, with discretization steps $\delta z = \delta r = 1$.
Consequently, continuous integration across equipotential surfaces foliating space is replaced by summation over discrete conductive layers, each bounded by two adjacent equipotential surfaces. 
Layers are indexed by the first equipotential surface intersecting the $z$-axis (see Figure~\ref{fig:integration_scheme}).

\begin{figure}[]
    \centering
    \includegraphics[width=\linewidth]{fig/resistance_integration.png}
    \caption{
        Numerical integration of local conductivity/resistance on a cylindrical lattice.
        In the exterior region, conductivities are integrated over half-cylinder layers (red fill), each indexed by $z$, with radius $r_{\text{ext}}$ and height $z_{\text{ext}}$, a single layer is highlighted with the darker red color.
        In the interior region, integration is performed over cylindrical disk layers (blue fill); a single layer is highlighted in darker blue.
        Dashed red lines illustrate half-cylinder surfaces that approximate the oblate spheroidal equipotentials shown in Figure~\ref{fig:coordinate_system}.
    }
    \label{fig:integration_scheme}
\end{figure}

In the interior region, this discretization is straightforwardly applied, giving the resistance of a discrete disk-shaped layer as:
\begin{equation}
    \varrho_{\text{int}}^{\text{lat}}(z) 
    =\left[\pi \sum_{r=0}^{r_{\text{p}}-1}(2r+1)\,\tilde{D}[r,z]\right]^{-1}.
\end{equation}

For numerical integration in the exterior region, we approximate the equipotential surfaces as half-cylindrical rather than oblate hemispheroidal layers (Figure~\ref{fig:integration_scheme}).
Thus, instead of oblate hemispheroid-shaped "bowls," the discretization yields nested cylindrical "buckets" or caps with increasing radius $r_{\text{ext}} = r_{\text{p}} + |z| - L/2$ and height $z_{\text{ext}} = |z| - L/2$.

The resistance of a cylindrical cap layer naturally underestimates the corresponding oblate hemispheroidal layer at the same position $z$.
To compensate, we introduce a correction factor $p(z)$, which equals the ratio of the resistances of an oblate spheroidal layer to a cylindrical-cap layer with identical local conductivity $\tilde{D}$:
\begin{gather}
    \label{eq:prefactor}
    \begin{aligned}
        p(z) &= r_{\text{p}} \left(z_{\text{ext}} \log{\frac{r_{\text{ext}}}{r_{\text{ext}} - 1}} + r_{\text{ext}}^{2}\right)^{-1} \times \\[4pt]
        &\quad\times\left[\operatorname{atan}{\frac{z_{\text{ext}}}{r_{\text{p}}}} - \operatorname{atan}{\frac{z_{\text{ext}} - 1}{r_{\text{p}}}}\right]^{-1}.
    \end{aligned}
\end{gather}

The resistance of a layer at position $z$ in the exterior region, computed on a cylindrical lattice, is:
\begin{equation}
    \begin{aligned}
        &\varrho_{\text{ext}}^{\text{lat}}(z) = 
        \frac{p(z)}{\pi}\left[
        \sum_{r=0}^{r_{\text{ext}}-1}(2r+1)\tilde{D}[r,z_a]\right. + \\[4pt]
        &\quad+\left. 2 \log\left(\frac{r_{\text{ext}}}{r_{\text{ext}}\!-\! 1}\right)\sum_{z^{\prime}=z_{a}}^{z_{b}}\tilde{D}[r_{\text{ext}}\!-\!1,z^{\prime}]
        \right]^{-1}
    \end{aligned}
\end{equation}
with indexing limits defined by:
\begin{equation*}
    \begin{cases}
        z_{a} = -z,\quad z_{b} = -L/2-1,&\text{if } z < -L/2,\\[4pt]
        z_{a} = L/2,\quad z_{b} = z-1,&\text{if } z > L/2.
    \end{cases}
\end{equation*}


To account for the resistance of the semi-infinite reservoirs beyond the integration domain on the cylindrical lattice, we evaluate the analytical expression for the exterior resistance, Eq.~\ref{eq:rho_ext}, from the integration boundary $z$ to infinity:

\begin{eqnarray}
    \label{eq:r_reservoir}
    R_{(z, \pm\infty)} = \pm\!\! \int\displaylimits_{z}^{\pm\infty}\!\! \varrho_(z') \, \text{d}z' = \frac{\arctan\left({r_{\text{p}}}/{z_{\text{ext}}} \right)}{2 D_0 \pi r_{\text{p}}^{0}}
\end{eqnarray}

Finally, the total resistance of the pore, integrated on the discrete cylindrical lattice, is given by:

\begin{eqnarray}
    R_{\text{int}}^{\text{lat}} = \sum_{z=-L/2}^{+L/2} \varrho_{\text{int}}^{\text{lat}}(z)
    \\
    R_{\text{ext}}^{\text{lat}} = R_{(z_{\text{left}}, -\infty)} + \sum\limits_{\mathclap{\substack{z \in [z_{\text{left}},-L/2)\\z \in (L/2, z_\text{right}]}}} \varrho_{z} + R_{(z_{\text{right}}, +\infty)}
    \\
    \label{eq:R_total}
    R_{\text{lat}} = R_{\text{ext}}^{\text{lat}} + R_{\text{int}}^{\text{lat}}
\end{eqnarray}
where $z_{\text{left}}, z_{\text{right}}$ is the limits of the discrete domain.

Eq.~(\ref{eq:R_total}) concludes the analytical estimation scheme, giving the total resistance $R_{\text{lat}}$ of the pore.


%%%%%%%%%%%%%%%%%%%%%%%%%%%%%%%%%%%%%%%%%%%%%%%%%%%%%%%%%%%%%%%%%%%%%%%%%%%%%%%%%%%%%%%%%%%%%%%%
% \pagebreak
\section{Validating Analytical Approaches with Numerical Simulations}


Applying Eqs.~(2,3) and the boundary conditions reduces the Smoluchowski equation Eq.~(1) in the stationary state to the generalized Laplace equation:
\begin{equation}
  \mathcal L\tilde c=0,
  \label{eq:laplace}
\end{equation}
with the Smoluchowski diffusion operator
\begin{equation*}
  \mathcal L=\nabla\!\cdot\!\bigl(\tilde D(\bm r)\nabla\bigr).
\end{equation*}

\textbf{Laplace Equation in Discrete Finite Domain}

To obtain a stationary solution of Eq.~\eqref{eq:laplace}, we consider a finite cylindrical domain large enough to exclude edge effects:
\begin{equation*}
  (r,z) \in \Omega = [0,r_{\max}]\times[z_{\min},0],
\end{equation*}
and discretize it on a regular cylindrical lattice with spacings $\delta r=\delta z$. 
Due to symmetry about the mid-plane $z = 0$, we restrict the domain accordingly with imposing suitable boundary condition.

The computational grid contains $N_r=r_{\max}/\delta r$ radial nodes and $N_z=(z_{\max}-z_{\min})/\delta z$ axial nodes, indexed by $i=0,\dots,N_r-1$ and $k=0,\dots,N_z-1$, respectively.
Physical coordinates are $r_i=i\delta r$ and $z_k=z_{\min}+k\delta z$ as shown in Figure~\ref{fig:stencil}a.
Each node defines a finite volume element with four faces, labeled $z\pm$ and $r\pm$, as illustrated in Figure~\ref{fig:stencil}c.
The finite volume element is indexed using its lower-left node ($r-, z-$).

Applying the divergence theorem to each finite volume element yields a discrete approximation to the Laplacian in the form of a finite volume stencil:
\begin{eqnarray}
    \begin{aligned} 
        \nabla_{i,k} \psi = 
        \\
        \lambda^{z} D^{z+}_{i,k} (\psi_{i,k} - \psi_{i,k+1}) +  \lambda^{z} D^{z-}_{i,k} (\psi_{i,k} - \psi_{i,k-1})\\
        \lambda^{r+}_{i} D^{r+}_{i,k} (\psi_{i,k} - \psi_{i+1,k}) +  \lambda^{r-}_{i} D^{r-}_{i,k} (\psi_{i,k} - \psi_{i-1,k})
    \end{aligned}
    \label{eq:FV_divergence}
\end{eqnarray}

where, $D^{r\pm}_{i,k}$ and $D^{z\pm}_{i,k}$ denote effective diffusion coefficient evaluated at finite volume element faces, as a harmonic averaging of values at adjacent finite volume elements elements, 
the finite volume expansion in the radial direction (see Figure~\ref{fig:stencil}b) and discretization step is accounted with coordinate-dependent prefactors:

\begin{eqnarray}
    \lambda^{z} = \frac{1}{\delta z^2}\\
    \lambda^{r+}_{i} = \frac{2 r_i + 2 \delta r}{2 r_i + 1} \frac {1}{\delta r^2}\\
    \lambda^{r-}_{i} = \frac{2 r_i}{2 r_i + 1} \frac {1}{\delta r^2}
\end{eqnarray}

\begin{figure}[]
  \includegraphics[width = \linewidth]{fig/stencil.png}
  \caption{\textbf{(a)} The setup of the discrete domain for the numerical solution.
  Following the analytical solution for the bare pore the source nodes are placed in the region circumferenced by the oblate semi-spheroid (orange fill).
  The effective shape of the impermeable membrane (light green hatched fill) is constructed from the elements the correspond to the physical membrane (dark green hatched fill).
  We exploit system symmetry by modeling only half of the system, by placing Dirichlet boundary condition on what would be the mid-plane.
  The Smoluchowski operator is defined with 5-point stencil on the finite volume elements (red element and with grey neighbors).
  The dots on the stencil denotes the indexing nodes.
  \textbf{(b)} The finite volume element, naturally, expands with the radial coordinate, with the size defined by the discretization steps $\delta r$ and $\delta z$.
  \textbf{(c)} The finite volume element shares common faces with the neighboring elements with faces labeled with the coordinate direction along the radial $r \pm$ and the axial $z\pm$ axes.
  }
  \label{fig:stencil}
\end{figure}




The discretized system is assembled into matrix form:
\begin{equation}
  \mathbf L\, \bm{\psi}=\bm b,
  \label{eq:matrix_form}
\end{equation}
where $\mathbf{L}$ applies the finite volume stencil from Eq.~\ref{eq:FV_divergence} and $\bm b$ enforces boundary conditions.

The two-dimensional grid of volume elements are flattened into vectors of length $N_rN_z$ reindexed with $m=iN_z+k$.
Under this mapping, the continuous operator $\mathcal L$ becomes a sparse $[N_rN_z\times N_rN_z]$ matrix $\mathbf L$, and the unknown values of $\psi(r_i, z_k)$ form the column $[N_rN_z]$ vector $\bm{\psi}$.

Once the linear system Eq.~\ref{eq:matrix_form} is solved for $\bm{\psi}$, the resulting vector is reshaped into the $rz$-grid, giving $\psi(r,z)$, recovering the colloid concentration $c(r,z)$.
This allows us to examine the non-equilibrium partitioning of colloids in the presence of position-dependent insertion free energy fora given boundary conditions.

\textbf{Impermeable Wall (No-flux Elements)}

The impermeable regions are inherited from the SF-SCF calculations and define the finite volume elements inaccessible to diffusing colloids.
For a given pore size we define the original impermeable walls grid elements as set:
\begin{eqnarray}
    \textbf{Wall}_{0} [i,k] = 
    \begin{cases}
        1 \quad \text{if } r_i \ge r_{\text{p}}^0 \text{ and } z_k \le -L_0/2 \\
        0 \quad \text{otherwise}
    \end{cases}
\end{eqnarray}

Due to the excluded volume effect, the effective shape is different, the space impermeable to a spherical particle is an effective pore with a radius smaller than the actual pore radius, $r_{\text{p}} = r_{\text{p}}^{0} - \frac{d}{2}$, and longer chanel, $L = L_{0} + d$, with rounded corners, as shown in Figure~\ref{fig:excluded_volume} (left).

This excluded volume region corresponds to a morphological dilation of the membrane body by the particle. In this case, the dilation is the set of all grid points that the center of a particle of diameter $d$ cannot enter without intersecting the wall.

The dilated impermeable region is computed via the morphological dilation operation:
\begin{eqnarray}
    \textbf{Wall} = \textbf{Wall}_{0} \oplus \textbf{Particle}
\end{eqnarray}

The structuring element $\textbf{Particle}$ is coarse-grained circle in $rz$-coordinate, see Figure~\ref{fig:excluded_volume} (right).
\begin{equation}
    \begin{gathered}
        \textbf{Particle}[i,k] = \\
        \begin{cases}
                1, & \text{if } \left( \dfrac{d}{2} - r_i\right)^2 + \left( \dfrac{d}{2} - z_k\right)^2 \le \dfrac{d^2}{4} \\[5pt]
                0, & \text{otherwise}
            \end{cases}
    \end{gathered}
\end{equation}

\begin{figure}[]
    \centering
    \includegraphics[width=\linewidth]{fig/excluded_volume.png}
    \caption{
        In the left frame, the effective pore shape is traced with a dashed red line
        The excluded volume is created by a spherical particle with diameter $d$ and is shown with a red shade.
        In the right frame, the excluded volume is shown on the regular lattice on the $rz$-plane circumferenced with dashed red line .
        % The excluded volume on the regular lattice is a result of a morphological binary dilation $\bm{W}^{\ast} = \bm{W} \bigoplus \bm{V}$.
        }
    \label{fig:excluded_volume}
\end{figure}


\textbf{Colloid Particles Source}

We place source nodes at a finite axial distance, chosen to align with an oblate spheroidal surface (an ellipse in the $rz$-plane) intersecting the $z$-axis at $z_{\text{min}} + \delta z$.
The ellipse has foci located at $(\pm r_{\text{p}}, -L/2)$, such that the pore rim acts as the focal circle.
This defines the major semi-axis $r_{\text{max}} - \delta r$ and hence the radial extent of the discrete computational domain:

\begin{equation}
    r_{\text{max}} - \delta r = \sqrt{(|z_{\text{min}}| - L/2 + \delta z)^2 + r_{\text{p}}^2}
\end{equation}

We then define a binary mask $\textbf{Source}[i,j]$, which identifies the grid nodes at the boundary of the domain that act as the colloid source (see Figure~\ref{fig:stencil}a).
These nodes lie outside the oblate ellipsoid defined by the above condition.
The set is constructed as:

\begin{eqnarray}
    \begin{gathered}
        \textbf{Source}[i,j] = \\
        \begin{cases}
        1, & 
        \begin{aligned}
            \text{if } &\left\lVert (r_i, z_k) - (r_{\text{p}}, -L/2) \right\rVert \\
            +  &\left\lVert (r_i, z_k) - (-r_{\text{p}}, -L/2) \right\rVert \ge 2(r_{\text{max}} - \delta r)
        \end{aligned} \\
        0, & \text{otherwise}
        \end{cases}
    \end{gathered}
\end{eqnarray}

This boundary condition ensures that the solution approximates that of an idealized system with a source located infinitely far away ($z \to -\infty$).
We effectively reproduce the shape of iso-concentration lines in the exterior region of a bare pore, thereby maintaining a realistic flux geometry.

This deviation can be corrected by analytically adding the missing reservoir resistance, as given by Eq.~\ref{eq:r_reservoir}, which accounts for the truncated semi-infinite space.

\textbf{Colloid Particles Sink}

To exploit the system's symmetry, we impose a Dirichlet boundary condition $\psi = \psi_{\text{sink}}$ at the $z+$ face of all finite volume elements with $k = N_z - 1$.
This corresponds to placing the absorbing boundary condition at the mid-plane $z = 0$.

Due to symmetry, there is no radial component of the flux at $z = 0$, so no flux crosses the $r-$ or $r+$ faces of these finite volume elements.

We define the set of finite volume elements that act as sinks as:

\begin{equation}
    \textbf{Sink}[i,k] =
    \begin{cases}
    1, & \text{if } k = N_z - 1, \\
    0, & \text{otherwise}.
    \end{cases}
\end{equation}

For these elements, the divergence term simplifies to account for the fixed potential at the $z+$ face and no flux at the radial faces:

\begin{equation}
\nabla_{i,k_0} \psi =
\lambda^{z} D_{i,k_0} (\psi_{i,k_0} - \psi_{\text{sink}})
+ 2 \lambda^{z} D_{i,k_0}^{z-} (\psi_{i,k_0} - \psi_{i,k_0-1})
\end{equation}


\textbf{Laplace operator matrix}

The matrix $\mathbf{L}$ is constructed using a five-point stencil that adapts near boundaries.
The diagonal entries are defined as
\begin{equation}
  \bm{L}_{m,m} = 
  \begin{cases}
    0 & \text{if } m \in \textbf{Wall}, \\
    1 & \text{if } m \in \textbf{Source}, \\
    - 2 \lambda^{z} D_m & \text{if } m \in \textbf{Sink}, \\
    -\!\!\!\sum\limits_{m' \in \mathcal{N}_m} \bm{L}_{m, m'} & \text{otherwise},
  \end{cases}
  \label{eq:L_diag}
\end{equation}
where $\mathcal{N}_m$ is the set of valid neighbor indices:
\begin{equation}
  \textbf{N}_{i,k} = \left( \{(i \pm 1, k),\; (i, k \pm 1)\} \right) \cap \Omega,
  \label{eq:neighbors}
\end{equation}
and node indices are flattened as $m = i N_z + k$, $m' = i' N_z + k'$.

The off-diagonal entries $\bm{L}_{m,m'}$ represent finite-volume approximations of the Laplace operator Eq.~(\ref{eq:FV_divergence}), that aware of the boundary conditions:

\begin{eqnarray}
    \bm{L}_{m,m+1} =
    \begin{cases}
        0 &  \text{if } m+1 \in \textbf{Wall} \\
        D^{z+} \lambda^{z} &  \text{otherwise}
    \end{cases}
    \\
    \bm{L}_{m,m-1} =
    \begin{cases}
        0 &  \text{if } m-1 \in \textbf{Wall} \\
        2 D^{z-} \lambda^{z} & \text{if } m \in \textbf{Sink}\\
        D^{z-} \lambda^{z} &  \text{otherwise}
    \end{cases}
    \\
    \bm{L}_{m,m \pm N_z} =
    \begin{cases}
        0 \quad \text{if } m \pm N_z \in \textbf{Wall} \text{ or } m \in \textbf{Sink} \\
        D^{r\pm} \lambda^{r\pm}  \text{otherwise}
    \end{cases}
\end{eqnarray}


The right-hand side vector $\bm{b}$ enforces the boundary condition from the source and the sink:
\begin{eqnarray}
    \bm{b}_m = 
    \begin{cases}
        \psi_{\text{source}} & \text{if } m \in \textbf{Source} \\
        \psi_{\text{sink}} & \text{if } m \in \textbf{Sink} \\
        0 & \text{otherwise}
    \end{cases}
\end{eqnarray}

The resulting sparse linear system in Eq.~\eqref{eq:matrix_form} is solved using the \texttt{scipy.sparse.linalg} module in Python.

\textbf{Extracting Pore Resistance}

The total colloid flux is computed by summing the fluxes through the $z+$ faces of the finite volume elements at $k = N_z - 1$, corresponding to the pore cross-section at the mid-plane $z = 0$.
Due to symmetry, the radial flux vanishes at this plane, so only the axial ($z$-direction) components contribute.

\begin{equation}
    J_{\text{num}} = 2\pi \sum_{i=0}^{N_r-1} D_{i,k'} \frac{2(\psi_{i,k'} - \psi_{\text{sink}})}{\delta z} (2i + 1) \delta r^2
\end{equation}
where $k' = N_z - 1$ is the axial index of the mid-plane.

To determine the total pore resistance, we prescribe boundary conditions $\psi_{\text{source}} = 1.0$ and $\psi_{\text{sink}} = 0.5$.
This corresponds to a concentration difference of $\Delta c = 1.0$ across the full system, assuming symmetry about the mid-plane.

\begin{equation}
    R_{\text{num}} = \frac{\Delta c}{J_{\text{num}}} + 2 R_{\left(\left|z_{\text{min}} - L/2\right|,\ -\infty\right)}
\end{equation}
Here, $R_{(|z_{\text{min}} - L/2|, -\infty)}$ is the analytically estimated resistance of the truncated semi-infinite reservoir, as given by Eq.~\ref{eq:r_reservoir}.



% \pagebreak
\section{Kinetics of Pore-Mediated Equilibration: Comparison to Experimental Data}



\begin{figure}[]
    \centering
    \includegraphics[width=0.9\linewidth]{fig/experiment_description.png}
    \caption{Reductionist view of relevant pore-mediated equilibration experiments from Refs.~\cite{Ribbeck2001, Mohr2009, Popken2015, Timney2016, Frey2018}.\\
    \textbf{Left:} A compartment with finite volume $V_{\text{in}}$ is separated from a finite ($V_\text{out}$) or semi-infinite reservoir ($V_\text{out}\to \infty$)  by an impermeable membrane (green contour) permeated by $N_{\text{pores}}$ polymer-filled pores. 
    Nanocolloids (yellow circles) are mobile particles, present at a concentration $c_{\text{out}}$ in the semi-infinite reservoir, and influx into the finite volume where the time-dependent concentration is $c_{\text{in}}(t)$.\\
    \textbf{Top right:} Isolated polymers phase-separate in a poor solvent with solvent strength $\chi_{\text{PS}}^{\text{FG}}$, forming polymer gel droplets (red irregular shapes) with polymer volume fraction $\phi_{\text{gel}}$.\\
    \textbf{Bottom right:} Nanocolloids equilibrate between the solvent and the polymer gel, characterized by the partition coefficient $\text{PC}_{\text{gel}} \equiv \left(c_{\text{in}}/c_{\text{out}}\right)_{\text{gel}}$.
    }
    \label{fig:experiments_overview}
\end{figure}
    
We adopt a reductionist view of NPC‐mediated transport (Figure \ref{fig:experiments_overview}, left). 
The nucleus is represented as a well-mixed compartment of volume $V_{\text{in}}$ bounded by an impermeable membrane that contains $N_{\text{pores}}$ identical, widely spaced cylindrical channels, each filled with a homogeneous polymer brush mimicking the FG-domain meshwork of nucleoporins. 

Colloid particles at concentration $c_{\text{out}} = c_{0}$ diffuse from an external reservoir, either finite ($V_{\text{out}}$) or effectively infinite ($V_{\text{out}}\!\to\!\infty$), into the 'nucleus' compartment, whose initial concentration is $c_{\text{in}}(0)=0$.
Because the combined pore volume is negligible ($N_{\text{pores}}V_{\text{pore}}\ll V_{\text{in}}$) the equilibration follows a first-order rate law.

\begin{eqnarray}
    \frac{\partial c}{\partial t} &=& k (c_{\text{out}} - c_{\text{in}}(t)) \\
    \frac{c_{\text{in}}(t)}{c_{\text{out}}} &=& 1 - e^{-kt} \label{eq:kinetics} \\
    k &=& \frac{N_{\text{pores}}}{R} V_{\text{in}}^{-1}
    \label{eq:rate_constant}
\end{eqnarray}
where $k$ is the rate constant ($\tau = 1/k$ is characteristic time).
The pore resistance $R$ is defined by Eq.~(14).

If both compartments change concentration during equilibration
(a two-compartment model, \textit{i.e.} exchange between the nucleus and cytoplasm), the rate constant becomes
\begin{equation}
  k = \frac{N_{\text{pores}}}{R}
      \bigl(V_{\text{in}}^{-1}+V_{\text{out}}^{-1}\bigr),
  \label{eq:rate_constant_2}
\end{equation}

FG-domains of nucleoporins are slightly hydrophobic. 
In solution they can therefore phase-separate into hydrogel particles (Figure~\ref{fig:experiments_overview}, top right), whereas when end-grafted to the pore walls they form a dense polymer brush that fills the NPC channel.
The equilibrium partition coefficient of a probe colloid in these isolated FG hydrogels (Figure~\ref{fig:experiments_overview}, lower right) reflects its interaction with the same domains in the grafted brush, and can be translated into an insertion free energy, and hence into a predicted pore resistance.

To capture this behaviour we treat the chemically diverse FG-domains as identical homopolymer chains with averaged properties.  
The same model reproduces both the hydrogel phase separation of ungrafted chains and the observed correlation between hydrogel
partitioning and NPC resistance.


\textbf{Extracting pore resistance from the experimental data.}
    
We compiled transport data for NPCs from HeLa, \textit{S. cerevisiae} and \textit{T. thermophilia} cells.  
The human cells studies used digitonin-permeabilised HeLa cells and monitored nuclear fluorescence while fluorescently tagged probes equilibrated with a large external reservoir 
\cite{Ribbeck2001,Mohr2009,Frey2018}.  
The yeast studies followed nuclear-cytoplasmic equilibration after
photobleaching a reporter protein \cite{Popken2015,Timney2016}.

Although yeast and human NPCs differ in their protein composition, the dimensions of the pore channel are similar and only slightly smaller for yeast NPC \cite{Yang1998}, allowing a direct comparison.  
Table~\ref{tbl:experimental} lists the experimental observables we extracted. 
To compare them with one another, and with our theoretical predictions, we convert each observable to the single-pore resistance $R_{\text{exp}}$.  
For data that report a kinetic constant $k$ (or its reciprocal time constant $\tau$) we use Eqs.~(\ref{eq:rate_constant}, \ref{eq:rate_constant_2}).

Popken \textit{et al.}\,\cite{Popken2015} measured the
nucleus-to-cytoplasm fluorescence ratio after $t =  1\text{h}$, which directly gives the concentration ratio
$c_{\text{in}}/c_{\text{out}}$.  
With finite nuclear and cytoplasmic volumes, the corresponding rate
constant from Eq.~\ref{eq:rate_constant_2} is

\begin{equation}
    k = t^{-1}\ln\left(
        \cfrac{\cfrac{c_{\text{in}}(t)}{c_{\text{out}}(t)} \cfrac{V_{\text{in}}}{V_{\text{out}}} +1}
             {1 - \cfrac{c_{\text{in}}(t)}{c_{\text{out}}(t)}}
        \right)
\end{equation}

The single-pore resistance extracted from Eq.~\ref{eq:rate_constant_2}
for nucleus-cytoplasm equilibration is
\begin{equation*}
  R_{\text{exp}}
  = \frac{N_{\text{pores}}}{k}
    \bigl(V_{\text{in}}^{-1}+V_{\text{out}}^{-1}\bigr),
\end{equation*}
whereas for digitonin-permeabilised nuclei equilibrating with a large
external reservoir (Eq.~\ref{eq:rate_constant}) it reduces to
\begin{equation*}
  R_{\text{exp}}
  = \frac{N_{\text{pores}}}{k}\,V_{\text{in}}^{-1}.
\end{equation*}

Because NPCs operate on the nanometer scale, the resulting resistances are extremely small, typically
$R\sim10^{20}\,\text{s}/\text{m}^{-3}$.  
To express these values more intuitively, we convert them to a
translocation rate per one NPC, defined as the number of molecules that cross one pore per second under a unit concentration gradient of $\Delta c = 1\;\mu\text{M}$ (cf.~\cite{Ribbeck2001}):
\begin{equation}
    \begin{array}{c}
        \text{Translocation rate} \\
        \text{per one NPC}\\
        \text{at} \, \Delta c = 1\;\mu\text{M}
    \end{array}
    \hspace{-1em}= \,\frac{N_{\mathrm{A}}}{R}\,10^{-3}\;\text{s}^{-1},
\end{equation}

where \(N_{\mathrm{A}}\) is Avogadro's number.

%\textbf{}

\textbf{Translocation rate versus inert probe colloid molar weight.}

% \begin{figure}[]
%     \centering
%     \includegraphics[width=3.2in]{fig/flux_vs_MW_SI.png}
%     \caption{%
%     Single-NPC translocation rate plotted against probe molar mass:
%     comparison of three models.  
%     \textit{Diffusive barrier} (gray)-
%     the FG mesh lowers the diffusivity but imposes no net insertion free
%     energy;  
%     \textit{Rigid barrier} (grey)-
%     the pore is treated as an impermeable channel $\approx$\,4-5\,nm in
%     diameter (cut-off $\approx$\,40-50\,kDa), so larger inert colloids are almost completely
%     excluded;  
%     \textit{Soft barrier} (orange, this work)-
%     the FG domains are modelled explicitly, combining size-dependent
%     diffusivity with the calculated insertion free energy.  
%     Shaded symbols represent experimental rates; shaded bands indicate parameter
%     uncertainty $1.0 \le \rho_{\text{probe}} \le 1.4\;\text{g/cm}^{-3}$.
%     }
%     \label{fig:flux_vs_MW}
% \end{figure}

We compare our theoretical predictions with experimental translocation data—both the original measurements of Ribbeck \textit{et al.}\,\cite{Ribbeck2001} and values we deduced from other studies \cite{Mohr2009,Popken2015,Timney2016,Frey2018} (see Table~\ref{tbl:experimental}).  
For the calculations we adopt the number of pores per nucleus, nuclear and cytoplasmic volume reported for digitonin-permeabilised HeLa cells \cite{Ribbeck2001} and for \textit{S.\,cerevisiae} nuclei \cite{Timney2016} (Table~\ref{tbl:experimental}), the inner diameter and the channel length of the pores is set to be $r_{\text{pore}}\approx 40 \text{nm}$ (Figure~1).
The buffer viscosity at 20$^\circ$C is taken as $\eta_{\text{S}} = 1.45\times10^{-3}\,\text{Pa\,s}$, matching the assay composition (20 mM HEPES-KOH, pH 7.4; 120 mM KOAc; 5 mM Mg(OAc)$_2$; 0.5 mM EGTA; 250 mM sucrose) \cite{Ribbeck2001}.

Throughout we assume the nuclear-pore brush is in a moderately poor solvent, $\chi_{\text{PS}} = 0.6$. \todo{[add reference or brief rationale]}

All probe particles are treated as spheres with densities in the range $1.0 \le \rho_{\text{probe}} \le 1.4\;\text{g/cm}^{-3}$.  Their
diameter (in nm) is estimated from the molar mass $M_{w}$ (in kDa) via
\begin{equation}
  d_{\text{probe}}
  = 10^{8}
    \left(
      \frac{6}{\pi}\,
      \frac{M_{w}}{N_{\text{A}}\rho_{\text{probe}}}
    \right)^{\!1/3},
  \label{eq:d_probe}
\end{equation}
where $N_{\text{A}}$ is Avogadro's number; the corresponding molecular masses are listed in Tables~\ref{tbl:inert_probes} and~\ref{tbl:attr_probes}.

In Figure 8a the scatter represent the experimental translocation rates extracted for inert probes (Table~\ref{tbl:inert_probes}).
The matching shaded band shows our theoretical prediction over the size range $1 \le d \le 10$ nm, bounded by the lowest and highest probe densities.
For reference we also plot the rates expected for a bare pore, using Eq.~(9), again with a band bracketing the density range.

% \todo{Here to write that rigid barrier does not fit the data well, and about correction of wall drag, as it is important here.
% They have similar comparison in Timney et al \cite{Timney2016}.}

% \begin{equation}
%     R = \frac{LK\{\frac{d}{2r_{\text{pore}}}\}}{D_0 \pi (r_{\text{p}}^0-r_{\text{p}})^{2}} + \frac{1}{2 D_0 (r_{\text{p}}^0-r_{\text{p}})}
%     \label{eq:resistance}
% \end{equation}
% where $K\{\frac{d}{2r_{\text{pore}}}\}$ account for wall drag and found from \cite{Haberman1958}.

% \todo{Also reduced by the meshwork diffusivity does not explain the trend, even if $\beta = 8$ instead of $\beta = 5.5$. We change $\beta$ to fit trend to smaller particles}

\textbf{Translocation rate versus partition coefficient in FG-domain hydrogel correlation.}

For particles with high affinity to the polymer (low $\chi_{\text{PC}}$), we expect enhanced permeability compared to inert particles, possibly even surpassing that of a bare pore.

\begin{figure}[]
    \centering
    \includegraphics[width=3.5in]{fig/flux_vs_PC_SI.png}
    \caption{%
    Gating of colloids by their affinity as in Figure~8b ($\rho_{\text{probe}} =1.3 \text{g/cm}^{-3}$).
    Shaded bands indicate parameter uncertainty $1.0 \le \rho_{\text{probe}} \le 1.4\;\text{g/cm}^{-3}$.
    }
    \label{fig:flux_vs_PC}
\end{figure}

Frey \textit{et al}~\cite{Frey2018} examined permeability of similarly sized colloidal particles with varying surface properties, ranging from inert to FG-domain-affine.
To characterize these surfaces, FG-domain hydrogels were prepared to study colloid partitioning.

The authors prepared droplets of Mac98A FG-domains via rapid dilution, inducing phase separation. 
%following the protocol of Ref.~\cite{Schmidt2015}.
The estimated protein concentration in the gel phase is $\approx 275 \, \text{mg}/\text{ml}$, which corresponds to a volume fraction $\phi_{\text{gel}} \approx 0.2$, assuming a typical dry protein density of $1.35 \, \text{g}/\text{ml}$.

Similarly, droplets of Nup116 FG-domains were prepared with an estimated intra-particle protein concentration of $\phi_{\text{gel}} \approx 400 \, \text{mg}/\text{ml}$, corresponding to $\phi_{\text{gel}} \approx 0.4$. A compromise value $\phi_{\text{gel}} = 0.3$ was used in \todo{Figure~8}.
Within the range $0.2 < \phi_{\text{gel}} < 0.4$, the results are not sensitive to the exact value of $\phi_{\text{gel}}$.

Upon dilution, FG-domains undergo phase separation due to chain cohesiveness, demixing into a dilute FG-poor phase and a condensed FG-rich gel.
The polymer concentration in the bulk phase, $\phi_{\text{gel}}^{\text{out}}$, approximately equals the critical concentration for phase separation.

For a two-component system (FG-domain + buffer), phase separation occurs at:
$
\chi_{\text{PS}}^{\text{FG}} > \frac{1}{2} + \frac{1}{\sqrt{N}} \gtrsim 0.5
$
according to mean-field theory (assuming $N \gg 1$). 
Both chemical potentials and osmotic pressures are equal in the rich and poor phases \cite{Vovk2016, Zilman2018}.

The estimated critical concentration for Nup116 and Nup98A FG-domains is $1 \, \mu\text{g}/\text{ml}$, corresponding to \mbox{$\phi_{\text{gel}}^{\text{out}} \approx 10^{-6}$} \cite{Schmidt2015}.
This extremely low value implies the osmotic pressure is effectively zero in both phases:

\begin{eqnarray}
    \Pi(\phi_{\text{gel}}) = \Pi(\phi_{\text{gel}}^{\text{out}}) \approx 0
\end{eqnarray}

This allows us to estimate the solvent quality $\chi_{\text{PS}}^{\text{FG}}$ using the condition for vanishing osmotic pressure Eq.~(16):

\begin{eqnarray}
    \chi_{\text{PS}}^{\text{FG}} = \frac{-\ln(1-\phi_{\text{gel}}) - \phi_{\text{gel}}}{\phi_{\text{gel}}^2}
\end{eqnarray}

Assuming that FG-domains in the gel are chemically similar to those in nuclear pores, we take the polymer-colloid interaction to be unchanged: $\chi_{\text{PC}}^{\text{gel}} \approx \chi_{\text{PC}}$.

Combining this information, we estimate the insertion free energy $\Delta F_{\text{gel}}$ for particles with different surface properties (i.e., varying $\chi_{\text{PC}}$) into an FG-domain gel of volume fraction $\phi_{\text{gel}}$.
Since the osmotic pressure is negligible, only the surface interaction contributes:

\begin{eqnarray}
    \Delta F_{\text{gel}} =
    \frac{\pi d^3}{6} \cdot \gamma\left(
    \phi_{\text{gel}}, \chi_{\text{PC}},
    \chi_{\text{PS}}^{\text{FG}}\{ \phi_{\text{gel}} \}
    \right) \\
    \text{PC}_{\text{gel}} = \left(\frac{c_{\text{in}}}{c_{\text{out}}}\right)_{\text{gel}} = e^{-\Delta F_{\text{gel}}}
\end{eqnarray}

Owing to the uncertainty in the volume fraction of the FG-particles, $\phi_{\text{gel}}$, we adopt a compromise value of $\phi_{\text{gel}} = 0.30$ in Figure 8b,.  
With an average probe molar mass of \(\sim\!28\;\text{kDa}\) the particle diameter is estimated to lie in the range  $4.0\;\text{nm} \lesssim d\cdot a \lesssim 4.5\;\text{nm}$.
Because our theoretical model is implemented for even integer diameters expressed in segment lengths ($a = 0.76\;\text{nm}$), we use the
nearest value, $d = 6$.




\begin{table*}[h]
\begin{minipage}{\linewidth}
\centering
\caption{Transport-related quantities extracted from the experimental studies.}
\resizebox{0.9\linewidth}{!}{
    \begin{tabular}{p{2.6cm}|p{2.6cm}|p{9cm}|p{0.7cm}|p{0.7cm}|p{0.4cm}}
        %\hline
        Study & Cell culture & Reported quantity & $\!\!N_{\text{pores}}$ & $V_{\text{in}}$ $[f\text{l}]$ & $\!V_{\text{out}}$ $[f\text{l}]$ \\
        \hline
        Ribbeck \textit{et al.}\cite{Ribbeck2001} & HeLa & Translocation rate per one NPC at $\Delta c = 1 \mu\text{M}$ & 2770 & 1130 & $\infty$ \\
        Mohr \textit{et al.}\quad\,\cite{Mohr2009} & HeLa & Rate constant $k$ & 2770 & 1130 & $\infty$ \\
        Popken \textit{et al.}\,\,\cite{Popken2015} & \textit{S. cerevisiae} & Nuclear/cytoplasm concentration $c_{\text{in}}/c_{\text{out}}$ at  $t\! =\!1\text{h}$  & 161 & 4.8 & 60 \\
        Timney \textit{et al.}\,\cite{Timney2016} & \textit{S. cerevisiae} & Characteristic time $\tau$ & 161 & 4.8 & 60 \\
        Frey \textit{et al.}\quad\;\;\,\cite{Frey2018} & HeLa, \newline \textit{S. cerevisiae}, \newline \textit{T. thermophilia} & Rate constant $k$ for HeLa, partitioning in FG hydrogel $P_{\text{gel}}$ for \textit{S. cerevisiae} Nup116 and \textit{T. thermophilia} Nup98A & 2770 & 1130 & $\infty$ \\
        %\hline
    \end{tabular}
    }
    \label{tbl:experimental}
\end{minipage}

\hfill
\vspace{1cm}

\begin{minipage}{0.48\linewidth}
\centering
% \begin{table}[h]
\caption{Translocation rates of inert probe colloids calculated from the experimental data.
Data marked with $^{*}$ are taken directly from the source without modification.}
\resizebox{\linewidth}{!}{
\begin{tabular}{p{3cm}|p{1cm}|p{2cm}|p{2.6cm}}
Probe & Molar weight [kDa] & Translocation rate $[\text{s}^{-1}]$ \mbox{per one NPC} \mbox{at $\Delta c = 1\mu\text{M}$}& Study \\
\hline
Fluorescein-Cys & 0.5 & 231 & Mohr \textit{et al} \\
11 aa peptide & 1.4 & 130 & Mohr \textit{et al} \\
Insulin & 5.8 & 59 & Mohr \textit{et al} \\
Aprotinin & 6.5 & 21.1 & Mohr \textit{et al} \\
% \todo{Profilin} & ? & 13.4 & Mohr \textit{et al} \\
Ubiquitin & 8.5 & 8.7 & Mohr \textit{et al} \\
z-domain & 8.2 & 9.9 & Mohr \textit{et al} \\
Thioredoxin & 13.9 & 5.0 & Mohr \textit{et al} \\
Lactalbumin & 14.2 & 3.54 & Mohr \textit{et al} \\
GFP & 27.0 & 0.50 & Mohr \textit{et al} \\
PBP & 37.0 & 0.064 & Mohr \textit{et al} \\
MBP & 43.0 & 0.054 & Mohr \textit{et al} \\
GFP-HIS & 26.8 & 1.11 & Timney \textit{et al} \\
GFP-1PrA & 34.2 & 0.268 & Timney \textit{et al} \\
GFP-2PrA & 40.7 & 0.146 & Timney \textit{et al} \\
GFP-3PrA & 46.8 & 0.092 & Timney \textit{et al} \\
GFP-4PrA & 53.6 & 0.067 & Timney \textit{et al} \\
GFP-6PrA & 66.8 & 0.040 & Timney \textit{et al} \\
GFP-1PrG & 34.7 & 0.83 & Timney \textit{et al} \\
GFP-2PrG & 42.3 & 0.25 & Timney \textit{et al} \\
MGM & 109 & 0.0091 & Popken \textit{et al} \\
MGM2 & 149 & 0.00250 & Popken \textit{et al} \\
MGM4 & 230 & 0.00116 & Popken \textit{et al} \\
MG3 & 122 & 0.0052 & Popken \textit{et al} \\
MG4 & 150 & 0.00368 & Popken \textit{et al} \\
MG5 & 177 & 0.00197 & Popken \textit{et al} \\
BSA & 68 & $<0.1^{*}$ & Ribbeck \textit{et al} \\
GFP & 29.0 & $2^{*}$ & Ribbeck \textit{et al} \\
mCherry & 28.0 & 0.140 & Frey \textit{et al} \\
EGFP & 28.0 & 0.49 & Frey \textit{et al} \\
efGFP\_8R & 30.0 & 2.66 & Frey \textit{et al} \\
sffrGFP4 18xR→K & 28.0 & 0.53 & Frey \textit{et al} \\
sffrGFP4 25xR→K & 27.0 & 0.224 & Frey \textit{et al} \\
MBP & 43.0 & 0.0112 & Frey \textit{et al} \\
MBP K→R & 43.0 & 0.45 & Frey \textit{et al} \\
\end{tabular}
}
\label{tbl:inert_probes}
% \end{table}
\end{minipage}
\hfill
\vspace{-0.5cm}
\begin{minipage}{0.48\linewidth}
\centering
% \begin{table}[h]
\caption{Translocation rates probe colloids with different surface features calculated from the experimental data and their partition coefficient in the FG-gel taken directly from the source without modification from Frey \textit{et al}.\cite{Frey2018}.
Probe colloids marked with $^{*}$ reported to form oligomers, molar weights presented for monomer state.
}
\resizebox{\linewidth}{!}{
\begin{tabular}{p{3cm}|p{1cm}|p{2cm}|p{1.2cm}|p{1.2cm}}
Probe & Molar weight [kDa] & Translocation rate $[\text{s}^{-1}]$ \mbox{per one NPC} \mbox{at $\Delta c = 1\mu\text{M}$}& PC in Mac98A & PC in Nup116 \\
\hline
% yNTF2 (dimer) & 28 & 84 & 290 & 1400 \\
% rNTF2 (dimer) & 28 & 95 & 3200 & 13000 \\
mCherry & 28 & 0.140 & - & 0.09 \\
EGFP & 28 & 0.49 & 0.11 & 0.33 \\
efGFP\_0W & 28 & 0.84 & 0.09 & 0.42 \\
efGFP\_3W* & 27.5 & 11.5 & 1.50 & 14 \\
efGFP\_5W* & 28 & 12.2 & 2.20 & 15 \\
efGFP\_8F* & 28.3 & 6.0 & 8.30 & 51 \\
efGFP\_8L* & 26.8 & 4.6 & 1.80 & 10 \\
efGFP\_8I* & 28.2 & 9.5 & 2.90 & 23 \\
efGFP\_8M* & 28.2 & 15.4 & 3 & 17 \\
efGFP\_8R & 30 & 2.66 & 0.50 & 1.90 \\
sffrGFP4 & 29 & 22.4 & 14 & 50 \\
sffrGFP4 & 29 & 22.4 & 14 & 50 \\
sffrGFP4 18xR→K & 28 & 0.53 & 0.12 & 0.31 \\
sffrGFP4 25xR→K & 27 & 0.22 & 0.06 & 0.10 \\
sffrGFP4 & 29 & 22.4 & 14 & 50 \\
sffrGFP4 & 29 & 22.4 & 14 & 50 \\
sffrGFP5 & 28 & 9.0 & 0.67 & 5.50 \\
sffrGFP6 & 29 & 39.2 & 100 & 160 \\
sffrGFP7 & 29 & 43 & 200 & 200 \\
GFP\_MaxR\_5W* & 28 & 116 & 2100 & 4000 \\
GFP\_MaxR\_8i* & 27.6 & 182 & 2000 & 4100 \\
GFPNTR\_2B7* & 27.1 & 224 & 1200 & 1600 \\
GFPNTR\_7B3* & 26.3 & 238 & 1700 & 1400 \\
GFPNTR\_3B1* & 28.5 & 60 & 290 & - \\
GFPNTR\_3B7* & 27.5 & 106 & 3000 & - \\
GFPNTR\_3B8* & 27.5 & 94 & 1800 & - \\
GFPNTR\_3B9 & 28.3 & 122 & 4800 & - \\
\end{tabular}
}
\label{tbl:attr_probes}
% \end{table}
\end{minipage}
\end{table*}


\begin{table*}[t]
\centering
\caption{Analysis of the content of intrinsically disordered FG nucleoporin material in NPCs.}
\label{tbl:idr_npc}
\resizebox{0.9\textwidth}{!}{%
\begin{tabular}{|l|c|c|c|}
\hline
FG nucleoporin in \textit{S. cerevisiae} & \begin{tabular}[c]{@{}c@{}}Number of amino acids\\ in IDR per protein\end{tabular} & \begin{tabular}[c]{@{}c@{}}Protein copy\\ number per NPC\end{tabular} & \begin{tabular}[c]{@{}c@{}}Number of amino acids\\ in IDRs per NPC\end{tabular} \\
\hline
Nsp1       & 617 & 48  & 29616  \\
Nup49      & 251 & 32  & 8032   \\
Nup57      & 255 & 32  & 8160   \\
Nup145N    & 433 & 16  & 6928   \\
Nup116\textsuperscript{d)}  & 960 & 16  & 15360  \\
Nup100     & 800 & 16  & 12800  \\
Nup60\textsuperscript{e)}   & 539 & 16  & 8624   \\
Nup1       & 857 & 16  & 13712  \\
Nup42      & 382 & 8   & 3056   \\
Nup159     & 685 & 16  & 10960  \\
\hline
\textbf{Total} &       & 216 & \textbf{117248} \\
\hline
\end{tabular}
}
\vspace{0.5em}
\begin{minipage}{0.9\textwidth}
\footnotesize
\textsuperscript{a)} Nup2 was excluded as this protein is non-essential. \\
\textsuperscript{b)} According to Yamada et al.~\cite{Yamada2010}. \\
\textsuperscript{c)} According to Kim et al.~\cite{Kim2018}. \\
\textsuperscript{d)} The full C-terminal region up to amino acid 960 was considered IDR. \\
\textsuperscript{e)} The full protein was considered disordered, based on prediction.
\end{minipage}
\end{table*}



% \pagebreak
\onecolumn
\subsection*{List of variables and abbreviations}
\todo{TO BE REVISED}

%
\begin{tabularx}{\linewidth}{l X}
    % \toprule
    % \textbf{Variable} & \textbf{Definition} \\
    % \midrule
    $a$ & Kuhn segment length \\
    $b_0$, $b_1$ & Polymer depletion/accumulation correction coefficients \\
    $c$ & Local concentration of diffusing colloid particles in the steady state \\
    $d$ & Diameter of the spherical or cylindrical colloid particle \\
    $D$ & Local diffusion coefficient of colloid particles \\
    $D_0$ & Diffusion coefficient of colloid particles in pure solvent \\
    $\delta r$, $\delta z$ & Size of the discretization step in radial and axial directions \\
    $\Delta F$ & Analytical insertion free energy penalty to place a spherical particle \\
    $\Delta F_{\text{cyl}}$ & Analytical insertion free energy penalty to place a cylindrical particle \\
    $\Delta F_{\text{cyl}}^{\text{osm}}$ & Osmotic contribution to $\Delta F_{\text{cyl}}$ \\
    $\Delta F_{\text{cyl}}^{\text{sur}}$ & Surface contribution to $\Delta F_{\text{cyl}}$ \\
    $\Delta F_{\text{osm}}$ & Osmotic contribution to $\Delta F$ for a spherical particle\\
    $\Delta F_{\text{SF-SCF}}$ & Insertion free energy penalty to place a cylindrical particle calculated using the Scheutjens-Fleer approach \\
    $\Delta F_{\text{sur}}$ & Surface contribution to $\Delta F$ for a spherical particle\\
    $\Delta \phi$ & Change in polymer segment volume concentration due to the presence of a colloid particle \\
    $p(z)$ & Prefactor introduced to correct the difference between the resistance calculated using half-cylinder shells and oblate spheroid shells \\
    $G$ & Statistical weight of a subchain \\
    $\gamma$ & Surface tension coefficient \\
    $h_{r}$, $h_{z}$, $h_{\theta}$ & Lam\'e coefficients (scale factors) in curvilinear coordinate transformations \\
    $i$, $k$ & Indices of the discretized grid in radial $r$ and axial $z$ directions, respectively \\
    $J$ & Net flux of colloid particles through the pore in the steady state \\
    $j$ & Colloid particle flux density in the steady state \\
    $k_B$ & Boltzmann constant \\
    $L_{0}$ & Membrane thickness \\
    $L$ & Effective length of the pore considering volume exclusion \\
    $N$ & Number of Kuhn segments in the brush-forming chains \\
    $\Pi$ & Flory mean-field local osmotic pressure \\
    $\phi$ & Local volume fraction of polymer segments in a polymer brush \\
    $\phi^{\ast}$ & Apparent local volume fraction of polymer segments \\
    $\phi^{\text{ins}}$ & Local volume fraction of polymer segments disturbed by an inserted particle \\
    $\psi$ & Scalar potential function introduced to express the steady state solution \\
    $\rho$ & Local resistivity (inverse conductivity) to colloid particle diffusion \\
    $r$ & Radial coordinate in cylindrical coordinates \\
    $r_{\text{ext}}$ & Base radius of half-cylinder shells used in resistance calculations\\
    $r_{\text{p}}^{0}$ & Radius of the pore \\
    $r_{\text{p}}$ & Effective radius of the pore considering volume exclusion \\
    $r_{c}$ & Radial coordinate of the colloid particle center in cylindrical coordinates \\
    $R$ & Total resistance of the pore to colloid particle diffusion in a semi-infinite solution \\
    $R_{0}$ & Total resistance of an empty pore (without polymer brush) \\
    $R_{(z, \pm\infty)}$ & Resistance from position $z$ to infinity along positive or negative $z$-direction \\
    $R_{\text{ext}}$ & Convergent flow contribution to the total resistance \\
    $R_{\text{int}}$ & Resistance contribution from the pore channel \\
    $\varrho_(z)$ & Resistance per unit length at position $z$ \\
    $\varrho^{0}(z)$ & Resistance per unit length at position $z$ for an empty pore \\
    $\sigma$ & Grafting density (number of polymer chains per unit area) \\
    $T$ & Temperature \\
    $\theta$ & Angular coordinate in cylindrical coordinates \\
    $u$ & Segment potential in Scheutjens-Fleer method\\
    $\xi$ & Correlation length in a semi-dilute polymer solution \\
    $z$ & Axial coordinate in cylindrical coordinates \\
    $z_a$, $z_b$ & Integration limits along the $z$-axis \\
    $\bm{1}_{V}$, $\bm{1}_{S}$ & Volume and surface indicator functions \\
    $\bm{e}_r$, $\bm{e}_z$ & Local covariant basis vectors in curvilinear coordinates \\
    $\bm{S}\{r_{c}\}$ & Particle surface projection matrix for a spherical particle with center at $r_{c}$ \\
    $\bm{S}_{\theta\downarrow}$ & Particle surface projection onto the $rz$-plane for a spherical particle with center at $r_{c}$ \\
    $\bm{V}$ & Boolean array representing the presence of the spherical particle in the discretized grid \\
    $\bm{V}\{r_{c}\}$ & Particle volume projection matrix for a spherical particle with center at $r_{c}$ \\
    $\bm{V}_{\theta\downarrow}$ & Particle volume projection onto the $rz$-plane for a spherical particle with center at $r_{c}$ \\
    $\chi_{\text{CS}}$ & Flory colloid-solvent interaction parameter \\
    $\chi_{\text{PC}}$ & Flory polymer-colloid interaction parameter \\
    $\chi_{\textrm{PS}}$ & Flory polymer-solvent interaction parameter \\
    %\bottomrule
\end{tabularx}
%

\printbibliography
\end{document}