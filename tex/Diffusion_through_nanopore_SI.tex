\documentclass[10pt, a4paper, twocolumn]{article}
\raggedbottom
\usepackage{geometry}
\geometry{
    a4paper,
    left= 15mm,
    right = 15mm,
    top=20mm,
    bottom = 25mm,
    }
\usepackage{caption}
\captionsetup[figure]{
    font=small, 
    labelfont=bf
    }

\usepackage{graphicx}
\usepackage{amsmath, amssymb, amsfonts, mathtools}
\usepackage{xcolor}
\usepackage{bm}
\usepackage{multicol}
\usepackage{ltablex}
\usepackage{wrapfig}
\usepackage{float}

% \usepackage[utf8]{inputenc}
\usepackage[T1]{fontenc}
\usepackage{lmodern}        % better fonts in T1

\usepackage{authblk}
\renewcommand\Authfont{\normalsize}
\renewcommand\Affilfont{\small}

\usepackage[
  backend=biber,
  style=nature,
  maxnames=12,  % limit names in citations
  doi=false,
  url=false,
  eprint=false,
  isbn=false,
]{biblatex}
\addbibresource{biblio.bib}

%\REMOVE BEFORE SUBMISSION
\usepackage[switch]{lineno}
\linenumbers


\usepackage{titlesec}
\titleformat{\section}[block]
  {\normalfont\bfseries\fontsize{12pt}{12pt}}  % Style
  {Supplementary Note \thesection.} % Label
  {0.5em}                 % Spacing between label and title
  {}                      % Before-code (empty)
  %[\vspace{0.5em}]        % After-code (optional vertical space)

%\REMOVE BEFORE SUBMISSION
\usepackage[
  author={},         % no author shown
  subject={},        % no subject
  date={},           % no date/timestamp
  color=yellow,      % note color (viewer-dependent rendering)
  icon=Note,         % classic sticky note icon
  open=false         % closed by default
]{pdfcomment}
\newcommand{\todo}[1]{\pdfcomment{#1}} %\REMOVE ALL \todo{...} BEFORE SUBMISSION

\newcommand\scalemath[2]{\scalebox{#1}{\mbox{\ensuremath{\displaystyle #2}}}}
\newcommand{\sign}{\text{sign}}

% Add S prefix to figures, equations, tables and references
\renewcommand{\thefigure}{S\arabic{figure}}
\renewcommand{\theequation}{S\arabic{equation}}
\renewcommand{\thetable}{S\arabic{table}}
\makeatletter
\renewcommand{\fnum@table}{\textbf{Table~\thetable}}
\makeatother
\DeclareFieldFormat{labelnumber}{S#1}


\addbibresource{biblio.bib}
\onecolumn

\title{SUPPLEMENTARY INFORMATION
\\A polymer filling enhances the rate and selectivity of colloid permeation across mesopores}

\author[1]{Mikhail Y. Laktionov}
\author[2]{Frans A. M. Leermakers}
\author[3]{Ralf P. Richter}
\author[4,5]{Leonid I. Klushin}
\author[1]{Oleg V. Borisov}

\affil[1]{CNRS, Université de Pau et des Pays de l'Adour, UMR 5254, 
Institut des Sciences Analytiques et de Physico-Chimie pour l'Environnement et les Matériaux, 64053 Pau, France}

\affil[2]{Physical Chemistry and Soft Matter, Wageningen University, Stippeneng 4, 6708 WE, Wageningen, The Netherlands}

\affil[3]{University of Leeds, School of Biomedical Sciences, Faculty of Biological Sciences, 
School of Physics and Astronomy, Faculty of Engineering and Physical Sciences, 
Astbury Centre for Structural Molecular Biology, 
and Bragg Center for Materials Research, Leeds, LS2 9JT, United Kingdom}

\affil[4]{Branch of Petersburg Nuclear Physics Institute 
named by B.P. Konstantinov of National Research Centre ''Kurchatov Institute'', 
Institute of Macromolecular Compounds, 199004 St. Petersburg, Russia}

\affil[5]{American University of Beirut, Department of Physics, Beirut 1107 2020, Lebanon}

\date{}

\begin{document}
\maketitle


\vspace{-8ex}
\begin{center}
\small{
All authors contributed equally to this work.\\
$^{*}$Corresponding author:  \href{mailto:r.richter@leeds.ac.uk}{r.richter@leeds.ac.uk}
}
\end{center}
\vspace{1ex}

\begin{center}
        \textbf{This file includes:} Supplementary Notes 1 to 7, Supplementary Tables S1 to S4, and Supplementary References.
\end{center}
\twocolumn


\pagebreak
%%%%%%%%
\section{Estimates of nucleopore geometry and polymer filling}
%%%%%%%%

\textbf{Nuclear pore geometry.}
According to recent structural work \cite{Zimmerli2021, Schuller2021}, the diameter, and to a lesser extent the height, of the inner FG-domain-filled channel of nuclear complexes can vary depending on the cellular environment and species.
Diameters between 40 and 80 nm, and heights between 30 and 70 nm, were reported.
In our model, we took diameter and height to extend over 56 Kuhn segments, corresponding to 42.6 nm.
This choice reflects that the plain cylinder shape with evenly distributed anchors for homopolymers in our model is a simplifcation of the more complex shape, and the heterogeneity in anchorage and composition of FG domains, in the NPC.

\bigskip\noindent
\textbf{Polymer filling.}
To estimate the amount of intrinsically disordered protein material in the NPC, we considered the copy number, and the size of the region predicted to be intrinsically disordered, for each FG nucleoporin in \textit{S. cerevisiae} yeast NPCs.
For the copy numbers, we used the data by Kim et al. \cite{Kim2018}, who assessed all yeast nucleoporins except Nup2 through quantitative mass spectrometry.
A large subset of nucleoporins (though missing some major FG nucleoporins) was independently analysed by Rajoo et al. \cite{Rajoo2018} through quantitative image analysis, with consistent results.
The size of the instrinsically disordered regions had been predicted by Yamada et al. \cite{Yamada2010}.
We included here the total mass of all intrinsically disordered regions irrespective of their FG motif content, as they contribute to the filling of the pore.
The total number of FG nucleoporins (excluding Nup2 as non-essential) per NPC thus is 216, and the total content in intrinsically disordered parts amounts to  $11.7 \times 10^4$ amino acids (Supplementary Table~\ref{tbl:idr_npc}).

In our model, we took a grafting density of $\sigma = 0.02$ and a degree of polymerization of $N = 300$.
With the chosen pore geometry, a Kuhn segment length of $a = 0.76$ nm, and 2 amino acids per Kuhn segment, this corresponds to 197 polymers and a total of $11.8 \times 10^4$ amino acids, in close agreement with the properties of the yeast NPC.
The total amount of polymer is also in close agreement with an estimated 14.5 MDa FG domain mass in human NPCs \cite{Ng2023}, which corresponds to $13.1 \times 10^4$ amino acids, considering an average mass per amino acid of 110 Da.


\pagebreak
%%%%%%%%
\section{Analytical estimation of the pore resistance, separating internal and external contributions}
%%%%%%%%

\begin{figure*}[h]
    \centering
    \includegraphics[width=0.85\linewidth]{fig/coordinate_system.png}
    \caption{%
        \textbf{Left:}
        Steady-state solution of the diffusion equation for a point-like particle diffusing through an empty cylindrical pore of finite thickness.
        Iso-concentration surfaces, $c = \text{const}$, are represented by contour lines with labeled concentration values.
        Blue and red axes indicate radial and axial coordinates, respectively.
        \textbf{Right:}
        Intrinsic orthogonal curvilinear coordinate system for the pore.
        Radial and axial coordinates are parameterized as $r'(r,z)$ and $z'(r,z)$, respectively.
        Solid lines indicate surfaces of rotation about the pore axis.
        Red lines correspond to surfaces of constant $z'$; blue lines correspond to surfaces of constant $r'$.
        Semi-planes with constant angular coordinate $\theta$ are not shown.
        Local basis vectors of the intrinsic coordinate system ($\hat{e}_r$, $\hat{e}_z$) are illustrated by arrows.
        Lamé coefficients are defined by the magnitudes of the local basis vectors as $h_r = |\hat{e}_r|$, $h_z = |\hat{e}_z|$, and $h_{\theta} = |\hat{e}_{\theta}|$.
        Pore radius $r_{\text{p}}^{0} = 20$ and thickness $L_{0} = 20$ were chosen for illustrative purposes (i.e., to render all numbers clearer than what would be the case with the parameters in Figure 1).
        The membrane is illustrated in striped green.
    }
    \label{fig:empty_pore_solution}    
\end{figure*}

\textbf{Development of the analytical model in continuous space.}
To construct an approximate analytical solution, we first consider a bare pore with a set geometry (Figure 1).
The diffusivity $D_0$ is constant throughout the solution.
Boundary conditions are set as $c(z=-\infty) = 1$ and $c(z=+\infty) = 0$.
The steady-state solution of the diffusion equation, defined by $\partial c/\partial t = 0$, produces the concentration profile shown in Figure~\ref{fig:empty_pore_solution} (left).
For a pore in an infinitely thin membrane, the iso-concentration surfaces are known to form oblate spheroids with the pore rim as their focal circle \cite{Cooke1966}.
This approximation remains valid for the exterior of the pore in a membrane of finite thickness, as considered here.
The solution in the pore lumen is approximated by equally spaced, disk-shaped iso-concentration surfaces. 
The flux density field is directly related to the concentration gradient via Fick's law \mbox{$\bm{j} = -D_0 \nabla c$}.

A polymer filling within the pore modifies the local diffusion coefficient $D$ and generates a free-energy landscape, resulting in an effective position-dependent diffusion coefficient (i.e., conductivity) $\tilde{D}(r,z) = D\,e^{-\Delta F/k_B T}$, or local resistivity $\rho = \tilde{D}^{-1}$.
Since the flux density $\bm{j}$ is a conservative vector field, we define a scalar potential function $\psi = c\,e^{\Delta F/k_B T}$, such that $\bm{j} = -\tilde{D}\nabla\psi$.
Consequently, the steady-state iso-concentration surfaces differ from the oblate spheroids of the bare pore; however, they retain a similar structure for iso-values of $\psi$ (Figure 6).

For the exterior region, we introduce an intrinsic curvilinear coordinate system $(r', z', \theta)$ aligned with the approximate iso-surfaces of $\psi$, as depicted in Figure~\ref{fig:empty_pore_solution} (right) and defined as follows.
Level sets of the potential function $\psi$ form a family of oblate hemispheroids $z'$, indexed by their intersection points with the $z$-axis:
\begin{gather}
    r' = \left\{(r,z) \mid \nabla f \cdot \nabla \psi = 0,\; f=f(r,0)\right\}
\end{gather}
The flux-density stream surfaces, perpendicular to $\psi$, form a family of hyperboloids of revolution $r'$, indexed by their intersection radii with the plane $z=0$:
\begin{gather}
    z' = \left\{(r,z) \mid \psi(r,z)=\psi(0,z)\right\}
\end{gather}
Half-planes of constant azimuthal angle $\theta$ remain unchanged from the original cylindrical coordinate system.

In the exterior region ($|z| > L/2$, where we consider the effective pore size to account for the finite colloid size), the intrinsic coordinate system $(r', z', \theta)$ is parameterized using the original cylindrical coordinates $(r,z,\theta)$ and the distance from the pore opening $z_{\text{ext}} = |z| - L/2$:
\begin{align}
    r'(r,z) &= r\sqrt{1 + \frac{z_{\text{ext}}^2}{r_{\text{p}}^2}},\\[4pt]
    z'(r,z) &= z_{\text{ext}}\frac{\sqrt{r_{\text{p}}^2 - r^2}}{r_{\text{p}}} + \text{sign}(z)\frac{L}{2}.
\end{align}
The corresponding Lam\'e  coefficients are:
\begin{align}
    h_r &= \frac{\sqrt{r_{\text{p}}^2 + z_{\text{ext}}^2 - r^2}}{\sqrt{r_{\text{p}}^2 - r^2}},\\[4pt]
    h_z &= \frac{\sqrt{r_{\text{p}}^2 + z_{\text{ext}}^2 - r^2}}{\sqrt{r_{\text{p}}^2 + z_{\text{ext}}^2}},\\[4pt]
    h_{\theta} &= \frac{r\sqrt{r_{\text{p}}^2 + z_{\text{ext}}^2}}{r_{\text{p}}},\\[4pt]
    \tilde{h}(r,z) &= h_r h_{\theta} h_z^{-1} = \frac{r}{r_{\text{p}}}\frac{r_{\text{p}}^2 + z_{\text{ext}}^2}{\sqrt{r_{\text{p}}^2 - r^2}}.
\end{align}

The conductivity integrated over the oblate hemispheroids in the exterior region is given by:
\begin{equation}
  \label{eq:rho_ext}
  \varrho_{\text{ext}}^{-1}(z)= 2\pi\int_{0}^{r_{\text{p}}^{}} 
  \tilde{D}\left( r'(r,z), z'(r,z) \right)\tilde{h}(r,z)\,dr.
\end{equation}
Inside the pore ($|z|\leq L/2$), assuming no significant radial flux, the conductivity of a disk cross-section at position $z$ is approximated as:
\begin{equation}
  \varrho_{\text{int}}^{-1}(z)= 2\pi\int_{0}^{r_{\text{p}}^{}} \tilde{D}(r,z)\,r\,dr.
\end{equation}
Therefore, the total resistances of the exterior and interior regions are respectively obtained by:
\begin{align}
   \label{eq:R_ext}
   R_{\text{ext}} &=2\int_{+L/2}^{+\infty}\varrho_{\text{ext}}(z)\,dz,\\[5pt]
   \label{eq:R_int}
   R_{\text{int}} &=\int_{-L/2}^{+L/2}\varrho_{\text{int}}(z)\,dz.
\end{align}

\bigskip

As a control, we revisit the resistance of a bare pore of finite thickness to the diffusion of a point-like colloid using Eqs.~(\ref{eq:R_ext},~\ref{eq:R_int}).
For a bare pore, the effective diffusion coefficient is simply $\tilde{D} = D_0$:
\begin{eqnarray}
    \varrho^{0}_{\text{int}}(z) &=& \frac{1}{D_0 \pi r_{\text{p}}^2},\\[4pt]
    \varrho^{0}_{\text{ext}}(z) &=& \frac{1}{2\pi D_0\left(z_{\text{ext}}^2 + r_{\text{p}}^2\right)}.
\end{eqnarray}
Integration over the full domains of $z$ yields the expected classical result \cite{Brunn1984} stated in Eq. (9):
\begin{eqnarray}
    \label{eq:r_empty}
    R_{\text{ext}}^{0} &=& 2 \int_{-\infty}^{-L/2} \varrho_{\text{ext}}^{0}(z)\,dz = \frac{1}{2 D_0 r_{\text{p}}},\\[4pt]
    R_{\text{int}}^{0} &=& \int_{-L/2}^{+L/2} \varrho_{\text{int}}^{0}(z)\,dz = \frac{L}{D_0 \pi r_{\text{p}}^2}.
\end{eqnarray}

\bigskip\noindent
\textbf{Application to a discrete cylindrical lattice.}
Our numerical approach inherently employs the discrete cylindrical lattice from the SF-SCF calculations (Supplementary Note 3), with discretization steps $\delta z = \delta r = 1$.
Consequently, continuous integration across equipotential surfaces foliating space is replaced by summation over discrete conductive layers, each bounded by two adjacent equipotential surfaces. 
Layers are indexed by the first equipotential surface intersecting the $z$-axis (see Figure~\ref{fig:integration_scheme}).

\begin{figure}[]
    \centering
    \includegraphics[width=\linewidth]{fig/resistance_integration.png}
    \caption{
        Schematic of the numerical integration of the local conductivity/resistance on a cylindrical lattice.    
        \textbf{Left:}
        In the exterior region, conductivities are integrated over nested cylindrical layers (red), each indexed by $z$.
        A selected layer with radius $r_{\text{ext}}$ and height $z_{\text{ext}}$ is highlighted with a darker red color.
        Dashed red lines illustrate the boundaries of the cylindrical layers that approximate the oblate hemispheroidal equipotentials shown in Figure~\ref{fig:empty_pore_solution}.
        In the interior region, integration is performed over disk-shaped layers (blue), with a selected layer highlighted in darker blue.
        The membrane is shown in striped green.
        \textbf{Right:}
        Perspective view to illustrate the nested cylindrical exterior layers (in different tones of red) and disk-shaped interior layers (in different tones of blue) in three dimensions.
    }
    \label{fig:integration_scheme}
\end{figure}

In the interior region, this discretization is straightforwardly applied, giving the resistance of a discrete disk-shaped layer as:
\begin{equation}
    \varrho_{\text{int}}^{\text{lat}}(z) 
    =\left[\pi \sum_{r=0}^{r_{\text{p}}-1}(2r+1)\,\tilde{D}[r,z]\right]^{-1}.
\end{equation}

For numerical integration in the exterior region, we approximate the equipotential surfaces as cylindrical rather than oblate hemispheroidal layers (Figure~\ref{fig:integration_scheme}).
Thus, instead of nested oblate hemispheroids (or "bowls"), the discretization yields nested cylinders (or "buckets") with increasing radius $r_{\text{ext}} = r_{\text{p}} + |z| - L/2$ and height $z_{\text{ext}} = |z| - L/2$.

The resistance of each cylindrical layer naturally underestimates the corresponding oblate hemispheroidal layer at the same position $z$.
To compensate, we introduce a correction factor $p(z)$, which equals the ratio of the resistances of an oblate spheroidal layer to a cylindrical layer with identical local conductivity $\tilde{D}$:
\begin{gather}
    \label{eq:prefactor}
    \begin{aligned}
        p(z) &= r_{\text{p}} \left(z_{\text{ext}} \log{\frac{r_{\text{ext}}}{r_{\text{ext}} - 1}} + r_{\text{ext}}^{2}\right)^{-1} \times \\[4pt]
        &\quad\times\left[\operatorname{atan}{\frac{z_{\text{ext}}}{r_{\text{p}}}} - \operatorname{atan}{\frac{z_{\text{ext}} - 1}{r_{\text{p}}}}\right]^{-1}.
    \end{aligned}
\end{gather}
The resistance of a layer at position $z$ in the exterior region, computed on a cylindrical lattice, then is:
\begin{equation}
    \begin{aligned}
        &\varrho_{\text{ext}}^{\text{lat}}(z) = 
        \frac{p(z)}{\pi}\left[
        \sum_{r=0}^{r_{\text{ext}}-1}(2r+1)\tilde{D}[r,z_a]\right. + \\[4pt]
        &\quad+\left. 2 \log\left(\frac{r_{\text{ext}}}{r_{\text{ext}}\!-\! 1}\right)\sum_{z^{\prime}=z_{a}}^{z_{b}}\tilde{D}[r_{\text{ext}}\!-\!1,z^{\prime}]
        \right]^{-1}
    \end{aligned}
\end{equation}
with indexing limits defined by:
\begin{equation*}
    \begin{cases}
        z_{a} = -z,\quad z_{b} = -L/2-1,&\text{if } z < -L/2,\\[4pt]
        z_{a} = L/2,\quad z_{b} = z-1,&\text{if } z > L/2.
    \end{cases}
\end{equation*}

To account for the resistance of the semi-infinite reservoirs beyond the integration domain on the cylindrical lattice, we evaluate the analytical expression for the exterior resistance, Eq.~\eqref{eq:rho_ext}, from the integration boundary $z$ to infinity:

\begin{eqnarray}
    \label{eq:r_reservoir}
    R_{(z, \pm\infty)} = \pm\!\! \int\displaylimits_{z}^{\pm\infty}\!\! \varrho_(z') \, \text{d}z' = \frac{\arctan\left({r_{\text{p}}}/{z_{\text{ext}}} \right)}{2 D_0 \pi r_{\text{p}}^{0}}
\end{eqnarray}

Finally, the total resistance of the pore, integrated on the discrete cylindrical lattice, is given by:

\begin{eqnarray}
    R_{\text{int}}^{\text{lat}} = \sum_{z=-L/2}^{+L/2} \varrho_{\text{int}}^{\text{lat}}(z)
    \\
    R_{\text{ext}}^{\text{lat}} = R_{(z_{\text{left}}, -\infty)} + \sum\limits_{\mathclap{\substack{z \in [z_{\text{left}},-L/2)\\z \in (L/2, z_\text{right}]}}} \varrho_{z} + R_{(z_{\text{right}}, +\infty)}
    \\
    \label{eq:R_total}
    R_{\text{lat}} = R_{\text{ext}}^{\text{lat}} + R_{\text{int}}^{\text{lat}}
\end{eqnarray}
where $z_{\text{left}}$ and $z_{\text{right}}$ are the limits of the discrete domain.

Equation~\eqref{eq:R_total} concludes the analytical estimation scheme, giving the total resistance $R_{\text{lat}}$ of the pore.


\pagebreak
%%%%%%%%
\section{Computing polymer density maps with the numerical Scheutjens-Fleer self-consistent field (SF-SCF) method}
%%%%%%%%

The numerical SF-SCF method was deployed with pore, polymer brush and colloid parameters as defined in Figure~\ref{fig:sf-scf_scheme}.
With this method, we (i) calculated the spatial polymer distribution $\phi(r,z)$ and (ii) evaluated the insertion free energy for cylindrical colloids into the brush.
The latter data served as a reference for the approximate analytical model.

\bigskip\noindent
\textbf{Main features of the SF-SCF method.}
The method is based on the minimization of the excess Helmholtz energy in the incompressible limit, that is, at each specified coordinate $(r,z)$ all volume fractions add up to unity.
The method uses lattice approximations, a (local) mean-field approximation with short-ranged interactions parameterized by Flory-Huggins interaction parameters, and employs the freely jointed chain (FJC) model for conformational degrees of freedom of polymer chains composed of segments that fit on the lattice sites. 

\bigskip\noindent
\textbf{Discretization and geometrical features of the pore.}
As our problem exhibits axial symmetry, space was discretized into a cylindrical lattice with a degenerate angular direction,
implemented as a homogeneously curved two-gradient lattice defined by longitudinal $z$ and radial $r$ coordinates (Figure~\ref{fig:sf-scf_scheme}).
This $(r,z)$ coordinate system is visualized as a two-dimensional Cartesian coordinate system (Figure~\ref{fig:sf-scf_scheme}, right); however, each element of the lattice represents a square toroid (of volume $2 \pi r a^3$) instead of a square (of area $a^2$).
The mean-field approximation is applied in the angular direction, meaning properties in the angular direction are uniform.

\begin{figure}[h]
    \centering
    \includegraphics[width = 0.95\linewidth]{fig/sf-scf_scheme.png}
    \caption{
    Left - Three-dimensional schematic of the lattice and geometrical features of the cylindrical pore model in the SF-SCF method.
    Right - Key with color code of the modeled objects (top) and their representation on a discrete two-gradient lattice (bottom).
    }
    \label{fig:sf-scf_scheme}
\end{figure}

The shape of the membrane and any colloids were fixed.
The membrane was modeled as a toroid with a rectangular cross-section of physical height $L_0$, a physical inner radius $r_{\text{p}}^0$, and an outer radius large enough to be effectively infinite with respect to the polymer distribution (Figure~\ref{fig:sf-scf_scheme}, green).
For $L_0/2 < z < L_0/2$, we refer to the coordinates $\{z_\text{p}\}$ inside the pore.
The colloid was modeled as a cylinder with height and diameter $d$ (Figure~\ref{fig:sf-scf_scheme}, yellow).
The membrane and colloid lattice elements were modeled as impermeable to the solvent and the polymers, illustrated as crossed-out cells in matching colors in Figure~\ref{fig:sf-scf_scheme} (right).

Each polymer chain was represented as a FJC with $s=1, \cdots, N$ segments of length $a$, modeled as a step-weighted random walk on the lattice (Figure~\ref{fig:sf-scf_scheme}, right; red circles with black connecting lines).
The first segment $s=1$ is constrained to the coordinate next to the pore wall (grafting condition).
The weights of each step direction were set according to the boundary conditions and the lattice curvature.
Steps into impermeable lattice elements had zero weight.
Other steps were weighted according to the change in volume per lattice element, with steps towards increasing $r$ consequently being favored, and steps towards decreasing $r$ being disfavored, compared to steps along the $z$-direction.
The resulting local polymer concentration $\phi$ is a weighted sum of all possible paths the chain can take (Eq.~\eqref{eq:sum_to_phi}).

\bigskip\noindent
\textbf{Minimization of excess Helmholtz energy.}
The excess Helmholtz energy was minimized through a Lagrangian with multipliers $\alpha(r,z)$:
\begin{equation}
    \label{eq:fe_lagrangian}
    \begin{aligned}
        &F[\bm{u}, \boldsymbol{\phi}, \boldsymbol{\alpha}] =\\
        &= F_{\text{mix}}[\bm{u}] - \sum\limits_{r,z} \sum\limits_X u_X(r, z) \phi_X(r, z) + \\
        &+ F_{\text{int}} [\boldsymbol{\phi}] 
        + \sum\limits_{r,z} \alpha(r, z) \left( \sum\limits_X \phi_X(r, z) - 1 \right),
    \end{aligned} 
\end{equation}
where $\phi_X(r,z)$ is the local volume concentration function of segment type $X$ (polymer, colloid, or solvent), $u_X(r, z)$ is the potential field of segment type $X$, the functional $F_{\text{mix}}[\bm{u}]$ is the mixing free energy expressed in segment potentials, and the functional $F_{\text{int}} [\boldsymbol{\phi}]$ is the interaction energy expressed in  segment densities.
The second therm $- u\cdot \phi$ transforms the free energy from potential space to density space.

The condition for the minimum of the functional is a system of three variational equations:
\begin{equation}
    \label{eq:energy_min_system}
    \begin{cases}
        \frac{\partial F}{\partial \boldsymbol{\alpha}} = 0 \\
        \frac{\partial F}{\partial \boldsymbol{\phi}} = 0 \\
        \frac{\partial F}{\partial \bm{u}} = 0 \\
    \end{cases}
\end{equation}
The first condition in Eq.~\eqref{eq:energy_min_system} ensures system incompressibility for each specified coordinate $(r,z)$.

The second condition in Eq.~\eqref{eq:energy_min_system} results in the segment potential field equation for a regular solution:
\begin{equation}
    \label{eq:u-phi}
    u_A(r, z) =\sum\limits_{B} \chi_{A,B} \left(\langle \phi_B(r,z)\rangle - \phi_B^b \right) + \alpha(r, z),
\end{equation}
where $\chi_{A,B}$ is the Flory interaction parameter between segments $A$ and $B$, and $\phi_B^b$ is the volume fraction of $B$ in the bulk (equal to 1 for the solvent and zero otherwise).
The angular brackets indicate the site average $\langle X(r,z)\rangle = \sum_{r'=r-1,\ r,\ r+1}\sum_{z'=z-1,\ z,\ z+1} \lambda(r',r,z',z) X(r',z')$ with a priori step probabilities $\lambda(r',r,z',z)=\lambda_{r'-r}(r) \lambda_{z'-z}(z) $ and individual steps obeying to $\lambda_{-1}(r)=\frac{1}{6}(1-\frac{1}{2r-1}$, $\lambda_1 (r)=\frac{1}{6}(1+\frac{1}{2r-1}$, $\lambda_{-1}(z)=\lambda_1(z)=1/6$ and $\lambda_0(z)=\lambda_0(r)=4/6$.
It is easily seen that the sum over the step probabilities equals unity (as it should): $\sum_z'\sum_r' \lambda(r',r,z',z) = 1$.

The third condition in Eq.~\eqref{eq:energy_min_system}, the minimization with respect to potentials, links the chain partition function with the local polymer concentration $\phi$ in a diffusion-like equation (Eq.~\eqref{eq:propagation}).

Any subchain of the FJC can be considered a Markov process starting at some segment $s_i$ at coordinates $r_i, z_i$ that goes through intermediate steps to segment $s_k$ at coordinates $r_k, z_k$ (Figure~\ref{fig:sf-scf_scheme}, bottom right).
Such a process has a statistical weight $G(\{r_k, z_k\}, s_k | \{r_i, z_i\}, s_i)$.

All the Markov processes that start with segment $s_i$ and end with segment $s_k$ at fixed coordinates $\{r, z\}$ are found as the sum over all possible and allowed starting coordinates:
\begin{equation}
    \label{eq:sum_to_phi}
    G(\{r, z\}, s_k | s_i) = \sum_{r^{\prime}, z^{\prime}} G(\{r, z\}, s_i | \{r^{\prime}, z^{\prime}\}, s_i)
\end{equation}

The statistical weight of all possible processes that start from segment $s_i$ and end with segment $s_k$ is the sum over all possible coordinates:
\begin{equation}
    G(s_k | s_i) = \sum_{r, z} G(\{r, z\}, s_k | s_i) \pi (2r-1)
\end{equation}
where the weight $\pi (2r-1)$ is the degeneracy in the radial direction.
When $s_i=1$ and $s_k=N$, the result contains the statistical weight of all possible and allowed conformations of the chain and is the single-chain partition function $G(N|1)$.

There are two initial (starting) conditions.
The first one, $G(\{r, z\}, 1|1)=G(r,z) \delta_{r_p,z_p}(r,z) $ is the initial condition of the Markov process which contains just one segment (starts and ends at segment $1$).
Here $\delta_{r_p,z_p}(r,z)=1$ when $r_p-r=1$ and $z \in \{z_p\}$ and zero otherwise (implementing the grafting of chains by segment 1 onto the pore wall).
The starting condition for the complementary statistical weights is $G(\{r,z\},N|N)=G(r,z)$, because these ends are not restricted. 

The segment potential $\bm{u}$ acts on the 'free' segment; thus, Boltzmann statistical weights are applied:
\begin{equation}
    G(r, z) = \exp(-u(r,z))
\end{equation}

Within the FJC approach, two complementary propagators are defined, which generate the statistical weights once integrated as described above:
\begin{align}
    G(r,z,s|1) &= G(r,z),\langle G(r,z,s-1|1)\rangle, \\
    G(r,z,s|N) &= G(r,z),\langle G(r,z,s+1|N)\rangle,
\end{align}
where the angular brackets denote averaging over all allowed orientations of the next segment, in analogy to the averaging used for the site fractions.

The volume density distribution of segment $s$ at coordinates $\{r, z\}$ is found from the composition law:
\begin{equation}
    \label{eq:propagation}
    \phi(\{r, z\}, s) = \\
    \frac{2 \pi r_\text{p}^0 L_0 \sigma}{G(N,1)} \frac{G(\{r, z\}, s | 1) G(\{r, z\}, s | N)}{G(r, z)},
\end{equation}
where $\sigma$ is the grafting density.

Finally, the volume concentration at coordinates $r, z$ is found as the sum over all chain segments:
\begin{equation}
    \phi(r, z) = \sum_{s=1}^{N} \phi(\{r, z\}, s)
\end{equation}

For the solvent, $S$, we may use
\begin{equation}
    \phi_S(r,z) =\varphi_S^\text{b} \exp (-u_S(r,z))
\end{equation}
where $\varphi_S^\text{b} = 1$ represents the plain bulk solution far from the polymer brush. 

\bigskip\noindent
\textbf{Numerical algorithm.}
The numerical algorithm solves the Scheutjens-Fleer system of nonlinear equations such that the segment potentials $\bm{u}$ are consistent with the volume concentrations $\boldsymbol{\phi}$.
The relationship between the segment potentials and the volume concentrations is defined in Eq.~\eqref{eq:u-phi}.
Hence, the SF-SCF scheme can be summarized as:
\begin{equation}
    \boldsymbol{u}[\boldsymbol{\phi}] \xleftrightarrow[]{\sum_{X} \phi_X = 1} \boldsymbol{\phi}[\boldsymbol{u}]
\end{equation}

The equations were solved by a Newton/quasi-Newton optimization routine which effectively leads to the minimization of the functional Eq.~\eqref{eq:fe_lagrangian}.
During a numerical iteration step the specified guess for the segment potentials $\bm{u}$ leads to an update of the volume concentrations $\bm{\phi}$.
It is checked whether these new volume fractions obey to the incompressibility limit, and the values of $\alpha(r, z)$ is adjusted if they do not.
The segment potentials are then re-evaluated, and a new guess generated when the input and output potentials differ.
The evaluation loop is repeated until the desired accuracy is reached, coresponding to a difference between input and output potentials and errors with respect to the incompressibility constraint of less than 7 significant digits.
From the resulting SCF solution, the excess free energy is then evaluated.

For the calculations, we used the package \emph{SFbox} developed at Wageningen University, and we refer to the literature for further computational details \cite{Evers1990b}.


\pagebreak
%%%%%%%%%%%%%%%%%%%%%%%%%%%%%%%%%%%%%%%%%%%%%%%%%%%%%%%%%%%%%%%%%%%%%%%%%%
\section{Extracting surface and volume contributions to the colloid insertion free energy from SF-SCF data}

The position-dependent insertion free energy can be calculated numerically using the two-gradient SF-SCF method, $\Delta F_{\text{SF-SCF}}(z_{\text{c}})$, with the restriction that the cylindrical colloid must be placed coaxially along the $z$-axis, with its center located at $z_{\text{c}}$ (see Figure~\ref{fig:sf-scf_scheme}).

In contrast, an analytical approach allows one to compute the insertion free energy for arbitrarily positioned colloids, as in Eq.~(15).
In this approach, the osmotic contribution can be obtained from the polymer volume fraction using the mean-field Flory expression (Eq.~(16)).
The interfacial term includes both entropic and enthalpic effects: 
on one hand, the presence of the colloid surface restricts polymer conformations, creating an entropic penalty; 
on the other hand, polymer-colloid contacts produce an enthalpic effect governed by the interaction parameter $\chi_{\text{ads}}$.
At $\chi_{\text{ads}} = \chi_{\text{ads}}^{\text{crit}}$, these entropic and enthalpic contributions cancel, so that the interfacial term vanishes.
Additional effects arise because the colloid perturbs the local polymer distribution, producing zones enriched or depleted in polymer concentration around its surface.
To capture these local perturbations we introduce phenomenological parameters $b_0$ and $b_1$ to the analytical model, leading to the expression in Eq.~(17).

The SF-SCF method directly provides estimates of the insertion free energy $\Delta F_{\text{SF-SCF}}$ without assumptions about the underlying contributions and without the need for phenomenological parameters.

A natural point of comparison between the two approaches is a cylindrical colloid aligned coaxially along the $z$-axis.
In this case, insertion free energies can be obtained by both SF-SCF and the analytical model, enabling direct validation.

We therefore use this configuration as a reference to determine $b_0$ and $b_1$ by SF-SCF results.
Specifically, we fit $b_0$ and $b_1$ by minimizing 
\begin{equation}
    \sum \bigl[\Delta F_{\text{SF-SCF}} - \Delta F_{\text{cyl}}(b_0,b_1)\bigr]^2,
    \label{eq:b0_b1_fit}
\end{equation}
where $\Delta F_{\text{cyl}}(b_0,b_1)$ is the analytical result for the coaxial cylindrical colloid case.

Below, we discuss both methods in detail and present the fitted values of $b_0$ and $b_1$.


\bigskip\noindent
\textbf{Analytical estimation of $\Delta F_{\text{cyl}}^{\text{osm}}$ and $\Delta F_{\text{cyl}}^{\text{sur}}$ for cylindrical colloids on a discrete lattice}
Following Eq.~(15) the insertion free energy a coaxial cylindrical colloid:
\begin{equation}
    \Delta F_{\text{cyl}}(b_0, b_1) = \Delta F_{\text{cyl}}^{\text{osm}} + \Delta F_{\text{cyl}}^{\text{sur}}(b_0, b_1).
\end{equation}
where $\Delta F_{\text{cyl}}(b_0, b_1)$ is essentially parametrized by $b_0$ and $b_1$ (see Eqs~(15,~17)).

In their continuous form, the contributions are:
\begin{equation}
    \Delta F_{\text{cyl}}^{\text{osm}}(z_{\text{c}}) = 2 \pi \int_{z_{\text{c}} - d/2}^{z_{\text{c}} + d/2} \int_{0}^{d/2} \Pi(r,z) \, r \, dr \, dz
\end{equation}
and
\begin{equation}\label{eq:continuous_surf_int}
    \begin{aligned}
        \Delta F_{\text{cyl}}^{\text{sur}}(z_{\text{c}}) = 2 \pi d \int_{z_{\text{c}} - d/2}^{z_{\text{c}} + d/2} \gamma(d/2,z) \, dz +\\
        + \pi \int_{0}^{d/2} \left[ \gamma(z_{\text{c}} - d/2, r) + \gamma(z_{\text{c}} + d/2,r) \right] dr,
    \end{aligned}
\end{equation}
where the first term in Eq.~\eqref{eq:continuous_surf_int} integrates over the lateral surface, and the second term integrates over the top and bottom faces, of the cylinder.

Following the lattice discretization of SF-SCF outputs with discretization steps $\delta r = \delta z = 1$, we use an indexing such that $0 \le i \le d/2-1$ iterates in the direction of the $r$-axis, and $0 \le k \le d-1$ iterates in the direction of the $z$-axis.
This corresponds to the physical coordinates $r,z \in [0, d/2]\times[z_{\text{c}} - d/2, z_{\text{c}} + d/2]$ (as illustrated in Figure~\ref{fig:sf-scf_scheme}) with $d$ being an even integer to match the lattice.

We define the volume projection matrix $\bm{V}_{\text{cyl}}[d/2 \times d]$ for a cylindrical colloid of size $d$, such that each element of the matrix equals the volume of the colloid contained within the corresponding lattice element:
\begin{equation}
    V_{\text{cyl}}[i, k] = \pi(2i + 1)
\end{equation}
Obviously, the sum of all matrix elements equals the volume of the cylinder.
\begin{equation*}
    \sum_{i=0}^{d/2 - 1} \sum_{k=0}^{d - 1} V_{\text{cyl}}[i, k] = \frac{\pi d^3}{4}
\end{equation*}

Analogously, we define the colloid surface projection matrix $\bm{S}_{\text{cyl}}[d/2 \times d]$, such that each element of the matrix equals the surface area of the colloid within the corresponding lattice element:
\begin{align}
    \begin{split}
        S_{\text{cyl}}[i, k] = 
        &\begin{cases}
            2 \pi i,   & \text{if } i = d/2 - 1 \\
            0,         & \text{otherwise}
        \end{cases} +
        \\
        &+
        \begin{cases}
            2 \pi (i + 1), & \text{if } k = 0 \text{ or } k = d - 1 \\
            0,             & \text{otherwise}
        \end{cases}
    \end{split}
\end{align}
Here, the first term accounts for the top and bottom faces, and the second term accounts for the lateral surface elements, of the cylinder.
Again, the sum of all matrix elements equals the surface area of the cylinder:
\begin{equation*}
    \sum_{i=0}^{d/2 - 1} \sum_{k=0}^{d - 1} S_{\text{cyl}}[i, k] = \frac{3 \pi d^2}{2}
\end{equation*}

Figure~\ref{fig:cylindrical_kernel_SI} provides a color-coded map of the volume $\bm{V}_{\text{cyl}}$ and surface $\bm{S}_{\text{cyl}}$ projection matrices for a selected colloid size ($d = 16$).

\begin{figure}[h]
    \centering
    \includegraphics[width = \linewidth]{fig/cylindrical_kernel_SI.png}
    \caption{
    Volume (left) and surface (right) projection matrices for a cylindrical colloid with diameter and height $d = 16$. The element values of the matrices are color-coded, with violet representing zero and yellow the highest values.
    }
    \label{fig:cylindrical_kernel_SI}
\end{figure}

To calculate the colloid insertion free energy, we integrated the osmotic pressure over the colloid volume and the surface tension over the colloid surface (Eq.~(15)).
The equivalent operation on a discrete lattice is the matrix dot product.
The two contributions to the insertion free energy are thus calculated as:
\begin{equation}
    \label{eq:cyl_fe_osm}
    \begin{split}
        \Delta F_{\text{cyl}}^{\text{osm}}(z_{\text{c}}) = \bm{V}_{\text{cyl}} \cdot \boldsymbol{\Pi}\{z_{\text{c}}\} \text{ and} \\
        \Delta F_{\text{cyl}}^{\text{sur}}(z_{\text{c}}) = \bm{S}_{\text{cyl}} \cdot \boldsymbol{\gamma}\{z_{\text{c}}\},
    \end{split}
\end{equation}
where the matrix elements for $\boldsymbol{\Pi}\{z_{\text{c}}\}$ and $\boldsymbol{\gamma}\{z_{\text{c}}\}$ run across $0 \leq i < d/2$ and $z_{\text{c}} - d/2 \leq k < z_{\text{c}} + d/2$.

% The discretized insertion free energy profile $\Delta F_{\text{cyl}}$ is obtained from a series of insertion free energy calculations across all possible colloid center positions $z_{\text{c}}$.
% Such a series of sequential integrations (matrix dot products) is equivalent to the convolution with the colloid volume/surface projection matrix acting as a kernel:
% \begin{equation*}
%     \Delta F_{\text{cyl}}^{\text{osm}} = \boldsymbol{\Pi} \ast \bm{V}_{\text{cyl}} \text{ and }
%     \Delta F_{\text{cyl}}^{\text{sur}} = \boldsymbol{\gamma} \ast \bm{S}_{\text{cyl}}.
% \end{equation*}
% Convolution was performed via fast Fourier transform for computational efficiency.

\bigskip\noindent
\textbf{Extracting insertion free energy from SF-SCF data}
As illustrated in Figure~\ref{fig:sf-scf_scheme}, a cylindrical colloid is positioned coaxially along the $z$-axis as impermeable lattice elements.
Consequently, the polymer-brush chains adjust to the available space and the polymer-colloid interaction strength, producing a change in the system's total free energy, $F_{\text{cyl}}(z_{\text{c}})$.

We performed a series of SF-SCF calculations to obtain free energy profiles $F_{\text{SF-SCF}}(z_{\text{c}})$ for a range of colloid diameters $d$, polymer-solvent interaction parameters $\chi_{\text{PS}}$, and polymer-colloid interaction parameters $\chi_{\text{PC}}$, using the pore geometry and brush parameters defined in Figure~1.  
To isolate the insertion free energy, we apply a ground-state correction and define
\begin{equation}
    \Delta F_{\text{SF-SCF}}(z_{\text{c}}) 
    = F_{\text{SF-SCF}}(z_{\text{c}}) - F_{\text{SF-SCF}}(z_{\text{c}}^{\text{bulk}}),
\end{equation}
where $F_{\text{SF-SCF}}(z_{\text{c}}^{\text{bulk}})$ is the system free energy when the colloid is placed far away from the pore.%, and $z_{\text{c}}$ denotes the colloid center coordinate along the $z$-axis.

\bigskip\noindent
\textbf{Fitting $b_0$ and $b_1$ to match analytical model to SF-SCF data}
The optimal coefficients $b_0$ and $b_1$ were found using the least-squares method, i.e., minimizing Eq.~(\ref{eq:b0_b1_fit}) across a range of $\chi_{\text{PS}} \in [0,1]$  and $\chi_{\text{PC}} \in [-2,0]$ for the relevant colloid positions $0 \leq |z_{\text{c}}| \leq 60$.
Fits were performed with small colloids ($d=4$), to focus on local effects and avoid added effects that may arise due to global perturbations of the polymer distribution in the pore.

\begin{figure}[]
    \centering
    \includegraphics[width = 0.9\linewidth]{fig/fit_SI.png}
    \caption{
    Comparison of $\Delta F_{\text{SF-SCF}}$ profiles (circles) with $\Delta F_{\text{cyl}}(b_0,b_1)$, for the best-fit values of $b_0 = 0.7$ and $b_1 = -0.3$ (optimally accounting for local perturbations in the polymer concentration; thick solid lines) and for $b_0 = 1.0$ and $b_1 = 0.0$ (neglecting any local perturbations in polymer concentration; thin dashed lines).
    Pore and brush parameters are as given in Figure 1; $d = 4$, $\chi_{\text{PS}} = 0.5$, and $\chi_{\text{PC}}$ values are color-coded and indicated on the right side of the figure.
    The light green hatched area marks values of $z_{\text{c}}$ that are located inside the pore lumen ($|z| \leq 26$).
    }
    \label{fig:fit_SI}
\end{figure}

In a $\theta$-solvent (chosen here as a representative case) and for the selected pore geometry and brush parameters (Figure 1), we selected three $\chi_{\text{PC}}$ values (0.00, -0.75 and -1.50) to span the regimes of net repulsion, near-critical adsorption, and net attraction, respectively.
Figure~\ref{fig:fit_SI} demonstrates that a satisfactory fit (thick solid line) to the $\Delta F_{\text{SF-SCF}}$ profiles (circles) could be obtained across all three representative datasets with a single parameter set, $b_0 = 0.7$ and $b_1 = -0.3$.
The insertion free energies are consistently close to zero for $\chi_{\text{PC}} = -0.75$, illustrating that this value is indeed near the critical condition, $\chi_{\text{PC}}^{\text{crit}} = \chi_{\text{crit}} + \chi_{\text{PS}} (1 - \phi)$, where polymer attraction and osmotic repulsion just cancel each other.
Consequently, the least-square fits considered the regimes of net repulsion and net attraction approximately equally through $\chi_{\text{PC}} = 0.00$ and $\chi_{\text{PC}} = -1.50$, respectively.
In contrast, neglect of local perturbations to the polymer concentration ($b_0 = 1.0$ and $b_1 = 0.0$) reproduces $\Delta F_{\text{SF-SCF}}$ rather poorly for repulsive and attractive colloids (Figure~\ref{fig:fit_SI}, thin dashed line), thus demonstrating the importance of the correction.

Although the fit was performed only for small colloids, the resulting parameters $b_{0}$ and $b_{1}$ still successfully account for the local perturbations to the polymer concentration when calculating the insertion free energy of larger colloids.
This is illustrated in Figure~\ref{fig:fe_scf_grid} for two selected $\chi_{\text{PC}}$ values (-0.5 and -1.0) and three selected $\chi_{\text{PS}}$ values (0.4, 0.5 and 0.6).

Deviations become notable, however, for $d = 16$ under conditions of strong repulsion or attraction.
This size regime thus marks the limit of validity of the local perturbation approximation.
Instead, attractive or repulsive interactions entail non-local changes that impact the polymer concentration across the entire pore cross-section.
As a consequence, the pore walls also influence colloid insertion. 

\begin{figure}[]
    \centering
    \includegraphics[width = 0.95\linewidth]{fig/fe_scf_grid2.png}
    \caption{ 
    Comparison of $\Delta F_{\text{SF-SCF}}$ (square symbols) with $\Delta F_{\text{cyl}}(b_0,b_1)$ profiles for the best-fit values $b_0 = 0.7$ and $b_1 = -0.3$ (solid lines), for $d = [8, 12, 16]$ (color coded).
    Pore and brush parameters are as given in Figure 1; $\chi_{\text{PC}} = -1.0$ (top row) and -0.5 (bottom row), and the solvent quality was varied near the $\theta$-point with $\chi_{\text{PS}} = [0.4, 0.5, 0.6]$, as indicated.
    The light green hatched area marks values of $z_{\text{c}}$ that are located inside the pore lumen ($|z| \leq 26$).
    \label{fig:fe_scf_grid}
    }
\end{figure}

% \todo{RR: Is the following paragraph needed? If we don't refer to this in the main text, then we better remove it. We can add this point back later, in case the reviewers will query it.}
% Another interesting effect can also observed for strongly attractive colloids, even when small.
% The SF-SCF results predict that, to reach the minimum of the insertion free energy $\Delta F_{\text{SF-SCF}}$, the polymer brush changes conformation to reach the colloid at a greater distance $|z_{\text{c}}|$ from the pore.
% In Figure~\ref{fig:fit_SI}, this effect is apparent for $\chi_{\text{PC}} = -1.5$, where the SF-SCF results (blue circles) predict systematically lower insertion free energies than the analytical approach (blue solid line) outside the pore ($|z_{\text{c}}| > 26$).
% This effect is mild and only slightly increases the region with negative insertion free energy.
% It may entail a slight reduction in the total resistance to diffusive transport: for pores with regions of negative insertion free energy, the resistance of the bulk solution limits the total resistance, and the capture of colloids by the polymers at a larger distance from the pore reduces the bulk resistivity.


\pagebreak
%%%%%%%%%%%%%%%%%%%%%%%%%%%%%%%%%%%%%%%%%%%%%%%%%%%%%%%%%%%%%%%%%%%%%%%%%%
\section{Computing insertion free energies for arbitrarily placed spherical colloids}

The SF-SCF method considered in the previous sections is limited to colloids moving along the main axis of the pore.
In reality, colloids may be located off the pore axis.
Here, we generalize the analytical approach (Supplementary Note 4) to calculate insertion free energies based on volume and surface contributions (assuming localized perturbations to the polymer concentration) to arbitrarily placed colloids.
In doing so, we also change the shape of the colloid, from a cylinder to a simpler sphere.

\bigskip\noindent
\textbf{Angular integration of colloid volume and surface onto the $rz$-plane in continuous space.}
As the polymer and pore geometrical features are uniform in the angular direction, we use $r, z$ cylindrical coordinates with a degenerate angular axis.
Any property such as the polymer volume fraction can hence be expressed as a function $f(r,z,\theta) = f(r,z)$.

The distance to the center of spherical body of a colloid is 
\begin{eqnarray}
\Delta_{\text{c}} = \sqrt{r^2 + r_{\text{c}}^2 - 2 r r_{\text{c}} \cos(\theta) + (z - z_{\text{c}})^2},
\end{eqnarray} 
where $r_{\text{c}}, z_{\text{c}}, \theta_{\text{c}}$ are the position of the center in cylindrical coordinates.
Without the loss of generality, we set $\theta_{\text{c}} = 0$.

The function $f(r,z)$ is integrated over the spherical volume as:
\begin{eqnarray}
    \label{eq:int_indicator_V}
    \begin{aligned}
        \int\limits_{V} f dV
        =\int\displaylimits_{0}^{+\infty} \int\displaylimits_{-\infty}^{+\infty} f(r, z) \int\displaylimits_{0}^{2\pi}  H(\Delta_{\text{c}} - d/2) r \text{d}r \text{d}z \text{d}\theta \\
        =\int\displaylimits_{0}^{+\infty} \int\displaylimits_{-\infty}^{+\infty} f(r, z)  V_{\theta \downarrow}(r,z) \text{d}r \text{d}z
    \end{aligned}
\end{eqnarray}
where $r \text{d}r \text{d}z \text{d}\theta$ is the differential volume element in cylindrical coordinates, $H(\Delta_{\text{c}} - d/2)$ is the Heaviside function, that evaluates to 1 inside the sphere.
$V_{\theta \downarrow}(r,z)$ is a projection of sphere volume on the $rz$-plane, with $\theta$ being the projecting direction.

Similarly, to compute surface integrals over the spherical colloid, the Dirac delta function $\delta(\Delta_{\text{c}} - d/2)$ is applied to restrict the integration domain to the spherical surface.
This allows to express the surface integral of a scalar function $f(r,z)$ as:
\begin{eqnarray}
    \label{eq:int_indicator_S}
    \begin{aligned}
        \int\limits_{S} f dS = \\
        = \int\displaylimits_{0}^{+\infty} \int\displaylimits_{-\infty}^{+\infty} \int\displaylimits_{0}^{2\pi} f(r, z) \delta(\Delta_{\text{c}} - d/2)  \frac{\Delta_{\text{c}}}{r_{\text{c}} |\sin\theta|} r \, \text{d}r \text{d}z \text{d}\theta=\\
        =\int\displaylimits_{0}^{+\infty} \int\displaylimits_{-\infty}^{+\infty} f(r, z)  S_{\theta \downarrow} \, \text{d}r \text{d}z
    \end{aligned}
\end{eqnarray}
where the factor $\frac{\Delta_{\text{c}}}{r_{\text{c}} |\sin\theta|}r \, dr dz$ corresponds to the surface area element expressed in cylindrical coordinates.

From Eqs.~(\ref{eq:int_indicator_V}, \ref{eq:int_indicator_S}) the volume $V_{\theta \downarrow}$ and surface $S_{\theta \downarrow}$ projections of a spherical body onto the $rz$-plane in cylindrical coordinates are:
\begin{gather}
    V_{\theta \downarrow}(r, z, r_{\text{c}}, z_{\text{c}}) = 2\int_{0}^{\pi} H\!\left( \Delta_{\text{center}} - {d}/{2} \right) r \, \text{d}\theta
    \\
    S_{\theta \downarrow}(r, z, r_{\text{c}}, z_{\text{c}}) = 2\int_{0}^{\pi}\delta(\Delta_{\text{c}} - d/2)  \frac{\Delta_{\text{c}}}{r_{\text{c}} |\sin\theta|} r \, \text{d}\theta
\end{gather}
These expressions describe the angular integration of the volume and surface projected onto the $rz$-plane, effectively reducing the 3D geometry to a two-gradient description suitable for cylindrical symmetry.

\begin{figure}[]
    \centering
    \includegraphics[width=0.95\linewidth]{fig/sphere_volume_and_surface_projection.png}
    \caption{
        \textbf{Left:}
        Surface $S_{\theta \downarrow}\{r_{\text{c}}\}$ (top) and volume $V_{\theta \downarrow}\{r_{\text{c}}\}$ (bottom) projections on the $rz$-plane for a spherical colloid with diameter $d = 8$, for a set of sphere center offsets $r_{\text{c}} = \{0, 2, 4, 6\}$ to the $z$-axis (shown side by side).
        The sphere centers are indicated with a red cross.
        The $r$-axis is shown with a blue arrow; the $z$-axis is omitted, as the center coordinates $z_{\text{c}}$ are arbitrary.
        \textbf{Right:}
        Discretization results of the surface and volume projection matrices $\bm{S}\{r_{\text{c}}\}$ and $\bm{V}\{r_{\text{c}}\}$ as heatmaps, for $r_{\text{c}} = 6$.
        Color codes are distinct for dsata on the left and right, as indicated.
        Intensities for surface and volume are normalized by $a^2$ and $a^3$, respectively.
    }
    \label{fig:sphere_volume_and_surface_projection}
\end{figure}

\bigskip\noindent
\textbf{Application to a discretized lattice.}
To find the elements of the projection matrices (each representing the volume or surface area of the spherical colloid intersecting a given lattice cell), the angularly projected volume and surface area are dicretized over finite lattice elements.
For each grid element indexed by $i, k$, corresponding to the domain $r \in [i, i + \delta r]$ and $z \in [k, k + \delta z]$, the matrix entries are defined as
\begin{eqnarray}
    V\{r_{\text{c}}\}{[i, k]} = \! \iint \limits_{i, k}^{\quad \substack{i+\delta r\\ k+\delta z}} \! V_{\theta \downarrow} (r, z, r_{\text{c}}, z_{\text{c}})\, \text{d}r \text{d}z
    \\
    S\{r_{\text{c}}\}{[i, k]} = \! \iint \limits_{i, k}^{\quad \substack{i+\delta r\\ k+\delta z}} \! S_{\theta \downarrow} (r, z, r_{\text{c}}, z_{\text{c}})\, \text{d}r \text{d}z
\end{eqnarray}
where $z_{\text{c}}$ has an arbitrary value.
The size of the projection matrices is $\min(d, r_{\text{c}} + d/2) \times d$.
In contrast to Cartesian projections, cylindrical projections depend on the radial coordinate of the center $r_{\text{c}}$.

Naturally, the sum of all matrix elements equals the surface and the volume of the sphere
\begin{eqnarray*}
    \mathop{\sum\sum}_{\mathclap{\substack{i \in [0, \min(r_{\text{c}}+d/2,d)-1] \\ k \in [0, d-1]}}}  V\{r_{\text{c}}\}{[i, k]} = \pi d^2,
    \\
    \mathop{\sum\sum}_{\mathclap{\substack{i \in [0, \min(r_{\text{c}}+d/2,d)-1] \\ k \in [0, d-1]}}}  S\{r_{\text{c}}\}{[i, k]} = \frac{\pi d^3}{6} .
\end{eqnarray*}

The discretized form of Eq.~(15) to calculate the osmotic term in the insertion free energy for a spherical colloid is:
\begin{eqnarray}
    \begin{aligned}
        \Delta F_{\text{osm}}(r_{\text{c}}, z_{\text{c}}) =\\
        = \mathop{\sum\sum}_{\mathclap{\substack{i \in [0, \min(r_{\text{c}}+d/2,d)-1] \\ k \in [0, d-1]}}} V{r_{\text{c}}}_{[i, k]} \cdot \Pi_{[\max(r_{\text{c}}-d/2,0)+i, z_{\text{c}}-d/2+k]} =\\[-15pt]
        = \bm{V}\{r_{\text{c}}\} \cdot \bm{\Pi}\{r_{\text{c}}, z_{\text{c}}\} \\[5pt]
        \text{where } \bm{\Pi}\{r_{\text{c}},z_{\text{c}}\} =\left(\bm{\Pi}_{i,k}\right) {\substack{\max(r_{\text{c}}d/2,0) \le i < r_{\text{c}}+d/2 \\ z_{\text{c}}-d/2 \le k < z_{\text{c}}+d/2}}
    \end{aligned}
\end{eqnarray}
Similarly, the surface term in the insertion free energy is the following matrix multiplication:
\begin{eqnarray}
    \begin{aligned}
        \Delta F_{\text{sur}}(r_{\text{c}}, z_{\text{c}}) = \bm{S}\{r_{\text{c}}\} \cdot \bm{\gamma}\{r_{\text{c}}, z_{\text{c}}\} \\[5pt]
        \text{where } \bm{\gamma}\{r_{\text{c}},z_{\text{c}}\} =\left(\bm{\gamma}_{i,k}\right) {\substack{\max(r_{\text{c}}-d/2,0) \le i < r_{\text{c}}+d/2 \\ z_{\text{c}}-d/2 \le k < z_{\text{c}}+d/2}}
    \end{aligned}
\end{eqnarray}

The function domain, values and discretization for $V_{\theta \downarrow}$ and $S_{\theta \downarrow}$ are exemplified in Figure~\ref{fig:sphere_volume_and_surface_projection} for a set of colloids with varying radial center positions $r_{\text{c}}$.

The method was used to compute the insertion free energy $\Delta F(r,z)$ from inherently discrete SF-SCF outputs $\Pi, \gamma$, preserving the cylindrical lattice.
Volume and surface projection matrices are explained geometrically in Figure~\ref{fig:spherical_kernel}.

\begin{figure}[H]
    \centering
    \includegraphics[width=0.9\linewidth]{fig/spherical_kernel.png}
    \caption{
        Illustration of a spherical colloid's volume and surface projection matrices in a cylindrical lattice.
        Showm on the left is the volume projection matrix $\bm{V}\{r_c\}$ for a spherical colloid with diameter $d = 12$ and $r_{\text{c}}= 8$.
        The colored tiles encode the matrix elements' values, where violet means zero and yellow represents the largest values.
        The geometrical meaning of the matrix element is the colloid volume (red body) or surface (green small patch) found in the domain $r,z \in [i, i + \delta r] \times [k, k + \delta z]$ (opaque blue toroid).
        The smaller drawing on the right complements the main drawing with a $z$-view for clarity, using consistent color-coding for volume and surface elements, with the pale blue annulus being the domain of lattice element.
        The yellow circle indicates the colloid cross-section within the selected $z$-slice, while the larger pale yellow circle shows the rest of the colloid body lying behind the plane of the cross-section.
    }
    \label{fig:spherical_kernel}
\end{figure}


\pagebreak
%%%%%%%%
\section{Validating analytical approaches with numerical simulations}
%%%%%%%%

Applying Eqs.~(2, 3) and the boundary conditions reduces the Smoluchowski equation (Eq.~(1)) in the stationary state to the generalized Laplace equation:
\begin{equation}
  \mathcal L\tilde c=0,
  \label{eq:laplace}
\end{equation}
with the Smoluchowski diffusion operator
\begin{equation*}
  \mathcal L=\nabla\!\cdot\!\bigl(\tilde D(\bm r)\nabla\bigr).
\end{equation*}

\bigskip\noindent
\textbf{Defining the Laplace equation in a finite, discretized domain.}
To obtain a stationary solution of Eq.~\eqref{eq:laplace}, we consider a finite cylindrical domain large enough to exclude edge effects:
\begin{equation*}
  (r,z) \in \Omega = [0,r_{\max}]\times[z_{\min},0],
\end{equation*}
and discretize it on a regular cylindrical lattice with spacings $\delta r=\delta z$. 
Due to symmetry about the mid-plane $z = 0$, we restrict the domain accordingly with suitable boundary conditions.

The computational grid contains $N_r=r_{\max}/\delta r$ radial nodes and $N_z=(z_{\max}-z_{\min})/\delta z$ axial nodes, indexed by $i=0,\dots,N_r-1$ and $k=0,\dots,N_z-1$, respectively.
Physical coordinates are $r_i=i\delta r$ and $z_k=z_{\min}+k\delta z$ as shown in Figure~\ref{fig:stencil}a.
Each node defines a finite volume element with four faces, labeled $z\pm$ and $r\pm$, as illustrated in Figure~\ref{fig:stencil}c.
The finite volume element is indexed using its lower-left node ($r-, z-$).

Applying the divergence theorem to each finite volume element yields a discrete approximation to the Laplacian in the form of a finite volume stencil:
\begin{eqnarray}
    \begin{aligned} 
        \nabla_{i,k} \psi = 
        \\
        \lambda^{z} D^{z+}_{i,k} (\psi_{i,k} - \psi_{i,k+1}) +  \lambda^{z} D^{z-}_{i,k} (\psi_{i,k} - \psi_{i,k-1})\\
        \lambda^{r+}_{i} D^{r+}_{i,k} (\psi_{i,k} - \psi_{i+1,k}) +  \lambda^{r-}_{i} D^{r-}_{i,k} (\psi_{i,k} - \psi_{i-1,k}),
    \end{aligned}
    \label{eq:FV_divergence}
\end{eqnarray}
where, $D^{r\pm}_{i,k}$ and $D^{z\pm}_{i,k}$ denote effective diffusion coefficient evaluated at finite volume element faces, as a harmonic averaging of values at adjacent finite volume elements. 
The finite volume expansion in the radial direction (see Figure~\ref{fig:stencil}b) and discretization steps are accounted for through coordinate-dependent prefactors:
\begin{eqnarray}
    \lambda^{z} = \frac{1}{\delta z^2}\\
    \lambda^{r+}_{i} = \frac{2 r_i + 2 \delta r}{2 r_i + 1} \frac {1}{\delta r^2}\\
    \lambda^{r-}_{i} = \frac{2 r_i}{2 r_i + 1} \frac {1}{\delta r^2}
\end{eqnarray}

\begin{figure}[]
    \includegraphics[width = \linewidth]{fig/stencil.png}
    \caption{
        \textbf{(a)} Discretized domain for numerical simulations.Following the analytical solution for the bare pore, the source nodes are placed in the region circumferenced by oblate hemispheroids (orange).
        The effective shape of the impermeable membrane wall is constructed from the real membrane wall (dark green) and excluded volume due to the finite colloid size (light green, see Figure~\ref{fig:excluded_volume}).
        Exploiting system symmetry, we modelled only half of the system, by placing Dirichlet boundary condition on the mid-plane  of the full system.
        The Smoluchowski operator is defined with a 5-point stencil on the finite volume elements (red element with grey neighbors).
        The dots on the stencil denote the indexing nodes.
        \textbf{(b)} The finite volume elements naturally expand with the radial coordinate, with the size defined by the discretization steps $\delta r$ and $\delta z$.
        \textbf{(c)} Each volume element shares common faces with the neighboring elements, labeled with the coordinate direction along the radial $r \pm$ and the axial $z\pm$ axes.
  }
  \label{fig:stencil}
\end{figure}

The discretized system is assembled into matrix form:
\begin{equation}
  \mathbf L\, \bm{\psi}=\bm b,
  \label{eq:matrix_form}
\end{equation}
where $\mathbf{L}$ applies the finite volume stencil from Eq.~\eqref{eq:FV_divergence} and $\bm b$ enforces boundary conditions.

The two-dimensional grid of volume elements are flattened into vectors of length $N_rN_z$ reindexed with $m=iN_z+k$.
Under this mapping, the continuous operator $\mathcal L$ becomes a sparse $[N_rN_z\times N_rN_z]$ matrix $\mathbf L$, and the unknown values of $\psi(r_i, z_k)$ form the column $[N_rN_z]$ vector $\bm{\psi}$.

Once Eq.~\eqref{eq:matrix_form} is solved for $\bm{\psi}$, the resulting vector is reshaped into the $rz$-grid, giving $\psi(r,z)$ and recovering the colloid concentration $c(r,z)$.
This enabled examination of the non-equilibrium partitioning of colloids in the presence of position-dependent insertion free energies for given boundary conditions.

\bigskip\noindent
\textbf{Defining the impermeable wall (no-flux elements).}
The impermeable regions are inherited from the SF-SCF calculations and define the finite volume elements inaccessible to diffusing colloids.
We define the original impermeable wall grid elements as:
\begin{eqnarray}
    \textbf{Wall}_{0} [i,k] = 
    \begin{cases}
        1 \quad \text{if } r_i \ge r_{\text{p}}^0 \text{ and } z_k \le -L_0/2 \\
        0 \quad \text{otherwise}
    \end{cases}
\end{eqnarray}

Due to excluded volume of the colloid, the effective pore shape is different.
The space impermeable to the centre of a spherical colloid of diameter $d$ defines an effective pore with radius $r_{\text{p}} = r_{\text{p}}^{0} - \frac{d}{2}$, length $L = L_{0} + d$, and rounded corners, as shown in Figure~\ref{fig:excluded_volume}a.
In discretized space, the dilated impermeable region is computed via the morphological dilation operation:
\begin{eqnarray}
    \textbf{Wall} = \textbf{Wall}_{0} \oplus \textbf{Colloid}
\end{eqnarray}
with the structuring element $\textbf{Colloid}$ being a coarse-grained circle in $rz$-coordinate, see Figure~\ref{fig:excluded_volume}b.
\begin{equation}
    \begin{gathered}
        \textbf{Colloid}[i,k] = \\
        \begin{cases}
                1, & \text{if } \left( \dfrac{d}{2} - r_i\right)^2 + \left( \dfrac{d}{2} - z_k\right)^2 \le \dfrac{d^2}{4} \\[5pt]
                0, & \text{otherwise}
            \end{cases}
    \end{gathered}
\end{equation}

\begin{figure}[]
    \centering
    \includegraphics[width=\linewidth]{fig/excluded_volume.png}
    \caption{
        \textbf{(a)} Effective pore size in real space. Perspective view of the real (green) and effective (red) pore shapes.
        In the cut through, the surface of the effective pore is traced with a dashed red line, and a colloid of size $d$ is shown in orange.
        The effective dilation of the pore is due to the finite size of the colloid.
        \textbf{(b)} Effective pore size in discretized space. $rz$ plane view of the real (dark green) and effective (light green, delimited with a red dashed line) membrane wall shapes.
        The coarse-grained colloid is shown in orange.
        }
    \label{fig:excluded_volume}
\end{figure}

\bigskip\noindent
\textbf{Defining the colloid source.}
We place source nodes at a finite axial distance, chosen to align with an oblate hemispheroidal surface (or a half ellipse in the $rz$-plane) intersecting the $z$-axis at $z_{\text{min}} + \delta z$.
The half ellipse has foci located at $(\pm r_{\text{p}}, -L/2)$, such that the pore rim acts as the focal circle.
This defines the major semi-axis $r_{\text{max}} - \delta r$ and hence the radial extent of the discrete computational domain:
\begin{equation}
    r_{\text{max}} - \delta r = \sqrt{(|z_{\text{min}}| - L/2 + \delta z)^2 + r_{\text{p}}^2}
\end{equation}
The binary mask $\textbf{Source}[i,j]$ identifies the grid nodes at the boundary of the domain that act as the colloid source (see Figure~\ref{fig:stencil}a).
These nodes lie outside the oblate ellipsoid defined by the above condition.
The set is constructed as:
\begin{eqnarray}
    \begin{gathered}
        \textbf{Source}[i,j] = \\
        \begin{cases}
        1, & 
        \begin{aligned}
            \text{if } &\left\lVert (r_i, z_k) - (r_{\text{p}}, -L/2) \right\rVert \\
            +  &\left\lVert (r_i, z_k) - (-r_{\text{p}}, -L/2) \right\rVert \ge 2(r_{\text{max}} - \delta r)
        \end{aligned} \\
        0, & \text{otherwise}
        \end{cases}
    \end{gathered}
\end{eqnarray}

This boundary condition ensures that the solution approximates that of an idealized system with a source located infinitely far away ($z \to -\infty$).
We effectively reproduce the shape of iso-concentration lines in the exterior region of a bare pore, thereby maintaining a realistic flux geometry.

\bigskip\noindent
\textbf{Defining the colloid sink}
Exploiting the system's symmetry, we impose a Dirichlet boundary condition $\psi = \psi_{\text{sink}}$ at the $z+$ face of all finite volume elements with $k = N_z - 1$.
This corresponds to placing the absorbing boundary condition at the mid-plane $z = 0$.
Due to symmetry, there is no radial component of the flux at $z = 0$, so no flux crosses the $r-$ or $r+$ faces of these finite volume elements.

We define the set of finite volume elements that act as sinks as:
\begin{equation}
    \textbf{Sink}[i,k] =
    \begin{cases}
    1, & \text{if } k = N_z - 1, \\
    0, & \text{otherwise}.
    \end{cases}
\end{equation}
For these elements, the divergence term simplifies to account for the fixed potential at the $z+$ face and no flux at the radial faces:
\begin{equation}
    \nabla_{i,k_0} \psi =
    \lambda^{z} D_{i,k_0} (\psi_{i,k_0} - \psi_{\text{sink}})
    + 2 \lambda^{z} D_{i,k_0}^{z-} (\psi_{i,k_0} - \psi_{i,k_0-1})
\end{equation}

\bigskip\noindent
\textbf{Constructing the Laplace operator matrix.}
The matrix $\mathbf{L}$ is constructed using a five-point stencil (Figure~\ref{fig:stencil}a) that adapts near boundaries.
The diagonal entries are defined as
\begin{equation}
  \bm{L}_{m,m} = 
  \begin{cases}
    0 & \text{if } m \in \textbf{Wall}, \\
    1 & \text{if } m \in \textbf{Source}, \\
    - 2 \lambda^{z} D_m & \text{if } m \in \textbf{Sink}, \\
    -\!\!\!\sum\limits_{m' \in \mathcal{N}_m} \bm{L}_{m, m'} & \text{otherwise},
  \end{cases}
  \label{eq:L_diag}
\end{equation}
where $\mathcal{N}_m$ is the set of valid neighbor indices:
\begin{equation}
  \mathcal{N}_{i,k} = \left( \{(i \pm 1, k),\; (i, k \pm 1)\} \right) \cap \Omega,
  \label{eq:neighbors}
\end{equation}
and node indices are flattened as $m = i N_z + k$ and $m' = i' N_z + k'$.

The off-diagonal entries $\bm{L}_{m,m'}$ represent finite-volume approximations of the Laplace operator (Eq.~\eqref{eq:FV_divergence}) taking account of the boundary conditions:
\begin{eqnarray}
    \bm{L}_{m,m+1} =
    \begin{cases}
        0 &  \text{if } m+1 \in \textbf{Wall} \\
        D^{z+} \lambda^{z} &  \text{otherwise}
    \end{cases}
    \\
    \bm{L}_{m,m-1} =
    \begin{cases}
        0 &  \text{if } m-1 \in \textbf{Wall} \\
        2 D^{z-} \lambda^{z} & \text{if } m \in \textbf{Sink}\\
        D^{z-} \lambda^{z} &  \text{otherwise}
    \end{cases}
    \\
    \bm{L}_{m,m \pm N_z} =
    \begin{cases}
        0 \quad \text{if } m \pm N_z \in \textbf{Wall} \text{ or } m \in \textbf{Sink} \\
        D^{r\pm} \lambda^{r\pm}  \text{otherwise}
    \end{cases}
\end{eqnarray}

The right-hand side vector $\bm{b}$ enforces the boundary condition from the source and the sink:
\begin{eqnarray}
    \bm{b}_m = 
    \begin{cases}
        \psi_{\text{source}} & \text{if } m \in \textbf{Source} \\
        \psi_{\text{sink}} & \text{if } m \in \textbf{Sink} \\
        0 & \text{otherwise}
    \end{cases}
\end{eqnarray}

The resulting sparse linear system in Eq.~\eqref{eq:matrix_form} is solved using the \texttt{scipy.sparse.linalg} package.


\bigskip\noindent
\textbf{Extracting pore resistance.}
The total colloid flux was computed by summing the fluxes through the $z+$ faces of the finite volume elements at $k = N_z - 1$, corresponding to the pore cross-section at the mid-plane $z = 0$.
Due to symmetry, the radial flux vanishes at this plane, so only the axial ($z$-direction) components contribute:
\begin{equation}
    J_{\text{num}} = 2\pi \sum_{i=0}^{N_r-1} D_{i,k'} \frac{2(\psi_{i,k'} - \psi_{\text{sink}})}{\delta z} (2i + 1) \delta r^2,
\end{equation}
where $k' = N_z - 1$ is the axial index of the mid-plane.

To determine the total pore resistance, we prescribe boundary conditions $\psi_{\text{source}} = 1.0$ and $\psi_{\text{sink}} = 0.5$.
This corresponds to a concentration difference of $\Delta c = 1.0$ across the full system, assuming symmetry about the mid-plane:
\begin{equation}
    R_{\text{num}} = \frac{\Delta c}{J_{\text{num}}} + 2 R_{\left(\left|z_{\text{min}} - L/2\right|,\ -\infty\right)}
\end{equation}
Here, $R_{(|z_{\text{min}} - L/2|, -\infty)}$ is the analytically estimated resistance of the truncated semi-infinite reservoir, as given by Eq.~\eqref{eq:r_reservoir}.


\pagebreak
%%%%%%%%
\section{Mapping our model to experimental data for biocolloid transport across NPCs}
%%%%%%%%

Figure~\ref{fig:experiments_overview} and Table~\ref{tbl:experimental} summarize the most relevant aspects of the published experimental quantifications of biocolloid transport rates across NPCs in the nuclear envelope of cells, along with partition coefficients for sticky colloids in phase-separated droplets of pure FG domain. 
The nucleus is represented as a well-mixed compartment of volume $V_\text{nucleus}$ bounded by an impermeable envelope that is perforated by $N_\text{NPC}$ nuclear pore complexes.
The NPCs are assumed to be sufficiently far apart for transport through each NPC to be independent.
Assuming (somewhat simplistically) that the nuclei are spherical, one can estimate a root-mean-square distances between NPCs of 0.44 $\mu$m for HeLa and 0.29 $\mu$m for yeast cells, based on the nuclear characteristics (Table~\ref{tbl:experimental}).
These values exceed the NPC transport channel diameter by at least 5-fold, and should indeed be sufficent for quasi-independent transport in most cases (as illustrated in Figure 6) \cite{Fabrikant1985}.
In mapping our model to the experimental data, we naturally ignore the chemical heterogeneity of FG domains and biocolloids.
Instead, we demonstrate that this level of detail is dispensible for a basic yet quantitative description of the NPC permselectivity barrier.

\bigskip\noindent
\textbf{Extracting NPC resistances from the experimental data}

\noindent
Table~\ref{tbl:experimental} highlights that the experimental conditions vary, and so do the reported transport rate observables.
To enable direct comparisons, we first converted all observables into resistances of individual NPCs.
To provide intuitively accessible numbers, we further considered translocation rates per NPC at a set probe colloid concentration of $\Delta c = 1 \mu \text{M}$.
The results are given in Tables~\ref{tbl:inert_probes} and \ref{tbl:attr_probes}, and Figure 8.

\begin{figure}[]
    \centering
    \includegraphics[width=0.9\linewidth]{fig/experiment_description.png}
    \caption{Reductionist view of relevant pore-mediated equilibration experiments from Refs.~\cite{Ribbeck2001, Mohr2009, Popken2015, Timney2016, Frey2018}.\\
    \textbf{Left:} The nucleus with finite volume $V_\text{nucleus}$ is separated from a finite cytosol ($V_\text{cytosol}$) or a quasi-infinite bulk solution by an impermeable envelope (green contour) perforated by $N_\text{NPC}$ FG-domain filled NPCs. 
    Biocolloids (yellow circles) are mobile, and a concentration difference $\Delta c$ drives their diffusive flux into the nucleus where they accumulate over time $c_\text{nucleus}(t)$.
    \textbf{Top right:} Isolated FG domain polymers phase-separate in a poor solvent with solvent strength $\chi_{\text{PS}}^{\text{FG}}$, forming droplets (red spheres) with polymer volume fraction $\phi_{\text{droplet}}$.
    \textbf{Bottom right:} Colloids equilibrate between the dilute and condensed droplet phases, characterized by the partition coefficient $P = c_\text{droplet}/c_\text{dilute}$.
    }
    \label{fig:experiments_overview}
\end{figure}

\bigskip\noindent
\textbf{Permeabilized human cells.}
In several studies \cite{Ribbeck2001, Mohr2009, Frey2018}, the plasma membrane of human HeLa cells was digitonin-permeabilized whilst leaving the nuclear membrane intact, fluorescent probe colloids were then delivered and the initial rate of probe concentration increase in the nucleus was quantified.
Here, the external solution represents a quasi-infinite reservoir with constant probe concentration $c_0$.
Under these conditions, the probe concentration in the nucleus initially increases linearly with time, $c_\text{nucleus} \approx k t c_0$ for $t \ll k^{-1}$, with the measured rate constant $k$.
From the rate constant, the experimentally determined resistance of each NPC in the nuclear envelope is obtained through: 
\begin{equation}
    R_\text{exp} = \frac{N_\text{NPC}}{k V_\text{nucleus}}.
    \label{eq:Exp_Resistance_1}
\end{equation}

\bigskip\noindent
\textbf{Intact yeast cells.}
Two studies \cite{Popken2015,Timney2016} photobleached the nuclear pool of fluorescent probe colloids in intact \textit{S. cerevisiae} cells, and then quantified the rate of influx of fluorescent probe molecules from the cytosol.
Here, the nuclear and cytosolic volumes are both finite, and the rate constant defining the transport process is given by
\begin{equation}
    k = t^{-1}\ln
    \left(
        \cfrac{1 + \cfrac{c_\text{nucleus}(t)}{c_\text{cytosol}(t)}
        \cfrac{V_\text{nucleus}}{V_\text{cytosol}}}
        {1 - \cfrac{c_\text{nucleus}(t)}{c_\text{cytosol}(t)}}
    \right)
\end{equation}
From the rate constant $k$, or equivalently the characteristic time $\tau = k^{-1}$, the resistance per NPC is obtained through:
\begin{equation}
    R_\text{exp} = \frac{N_\text{NPC}}{k} 
    \left(\frac{1}{V_\text{nucleus}} + \frac{1}{V_\text{cytosol}}\right)
    \label{eq:Exp_Resistance_2}
\end{equation} 

\bigskip\noindent
\textbf{Determining translocation rates per NPC.}
From the expirimentally determined NPC resistances, the translocation rates per NPC for a given molar probe concentration are readily obtained as:
\begin{equation}
    J_\text{exp} = \Delta c N_\text{A} / R_\text{exp}
    \label{eq:Translocation_rate}
\end{equation}

Naturally, this relation was also used to convert the theoretically predicted pore resistances into translocation rates.
Here, the solvent viscosity and temperature (which affect $D_0$) were set to $\eta_{\text{S}} = 1.45 \times 10^{-3} \text{ Pa s}$ and $T = 293 \text{ K}$, respectively, matching the conditions used for the experiments with HeLa cells \cite{Ribbeck2001}.

\bigskip\noindent
\textbf{Mapping model and experimental parameters for the transport of sticky colloids} 

Frey et al. \cite{Frey2018} used probe protein variants (GFP and mCherry) with varying FG domain attraction but similar size, to correlate their translocation rates across HeLa cell NPCs with their partition coefficients $P$ in phase-separated droplets of pure FG domains of either Nup116 from \textit{S. cerevisiae} or Nup98A from \textit{T. thermophilia}.

To compare these data with the predictions of our model, we estimated the effective solvent quality $\chi_\text{PS}^\text{FG}$ of the pure FG domain droplets, and the effective interaction strength $\chi_\text{PC}^\text{FG}$  between the colloids and the pure FG domains, as follows.

\bigskip\noindent
\textbf{Estimating $\chi_\text{PS}^\text{FG}$.} 
Nup98A and Nup116 FG domains undergo spontaneous phase separation due to chain cohesiveness, demixing into a dilute FG-poor phase and condensed FG-rich droplets.
The polymer concentration in the dilute phase, $\phi_\text{FG dilute}$, approximately equals the critical concentration for phase separation, and both chemical potentials and osmotic pressures are equal in the dilute and condensed phases \cite{Vovk2016, Zilman2018}.

The estimated critical concentration of Nup116 and Nup98A FG domains is $1 \mu\text{g}/\text{ml}$, corresponding to a FG domain volume fraction in the dilute phase of $\phi_\text{FG dilute} \approx 10^{-6}$ \cite{Schmidt2015}.
This extremely low value implies that the osmotic pressure is effectively zero, and allows to estimate the solvent quality from the condition for vanishing osmotic pressure (Eq.~(16)):
\begin{eqnarray}
    \chi_{\text{PS}}^{\text{FG}} = \frac{-\ln(1-\phi_\text{FG droplet}) - \phi_\text{FG droplet}}{\phi_\text{FG droplet}^2}
\end{eqnarray}

The concentration of Nup98A in phase-separated droplets was in a later study \cite{Ng2023} quantifed at 474 $\pm$ 32 mg/mL, and corresponds to a volume fraction $\phi_\text{Nup98A droplet}$ between 0.31 and 0.51 considering the range of plausible densities ($1.0 \text{ g/mL} < \rho < 1.4 \text{ g/mL}$, taking into account that some solvent may contribute to the effective FG domain volume).
The concentration of Nup116 was not quantified, but it is likely that it is lower than for Nup98A.
\todo{RR: Mikhail, you wrote that $c_\text{FG droplet} \approx 400 \text{mg/ml}$ was estimated, but I could not find this info anywhere. Can you please advise where to find it? We should reference it.
ML: Schmidt2015 estimated Nup98 200-300mg/ml among several taxa, and Nup116 particles to be approx 350 mg/ml for Saccharomyces cerevisiae,  Frey2007 thresholds by 200 mg/ml minimum for FG-domain}


For simplicity, we used $\chi_{\text{PS}}^{\text{FG}} = 0.6$ to represent the Nup98A and Nup116 droplets, consistent with our assumption for all NPCs.
This corresponds to a volume fraction of $\phi_\text{FG droplet} = 0.3$ \todo{ML: 0.245 to be precise, though it should not be important as we show in Fig S12}, i.e., between the real values for Nup98A and Nup116.
% Within the range $0.2 < \phi_\text{FG droplet} < 0.$, or $0.58 < \chi_{\text{PS}}^{\text{FG}} < 0.69$, the dependence of the transport rate on the partition coefficient $P$ is not very sensitive to the exact value of $\phi_\text{FG droplet}$ (Figure~\ref{fig:flux_vs_PC}).
Within the range $0.2 < \phi_\text{FG droplet} < 0.$, or $0.58 < \chi_{\text{PS}}^{\text{FG}} < 0.77$, the dependence of the transport rate on the partition coefficient $P$ is not very sensitive to the exact value of $\phi_\text{FG droplet}$ (Figure~\ref{fig:flux_vs_PC}).
\todo{RR: We may want to state briefly what Figure S12 shows in terms of the magnitude of variations.}

\bigskip\noindent
\textbf{Estimating $\chi_\text{PC}^\text{FG}$ for each probe protein.} 
The effective volume of the probe proteins was determined from the molecular mass $M_w$ as $V_\text{probe} = M_w / (N_\text{A} \rho)$, with Avogadro's number $N_\text{A}$ and $1.0 \text{ g/mL} < \rho < 1.4 \text{ g/mL}$.
The effective size of the probes was taken as the diameter of a sphere of equivalent volume:
\begin{eqnarray}
    d_\text{probe} =
    \left( \frac{6}{\pi} V_\text{probe} \right)^{1/3} =
    \left( \frac{6}{\pi} \frac{M_w}{N_\text{A} \rho} \right)^{1/3}
\end{eqnarray}
The molecular mass of probe proteins in this dataset was kept constant at $\sim$ 28 kDa (Table~\ref{tbl:attr_probes}), corresponding to an effective probe diameter between 4.0 and 4.5 nm, or between 5.2 and 5.9 when normalized by the segment length $a = 0.76 \text{ nm}$.
To simulate this system, we used $d_\text{probe} = 6$, i.e., the nearest even integer value compatible with our discretized model.

Since the osmotic pressure in the pure FG droplets is negligible, only surface interactions contribute to the insertion free energy $\Delta F_\text{FG droplet}$, such that:
\begin{equation}
    \Delta F_\text{FG droplet} =
    \frac{\pi d_\text{probe}^2}{6} \cdot \gamma\left(
    \phi_\text{FG droplet}, \chi_\text{PC}^\text{FG},
    \chi_\text{PS}^\text{FG}
    \right)
    \label{eq:FG_droplet_IFE}
\end{equation}
\begin{equation}
    P = \frac{c_\text{droplet}}{c_\text{dilute}} = e^{-\Delta F_\text{FG droplet}}
    \label{eq:FG_droplet_P}
\end{equation}
This set of equations enabled mapping of $\chi_\text{PC}^\text{FG}$ values to the measured partition coefficients $P$ between the droplet and dilute phases for each probe protein (Figure 8b), with the established estimates of $\phi_\text{FG droplet} = 0.3$, $\chi_{\text{PS}}^{\text{FG}} = 0.6$, and $d_\text{probe} = 6$.
% \todo{RR: Mikhail, please check if the values stated here are all correct.}

\begin{figure}[]
    \centering
    \includegraphics[width=3.5in]{fig/flux_vs_PC_SI.png}
    \caption{
    Effect of the polymer volume fraction in the FG domain droplet, $\phi_\text{FG droplet}$, on the relationship between the translocation rate across NPCs and the partitioning into FG domain droplets. 
    Theoretical predictions are presented as in Figure~8b for a colloid surface with a single sticky patch.
    % \todo{RR: Mikhail, please check if correct and amend as needed.}
    All parameters are as in Figure~8b, except for $\phi_\text{FG droplet}$ which was varied between 0.2 and 0.5 (as indicated), corresponding to $\chi_{\text{PS}}^{\text{FG}}$ between 0.58 and 0.77.
    $\chi_\text{PC}^\text{FG}$ values (not shown) were mapped through Eqs.~(\ref{eq:FG_droplet_IFE}-\ref{eq:FG_droplet_P}) to compute the translocation rates.
    % \todo{RR: Mikhail, the figure needs adjusting, for consistency of axis labels. I propose we remove the experimental data here. I have here tentatively suggested that we use a volume fraction range from 0.2 to 0.5 - let's review this once we have agreed on the estimates for Nup116.}
    }
    \label{fig:flux_vs_PC}
\end{figure}


\pagebreak
\onecolumn
%%%%%%%%
\section*{Supplementary Tables}
%%%%%%%%

\begin{table*}[h]
\begin{minipage}{\linewidth}
\centering
\caption{Translocation rates of non-sticky peptides and globular proteins across NPCs, calculated from the experimental data reported in the literature.
See references for details on the probe colloid constructs.}
\label{tbl:inert_probes}
\begin{tabular}{p{6cm}|p{2cm}|p{3.7cm}|p{2cm}}
Probe colloid & $M_w$ [kDa] & Translocation rate per NPC at $\Delta c = 1\mu\text{M}$ $[\text{s}^{-1}]$ & Reference \\
\hline
Fluorescein-cysteine & 0.5 & 231 & \cite{Mohr2009} \textsuperscript{a)} \\
11 amino acid peptide & 1.4 & 130 &  \\
Insulin & 5.8 & 59 &  \\
Aprotinin & 6.5 & 21.1 &  \\
Ubiquitin & 8.5 & 8.7 &  \\
Protein A z-domain & 8.2 & 9.9 &  \\
Thioredoxin & 13.9 & 5.0 &  \\
Lactalbumin & 14.2 & 3.54 &  \\
Green fluorscent protein (GFP) & 27.0 & 0.50 &  \\
Phosphate binding potein (PBP) & 37.0 & 0.064 &  \\
Maltose binding protein (MBP) & 43.0 & 0.054 &  \\
\hline
GFP & 26.8 & 1.11 & \cite{Timney2016} \textsuperscript{b)} \\
GFP-Protein A domain (PrA) & 34.2 & 0.268 &  \\
GFP-2$\times$PrA & 40.7 & 0.146 &  \\
GFP-3$\times$PrA & 46.8 & 0.092 &  \\
GFP-4$\times$PrA & 53.6 & 0.067 &  \\
GFP-6$\times$PrA & 66.8 & 0.040 &  \\
GFP-Protein G C2 domain & 34.7 & 0.83 &  \\
GFP-Protein G C2 and C3 domains & 42.3 & 0.25 &  \\
\hline
MBP-GFP-MBP & 109 & 0.0091 & \cite{Popken2015} \textsuperscript{b)} \\
MBP-GFP-2$\times$MBP & 149 & 0.00250 &  \\
MBP-GFP-4$\times$MBP & 230 & 0.00116 &  \\
MBP-3$\times$GFP & 122 & 0.0052 &  \\
MBP-4$\times$GFP & 150 & 0.00368 &  \\
MBP-5$\times$GFP & 177 & 0.00197 &  \\
\hline
%Bovine serum albumin (BSA) & 68 & $<0.1$ \textsuperscript{c)} & \cite{Ribbeck2001} \textsuperscript{d)} \\
%GFP & 29.0 & 2 \textsuperscript{c)} &  \\
%\hline
\todo{RR: Mikhail, I have here removed the data by Ribbeck et al. as these do not add any tangible data as far as I can see - BSA had an upper limit, which I suppose you did not include in Figure 8, and GFP data duplicated what was done in subsewquent studies by the same lab, albeit with slightly different results. See what you think, and let me know if you have a good reason to keep the data in!
ML: the goal was to give a sence of variations between experimental studies. There is data for BSA and GFP from Ribbeck et al. in Figure 8a (blue squares), should we remove it?
}
mCherry & 28.0 & 0.140 & \cite{Frey2018} \textsuperscript{d)} \\
GFP & 28.0 & 0.49 &  \\
efGFP\_8R & 30.0 & 2.66 &  \\
sffrGFP4 with 18$\times$R→K mutations & 28.0 & 0.53 &  \\
sffrGFP4 with 25$\times$R→K mutations & 27.0 & 0.224 &  \\
MBP & 43.0 & 0.0112 &  \\
MBP with K→R mutations & 43.0 & 0.45 &  \\
\end{tabular}
\end{minipage}
\vspace{0.5em}
\begin{minipage}{\textwidth}
\footnotesize
\textsuperscript{a)} Data for all probes in this source were included, except the profilin construct for which the molecular mass is unclear: the Stokes radius stated in the paper appears to small for the full human profilin 1 protein (15 kDa), suggesting only a part of the protein may have been used.
\textsuperscript{b)} Data for all probes in this source were included.
\textsuperscript{c)} Selected data from this source were included, representing inert probes.
%\textsuperscript{d)} Data taken directly from the source without modification.
\end{minipage}
\end{table*}


\begin{table*}[h]
\begin{minipage}{\linewidth}
\centering
\caption{Translocation rates across NPCs, and partition coefficients $P$ in phase-separated droplets of pure FG domains, of mCherry and variants of GFP with modified surface features, from Frey et al. \cite{Frey2018}.
See original paper for details on the probe protein constructs.}
\label{tbl:attr_probes}
\begin{tabular}{p{4cm}|p{2cm}|p{4.1cm}|p{2cm}|p{2cm}}
Probe protein & Protomer $M_w$ [kDa] & Translocation rate per NPC at $\Delta c = 1\mu\text{M}$ $[\text{s}^{-1}]$ \textsuperscript{a)} & $P$ in Nup98A droplets \textsuperscript{b)} & $P$ in Nup116 droplets \textsuperscript{b)} \\
\hline
mCherry & 28 & 0.140 & - & 0.09 \\
GFP & 28 & 0.49 & 0.11 & 0.33 \\
efGFP\_0W & 28 & 0.84 & 0.09 & 0.42 \\
efGFP\_3W & 27.5 & 11.5 & 1.50 & 14 \\
efGFP\_5W & 28 & 12.2 & 2.20 & 15 \\
efGFP\_8F & 28.3 & 6.0 & 8.30 & 51 \\
efGFP\_8L & 26.8 & 4.6 & 1.80 & 10 \\
efGFP\_8I & 28.2 & 9.5 & 2.90 & 23 \\
efGFP\_8M & 28.2 & 15.4 & 3 & 17 \\
efGFP\_8R & 30 & 2.66 & 0.50 & 1.90 \\
sffrGFP4 & 29 & 22.4 & 14 & 50 \\
sffrGFP4 & 29 & 22.4 & 14 & 50 \\
sffrGFP4 18xR→K & 28 & 0.53 & 0.12 & 0.31 \\
sffrGFP4 25xR→K & 27 & 0.22 & 0.06 & 0.10 \\
sffrGFP4 & 29 & 22.4 & 14 & 50 \\
sffrGFP4 & 29 & 22.4 & 14 & 50 \\
sffrGFP5 & 28 & 9.0 & 0.67 & 5.50 \\
sffrGFP6 & 29 & 39.2 & 100 & 160 \\
sffrGFP7 & 29 & 43 & 200 & 200 \\
GFP\_MaxR\_5W & 28 & 116 & 2100 & 4000 \\
GFP\_MaxR\_8i & 27.6 & 182 & 2000 & 4100 \\
GFPNTR\_2B7 & 27.1 & 224 & 1200 & 1600 \\
GFPNTR\_7B3 & 26.3 & 238 & 1700 & 1400 \\
GFPNTR\_3B1 & 28.5 & 60 & 290 & - \\
GFPNTR\_3B7 & 27.5 & 106 & 3000 & - \\
GFPNTR\_3B8 & 27.5 & 94 & 1800 & - \\
GFPNTR\_3B9 & 28.3 & 122 & 4800 & - \\
\end{tabular}
\end{minipage}
\vspace{0.5em}
\begin{minipage}{\textwidth}
\footnotesize
\textsuperscript{a)} Translocation rates were calculated from the data in the original paper, as described in Supplementary Note 7.
\textsuperscript{b)} Partition coefficients are reproduced from the original paper.
\end{minipage}
\end{table*}


\begin{table*}[h]
\begin{minipage}{\linewidth}
\centering
\caption{Analysis of the content of intrinsically disordered regions (IDRs) in FG nucleoporins in yest NPCs.}
\label{tbl:idr_npc}
\begin{tabular}{p{4cm}|p{4cm}|p{4cm}|p{4cm}}
FG nucleoporin in \textit{S. cerevisiae} \textsuperscript{a)} & Number of amino acids in IDR per protein \textsuperscript{b)} & Protein copy number per NPC \textsuperscript{c)} & Number of amino acids in IDRs per NPC \\
\hline
Nsp1       & 617 & 48  & 29616 \\
Nup49      & 251 & 32  & 8032  \\
Nup57      & 255 & 32  & 8160  \\
Nup145N    & 433 & 16  & 6928  \\
Nup116 \textsuperscript{d)}  & 960 & 16  & 15360 \\
Nup100     & 800 & 16  & 12800 \\
Nup60 \textsuperscript{e)}   & 539 & 16  & 8624  \\
Nup1       & 857 & 16  & 13712 \\
Nup42      & 382 & 8   & 3056  \\
Nup159     & 685 & 16  & 10960 \\
\hline
Total      &     & 216 & 117248\\
\end{tabular}
\end{minipage}
\vspace{0.5em}
\begin{minipage}{\textwidth}
\footnotesize
\textsuperscript{a)} Nup2 was excluded as this protein is non-essential.
\textsuperscript{b)} According to Yamada et al.~\cite{Yamada2010}.
\textsuperscript{c)} According to Kim et al.~\cite{Kim2018}.
\textsuperscript{d)} The full C-terminal region up to amino acid 960 was considered IDR.
\textsuperscript{e)} The full protein was considered disordered, based on prediction.
\end{minipage}
\end{table*}


\begin{table*}[h]
\begin{minipage}{\linewidth}
\centering
\caption{Transport-related quantities extracted from the experimental studies.}
\label{tbl:experimental}
\begin{tabular}{p{2.8cm}|p{7cm}|p{2cm}|p{1cm}|p{1cm}|p{1cm}}
Study & Reported quantity & Nuclei from & $N_\text{NPC}$ & $V_\text{nucleus}$ [fL] & $V_\text{cytosol}$ [fL] \\
\hline
Ribbeck et al. \cite{Ribbeck2001} & Translocation rate per NPC at $\Delta c = 1 \mu\text{M}$ & HeLa cells & 2770 & 1130 & $\infty$ \\
Mohr et al. \cite{Mohr2009} & Rate constant $k$ &  &  &  &  \\
Frey et al. \cite{Frey2018} & Rate constant $k$; \newline  correlated with partitioning coefficient in phase-separated droplets of  \textit{S. cerevisiae} Nup116 and \textit{T. thermophilia} Nup98A &  &  &  &  \\
\hline
Popken et al. \cite{Popken2015} & Nuclear/cytoplasm concentration $c_{\text{in}}/c_{\text{out}}$ at $t = 1 \text{h}$ & \textit{S. cerevisiae} & 161 & 4.8 & 60 \\
Timney et al. \cite{Timney2016} & Characteristic time $\tau$ &  &  &  &  \\
\end{tabular}
\end{minipage}
\end{table*}
\pagebreak
\onecolumn

\printbibliography
\end{document}