\documentclass[12pt, a4paper]{article}
\usepackage{graphicx}
\usepackage{amsmath, amssymb, amsfonts, mathtools}
\usepackage{subcaption}
\usepackage{xcolor}
\usepackage{bm}

\begin{document}
\section{Linking local resistivity to global transport}

% Having defined the position-dependent insertion free energy and colloid diffusivity, we can develop an analytical method to estimate the total resistance of the brush-filled pore to the diffusive flow of colloids.

Diffusive transport in the presence of an external force generated by the insertion free energy $\Delta F$ in a position-dependent diffusion coefficient is described by the Smoluchowski equation,
\begin{equation}
    \frac{\partial c(\bm{r})}{\partial t}=-\nabla \cdot \Bigl(D(\bm{r}) \Bigl[\nabla c({\bm{r}})+c({\bm{r}})\nabla\bigl(\Delta F(\bm{r})\bigr)\Bigr]\Bigr),
    \label{eq:Smoluch}
\end{equation}

The next function transformation defined:
\begin{eqnarray}
    \begin{gathered}
        \tilde{c}(\bm{r}) = c(\bm{r})\exp(\Delta F(\bm{r}))
        \\
        \tilde{D}(\bm{r}) = D(\bm{r})\exp(-\Delta F(\bm{r})) 
    \end{gathered}
    \label{eq:transform}
\end{eqnarray}
where $\tilde{c}(\bm{r})$ and $\tilde{D}(\bm{r})$ are modified colloid concentration and modified position-dependent diffusion coefficient, respectively.

In our case the boundary conditions for the modified concentration are $\tilde{c}(z\rightarrow -\infty)=\Delta c$ and $\tilde{c}(z\rightarrow +\infty)=0$ since the insertion free energy $\Delta F$ vanishes far away from the pore.

With the defined transformations Eq.~\ref{eq:transform} and boundary conditions, modified concentration is divergence free $\nabla \tilde{c}(\bm{r}) = 0$ and the Smoluchowski equation (Eq.~(\ref{eq:Smoluch})) have a from of a generalized Laplace equation.
\begin{eqnarray}
    \mathcal{L} \tilde{c} = 0
    \label{eq:laplace}
\end{eqnarray}
where $\mathcal{L}  = \nabla \tilde{D}(\bm{r}) \nabla$ is Smoluchowski diffusion operator. 

If stationary modified concentration $\tilde{c}(\bm{r})$ is known, this recovers flux density field
with the flux density tracking a Fickian form:
\begin{eqnarray}
    %\bm{j}(\bm{r}) = \mathcal{J} \tilde{c}(\bm{r})
    \bm{j}(\bm{r}) = -\tilde{D}(\bm{r}) \nabla \tilde{c}(\bm{r})
\end{eqnarray}
%where $\mathcal{J}  = -\tilde{D}(\bm{r}) \nabla$ is Smoluchowski flux operator.


At the central crossection of the pore $z=0$, the flux density has only axial component to it $j=j_z$, thus the total flux through the pore is found by integrating flux density over the central crossection:
\begin{eqnarray}
    %\tilde{D}(r',z=0)\frac{\partial \tilde{c}(r',z=0)}
    J = 2\pi\int_{0}^{r_p}j(r',z=0) r' \text{d}r'
\end{eqnarray}

Finally, as we assume in the pore the flux lines to be parallel with teh resistance of the internal region:
\begin{eqnarray}
    R_{\text{int}} = 2 \pi \int\limits_{-L/2}^{L/2} \Bigg[\int_{0}^{r_p} \tilde{D}(r',z') r' \text{d}r' \Bigg]^{-1} \text{d}z'
\end{eqnarray}
the rest of the resistance is associated with modulated by the fringes resistance of the external region, as in Eq.~{eq.num}.
\begin{eqnarray}
    R_{\text{ext}} = \Delta c/J - R_{\text{int}}
\end{eqnarray}
where  $\Delta c/J$ is the total resistance of the system as in Eq.~(eq.num)

To find stationary solution of Smoluchowski equation we select large enough finite domain  $(r,z) \in [0, r_{\text{max}}] \times [z_{\text{min}}, z_{\text{max}}]$ the continuous space is discretized to a cylindrical regular lattice with $\Delta z, \Delta r$ with $N_r = r_{\text{max}}/\Delta r$ points in radial direction and $N_z = (z_{\text{max}} - z_{\text{min}})/\Delta z$ points in axial direction $N_r \times N_z$, indexed with $i \in [0, N_r]$ and $k \in [0, N_z]$ respectively, such that $r_i = i$ and $z_k =  z_{\text{min}} + k$.

We map all position-dependent properties defined on two-dimmension discrete domain $(i, k) \in [0, N_r]\times[0, N_z]$ by flattening them in vectors reindexed with $m$ as $m = i N_z + k$, then in the discrete domain the Smoluchowski diffusion operator $\mathcal{L}$ takes a sparse matrix form $\mathbf{L}$ with dimensions $[N_r N_z \times N_r N_z]$,  flattened vector of unknown modified concentrations $\tilde{\bm{c}}$ with dimmensions $[N_r N_z]$

the Eq.~(\ref{eq:laplace}) is rewritten in the matrix form as system of linear equations:
\begin{eqnarray}
    \mathbf{L} \tilde{\bm{c}} = \bm{b}
    \label{eq:matrix_form}
\end{eqnarray}
where vector $\bm{b}$ enforces the boundary conditions.
% We set the boundary conditions Dirichlet boundary conditions at the edges of axial coordinates $\tilde{c}_{i,k=0} = c_0$ and $\tilde{c}_{i,k=N_z-1} = 0$, and Neumann (reflective) boundary condition at the edge of radial coordinates $\tilde{c}_{i=N_r-1,k} = \tilde{c}_{i=N_r-2,k}$.

The boundary conditions and the effects of impermeable membrane are reflected in rhs vector $b$, and operator matrix $\mathbf{L}$ (see Supplementary Method).
From the solution $\tilde{c}_m$ we recover modified concentration reshaped to $\tilde{c}_{r,z}$ and physical concentration $\tilde{c}_{r,z} \exp{-\Delta F_{r,z}}$.

with the total colloid flux calculated from the discrete values of $\tilde{\bm{c}}$ vector:
\begin{eqnarray}
    J \approx 2 \pi \sum_{i=0}^{N_r} \tilde{D}_{N_z(i+1/2)} \frac{\tilde{c}_{N_z(i+1/2)} - \tilde{c}_{N_z(i+1/2)}}{\Delta z} i \Delta r
\end{eqnarray}





\end{document}