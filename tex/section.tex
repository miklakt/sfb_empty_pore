\documentclass[12pt, a4paper]{article}
\usepackage{graphicx}
\usepackage{amsmath, amssymb, amsfonts, mathtools}
\usepackage{subcaption}
\usepackage{xcolor}
\usepackage{bm}

\begin{document}

\subsection*{Linking Local Resistivity to Global Transport}

Diffusive transport in the presence of an external force originating from the insertion free energy $\Delta F$ and a position-dependent diffusion coefficient $D(\bm r)$ is described by Smoluchowski equation
\begin{equation}
  \frac{\partial c(\bm r)}{\partial t}
  = -\nabla\!\cdot\!\Bigl[
     D(\bm r)\Bigl(\nabla c(\bm r)+c(\bm r)\nabla\Delta F(\bm r)\Bigr)
     \Bigr].
  \label{eq:Smoluch}
\end{equation}
% with the flux density
% \begin{equation}
%     \bm{j}=- D(\bm{r})\left[\nabla c({\bm{r}})+c({\bm{r}}) \nabla \left(\Delta F(\bm{r})\right)\right]
%     \label{eq:j}
% \end{equation}

We introduce the transformations
\begin{eqnarray}
  \tilde c(\bm r)=c(\bm r)\,e^{\Delta F(\bm r)}, \label{eq:transform_1}\\
  \tilde D(\bm r)=D(\bm r)\,e^{-\Delta F(\bm r)}, \label{eq:transform_2}
\end{eqnarray}
which define a modified concentration $\tilde c(\bm r)$ and a modified diffusion coefficient $\tilde D(\bm r)$.

The modified diffusion coefficient $\tilde D(\bm r)$ can be interpreted as the position-dependent local conductivity.
Because the polymer volume fraction $\phi(\bm{r})$ varies throughout the pore, the local modified diffusion coefficient $\tilde{D}(r,z)$ is likewise position-dependent.
Figure~3a %\ref{fig:D_fe_map}a 
separates the two underlying contributions:
(i) the reduction of the colloid diffusivity in a homogeneous polymer mesh of concentration $\phi$ and
(ii) the net insertion free energy.\\
Adding these curves gives the net effect on resistance,
\begin{equation*}
    -\ln\!\bigl[\tilde {D}(\phi)/D_{0}\bigr]= -\ln\!\bigl[D(\phi)/D_{0}\bigr] + \Delta F(\phi),
\end{equation*}
as stated in Eq.~(\ref{eq:transform_2}). 
Only for sufficiently attractive particles, where $\Delta F(\phi)$ is negative enough, does the sum become positive, $\ln\!\bigl[\tilde {D}(\phi)/D_{0}\bigr] > 0$, indicating that the brush locally reduces the resistance.  
Otherwise the net effect remains positive and the resistance increases.

In our case the boundary conditions for the modified concentration are 
\begin{equation*}
  \tilde c(z\to-\infty)=\Delta c,\qquad
  \tilde c(z\to+\infty)=0.
\end{equation*}
since the insertion free energy $\Delta F$ vanishes far away from the pore.

Applying Eq.~(\ref{eq:transform_1},\ref{eq:transform_2}) and the boundary conditions reduces the Smoluchowski equation Eq.~\eqref{eq:Smoluch} in the stationary state to the generalized Laplace equation
\begin{equation}
  \mathcal L\tilde c=0,
  \label{eq:laplace}
\end{equation}
with the Smoluchowski diffusion operator
\begin{equation*}
  \mathcal L=\nabla\!\cdot\!\bigl(\tilde D(\bm r)\nabla\bigr).
\end{equation*}

Under stationary conditions, the flux density is divergence free. 
Once the stationary field $\tilde c(\bm r)$ is known, the flux density is found from the generalized Fickian equation:
\begin{equation}
  \bm j(\bm r)=-\tilde D(\bm r)\nabla\tilde c(\bm r).
\end{equation}

At the mid-plane $z=0$ the flux is purely axial, $j=j_z$. The total pore flux is therefore
\begin{equation}
  J=2\pi\int_{0}^{r_p} j(r',z=0)\,r'\,\mathrm dr'.
\end{equation}

Assuming parallel flux lines inside the pore, the isosurfaces $\tilde c(r,z)$ are taken as  discs of radius $r_{\text{p}}$ normal to the pore axis.
Analogous to a set of resistors connected in parallel, the total conductivity of a layer between two adjacent equipotential surfaces is obtained by integration of local conductivities $\tilde D(r,z)$ over the layer.
On the other hand, since the consecutive layers are connected in series, their total resistance can be found by appropriate integration.
Within the pore, $|z|\leq L/2$, the resistance of the interior region.
\begin{equation}
  R_{\text{int}}
   =2\pi\int_{-L/2}^{L/2}
     \Bigl[\,
       \int_{0}^{r_p}\tilde D(r',z')\,r'\,\mathrm dr'
     \Bigr]^{-1}\!\mathrm dz'.
\end{equation}

The remaining resistance is attributed to the exterior resistance modulated by loose fringes, as in Eq.~(?),
\begin{equation}
  R_{\text{ext}}=\frac{\Delta c}{J}-R_{\text{int}},
\end{equation}
where $\Delta c/J$ is the total resistance Eq.~(?).

\textbf{Numerical solution}

To obtain a stationary solution of Eq.~\eqref{eq:laplace} we consider a large enough to exclude edge effects finite domain
\begin{equation*}
  (r,z)\in[0,r_{\max}]\times[z_{\min},z_{\max}],
\end{equation*}
and discretize it on a regular cylindrical lattice with spacings $\Delta r=\Delta z$.
The grid contains $N_r=r_{\max}/\Delta r+1$ radial nodes and $N_z=(z_{\max}-z_{\min})/\Delta z+1$ axial nodes, indexed by $i=0,\dots,N_r-1$ and $k=0,\dots,N_z-1$, respectively. Physical coordinates are $r_i=i\Delta r$ and $z_k=z_{\min}+k\Delta z$.

All fields defined on the two-dimensional grid are flattened into vectors of length $N_rN_z$ reindexed with $m=iN_z+k$.
Under this mapping, the continuous operator $\mathcal L$ becomes a sparse $[N_rN_z\times N_rN_z]$ matrix $\mathbf L$, and the unknown values of $\tilde c(r_i, z_i)$ form the column $[N_rN_z]$ vector $\tilde{\bm c}$.

Equation~\eqref{eq:laplace} is then written in matrix form as
\begin{equation}
  \mathbf L\,\tilde{\bm c}=\bm b,
  \label{eq:matrix_form}
\end{equation}
where the right-hand side $\bm b$ enforces (the Dirichlet and Neumann) boundary conditions.
For the nodes that fall to impermeable membrane no-flux conditions are set (see Supplementary Method).

After solving for $\tilde{\bm c}$ we reshape the vector back to $\tilde{c}(r,z)$ and retrieve the physical concentration $c(r,z)=\tilde c_(r,z)\,e^{-\Delta F_(r,z)}$.

The total colloid flux is approximated on the grid by
\begin{equation}
  J\approx 2\pi\sum_{i=0}^{N_r-1}
      \tilde D_{i N_z + k_0}\,
      \frac{\tilde c_{i N_z + k_0}-\tilde c_{i N_z + k_0}}{\Delta z}\,
      r_i\,\Delta r,
\end{equation}
where $k_0$ denotes the index corresponding to $z=0$.


\end{document}
